\documentclass{article}
\usepackage{amsmath}
\begin{document}

\section*{Lemma 1}
\begin{lemmatheorembox}
\textbf{Name:} Rule of product
\textbf{Statement:} We have \( |S_{1}|\cdot |S_{2}|\cdots |S_{n}| = |S_{1}\times S_{2}\times \cdots \times S_{n}| \).
\textbf{General Topic:} Combinatorics
\textbf{URL:} https://en.wikipedia.org/wiki/Rule_of_product
\end{lemmatheorembox}

\textbf{Usage count:} 171

\section*{Lemma 2}
\begin{lemmatheorembox}
\textbf{Name:} Pythagorean theorem
\textbf{Statement:} a^2 + b^2 = c^2 .
\textbf{General Topic:} Euclidean geometry
\textbf{URL:} https://en.wikipedia.org/wiki/Pythagorean_theorem
\end{lemmatheorembox}

\textbf{Usage count:} 144

\section*{Lemma 3}
\begin{lemmatheorembox}
\textbf{Name:} Fundamental theorem of arithmetic
\textbf{Statement:} every integer greater than 1 is either prime or can be represented uniquely as a product of prime numbers, up to the order of the factors.
\textbf{General Topic:} Number theory
\textbf{URL:} https://en.wikipedia.org/wiki/Fundamental_theorem_of_arithmetic
\end{lemmatheorembox}

\textbf{Usage count:} 121

\section*{Lemma 4}
\begin{lemmatheorembox}
\textbf{Name:} Geometric series
\textbf{Statement:} The partial sum of the first n+1 terms of a geometric series, up to and including the r\textasciicircum n term, S\_n = ar\textasciicircum 0 + ar\textasciicircum 1 + \textbackslash cdots + ar\textasciicircum n = \textbackslash sum\_\{k=0\}\textasciicircum n ar\textasciicircum k, is given by the closed form S\_n = \textbackslash begin\{cases\} a(n+1) & r=1 \textbackslash\textbackslash  a\textbackslash left(\textbackslash frac\{1-r\textasciicircum\{n+1\}\}\{1-r\}\textbackslash right) & \textbackslash text\{otherwise\} \textbackslash end\{cases\} where r is the common ratio.
\textbf{General Topic:} Series
\textbf{URL:} https://en.wikipedia.org/wiki/Geometric_series
\end{lemmatheorembox}

\textbf{Usage count:} 93

\section*{Lemma 5}
\begin{lemmatheorembox}
\textbf{Name:} Binomial coefficient
\textbf{Statement:} In combinatorics the symbol \tbinom{n}{k} is usually read as "n choose k" because there are \tbinom{n}{k} ways to choose an (unordered) subset of k elements from a fixed set of n elements.
\textbf{General Topic:} Combinatorics
\textbf{URL:} https://en.wikipedia.org/wiki/Binomial_coefficient
\end{lemmatheorembox}

\textbf{Usage count:} 92

\section*{Lemma 6}
\begin{lemmatheorembox}
\textbf{Name:} Mathematical induction
\textbf{Statement:} These two steps establish that the statement holds for every natural number n.
\textbf{General Topic:} Proof techniques
\textbf{URL:} https://en.wikipedia.org/wiki/Mathematical_induction
\end{lemmatheorembox}

\textbf{Usage count:} 85

\section*{Lemma 7}
\begin{lemmatheorembox}
\textbf{Name:} Addition principle
\textbf{Statement:} In mathematical terms, the addition principle states that, for disjoint sets A and B, we have $|A\cup B|=|A|+|B|$, provided that the intersection of the sets is without any elements.
\textbf{General Topic:} Combinatorics
\textbf{URL:} https://en.wikipedia.org/wiki/Addition_principle
\end{lemmatheorembox}

\textbf{Usage count:} 68

\section*{Lemma 8}
\begin{lemmatheorembox}
\textbf{Name:} Area of a triangle
\textbf{Statement:} Substituting this in the area formula \({\tfrac {1}{2}}bh\) derived above, the area of the triangle can be expressed as: \({\tfrac {1}{2}}ab\sin \gamma ={\tfrac {1}{2}}bc\sin \alpha ={\tfrac {1}{2}}ca\sin \beta\).
\textbf{General Topic:} Euclidean geometry
\textbf{URL:} https://en.wikipedia.org/wiki/Area_of_a_triangle
\end{lemmatheorembox}

\textbf{Usage count:} 61

\section*{Lemma 9}
\begin{lemmatheorembox}
\textbf{Name:} Pigeonhole principle
\textbf{Statement:} If there are more items than boxes holding them, one box must contain at least two items
\textbf{General Topic:} Combinatorics
\textbf{URL:} https://en.wikipedia.org/wiki/Pigeonhole_principle
\end{lemmatheorembox}

\textbf{Usage count:} 57

\section*{Lemma 10}
\begin{lemmatheorembox}
\textbf{Name:} Binomial theorem
\textbf{Statement:} \((x+y)^{n}=\sum _{k=0}^{n}{\binom {n}{k}}x^{n-k}y^{k}=\sum _{k=0}^{n}{\binom {n}{k}}x^{k}y^{n-k}.\)
\textbf{General Topic:} Algebra
\textbf{URL:} https://en.wikipedia.org/wiki/Binomial_theorem
\end{lemmatheorembox}

\textbf{Usage count:} 53

\section*{Lemma 11}
\begin{lemmatheorembox}
\textbf{Name:} Linearity of expectation
\textbf{Statement:} The expected value operator (or expectation operator) \operatorname{E}[\cdot] is linear in the sense that, for any random variables X and Y, and a constant a,
\begin{align}
\operatorname{E}[X+Y]&=\operatorname{E}[X]+\operatorname{E}[Y],\\
\operatorname{E}[aX]&=a\operatorname{E}[X],
\end{align}
whenever the right-hand side is well-defined. By induction, this means that the expected value of the sum of any finite number of random variables is the sum of the expected values of the individual random variables, and the expected value scales linearly with a multiplicative constant. Symbolically, for N random variables X_{i} and constants a_{i} (1\leq i \leq N), we have \operatorname{E}\left[\sum_{i=1}^{N}a_{i}X_{i}\right] = \sum_{i=1}^{N}a_{i}\operatorname{E}[X_{i}].
\textbf{General Topic:} Probability theory
\textbf{URL:} https://en.wikipedia.org/wiki/Expected_value
\end{lemmatheorembox}

\textbf{Usage count:} 53

\section*{Lemma 12}
\begin{lemmatheorembox}
\textbf{Name:} Vieta's formulas
\textbf{Statement:} The roots \(r_{1},r_{2}\) of the quadratic polynomial \(P(x)=ax^{2}+bx+c\) satisfy \(r_{1}+r_{2}=-{\frac {b}{a}},\quad r_{1}r_{2}={\frac {c}{a}}.\)
\textbf{General Topic:} Algebra
\textbf{URL:} https://en.wikipedia.org/wiki/Vieta%27s_formulas
\end{lemmatheorembox}

\textbf{Usage count:} 51

\section*{Lemma 13}
\begin{lemmatheorembox}
\textbf{Name:} AA postulate
\textbf{Statement:} In Euclidean geometry, the AA postulate states that two triangles are similar if they have two corresponding angles congruent.
\textbf{General Topic:} Euclidean geometry
\textbf{URL:} https://en.wikipedia.org/wiki/AA_postulate
\end{lemmatheorembox}

\textbf{Usage count:} 50

\section*{Lemma 14}
\begin{lemmatheorembox}
\textbf{Name:} Trigonometric identities
\textbf{Statement:} \(\cos(\alpha+\beta)=\cos\alpha\cos\beta-\sin\alpha\sin\beta.\)
\textbf{General Topic:} Trigonometry
\textbf{URL:} https://en.wikipedia.org/wiki/List_of_trigonometric_identities
\end{lemmatheorembox}

\textbf{Usage count:} 50

\section*{Lemma 15}
\begin{lemmatheorembox}
\textbf{Name:} Quadratic formula
\textbf{Statement:} \(x = \frac{-b \pm \sqrt{b^2 - 4ac}}{2a},\)
\textbf{General Topic:} Elementary algebra
\textbf{URL:} https://en.wikipedia.org/wiki/Quadratic_formula
\end{lemmatheorembox}

\textbf{Usage count:} 49

\section*{Lemma 16}
\begin{lemmatheorembox}
\textbf{Name:} Euclid's lemma
\textbf{Statement:} Euclid's lemma—If a prime p divides the product ab of two integers a and b, then p must divide at least one of those integers a or b.
\textbf{General Topic:} Number theory
\textbf{URL:} https://en.wikipedia.org/wiki/Euclid%27s_lemma
\end{lemmatheorembox}

\textbf{Usage count:} 47

\section*{Lemma 17}
\begin{lemmatheorembox}
\textbf{Name:} Inclusion–exclusion principle
\textbf{Statement:} \(\left| \bigcup_{i=1}^n A_i\right| = \sum_{\emptyset\neq J\subseteq\{1,\ldots,n\}}(-1)^{|J|+1}\left |\bigcap_{j\in J} A_j\right|.\)
\textbf{General Topic:} Combinatorics
\textbf{URL:} https://en.wikipedia.org/wiki/Inclusion%E2%80%93exclusion_principle
\end{lemmatheorembox}

\textbf{Usage count:} 47

\section*{Lemma 18}
\begin{lemmatheorembox}
\textbf{Name:} AM--GM inequality
\textbf{Statement:} The simplest non-trivial case is for two non-negative numbers \(x\) and \(y\), that is, \(\frac{x+y}{2} \ge \sqrt{xy}\) with equality if and only if \(x = y\).
\textbf{General Topic:} Inequalities
\textbf{URL:} https://en.wikipedia.org/wiki/AM%E2%80%93GM_inequality
\end{lemmatheorembox}

\textbf{Usage count:} 45

\section*{Lemma 19}
\begin{lemmatheorembox}
\textbf{Name:} Inscribed angle
\textbf{Statement:} The measure of an inscribed angle is half the measure of the central angle that subtends the same arc.
\textbf{General Topic:} Euclidean geometry
\textbf{URL:} https://en.wikipedia.org/wiki/Inscribed_angle
\end{lemmatheorembox}

\textbf{Usage count:} 45

\section*{Lemma 20}
\begin{lemmatheorembox}
\textbf{Name:} Chinese remainder theorem
\textbf{Statement:} has a solution, and any two solutions, say x _{1} and x _{2}, are congruent modulo N, that is, x _{1} ≡ x _{2} (mod N ).
\textbf{General Topic:} Number theory
\textbf{URL:} https://en.wikipedia.org/wiki/Chinese_remainder_theorem
\end{lemmatheorembox}

\textbf{Usage count:} 44

\section*{Lemma 21}
\begin{lemmatheorembox}
\textbf{Name:} Triangle inequality
\textbf{Statement:} In a metric space \(M\) with metric \(d\), the triangle inequality is a requirement upon distance: \(d(A,\ C)\leq d(A,\ B)+d(B,\ C)\ ,\) for all points \(A\), \(B\), and \(C\) in \(M\). That is, the distance from \(A\) to \(C\) is at most as large as the sum of the distance from \(A\) to \(B\) and the distance from \(B\) to \(C\).
\textbf{General Topic:} Metric spaces
\textbf{URL:} https://en.wikipedia.org/wiki/Triangle_inequality
\end{lemmatheorembox}

\textbf{Usage count:} 43

\section*{Lemma 22}
\begin{lemmatheorembox}
\textbf{Name:} Law of cosines
\textbf{Statement:} For a triangle with sides \(a\), \(b\), and \(c\), opposite respective angles \(\alpha\), \(\beta\), and \(\gamma\) (see Fig. 1), the law of cosines states:
\begin{align}
c^2 &= a^2 + b^2 - 2ab\cos\gamma, \\[3mu]
a^2 &= b^2+c^2-2bc\cos\alpha, \\[3mu]
b^2 &= a^2+c^2-2ac\cos\beta.
\end{align}
\textbf{General Topic:} Trigonometry
\textbf{URL:} https://en.wikipedia.org/wiki/Law_of_cosines
\end{lemmatheorembox}

\textbf{Usage count:} 40

\section*{Lemma 23}
\begin{lemmatheorembox}
\textbf{Name:} Divisor function
\textbf{Statement:} \(\sigma_x(n) = \prod_{i=1}^r \sum_{j=0}^{a_i} p_i^{j x} = \prod_{i=1}^r \left (1 + p_i^x + p_i^{2x} + \cdots + p_i^{a_i x} \right ).\)
\textbf{General Topic:} Number theory
\textbf{URL:} https://en.wikipedia.org/wiki/Divisor_function
\end{lemmatheorembox}

\textbf{Usage count:} 35

\section*{Lemma 24}
\begin{lemmatheorembox}
\textbf{Name:} Euclidean division
\textbf{Statement:} Given two integers \(a\) and \(b\), with \(b\neq 0\), there exist unique integers \(q\) and \(r\) such that \(a=bq+r\), and \(0\leq r<|b|\), where \(|b|\) denotes the absolute value of \(b\).
\textbf{General Topic:} Arithmetic
\textbf{URL:} https://en.wikipedia.org/wiki/Euclidean_division
\end{lemmatheorembox}

\textbf{Usage count:} 34

\section*{Lemma 25}
\begin{lemmatheorembox}
\textbf{Name:} Divisibility rule
\textbf{Statement:} The last digit is 0.
\textbf{General Topic:} Number theory
\textbf{URL:} https://en.wikipedia.org/wiki/Divisibility_rule
\end{lemmatheorembox}

\textbf{Usage count:} 34

\section*{Lemma 26}
\begin{lemmatheorembox}
\textbf{Name:} Similarity (geometry)
\textbf{Statement:} Two triangles, △ABC and △A'B'C' are similar if and only if corresponding angles have the same measure: this implies that they are similar if and only if the lengths of corresponding sides are proportional.[ 1 ]
\textbf{General Topic:} Euclidean geometry
\textbf{URL:} https://en.wikipedia.org/wiki/Similarity_(geometry)
\end{lemmatheorembox}

\textbf{Usage count:} 32

\section*{Lemma 27}
\begin{lemmatheorembox}
\textbf{Name:} Factor theorem
\textbf{Statement:} Specifically, if \(f(x)\) is a (univariate) polynomial, then \(x-a\) is a factor of \(f(x)\) if and only if \(f(a)=0\) (that is, \(a\) is a root of the polynomial).
\textbf{General Topic:} Algebra
\textbf{URL:} https://en.wikipedia.org/wiki/Factor_theorem
\end{lemmatheorembox}

\textbf{Usage count:} 30

\section*{Lemma 28}
\begin{lemmatheorembox}
\textbf{Name:} Difference of two squares
\textbf{Statement:} The difference of two squares is a squared (multiplied by itself) number subtracted from another squared number. It refers to the identity $a^2 - b^2 = (a - b)(a + b)$.
\textbf{General Topic:} Algebra
\textbf{URL:} https://en.wikipedia.org/wiki/Difference_of_two_squares
\end{lemmatheorembox}

\textbf{Usage count:} 30

\section*{Lemma 29}
\begin{lemmatheorembox}
\textbf{Name:} Law of total probability
\textbf{Statement:} The law of total probability is a theorem that states, in its discrete case, if $\left\{{B_{n}:n=1,2,3,\ldots }\right\}$ is a finite or countably infinite set of mutually exclusive and collectively exhaustive events, then for any event $A$ \[P(A)=\sum _{n}P(A\cap B_{n})\] or, alternatively, \[P(A)=\sum _{n}P(A\mid B_{n})P(B_{n}),\] where, for any $n$, if $P(B_{n})=0$, then these terms are simply omitted from the summation since $P(A\mid B_{n})$ is finite.
\textbf{General Topic:} Probability theory
\textbf{URL:} https://en.wikipedia.org/wiki/Law_of_total_probability
\end{lemmatheorembox}

\textbf{Usage count:} 29

\section*{Lemma 30}
\begin{lemmatheorembox}
\textbf{Name:} Triangular number
\textbf{Statement:} {\displaystyle {\begin{aligned}T_{n}&=\sum _{k=1}^{n}k=1+2+\dotsb +n\\&={\frac {n^{2}+n{\vphantom {(n+1)}}}{2}}={\frac {n(n+1)}{2}}\\&={n+1 \choose 2}\end{aligned}}}
\textbf{General Topic:} Elementary mathematics
\textbf{URL:} https://en.wikipedia.org/wiki/Triangular_number
\end{lemmatheorembox}

\textbf{Usage count:} 28

\section*{Lemma 31}
\begin{lemmatheorembox}
\textbf{Name:} Distributive property
\textbf{Statement:} $x\cdot (y+z)=x\cdot y+x\cdot z$
\textbf{General Topic:} Elementary algebra
\textbf{URL:} https://en.wikipedia.org/wiki/Distributive_property
\end{lemmatheorembox}

\textbf{Usage count:} 28

\section*{Lemma 32}
\begin{lemmatheorembox}
\textbf{Name:} Arithmetic progression
\textbf{Statement:} \(S_n=\frac{n}{2}(a+a_n).\)
\textbf{General Topic:} Sequences and series
\textbf{URL:} https://en.wikipedia.org/wiki/Arithmetic_progression
\end{lemmatheorembox}

\textbf{Usage count:} 27

\section*{Lemma 33}
\begin{lemmatheorembox}
\textbf{Name:} Tangent lines to circles
\textbf{Statement:} The radius of a circle is perpendicular to the tangent line through its endpoint on the circle's circumference. Conversely, the perpendicular to a radius through the same endpoint is a tangent line.
\textbf{General Topic:} Euclidean geometry
\textbf{URL:} https://en.wikipedia.org/wiki/Tangent_lines_to_circles
\end{lemmatheorembox}

\textbf{Usage count:} 27

\section*{Lemma 34}
\begin{lemmatheorembox}
\textbf{Name:} Permutations of multisets
\textbf{Statement:} If M is a finite multiset, then a multiset permutation is an ordered arrangement of elements of M in which each element appears a number of times equal exactly to its multiplicity in M. An anagram of a word having some repeated letters is an example of a multiset permutation. If the multiplicities of the elements of M (taken in some order) are $m_{1}$, $m_{2}$, ..., $m_{l}$ and their sum (that is, the size of M) is n, then the number of multiset permutations of M is given by the multinomial coefficient, ${n \choose m_{1},m_{2},\ldots ,m_{l}}={\frac {n!}{m_{1}!\,m_{2}!\,\cdots \,m_{l}!}}={\frac {\left(\sum _{i=1}^{l}{m_{i}}\right)!}{\prod _{i=1}^{l}{m_{i}!}}}.$
\textbf{General Topic:} Combinatorics
\textbf{URL:} https://en.wikipedia.org/wiki/Permutation
\end{lemmatheorembox}

\textbf{Usage count:} 27

\section*{Lemma 35}
\begin{lemmatheorembox}
\textbf{Name:} Law of sines
\textbf{Statement:} In trigonometry, the law of sines (sometimes called the sine formula or sine rule) is a mathematical equation relating the lengths of the sides of any triangle to the sines of its angles. According to the law, \(\frac{a}{\sin{\alpha}} \,=\, \frac{b}{\sin{\beta}} \,=\, \frac{c}{\sin{\gamma}} \,=\, 2R,\) where a, b, and c are the lengths of the sides of a triangle, and \(\alpha\), \(\beta\), and \(\gamma\) are the opposite angles (see figure 2), while R is the radius of the triangle's circumcircle.
\textbf{General Topic:} Trigonometry
\textbf{URL:} https://en.wikipedia.org/wiki/Law_of_sines
\end{lemmatheorembox}

\textbf{Usage count:} 26

\section*{Lemma 36}
\begin{lemmatheorembox}
\textbf{Name:} Exponentiation
\textbf{Statement:} The associativity of multiplication implies that for any positive integers m and n, $b^{m+n}=b^{m}\cdot b^{n}$, and $(b^{m})^{n}=b^{mn}$.
\textbf{General Topic:} Arithmetic
\textbf{URL:} https://en.wikipedia.org/wiki/Exponentiation
\end{lemmatheorembox}

\textbf{Usage count:} 26

\section*{Lemma 37}
\begin{lemmatheorembox}
\textbf{Name:} Floor and ceiling functions
\textbf{Statement:} for any real number \(x\), there are unique integers \(m\) and \(n\) satisfying the equation \(x-1<m\leq x\leq n<x+1\).
\textbf{General Topic:} Real analysis
\textbf{URL:} https://en.wikipedia.org/wiki/Floor_and_ceiling_functions
\end{lemmatheorembox}

\textbf{Usage count:} 25

\section*{Lemma 38}
\begin{lemmatheorembox}
\textbf{Name:} Law of total expectation
\textbf{Statement:} The proposition in probability theory known as the law of total expectation, the law of iterated expectations (LIE), Adam's law, the tower rule, and the smoothing property of conditional expectation, among other names, states that if $X$ is a random variable whose expected value $\operatorname{E}(X)$ is defined, and $Y$ is any random variable on the same probability space, then
\[
\operatorname{E}(X)=\operatorname{E}(\operatorname{E}(X\mid Y)),
\]
i.e., the expected value of the conditional expected value of $X$ given $Y$ is the same as the expected value of $X$. One special case states that if $\left\{A_i\right\}$ is a finite or countable partition of the sample space, then
\[
\operatorname{E}(X)=\sum_i{\operatorname{E}(X\mid A_i)\operatorname{P}(A_i)}.
\]
\textbf{General Topic:} Probability theory
\textbf{URL:} https://en.wikipedia.org/wiki/Law_of_total_expectation
\end{lemmatheorembox}

\textbf{Usage count:} 24

\section*{Lemma 39}
\begin{lemmatheorembox}
\textbf{Name:} Sum of angles of a triangle
\textbf{Statement:} In a Euclidean space, the sum of angles of a triangle equals a straight angle (180 degrees, π radians, two right angles, or a half-turn).
\textbf{General Topic:} Geometry
\textbf{URL:} https://en.wikipedia.org/wiki/Sum_of_angles_of_a_triangle
\end{lemmatheorembox}

\textbf{Usage count:} 24

\section*{Lemma 40}
\begin{lemmatheorembox}
\textbf{Name:} Modular arithmetic
\textbf{Statement:} If \(a_{1}\equiv b_{1}\pmod m\) and \(a_{2}\equiv b_{2}\pmod m\), or if \(a\equiv b\pmod m\), then: \(a^{k}\equiv b^{k}\pmod m\) for any non-negative integer \(k\) (compatibility with exponentiation).
\textbf{General Topic:} Number Theory
\textbf{URL:} https://en.wikipedia.org/wiki/Modular_arithmetic
\end{lemmatheorembox}

\textbf{Usage count:} 23

\section*{Lemma 41}
\begin{lemmatheorembox}
\textbf{Name:} Zero-product property
\textbf{Statement:} The zero-product property states that if the product of two factors is zero, at least one of the factors must be zero.
\textbf{General Topic:} Algebra
\textbf{URL:} https://en.wikipedia.org/wiki/Zero-product_property
\end{lemmatheorembox}

\textbf{Usage count:} 23

\section*{Lemma 42}
\begin{lemmatheorembox}
\textbf{Name:} Parity (mathematics)
\textbf{Statement:} even ± even = even; even ± odd = odd; odd ± odd = even;
\textbf{General Topic:} Number Theory
\textbf{URL:} https://en.wikipedia.org/wiki/Parity_(mathematics)
\end{lemmatheorembox}

\textbf{Usage count:} 23

\section*{Lemma 43}
\begin{lemmatheorembox}
\textbf{Name:} Fermat's little theorem
\textbf{Statement:} In number theory, Fermat's little theorem states that if p is a prime number, then for any integer a, the number a\textsuperscript{p} − a is an integer multiple of p. In the notation of modular arithmetic, this is expressed as a\textsuperscript{p} \equiv a \pmod p.
\textbf{General Topic:} Number theory
\textbf{URL:} https://en.wikipedia.org/wiki/Fermat%27s_little_theorem
\end{lemmatheorembox}

\textbf{Usage count:} 22

\section*{Lemma 44}
\begin{lemmatheorembox}
\textbf{Name:} Isosceles triangle theorem
\textbf{Statement:} the theorem that the angles opposite the equal sides of an isosceles triangle are themselves equal
\textbf{General Topic:} Geometry
\textbf{URL:} https://en.wikipedia.org/wiki/Pons_asinorum
\end{lemmatheorembox}

\textbf{Usage count:} 22

\section*{Lemma 45}
\begin{lemmatheorembox}
\textbf{Name:} Intermediate value theorem
\textbf{Statement:} Version I. if \(u\) is a number between \(f(a)\) and \(f(b)\), that is, \(\min(f(a),f(b))<u<\max(f(a),f(b)),\) then there is a \(c\in (a,b)\) such that \(f(c)=u\).
\textbf{General Topic:} Mathematical analysis
\textbf{URL:} https://en.wikipedia.org/wiki/Intermediate_value_theorem
\end{lemmatheorembox}

\textbf{Usage count:} 21

\section*{Lemma 46}
\begin{lemmatheorembox}
\textbf{Name:} Thales's theorem
\textbf{Statement:} In geometry, Thales's theorem states that if A, B, and C are distinct points on a circle where the line AC is a diameter, the angle \(\angle ABC\) is a right angle.
\textbf{General Topic:} Euclidean geometry
\textbf{URL:} https://en.wikipedia.org/wiki/Thales%27s_theorem
\end{lemmatheorembox}

\textbf{Usage count:} 21

\section*{Lemma 47}
\begin{lemmatheorembox}
\textbf{Name:} Root of unity
\textbf{Statement:} From the summation formula follows an orthogonality relationship: for j = 1, \ldots, n and j′ = 1, \ldots, n \[\sum_{k=1}^{n}{\overline {z^{j\cdot k}}}\cdot z^{j'\cdot k}=n\cdot \delta _{j,j'}\] where δ is the Kronecker delta and z is any primitive n th root of unity.
\textbf{General Topic:} Complex analysis
\textbf{URL:} https://en.wikipedia.org/wiki/Root_of_unity
\end{lemmatheorembox}

\textbf{Usage count:} 21

\section*{Lemma 48}
\begin{lemmatheorembox}
\textbf{Name:} Telescoping series
\textbf{Statement:} Telescoping sums are finite sums in which pairs of consecutive terms partly cancel each other, leaving only parts of the initial and final terms. Let $a_{n}$ be the elements of a sequence of numbers. Then $\sum _{n=1}^{N}\left(a_{n}-a_{n-1}\right)=a_{N}-a_{0}.$ If $a_{n}$ converges to a limit $L$, the telescoping series gives: $\sum _{n=1}^{\infty }\left(a_{n}-a_{n-1}\right)=L-a_{0}.$
\textbf{General Topic:} Mathematical analysis
\textbf{URL:} https://en.wikipedia.org/wiki/Telescoping_series
\end{lemmatheorembox}

\textbf{Usage count:} 21

\section*{Lemma 49}
\begin{lemmatheorembox}
\textbf{Name:} Parallelogram
\textbf{Statement:} In Euclidean geometry, a parallelogram is a simple (non-self-intersecting) quadrilateral with two pairs of parallel sides. The opposite or facing sides of a parallelogram are of equal length and the opposite angles of a parallelogram are of equal measure.
\textbf{General Topic:} Euclidean geometry
\textbf{URL:} https://en.wikipedia.org/wiki/Parallelogram
\end{lemmatheorembox}

\textbf{Usage count:} 20

\section*{Lemma 50}
\begin{lemmatheorembox}
\textbf{Name:} Congruence relation
\textbf{Statement:} Congruence modulo n (for a fixed n) is compatible with both addition and multiplication on the integers.
\textbf{General Topic:} Abstract algebra
\textbf{URL:} https://en.wikipedia.org/wiki/Congruence_relation
\end{lemmatheorembox}

\textbf{Usage count:} 20

\section*{Lemma 51}
\begin{lemmatheorembox}
\textbf{Name:} Least common multiple
\textbf{Statement:} A least common multiple of a and b is a common multiple that is minimal, in the sense that for any other common multiple n of a and b, m divides n.
\textbf{General Topic:} Number theory
\textbf{URL:} https://en.wikipedia.org/wiki/Least_common_multiple
\end{lemmatheorembox}

\textbf{Usage count:} 20

\section*{Lemma 52}
\begin{lemmatheorembox}
\textbf{Name:} Catalan number
\textbf{Statement:} The Catalan numbers are a sequence of natural numbers that occur in various counting problems, often involving recursively defined objects.
\textbf{General Topic:} Combinatorics
\textbf{URL:} https://en.wikipedia.org/wiki/Catalan_number
\end{lemmatheorembox}

\textbf{Usage count:} 20

\section*{Lemma 53}
\begin{lemmatheorembox}
\textbf{Name:} Arithmetic mean
\textbf{Statement:} The arithmetic mean of a set of observed data is equal to the sum of the numerical values of each observation, divided by the total number of observations.
\textbf{General Topic:} Mathematics and statistics
\textbf{URL:} https://en.wikipedia.org/wiki/Arithmetic_mean
\end{lemmatheorembox}

\textbf{Usage count:} 20

\section*{Lemma 54}
\begin{lemmatheorembox}
\textbf{Name:} Euler's theorem
\textbf{Statement:} if n and a are coprime positive integers, then a^{\varphi (n)} is congruent to 1 modulo n,
\textbf{General Topic:} Number theory
\textbf{URL:} https://en.wikipedia.org/wiki/Euler%27s_theorem
\end{lemmatheorembox}

\textbf{Usage count:} 19

\section*{Lemma 55}
\begin{lemmatheorembox}
\textbf{Name:} Euler's product formula
\textbf{Statement:} It states
\[
\varphi(n) =n \prod_{p\mid n} \left(1-\frac{1}{p}\right),
\]
where the product is over the distinct prime numbers dividing $n$.
\textbf{General Topic:} Number theory
\textbf{URL:} https://en.wikipedia.org/wiki/Euler%27s_totient_function
\end{lemmatheorembox}

\textbf{Usage count:} 19

\section*{Lemma 56}
\begin{lemmatheorembox}
\textbf{Name:} Pythagorean trigonometric identity
\textbf{Statement:} The two identities \begin{align} 1 + \tan^2 \theta &= \sec^2 \theta \\ 1 + \cot^2 \theta &= \csc^2 \theta \end{align} are also called Pythagorean trigonometric identities.
\textbf{General Topic:} Trigonometry
\textbf{URL:} https://en.wikipedia.org/wiki/Pythagorean_trigonometric_identity
\end{lemmatheorembox}

\textbf{Usage count:} 18

\section*{Lemma 57}
\begin{lemmatheorembox}
\textbf{Name:} Heron's formula
\textbf{Statement:} In geometry, Heron's formula (or Hero's formula) gives the area of a triangle in terms of the three side lengths $a$, $b$, $c$. Letting $s$ be the semiperimeter of the triangle, $s={\tfrac {1}{2}}(a+b+c)$, the area $A$ is $A={\sqrt {s(s-a)(s-b)(s-c)}}$.
\textbf{General Topic:} Geometry
\textbf{URL:} https://en.wikipedia.org/wiki/Heron%27s_formula
\end{lemmatheorembox}

\textbf{Usage count:} 18

\section*{Lemma 58}
\begin{lemmatheorembox}
\textbf{Name:} Commutative property
\textbf{Statement:} A binary operation is commutative if changing the order of the operands does not change the result.
\textbf{General Topic:} Algebra
\textbf{URL:} https://en.wikipedia.org/wiki/Commutative_property
\end{lemmatheorembox}

\textbf{Usage count:} 17

\section*{Lemma 59}
\begin{lemmatheorembox}
\textbf{Name:} Cauchy–Schwarz inequality
\textbf{Statement:} The Cauchy–Schwarz inequality states that for all vectors \(\mathbf{u}\) and \(\mathbf{v}\) of an inner product space \(\left|\langle \mathbf{u} ,\mathbf{v} \rangle \right|^{2}\leq \langle \mathbf{u} ,\mathbf{u} \rangle \cdot \langle \mathbf{v} ,\mathbf{v} \rangle\).
\textbf{General Topic:} Inequality
\textbf{URL:} https://en.wikipedia.org/wiki/Cauchy%E2%80%93Schwarz_inequality
\end{lemmatheorembox}

\textbf{Usage count:} 16

\section*{Lemma 60}
\begin{lemmatheorembox}
\textbf{Name:} Square root
\textbf{Statement:} Every positive number \(x\) has two square roots: \(\sqrt{x}\) (which is positive) and \(-\sqrt{x}\) (which is negative).
\textbf{General Topic:} Elementary algebra
\textbf{URL:} https://en.wikipedia.org/wiki/Square_root
\end{lemmatheorembox}

\textbf{Usage count:} 16

\section*{Lemma 61}
\begin{lemmatheorembox}
\textbf{Name:} Cyclic quadrilateral
\textbf{Statement:} A convex quadrilateral is cyclic if and only if its opposite angles are supplementary.
\textbf{General Topic:} Euclidean geometry
\textbf{URL:} https://en.wikipedia.org/wiki/Cyclic_quadrilateral
\end{lemmatheorembox}

\textbf{Usage count:} 15

\section*{Lemma 62}
\begin{lemmatheorembox}
\textbf{Name:} Angle bisector theorem
\textbf{Statement:} {\displaystyle {\frac {|BD|} {|CD|}}={\frac {|AB|}{|AC|}},}
\textbf{General Topic:} Euclidean geometry
\textbf{URL:} https://en.wikipedia.org/wiki/Angle_bisector_theorem
\end{lemmatheorembox}

\textbf{Usage count:} 15

\section*{Lemma 63}
\begin{lemmatheorembox}
\textbf{Name:} Cancellation property
\textbf{Statement:} An element a in a magma (M, ∗) has the left cancellation property (or is left-cancellative) if for all b and c in M, a ∗ b = a ∗ c always implies that b = c.
\textbf{General Topic:} Abstract algebra
\textbf{URL:} https://en.wikipedia.org/wiki/Cancellation_property
\end{lemmatheorembox}

\textbf{Usage count:} 15

\section*{Lemma 64}
\begin{lemmatheorembox}
\textbf{Name:} Combination
\textbf{Statement:} If the set has n elements, the number of k-combinations, denoted by C(n,k) or C_{k}^{n}, is equal to the binomial coefficient:
\[
\binom nk = \frac{n(n-1)\dotsb (n-k+1)}{k(k-1)\dotsb 1},
\]
which using factorial notation can be compactly expressed as
\[
\binom{n}{k} = \frac{n!}{k! (n-k)!}
\]
whenever n\geq k\geq 0.
\textbf{General Topic:} Combinatorics
\textbf{URL:} https://en.wikipedia.org/wiki/Combination
\end{lemmatheorembox}

\textbf{Usage count:} 15

\section*{Lemma 65}
\begin{lemmatheorembox}
\textbf{Name:} Independence (probability theory)
\textbf{Statement:} $\mathrm{P}(A\cap B)=\mathrm{P}(A)\mathrm{P}(B)$
\textbf{General Topic:} Probability theory
\textbf{URL:} https://en.wikipedia.org/wiki/Independence_(probability_theory)
\end{lemmatheorembox}

\textbf{Usage count:} 15

\section*{Lemma 66}
\begin{lemmatheorembox}
\textbf{Name:} Fundamental theorem of algebra
\textbf{Statement:} Every univariate polynomial of positive degree with complex coefficients has at least one complex root.
\textbf{General Topic:} Algebra
\textbf{URL:} https://en.wikipedia.org/wiki/Fundamental_theorem_of_algebra
\end{lemmatheorembox}

\textbf{Usage count:} 14

\section*{Lemma 67}
\begin{lemmatheorembox}
\textbf{Name:} principle of strong induction
\textbf{Statement:} \(\left[\left(P(0)\land \forall k\left(\left(\forall j\,(0\leq j\leq k)\rightarrow P(j)\right)\rightarrow P(k+1)\right)\right)\right]\rightarrow \forall n\in \mathbb {Z} _{\geq 0},\,P(n)\).
\textbf{General Topic:} Proof techniques
\textbf{URL:} https://en.wikipedia.org/wiki/Well-ordering_principle
\end{lemmatheorembox}

\textbf{Usage count:} 14

\section*{Lemma 68}
\begin{lemmatheorembox}
\textbf{Name:} Independence (probability theory)
\textbf{Statement:} \(\mathrm {P} (A\cap B)=\mathrm {P} (A)\mathrm {P} (B)\)
\textbf{General Topic:} Probability theory
\textbf{URL:} https://en.wikipedia.org/wiki/Independence_%28probability_theory%29
\end{lemmatheorembox}

\textbf{Usage count:} 14

\section*{Lemma 69}
\begin{lemmatheorembox}
\textbf{Name:} Legendre's formula
\textbf{Statement:} For any prime number $p$ and any positive integer $n$, let $\nu_p(n)$ be the exponent of the largest power of $p$ that divides $n$ (that is, the $p$-adic valuation of $n$). Then
\[
\nu_p(n!) = \sum_{i=1}^{\infty} \left\lfloor \frac{n}{p^i} \right\rfloor,
\]
where $\lfloor x \rfloor$ is the floor function.
\textbf{General Topic:} Number theory
\textbf{URL:} https://en.wikipedia.org/wiki/Legendre%27s_formula
\end{lemmatheorembox}

\textbf{Usage count:} 14

\section*{Lemma 70}
\begin{lemmatheorembox}
\textbf{Name:} Power
\textbf{Statement:} $\log_b\left(x^p\right) = p \log_b x$
\textbf{General Topic:} Logarithms
\textbf{URL:} https://en.wikipedia.org/wiki/Logarithm
\end{lemmatheorembox}

\textbf{Usage count:} 14

\section*{Lemma 71}
\begin{lemmatheorembox}
\textbf{Name:} Associative property
\textbf{Statement:} Formally, a binary operation $\ast$ on a set $S$ is called associative if it satisfies the associative law: $(x\ast y)\ast z=x\ast (y\ast z)$, for all $x,y,z$ in $S$.
\textbf{General Topic:} Elementary algebra
\textbf{URL:} https://en.wikipedia.org/wiki/Associative_property
\end{lemmatheorembox}

\textbf{Usage count:} 13

\section*{Lemma 72}
\begin{lemmatheorembox}
\textbf{Name:} Equilateral triangle

\textbf{Statement:} The area of an equilateral triangle with edge length \(a\) is \(T={\frac {\sqrt {3}}{4}}a^{2}.\)

\textbf{General Topic:} Geometry

\textbf{URL:} https://en.wikipedia.org/wiki/Equilateral_triangle
\end{lemmatheorembox}

\textbf{Usage count:} 13

\section*{Lemma 73}
\begin{lemmatheorembox}
\textbf{Name:} Geometric progression
\textbf{Statement:} Geometric sequences also satisfy the nonlinear recurrence relation $a_{n}=a_{n-1}^{2}/a_{n-2}$ for every integer $n>2$.
\textbf{General Topic:} Sequences
\textbf{URL:} https://en.wikipedia.org/wiki/Geometric_progression
\end{lemmatheorembox}

\textbf{Usage count:} 13

\section*{Lemma 74}
\begin{lemmatheorembox}
\textbf{Name:} Isosceles triangle
\textbf{Statement:} For any isosceles triangle, the following six line segments coincide:
* the altitude, a line segment from the apex perpendicular to the base,
* the angle bisector from the apex to the base,
* the median from the apex to the midpoint of the base,
* the perpendicular bisector of the base within the triangle,
* the segment within the triangle of the unique axis of symmetry of the triangle, and
* the segment within the triangle of the Euler line of the triangle, except when the triangle is equilateral.
\textbf{General Topic:} Euclidean geometry
\textbf{URL:} https://en.wikipedia.org/wiki/Isosceles_triangle
\end{lemmatheorembox}

\textbf{Usage count:} 13

\section*{Lemma 75}
\begin{lemmatheorembox}
\textbf{Name:} Binet's formula
\textbf{Statement:} $F_{n}={\frac {\varphi ^{n}-\psi ^{n}}{\varphi -\psi }}={\frac {\varphi ^{n}-\psi ^{n}}{\sqrt {5}}},$
\textbf{General Topic:} Number theory
\textbf{URL:} https://en.wikipedia.org/wiki/Fibonacci_sequence
\end{lemmatheorembox}

\textbf{Usage count:} 13

\section*{Lemma 76}
\begin{lemmatheorembox}
\textbf{Name:} Degree sum formula
\textbf{Statement:} The degree sum formula states that $\sum_{v\in V} \deg v = 2|E|$, where V is the set of nodes (or vertices) in the graph and E is the set of edges in the graph.
\textbf{General Topic:} Graph theory
\textbf{URL:} https://en.wikipedia.org/wiki/Handshaking_lemma
\end{lemmatheorembox}

\textbf{Usage count:} 12

\section*{Lemma 77}
\begin{lemmatheorembox}
\textbf{Name:} Modular multiplicative inverse
\textbf{Statement:} The previous result says that a solution exists if and only if gcd(a, m) = 1, that is, a and m must be relatively prime (i.e. coprime).
\textbf{General Topic:} Modular arithmetic
\textbf{URL:} https://en.wikipedia.org/wiki/Modular_multiplicative_inverse
\end{lemmatheorembox}

\textbf{Usage count:} 12

\section*{Lemma 78}
\begin{lemmatheorembox}
\textbf{Name:} Linear recurrence with constant coefficients
\textbf{Statement:} Different solutions are obtained depending on the nature of the roots: If these roots are distinct, we have the general solution \(a_{n}=C\lambda _{1}^{n}+D\lambda _{2}^{n}\).
\textbf{General Topic:} Recurrence relations
\textbf{URL:} https://en.wikipedia.org/wiki/Linear_recurrence_with_constant_coefficients
\end{lemmatheorembox}

\textbf{Usage count:} 12

\section*{Lemma 79}
\begin{lemmatheorembox}
\textbf{Name:} Euler's formula
\textbf{Statement:} Euler's formula states that, for any real number x, one has
\[
e^{i x} = \cos x + i \sin x,
\]
where e is the base of the natural logarithm, i is the imaginary unit, and cos and sin are the trigonometric functions cosine and sine respectively.
\textbf{General Topic:} Complex analysis
\textbf{URL:} https://en.wikipedia.org/wiki/Euler%27s_formula
\end{lemmatheorembox}

\textbf{Usage count:} 12

\section*{Lemma 80}
\begin{lemmatheorembox}
\textbf{Name:} Power of a point
\textbf{Statement:} |PS_1|\cdot|PS_2|=\Pi(P)
\textbf{General Topic:} Euclidean geometry
\textbf{URL:} https://en.wikipedia.org/wiki/Power_of_a_point
\end{lemmatheorembox}

\textbf{Usage count:} 11

\section*{Lemma 81}
\begin{lemmatheorembox}
\textbf{Name:} Cyclic group  
\textbf{Statement:} For a prime number p, the group (Z/pZ)× is always cyclic, consisting of the non-zero elements of the finite field of order p.  
\textbf{General Topic:} Group Theory  
\textbf{URL:} https://en.wikipedia.org/wiki/Cyclic_group
\end{lemmatheorembox}

\textbf{Usage count:} 11

\section*{Lemma 82}
\begin{lemmatheorembox}
\textbf{Name:} Faulhaber's formula

\textbf{Statement:} \(\displaystyle \sum_{k=1}^n k^{p} = \frac{1}{p+1} \sum_{r=0}^p \binom{p+1}{r} B_r n^{p+1-r} .\)

\textbf{General Topic:} Number Theory

\textbf{URL:} https://en.wikipedia.org/wiki/Faulhaber%27s_formula
\end{lemmatheorembox}

\textbf{Usage count:} 11

\section*{Lemma 83}
\begin{lemmatheorembox}
\textbf{Name:} Greatest common divisor
\textbf{Statement:} If m is a positive integer, then gcd(m⋅a, m⋅b) = m⋅gcd(a, b).
\textbf{General Topic:} Number theory
\textbf{URL:} https://en.wikipedia.org/wiki/Greatest_common_divisor
\end{lemmatheorembox}

\textbf{Usage count:} 11

\section*{Lemma 84}
\begin{lemmatheorembox}
\textbf{Name:} Complementary event
\textbf{Statement:} \(P(A^{c})=1-P(A).\)
\textbf{General Topic:} Probability theory
\textbf{URL:} https://en.wikipedia.org/wiki/Complementary_event
\end{lemmatheorembox}

\textbf{Usage count:} 11

\section*{Lemma 85}
\begin{lemmatheorembox}
\textbf{Name:} Theorem one
\textbf{Statement:} For any pair of positive integers n and k, the number of k-tuples of positive integers whose sum is n is equal to the number of (k − 1)-element subsets of a set with n − 1 elements.
\textbf{General Topic:} Combinatorics
\textbf{URL:} https://en.wikipedia.org/wiki/Stars_and_bars_(combinatorics)
\end{lemmatheorembox}

\textbf{Usage count:} 11

\section*{Lemma 86}
\begin{lemmatheorembox}
\textbf{Name:} Subtraction principle
\textbf{Statement:} Similarly, for a given finite set S, and given another set A, if A\subset S, then |A^{c}|=|S|-|A|.
\textbf{General Topic:} Combinatorics
\textbf{URL:} https://en.wikipedia.org/wiki/Addition_principle#Subtraction_principle
\end{lemmatheorembox}

\textbf{Usage count:} 11

\section*{Lemma 87}
\begin{lemmatheorembox}
\textbf{Name:} De Moivre's formula
\textbf{Statement:} In mathematics, de Moivre's formula (also known as de Moivre's theorem and de Moivre's identity) states that for any real number x and integer n it is the case that
\[
\big(\cos x + i \sin x\big)^n = \cos nx + i \sin nx,
\]
where i is the imaginary unit.
\textbf{General Topic:} Complex numbers
\textbf{URL:} https://en.wikipedia.org/wiki/De_Moivre%27s_formula
\end{lemmatheorembox}

\textbf{Usage count:} 11

\section*{Lemma 88}
\begin{lemmatheorembox}
\textbf{Name:} Fubini's theorem
\textbf{Statement:} A related theorem is often called Fubini's theorem for infinite series, although it is due to Alfred Pringsheim. It states that if $\{a_{m,n}\}_{m=1,n=1}^{\infty}$ is a double-indexed sequence of real numbers, and if $\displaystyle \sum_{(m,n)\in \mathbb {N} \times \mathbb {N} }a_{m,n}$ is absolutely convergent, then
\[
\sum_{(m,n)\in\mathbb {N} \times \mathbb {N} }a_{m,n} = \sum_{m=1}^{\infty}\sum_{n=1}^{\infty} a_{m,n} = \sum_{n=1}^{\infty} \sum_{m=1}^{\infty} a_{m,n}.
\]
\textbf{General Topic:} Measure theory
\textbf{URL:} https://en.wikipedia.org/wiki/Fubini%27s_theorem
\end{lemmatheorembox}

\textbf{Usage count:} 11

\section*{Lemma 89}
\begin{lemmatheorembox}
\textbf{Name:} Primitive root modulo n
\textbf{Statement:} A primitive root exists if and only if n is 1, 2, 4, p^k or 2p^k, where p is an odd prime and k > 0. For all other values of n the multiplicative group of integers modulo n is not cyclic.
\textbf{General Topic:} Number theory
\textbf{URL:} https://en.wikipedia.org/wiki/Primitive_root_modulo_n
\end{lemmatheorembox}

\textbf{Usage count:} 10

\section*{Lemma 90}
\begin{lemmatheorembox}
\textbf{Name:} Pascal's rule
\textbf{Statement:} \({\tbinom {n}{k}}={\tbinom {n-1}{k-1}}+{\tbinom {n-1}{k}}\)
\textbf{General Topic:} Combinatorics
\textbf{URL:} https://en.wikipedia.org/wiki/Pascal%27s_rule
\end{lemmatheorembox}

\textbf{Usage count:} 10

\section*{Lemma 91}
\begin{lemmatheorembox}
\textbf{Name:} Circumcircle
\textbf{Statement:} The diameter of the circumcircle, called the circumdiameter and equal to twice the circumradius, can be computed as the length of any side of the triangle divided by the sine of the opposite angle: $\text{diameter}={\frac {a}{\sin A}}={\frac {b}{\sin B}}={\frac {c}{\sin C}}.$ As a consequence of the law of sines, it does not matter which side and opposite angle are taken: the result will be the same. The diameter of the circumcircle can also be expressed as $\begin{aligned}{\text{diameter}}&{}={\frac {abc}{2\cdot {\text{area}}}}={\frac {|AB||BC||CA|}{2|\Delta ABC|}}\\[5pt]&{}={\frac {abc}{2{\sqrt {s(s-a)(s-b)(s-c)}}}}\\[5pt]&{}={\frac {2abc}{\sqrt {(a+b+c)(-a+b+c)(a-b+c)(a+b-c)}}}\end{aligned}$ where $a$, $b$, $c$ are the lengths of the sides of the triangle and $s={\tfrac {a+b+c}{2}}$ is the semiperimeter. The expression $\scriptstyle {\sqrt {s(s-a)(s-b)(s-c)}}$ above is the area of the triangle, by Heron's formula.
\textbf{General Topic:} Geometry
\textbf{URL:} https://en.wikipedia.org/wiki/Circumcircle
\end{lemmatheorembox}

\textbf{Usage count:} 10

\section*{Lemma 92}
\begin{lemmatheorembox}
\textbf{Name:} Tangent--secant theorem
\textbf{Statement:} Given a secant intersecting the circle at points $G_1$ and $G_2$ and a tangent intersecting the circle at point $T$ and given that $G_1G_2$ and $PT$ intersect at point $P$, the following equation holds: \[|PT|^2=|PG_1|\cdot|PG_2|.\]
\textbf{General Topic:} Euclidean geometry
\textbf{URL:} https://en.wikipedia.org/wiki/Tangent%E2%80%93secant_theorem
\end{lemmatheorembox}

\textbf{Usage count:} 10

\section*{Lemma 93}
\begin{lemmatheorembox}
\textbf{Name:} Multinomial theorem
\textbf{Statement:} $(x_1 + x_2 + \cdots + x_m)^n = \sum_{\substack{k_1+k_2+\cdots+k_m=n \\ k_1, k_2, \cdots, k_m \geq 0}} {n \choose k_1, k_2, \ldots, k_m} x_1^{k_1} \cdot x_2^{k_2} \cdots x_m^{k_m}$, where ${n \choose k_1, k_2, \ldots, k_m} = \frac{n!}{k_1!\, k_2! \cdots k_m!}$.
\textbf{General Topic:} Combinatorics
\textbf{URL:} https://en.wikipedia.org/wiki/Multinomial_theorem
\end{lemmatheorembox}

\textbf{Usage count:} 10

\section*{Lemma 94}
\begin{lemmatheorembox}
\textbf{Name:} Factorial
\textbf{Statement:} there are n! different ways of arranging n distinct objects into a sequence.
\textbf{General Topic:} Combinatorics
\textbf{URL:} https://en.wikipedia.org/wiki/Factorial
\end{lemmatheorembox}

\textbf{Usage count:} 10

\section*{Lemma 95}
\begin{lemmatheorembox}
\textbf{Name:} Finite difference
\textbf{Statement:} For a given polynomial of degree $n \geq 1$, expressed in the function $P(x)$, with real numbers $a \neq 0$ and $b$ and lower order terms (if any) marked as l.o.t.: $P(x)=ax^{n}+\;bx^{n-1}+~l.o.t.$ After $n$ pairwise differences, the following result can be achieved, where $h \neq 0$ is a real number marking the arithmetic difference: $\Delta _{h}^{n}[P](x)=ah^{n}n!$ Only the coefficient of the highest-order term remains. As this result is constant with respect to $x$, any further pairwise differences will have the value $0$.
\textbf{General Topic:} Numerical analysis
\textbf{URL:} https://en.wikipedia.org/wiki/Finite_difference
\end{lemmatheorembox}

\textbf{Usage count:} 10

\section*{Lemma 96}
\begin{lemmatheorembox}
\textbf{Name:} Homothety
\textbf{Statement:} In mathematics, a homothety (or homothecy, or homogeneous dilation) is a transformation of an affine space determined by a point S called its center and a nonzero number k called its ratio, which sends point X to a point X′ by the rule,[ 1 ] \(\overrightarrow{SX'}=k\overrightarrow{SX}\) for a fixed number \(k\neq 0\).
\textbf{General Topic:} Geometry
\textbf{URL:} https://en.wikipedia.org/wiki/Homothety
\end{lemmatheorembox}

\textbf{Usage count:} 9

\section*{Lemma 97}
\begin{lemmatheorembox}
\textbf{Name:} Rearrangement inequality
\textbf{Statement:} \(x_{1}y_{n}+\cdots +x_{n}y_{1}\leq x_{1}y_{\sigma (1)}+\cdots +x_{n}y_{\sigma (n)}\leq x_{1}y_{1}+\cdots +x_{n}y_{n}.\)
\textbf{General Topic:} Inequalities
\textbf{URL:} https://en.wikipedia.org/wiki/Rearrangement_inequality
\end{lemmatheorembox}

\textbf{Usage count:} 9

\section*{Lemma 98}
\begin{lemmatheorembox}
\textbf{Name:} Bézout's identity
\textbf{Statement:} Bézout's identity—Let a and b be integers with greatest common divisor d. Then there exist integers x and y such that ax + by = d. Moreover, the integers of the form az + bt are exactly the multiples of d.
\textbf{General Topic:} Number Theory
\textbf{URL:} https://en.wikipedia.org/wiki/B%C3%A9zout%27s_identity
\end{lemmatheorembox}

\textbf{Usage count:} 9

\section*{Lemma 99}
\begin{lemmatheorembox}
\textbf{Name:} Fundamental theorem of calculus
\textbf{Statement:} Let f be a continuous real-valued function defined on a closed interval [a,b]. Let F be the function defined, for all x in [a,b], by
F(x) = \int_a^x f(t)\, dt.
Then F is uniformly continuous on [a,b] and differentiable on the open interval (a,b), and
F'(x) = f(x)
for all x in (a,b) so F is an antiderivative of f.
\textbf{General Topic:} Calculus
\textbf{URL:} https://en.wikipedia.org/wiki/Fundamental_theorem_of_calculus
\end{lemmatheorembox}

\textbf{Usage count:} 9

\section*{Lemma 100}
\begin{lemmatheorembox}
\textbf{Name:} Inversion in a circle
\textbf{Statement:} In the plane, the inverse of a point P with respect to a reference circle (Ø) with center O and radius r is a point P', lying on the ray from O through P such that \(OP\cdot OP^{\prime }=r^{2}\).
\textbf{General Topic:} Inversive geometry
\textbf{URL:} https://en.wikipedia.org/wiki/Inversive_geometry
\end{lemmatheorembox}

\textbf{Usage count:} 9

\section*{Lemma 101}
\begin{lemmatheorembox}
\textbf{Name:} Exponential function
\textbf{Statement:} The exponential function is the sum of the power series
\[
\begin{aligned}
\exp(x)&=1+x+{\frac {x^{2}}{2!}}+{\frac {x^{3}}{3!}}+\cdots \\
&=\sum _{n=0}^{\infty }{\frac {x^{n}}{n!}},
\end{aligned}
\]
\textbf{General Topic:} Mathematical analysis
\textbf{URL:} https://en.wikipedia.org/wiki/Exponential_function
\end{lemmatheorembox}

\textbf{Usage count:} 9

\section*{Lemma 102}
\begin{lemmatheorembox}
\textbf{Name:} Menelaus's theorem
\textbf{Statement:} Suppose we have a triangle △ABC, and a transversal line that crosses BC, AC, AB at points D, E, F respectively, with D, E, F distinct from A, B, C. A weak version of the theorem states that
\[
\left|\frac{\overline{AF}}{\overline{FB}}\right| \times \left|\frac{\overline{BD}}{\overline{DC}}\right| \times \left|\frac{\overline{CE}}{\overline{EA}}\right| = 1,
\]
where "| |" denotes absolute value (i.e., all segment lengths are positive).
\textbf{General Topic:} Euclidean geometry
\textbf{URL:} https://en.wikipedia.org/wiki/Menelaus%27s_theorem
\end{lemmatheorembox}

\textbf{Usage count:} 9

\section*{Lemma 103}
\begin{lemmatheorembox}
\textbf{Name:} Area of a circle
\textbf{Statement:} In geometry, the area enclosed by a circle of radius r is π r^{2}.
\textbf{General Topic:} Geometry
\textbf{URL:} https://en.wikipedia.org/wiki/Area_of_a_circle
\end{lemmatheorembox}

\textbf{Usage count:} 9

\section*{Lemma 104}
\begin{lemmatheorembox}
\textbf{Name:} Power set
\textbf{Statement:} Assuming $|S|=n$: $\left|2^{S}\right|=2^{n}=\sum _{k=0}^{n}{\binom {n}{k}}$.
\textbf{General Topic:} Set theory
\textbf{URL:} https://en.wikipedia.org/wiki/Power_set
\end{lemmatheorembox}

\textbf{Usage count:} 9

\section*{Lemma 105}
\begin{lemmatheorembox}
\textbf{Name:} Divisor
\textbf{Statement:} An integer n is divisible by a nonzero integer m if there exists an integer k such that n=km.
\textbf{General Topic:} Number theory
\textbf{URL:} https://en.wikipedia.org/wiki/Divisor
\end{lemmatheorembox}

\textbf{Usage count:} 9

\section*{Lemma 106}
\begin{lemmatheorembox}
\textbf{Name:} Function application
\textbf{Statement:} For every a and b, with some function f(x), if a = b, then f(a)=f(b).
\textbf{General Topic:} Mathematical logic
\textbf{URL:} https://en.wikipedia.org/wiki/Equality_(mathematics)
\end{lemmatheorembox}

\textbf{Usage count:} 9

\section*{Lemma 107}
\begin{lemmatheorembox}
\textbf{Name:} Extreme value theorem
\textbf{Statement:} That is, there exist numbers \(c\) and \(d\) in \([a,b]\) such that: \(f(c)\leq f(x)\leq f(d)\quad \forall x\in [a,b]\).
\textbf{General Topic:} Real analysis
\textbf{URL:} https://en.wikipedia.org/wiki/Extreme_value_theorem
\end{lemmatheorembox}

\textbf{Usage count:} 8

\section*{Lemma 108}
\begin{lemmatheorembox}
\textbf{Name:} p-adic valuation
\textbf{Statement:} Some properties are:

\nu_p(r\cdot s) = \nu_p(r) + \nu_p(s)

\nu_p(r+s) \geq \min\bigl\{ \nu_p(r), \nu_p(s)\bigr\}

Moreover, if \nu_p(r) \neq \nu_p(s), then

\nu_p(r+s)= \min\bigl\{ \nu_p(r), \nu_p(s)\bigr\}

where \min is the minimum (i.e. the smaller of the two).
\textbf{General Topic:} Number Theory
\textbf{URL:} https://en.wikipedia.org/wiki/P-adic_valuation
\end{lemmatheorembox}

\textbf{Usage count:} 8

\section*{Lemma 109}
\begin{lemmatheorembox}
\textbf{Name:} Euclidean algorithm
\textbf{Statement:} The Euclidean algorithm is based on the principle that the greatest common divisor of two numbers does not change if the larger number is replaced by its difference with the smaller number.
\textbf{General Topic:} Number theory
\textbf{URL:} https://en.wikipedia.org/wiki/Euclidean_algorithm
\end{lemmatheorembox}

\textbf{Usage count:} 8

\section*{Lemma 110}
\begin{lemmatheorembox}
\textbf{Name:} Rational root theorem
\textbf{Statement:} In algebra, the rational root theorem (or rational root test, rational zero theorem, rational zero test or p/q theorem) states a constraint on rational solutions of a polynomial equation \(a_{n}x^{n}+a_{n-1}x^{n-1}+\cdots +a_{0}=0\) with integer coefficients \(a_i\in\mathbb{Z}\) and \(a_{0},a_{n}\neq 0\). The theorem states that each rational solution \(x=\tfrac{p}{q}\) written in lowest terms (that is, \(p\) and \(q\) are relatively prime), satisfies: \(p\) is an integer factor of the constant term \(a_{0}\), and \(q\) is an integer factor of the leading coefficient \(a_{n}\).
\textbf{General Topic:} Algebra
\textbf{URL:} https://en.wikipedia.org/wiki/Rational_root_theorem
\end{lemmatheorembox}

\textbf{Usage count:} 8

\section*{Lemma 111}
\begin{lemmatheorembox}
\textbf{Name:} Integration by substitution
\textbf{Statement:} Let g:[a,b]\to I be a differentiable function with a continuous derivative, where I \subset \mathbb{R} is an interval. Suppose that f:I\to\mathbb{R} is a continuous function. Then:
\int_a^b f(g(x))\cdot g'(x)\, dx = \int_{g(a)}^{g(b)} f(u)\ du.
\textbf{General Topic:} Calculus
\textbf{URL:} https://en.wikipedia.org/wiki/Integration_by_substitution
\end{lemmatheorembox}

\textbf{Usage count:} 8

\section*{Lemma 112}
\begin{lemmatheorembox}
\textbf{Name:} Intercept theorem
\textbf{Statement:} The ratio of the two segments on the same ray starting at $S$ equals the ratio of the segments on the parallels: $\frac {|SA|}{|SB|}=\frac {|SC|}{|SD|}=\frac {|AC|}{|BD|}$.
\textbf{General Topic:} Elementary geometry
\textbf{URL:} https://en.wikipedia.org/wiki/Intercept_theorem
\end{lemmatheorembox}

\textbf{Usage count:} 8

\section*{Lemma 113}
\begin{lemmatheorembox}
\textbf{Name:} Ptolemy's theorem
\textbf{Statement:} If the vertices of the cyclic quadrilateral are A, B, C, and D in order, then the theorem states that: \(AC\cdot BD = AB\cdot CD+BC\cdot AD\).
\textbf{General Topic:} Euclidean geometry
\textbf{URL:} https://en.wikipedia.org/wiki/Ptolemy%27s_theorem
\end{lemmatheorembox}

\textbf{Usage count:} 8

\section*{Lemma 114}
\begin{lemmatheorembox}
\textbf{Name:} Equidistant
\textbf{Statement:} In two-dimensional Euclidean geometry, the locus of points equidistant from two given (different) points is their perpendicular bisector.
\textbf{General Topic:} Geometry
\textbf{URL:} https://en.wikipedia.org/wiki/Equidistant
\end{lemmatheorembox}

\textbf{Usage count:} 8

\section*{Lemma 115}
\begin{lemmatheorembox}
\textbf{Name:} Angle bisector
\textbf{Statement:} An angle bisector divides the angle into two angles with equal measures.
\textbf{General Topic:} Geometry
\textbf{URL:} https://en.wikipedia.org/wiki/Bisection
\end{lemmatheorembox}

\textbf{Usage count:} 8

\section*{Lemma 116}
\begin{lemmatheorembox}
\textbf{Name:} Conditional probability
\textbf{Statement:} $P(A\mid B)={\frac {P(A\cap B)}{P(B)}}.$
\textbf{General Topic:} Probability theory
\textbf{URL:} https://en.wikipedia.org/wiki/Conditional_probability
\end{lemmatheorembox}

\textbf{Usage count:} 8

\section*{Lemma 117}
\begin{lemmatheorembox}
\textbf{Name:} Finite set
\textbf{Statement:} A finite set with $n$ elements has $2^{n}$ distinct subsets. That is, the power set $\wp (S)$ of a finite set $S$ is finite, with cardinality $2^{|S|}$.
\textbf{General Topic:} Set theory
\textbf{URL:} https://en.wikipedia.org/wiki/Finite_set
\end{lemmatheorembox}

\textbf{Usage count:} 8

\section*{Lemma 118}
\begin{lemmatheorembox}
\textbf{Name:} Summation
\textbf{Statement:} $\sum_{i=1}^{n} i=\frac{n(n+1)}{2}$.
\textbf{General Topic:} Mathematical Analysis
\textbf{URL:} https://en.wikipedia.org/wiki/Summation
\end{lemmatheorembox}

\textbf{Usage count:} 8

\section*{Lemma 119}
\begin{lemmatheorembox}
\textbf{Name:} Angle
\textbf{Statement:} Two angles that sum to a straight angle (turn, 180°, or radians) are called supplementary angles.
\textbf{General Topic:} Euclidean geometry
\textbf{URL:} https://en.wikipedia.org/wiki/Angle
\end{lemmatheorembox}

\textbf{Usage count:} 8

\section*{Lemma 120}
\begin{lemmatheorembox}
\textbf{Name:} Square pyramidal number
\textbf{Statement:} {\displaystyle P_{n}=\sum _{k=1}^{n}k^{2}=1+4+9+\cdots +n^{2},}\quad {\displaystyle P_{n}={\frac {n(n+1)(2n+1)}{6}}={\frac {2n^{3}+3n^{2}+n}{6}}={\frac {n^{3}}{3}}+{\frac {n^{2}}{2}}+{\frac {n}{6}}.}
\textbf{General Topic:} Figurate numbers
\textbf{URL:} https://en.wikipedia.org/wiki/Square_pyramidal_number
\end{lemmatheorembox}

\textbf{Usage count:} 7

\section*{Lemma 121}
\begin{lemmatheorembox}
\textbf{Name:} Pick's theorem
\textbf{Statement:} Suppose that a polygon has integer coordinates for all of its vertices. Let $i$ be the number of integer points interior to the polygon, and let $b$ be the number of integer points on its boundary (including both vertices and points along the sides). Then the area $A$ of this polygon is:
\[
A=i+{\frac {b}{2}}-1.
\]
\textbf{General Topic:} Geometry
\textbf{URL:} https://en.wikipedia.org/wiki/Pick%27s_theorem
\end{lemmatheorembox}

\textbf{Usage count:} 7

\section*{Lemma 122}
\begin{lemmatheorembox}
\textbf{Name:} Complex conjugate
\textbf{Statement:} This allows easy computation of the multiplicative inverse of a complex number given in rectangular coordinates: $z^{-1} = \frac{\overline{z}}{{\left| z \right|}^2},\quad \text{ for all } z \neq 0.$
\textbf{General Topic:} Complex numbers
\textbf{URL:} https://en.wikipedia.org/wiki/Complex_conjugate
\end{lemmatheorembox}

\textbf{Usage count:} 7

\section*{Lemma 123}
\begin{lemmatheorembox}
\textbf{Name:} Perpendicular
\textbf{Statement:} If two lines (a and b) are both perpendicular to a third line (c), all of the angles formed along the third line are right angles. Therefore, in Euclidean geometry, any two lines that are both perpendicular to a third line are parallel to each other, because of the parallel postulate.
\textbf{General Topic:} Geometry
\textbf{URL:} https://en.wikipedia.org/wiki/Perpendicular
\end{lemmatheorembox}

\textbf{Usage count:} 7

\section*{Lemma 124}
\begin{lemmatheorembox}
\textbf{Name:} Similarity (geometry)
\textbf{Statement:} If two angles of a triangle have measures equal to the measures of two angles of another triangle, then the triangles are similar. Corresponding sides of similar polygons are in proportion, and corresponding angles of similar polygons have the same measure.
\textbf{General Topic:} Euclidean geometry
\textbf{URL:} https://en.wikipedia.org/wiki/Similarity_%28geometry%29
\end{lemmatheorembox}

\textbf{Usage count:} 7

\section*{Lemma 125}
\begin{lemmatheorembox}
\textbf{Name:} Intersecting chords theorem
\textbf{Statement:} More precisely, for two chords AC and BD intersecting in a point S the following equation holds: |AS|\cdot|SC|=|BS|\cdot|SD|
\textbf{General Topic:} Euclidean geometry
\textbf{URL:} https://en.wikipedia.org/wiki/Intersecting_chords_theorem
\end{lemmatheorembox}

\textbf{Usage count:} 7

\section*{Lemma 126}
\begin{lemmatheorembox}
\textbf{Name:} Cyclotomic polynomial
\textbf{Statement:} Its roots are all n-th primitive roots of unity
\textbf{General Topic:} Number theory
\textbf{URL:} https://en.wikipedia.org/wiki/Cyclotomic_polynomial
\end{lemmatheorembox}

\textbf{Usage count:} 7

\section*{Lemma 127}
\begin{lemmatheorembox}
\textbf{Name:} Centroid
\textbf{Statement:} If the set of vertices of a simplex is {v_0,\ldots,v_n}, then considering the vertices as vectors, the centroid is

C = \frac{1}{n+1}\sum_{i=0}^n v_i.
\textbf{General Topic:} Geometry
\textbf{URL:} https://en.wikipedia.org/wiki/Centroid
\end{lemmatheorembox}

\textbf{Usage count:} 7

\section*{Lemma 128}
\begin{lemmatheorembox}
\textbf{Name:} Rule of sum
\textbf{Statement:} More formally, the sum of the sizes of two disjoint sets is equal to the size of their union.
\textbf{General Topic:} Combinatorics
\textbf{URL:} https://en.wikipedia.org/wiki/Combinatorial_principles
\end{lemmatheorembox}

\textbf{Usage count:} 7

\section*{Lemma 129}
\begin{lemmatheorembox}
\textbf{Name:} Trigonometric functions
\textbf{Statement:} $\tan(z+\pi )=\tan(z),\quad \cot(z+\pi )=\cot(z).$
\textbf{General Topic:} Trigonometry
\textbf{URL:} https://en.wikipedia.org/wiki/Trigonometric_functions
\end{lemmatheorembox}

\textbf{Usage count:} 7

\section*{Lemma 130}
\begin{lemmatheorembox}
\textbf{Name:} Cyclic permutation
\textbf{Statement:} One of the basic results on symmetric groups is that any permutation can be expressed as the product of disjoint cycles (more precisely: cycles with disjoint orbits); such cycles commute with each other, and the expression of the permutation is unique up to the order of the cycles.
\textbf{General Topic:} Group Theory
\textbf{URL:} https://en.wikipedia.org/wiki/Cyclic_permutation
\end{lemmatheorembox}

\textbf{Usage count:} 7

\section*{Lemma 131}
\begin{lemmatheorembox}
\textbf{Name:} Monotonic function
\textbf{Statement:} Functions that are strictly monotone are one-to-one (because for x not equal to y, either x < y or x > y and so, by monotonicity, either f\!\left(x\right) < f\!\left(y\right) or f\!\left(x\right) > f\!\left(y\right), thus f\!\left(x\right) \neq f\!\left(y\right).
\textbf{General Topic:} Calculus and Analysis
\textbf{URL:} https://en.wikipedia.org/wiki/Monotonic_function
\end{lemmatheorembox}

\textbf{Usage count:} 7

\section*{Lemma 132}
\begin{lemmatheorembox}
\textbf{Name:} Continuous uniform distribution
\textbf{Statement:} The probability that a continuously uniformly distributed random variable falls within any interval of fixed length is independent of the location of the interval itself (but it is dependent on the interval size $(\ell)$), so long as the interval is contained in the distribution's support. Indeed, if $X\sim U(a,b)$ and if $[x,x+\ell ]$ is a subinterval of $[a,b]$ with fixed $\ell >0$, then: $\Pr \big( X\in [x,x+\ell ] \big)=\int _{x}^{x+\ell }{\frac {dy}{b-a}}={\frac {\ell }{b-a}},$ which is independent of $x$.
\textbf{General Topic:} Probability theory
\textbf{URL:} https://en.wikipedia.org/wiki/Continuous_uniform_distribution
\end{lemmatheorembox}

\textbf{Usage count:} 7

\section*{Lemma 133}
\begin{lemmatheorembox}
\textbf{Name:} Properties of equality
\textbf{Statement:} if a = b then a + c = b + c and ac = bc;
\textbf{General Topic:} Elementary algebra
\textbf{URL:} https://en.wikipedia.org/wiki/Elementary_algebra#Properties_of_equality
\end{lemmatheorembox}

\textbf{Usage count:} 7

\section*{Lemma 134}
\begin{lemmatheorembox}
\textbf{Name:} Apollonius's theorem
\textbf{Statement:} In any triangle ABC, if AD is a median (|BD| = |CD|), then |AB|^{2}+|AC|^{2}=2(|BD|^{2}+|AD|^{2}).
\textbf{General Topic:} Euclidean geometry
\textbf{URL:} https://en.wikipedia.org/wiki/Apollonius%27s_theorem
\end{lemmatheorembox}

\textbf{Usage count:} 6

\section*{Lemma 135}
\begin{lemmatheorembox}
\textbf{Name:} e (mathematical constant)
\textbf{Statement:} The number e is the limit \(\lim _{n\to \infty }\left(1+{\frac {1}{n}}\right)^{n}\), an expression that arises in the computation of compound interest.
\textbf{General Topic:} Mathematical analysis
\textbf{URL:} https://en.wikipedia.org/wiki/E_%28mathematical_constant%29
\end{lemmatheorembox}

\textbf{Usage count:} 6

\section*{Lemma 136}
\begin{lemmatheorembox}
\textbf{Name:} Wilson's theorem
\textbf{Statement:} In algebra and number theory, Wilson's theorem states that a natural number n > 1 is a prime number if and only if the product of all the positive integers less than n is one less than a multiple of n. That is (using the notations of modular arithmetic), the factorial (n - 1)! = 1 × 2 × 3 × ⋯ × (n - 1) satisfies (n-1)!\ \equiv\; -1{\pmod {n}} exactly when n is a prime number. In other words, any integer n > 1 is a prime number if, and only if, (n − 1)!\ +\ 1 is divisible by n.
\textbf{General Topic:} Number theory
\textbf{URL:} https://en.wikipedia.org/wiki/Wilson%27s_theorem
\end{lemmatheorembox}

\textbf{Usage count:} 6

\section*{Lemma 137}
\begin{lemmatheorembox}
\textbf{Name:} Radical axis
\textbf{Statement:} \detokenize{If the circles have two points in common, the radical axis is the common secant line of the circles. In any case the radical axis is a line perpendicular to \overline{M_1M_2}.}
\textbf{General Topic:} Euclidean geometry
\textbf{URL:} https://en.wikipedia.org/wiki/Radical_axis
\end{lemmatheorembox}

\textbf{Usage count:} 6

\section*{Lemma 138}
\begin{lemmatheorembox}
\textbf{Name:} Sine and cosine
\textbf{Statement:} both sine and cosine functions have the range between \(-1\leq y\leq 1\).
\textbf{General Topic:} Trigonometry
\textbf{URL:} https://en.wikipedia.org/wiki/Sine_and_cosine
\end{lemmatheorembox}

\textbf{Usage count:} 6

\section*{Lemma 139}
\begin{lemmatheorembox}
\textbf{Name:} Reflection (mathematics)  
\textbf{Statement:} In mathematics, a reflection (also spelled reflexion) is a mapping from a Euclidean space to itself that is an isometry with a hyperplane as the set of fixed points; this set is called the axis (in dimension 2) or plane (in dimension 3) of reflection.  
\textbf{General Topic:} Euclidean Geometry  
\textbf{URL:} https://en.wikipedia.org/wiki/Reflection_%28mathematics%29
\end{lemmatheorembox}

\textbf{Usage count:} 6

\section*{Lemma 140}
\begin{lemmatheorembox}
\textbf{Name:} Incenter
\textbf{Statement:} The incenter may be equivalently defined as the point where the internal angle bisectors of the triangle cross, as the point equidistant from the triangle's sides, as the junction point of the medial axis and innermost point of the grassfire transform of the triangle, and as the center point of the inscribed circle of the triangle.
\textbf{General Topic:} Euclidean geometry
\textbf{URL:} https://en.wikipedia.org/wiki/Incenter
\end{lemmatheorembox}

\textbf{Usage count:} 6

\section*{Lemma 141}
\begin{lemmatheorembox}
\textbf{Name:} Pyramid (geometry)
\textbf{Statement:} The volume of a pyramid is the one-third product of the base's area and the height. The pyramid height is defined as the length of the line segment between the apex and its orthogonal projection on the base. Given that $B$ is the base's area and $h$ is the height of a pyramid, the volume of a pyramid is: $V={\frac {1}{3}}Bh$.
\textbf{General Topic:} Geometry
\textbf{URL:} https://en.wikipedia.org/wiki/Pyramid_(geometry)
\end{lemmatheorembox}

\textbf{Usage count:} 6

\section*{Lemma 142}
\begin{lemmatheorembox}
\textbf{Name:} Midpoint theorem (triangle)
\textbf{Statement:} The midpoint theorem, midsegment theorem, or midline theorem states that if the midpoints of two sides of a triangle are connected, then the resulting line segment will be parallel to the third side and have half of its length.
\textbf{General Topic:} Euclidean geometry
\textbf{URL:} https://en.wikipedia.org/wiki/Midpoint_theorem_(triangle)
\end{lemmatheorembox}

\textbf{Usage count:} 6

\section*{Lemma 143}
\begin{lemmatheorembox}
\textbf{Name:} Principle of indifference
\textbf{Statement:} The principle of indifference states that in the absence of any relevant evidence, agents should distribute their credence (or "degrees of belief") equally among all the possible outcomes under consideration.
\textbf{General Topic:} Bayesian probability
\textbf{URL:} https://en.wikipedia.org/wiki/Principle_of_indifference
\end{lemmatheorembox}

\textbf{Usage count:} 6

\section*{Lemma 144}
\begin{lemmatheorembox}
\textbf{Name:} Linearity of expectation
\textbf{Statement:} The expected value operator (or expectation operator) $\operatorname{E}[\cdot]$ is linear in the sense that, for any random variables $X$ and $Y$, and a constant $a$,
\begin{align}
\operatorname{E}[X+Y] &= \operatorname{E}[X]+\operatorname{E}[Y],\\
\operatorname{E}[aX] &= a\operatorname{E}[X],
\end{align}
whenever the right-hand side is well-defined.
\textbf{General Topic:} Probability theory
\textbf{URL:} https://en.wikipedia.org/wiki/Expected_value#Properties
\end{lemmatheorembox}

\textbf{Usage count:} 6

\section*{Lemma 145}
\begin{lemmatheorembox}
\textbf{Name:} Intersecting secants theorem
\textbf{Statement:} For two lines AD and BC that intersect each other at P and for which A, B, C, D all lie on the same circle, the following equation holds: $|PA|\cdot |PD|=|PB|\cdot |PC|$
\textbf{General Topic:} Euclidean geometry
\textbf{URL:} https://en.wikipedia.org/wiki/Intersecting_secants_theorem
\end{lemmatheorembox}

\textbf{Usage count:} 6

\section*{Lemma 146}
\begin{lemmatheorembox}
\textbf{Name:} Area of a triangle
\textbf{Statement:} T = \tfrac{1}{2}bh.
\textbf{General Topic:} Geometry
\textbf{URL:} https://en.wikipedia.org/wiki/Triangle
\end{lemmatheorembox}

\textbf{Usage count:} 6

\section*{Lemma 147}
\begin{lemmatheorembox}
\textbf{Name:} Secant--secant theorem
\textbf{Statement:} If two secants are inscribed in the circle as shown at right, then the measurement of angle A is equal to one half the difference of the measurements of the enclosed arcs ($\overset{\frown}{DE}$ and $\overset{\frown}{BC}$). That is, $2\angle {CAB}=\angle {DOE}-\angle {BOC}$, where O is the centre of the circle (secant--secant theorem).
\textbf{General Topic:} Euclidean geometry
\textbf{URL:} https://en.wikipedia.org/wiki/Circle
\end{lemmatheorembox}

\textbf{Usage count:} 6

\section*{Lemma 148}
\begin{lemmatheorembox}
\textbf{Name:} Lattice path
\textbf{Statement:} The number of lattice paths from $(0,0)$ to $(n,k)$ is equal to the binomial coefficient $\binom{n+k}{n}$.
\textbf{General Topic:} Combinatorics
\textbf{URL:} https://en.wikipedia.org/wiki/Lattice_path
\end{lemmatheorembox}

\textbf{Usage count:} 6

\section*{Lemma 149}
\begin{lemmatheorembox}
\textbf{Name:} Sum of two cubes
\textbf{Statement:} Every sum of cubes may be factored according to the identity
\[
a^{3}+b^{3}=(a+b)(a^{2}-ab+b^{2})
\]
and
\[
a^{3}-b^{3}=(a-b)(a^{2}+ab+b^{2}).
\]
\textbf{General Topic:} Elementary algebra
\textbf{URL:} https://en.wikipedia.org/wiki/Sum_of_two_cubes
\end{lemmatheorembox}

\textbf{Usage count:} 6

\section*{Lemma 150}
\begin{lemmatheorembox}
\textbf{Name:} Segment addition postulate
\textbf{Statement:} In geometry, the segment addition postulate states that given 2 points A and C, a third point B lies on the line segment AC if and only if the distances between the points satisfy the equation AB + BC = AC.
\textbf{General Topic:} Elementary geometry
\textbf{URL:} https://en.wikipedia.org/wiki/Segment_addition_postulate
\end{lemmatheorembox}

\textbf{Usage count:} 6

\section*{Lemma 151}
\begin{lemmatheorembox}
\textbf{Name:} Discriminant
\textbf{Statement:} If a, b, c are rational numbers, then the discriminant is the square of a rational number if and only if the two roots are rational numbers.
\textbf{General Topic:} Algebra
\textbf{URL:} https://en.wikipedia.org/wiki/Discriminant
\end{lemmatheorembox}

\textbf{Usage count:} 6

\section*{Lemma 152}
\begin{lemmatheorembox}
\textbf{Name:} Incircle and excircles
\textbf{Statement:} The tangency points of the incircle divide the sides into segments of lengths $s-a$ from $A$, $s-b$ from $B$, and $s-c$ from $C$ (see Tangent lines to a circle).
\textbf{General Topic:} Euclidean geometry
\textbf{URL:} https://en.wikipedia.org/wiki/Incircle_and_excircles
\end{lemmatheorembox}

\textbf{Usage count:} 6

\section*{Lemma 153}
\begin{lemmatheorembox}
\textbf{Name:} Determinant
\textbf{Statement:} The absolute value of ad − bc is the area of the parallelogram, and thus represents the scale factor by which areas are transformed by A.
\textbf{General Topic:} Linear Algebra
\textbf{URL:} https://en.wikipedia.org/wiki/Determinant
\end{lemmatheorembox}

\textbf{Usage count:} 6

\section*{Lemma 154}
\begin{lemmatheorembox}
\textbf{Name:} Rule of division (combinatorics)
\textbf{Statement:} In combinatorics, the rule of division is a counting principle. It states that there are n/d ways to do a task if it can be done using a procedure that can be carried out in n ways, and for each way w, exactly d of the n ways correspond to the way w.
\textbf{General Topic:} Combinatorics
\textbf{URL:} https://en.wikipedia.org/wiki/Rule_of_division_(combinatorics)
\end{lemmatheorembox}

\textbf{Usage count:} 6

\section*{Lemma 155}
\begin{lemmatheorembox}
\textbf{Name:} Additive inverse
\textbf{Statement:} Given an algebraic structure defined under addition \((S, +)\) with an additive identity \(e \in S\), an element \(x \in S\) has an additive inverse \(y\) if and only if \(y \in S\), \(x + y = e\), and \(y + x = e\).
\textbf{General Topic:} Abstract algebra
\textbf{URL:} https://en.wikipedia.org/wiki/Additive_inverse
\end{lemmatheorembox}

\textbf{Usage count:} 6

\section*{Lemma 156}
\begin{lemmatheorembox}
\textbf{Name:} Inequality (mathematics)
\textbf{Statement:} If $a \leq b$ and $c > 0$, then $ac \leq bc$ and $a/c \leq b/c$.

If $a \leq b$ and $c < 0$, then $ac \geq bc$ and $a/c \geq b/c$.
\textbf{General Topic:} Inequalities
\textbf{URL:} https://en.wikipedia.org/wiki/Inequality_(mathematics)
\end{lemmatheorembox}

\textbf{Usage count:} 6

\section*{Lemma 157}
\begin{lemmatheorembox}
\textbf{Name:} Binomial distribution
\textbf{Statement:} If X \sim B(n, p), that is, X is a binomially distributed random variable, n being the total number of experiments and p the probability of each experiment yielding a successful result, then the expected value of X is: \(\operatorname {E} [X]=np.\)
\textbf{General Topic:} Probability theory
\textbf{URL:} https://en.wikipedia.org/wiki/Binomial_distribution
\end{lemmatheorembox}

\textbf{Usage count:} 5

\section*{Lemma 158}
\begin{lemmatheorembox}
\textbf{Name:} Hockey-stick identity
\textbf{Statement:} In combinatorics, the hockey-stick identity, Christmas stocking identity, boomerang identity, Fermat's identity or Chu's Theorem, states that if \(n \geq r \geq 0\) are integers, then
\[
{\binom {r}{r}}+{\binom {r+1}{r}}+{\binom {r+2}{r}}+\cdots +{\binom {n}{r}}={\binom {n+1}{r+1}}.
\]
\textbf{General Topic:} Combinatorics
\textbf{URL:} https://en.wikipedia.org/wiki/Hockey-stick_identity
\end{lemmatheorembox}

\textbf{Usage count:} 5

\section*{Lemma 159}
\begin{lemmatheorembox}
\textbf{Name:} Discriminant
\textbf{Statement:} If the discriminant is negative, then there are no real roots.
\textbf{General Topic:} Algebra
\textbf{URL:} https://en.wikipedia.org/wiki/Quadratic_equation
\end{lemmatheorembox}

\textbf{Usage count:} 5

\section*{Lemma 160}
\begin{lemmatheorembox}
\textbf{Name:} Semiperimeter
\textbf{Statement:} The area $A$ of any triangle is the product of its inradius (the radius of its inscribed circle) and its semiperimeter: $A=rs$.
\textbf{General Topic:} Geometry
\textbf{URL:} https://en.wikipedia.org/wiki/Semiperimeter
\end{lemmatheorembox}

\textbf{Usage count:} 5

\section*{Lemma 161}
\begin{lemmatheorembox}
\textbf{Name:} Shoelace formula
\textbf{Statement:} The triangle formula sums up the oriented areas $A_i$ of triangles $OP_iP_{i+1}$:
\[
\begin{aligned}
A&={\frac {1}{2}}\sum _{i=1}^{n}(x_{i}y_{i+1}-x_{i+1}y_{i})={\frac {1}{2}}\sum _{i=1}^{n}{\begin{vmatrix}x_{i}&x_{i+1}\\y_{i}&y_{i+1}\end{vmatrix}}={\frac {1}{2}}\sum _{i=1}^{n}{\begin{vmatrix}x_{i}&y_{i}\\x_{i+1}&y_{i+1}\end{vmatrix}}\\
&={\frac {1}{2}}{\Big (}x_{1}y_{2}-x_{2}y_{1}+x_{2}y_{3}-x_{3}y_{2}+\cdots +x_{n}y_{1}-x_{1}y_{n}{\Big )}
\end{aligned}
\]
\textbf{General Topic:} Geometry
\textbf{URL:} https://en.wikipedia.org/wiki/Shoelace_formula
\end{lemmatheorembox}

\textbf{Usage count:} 5

\section*{Lemma 162}
\begin{lemmatheorembox}
\textbf{Name:} root of unity filter
\textbf{Statement:} \sum_{m=0}^\infty a_{qm+p}\cdot z^{qm+p} = \frac{1}{q}\cdot \sum_{k=0}^{q-1} \omega^{-kp}\cdot f(\omega^k\cdot z),
\textbf{General Topic:} Generating functions
\textbf{URL:} https://en.wikipedia.org/wiki/Series_multisection
\end{lemmatheorembox}

\textbf{Usage count:} 5

\section*{Lemma 163}
\begin{lemmatheorembox}
\textbf{Name:} Stewart's theorem
\textbf{Statement:} Let a, b, c be the lengths of the sides of a triangle. Let d be the length of a cevian to the side of length a. If the cevian divides the side of length a into two segments of length m and n, with m adjacent to c and n adjacent to b, then Stewart's theorem states that $b^{2}m+c^{2}n=a(d^{2}+mn)$.
\textbf{General Topic:} Euclidean geometry
\textbf{URL:} https://en.wikipedia.org/wiki/Stewart%27s_theorem
\end{lemmatheorembox}

\textbf{Usage count:} 5

\section*{Lemma 164}
\begin{lemmatheorembox}
\textbf{Name:} Congruence (geometry)
\textbf{Statement:} ASA (angle-side-angle): If two pairs of angles of two triangles are equal in measurement, and the included sides are equal in length, then the triangles are congruent.
\textbf{General Topic:} Euclidean geometry
\textbf{URL:} https://en.wikipedia.org/wiki/Congruence_%28geometry%29
\end{lemmatheorembox}

\textbf{Usage count:} 5

\section*{Lemma 165}
\begin{lemmatheorembox}
\textbf{Name:} Square (algebra)
\textbf{Statement:} The square function preserves the order of positive numbers: larger numbers have larger squares.
\textbf{General Topic:} Elementary algebra
\textbf{URL:} https://en.wikipedia.org/wiki/Square_(algebra)
\end{lemmatheorembox}

\textbf{Usage count:} 5

\section*{Lemma 166}
\begin{lemmatheorembox}
\textbf{Name:} SSS (side-side-side)
\textbf{Statement:} SSS (side-side-side): If three pairs of sides of two triangles are equal in length, then the triangles are congruent.
\textbf{General Topic:} Euclidean geometry
\textbf{URL:} https://en.wikipedia.org/wiki/Congruence_(geometry)
\end{lemmatheorembox}

\textbf{Usage count:} 5

\section*{Lemma 167}
\begin{lemmatheorembox}
\textbf{Name:} Square--cube law
\textbf{Statement:} $V_2=V_1\left(\frac{\ell_2}{\ell_1}\right)^3$
\textbf{General Topic:} Geometry
\textbf{URL:} https://en.wikipedia.org/wiki/Square%E2%80%93cube_law
\end{lemmatheorembox}

\textbf{Usage count:} 5

\section*{Lemma 168}
\begin{lemmatheorembox}
\textbf{Name:} Bienaymé formula
\textbf{Statement:} One reason for the use of the variance in preference to other measures of dispersion is that the variance of the sum (or the difference) of uncorrelated random variables is the sum of their variances: \(\operatorname{Var}\left(\sum_{i=1}^n X_i\right) = \sum_{i=1}^n \operatorname{Var}(X_i).\)
\textbf{General Topic:} Probability theory / Statistics
\textbf{URL:} https://en.wikipedia.org/wiki/Variance
\end{lemmatheorembox}

\textbf{Usage count:} 5

\section*{Lemma 169}
\begin{lemmatheorembox}
\textbf{Name:} Orbit–stabilizer theorem
\textbf{Statement:} Therefore f induces a bijection between the set $G / G_{x}$ of cosets for the stabilizer subgroup and the orbit $G\cdot x$, which sends $gG_{x} \mapsto g\cdot x$.
\textbf{General Topic:} Group theory
\textbf{URL:} https://en.wikipedia.org/wiki/Group_action#Orbit%E2%80%93stabilizer_theorem
\end{lemmatheorembox}

\textbf{Usage count:} 5

\section*{Lemma 170}
\begin{lemmatheorembox}
\textbf{Name:} Probability axioms
\textbf{Statement:} This is the assumption of $\sigma$-additivity: Any countable sequence of disjoint sets (synonymous with mutually exclusive events) $E_{1},E_{2},\ldots$ satisfies $P\left(\bigcup _{i=1}^{\infty }E_{i}\right)=\sum _{i=1}^{\infty }P(E_{i}).$
\textbf{General Topic:} Probability theory
\textbf{URL:} https://en.wikipedia.org/wiki/Probability_axioms
\end{lemmatheorembox}

\textbf{Usage count:} 5

\section*{Lemma 171}
\begin{lemmatheorembox}
\textbf{Name:} Axiom of induction
\textbf{Statement:} $\forall P\,{\Bigl (}P(0)\land \forall k{\bigl (}P(k)\to P(k+1){\bigr )}\to \forall n\,{\bigl (}P(n){\bigr )}{\Bigr )}$
\textbf{General Topic:} Mathematical logic
\textbf{URL:} https://en.wikipedia.org/wiki/Mathematical_induction#Formalization
\end{lemmatheorembox}

\textbf{Usage count:} 5

\section*{Lemma 172}
\begin{lemmatheorembox}
\textbf{Name:} Area
\textbf{Statement:} The triangle of largest area of all those inscribed in a given circle is equilateral; and the triangle of smallest area of all those circumscribed around a given circle is equilateral.
\textbf{General Topic:} Geometry
\textbf{URL:} https://en.wikipedia.org/wiki/Area
\end{lemmatheorembox}

\textbf{Usage count:} 5

\section*{Lemma 173}
\begin{lemmatheorembox}
\textbf{Name:} Chain rule
\textbf{Statement:} h'(x) = f'(g(x)) g'(x).
\textbf{General Topic:} Calculus
\textbf{URL:} https://en.wikipedia.org/wiki/Chain_rule
\end{lemmatheorembox}

\textbf{Usage count:} 5

\section*{Lemma 174}
\begin{lemmatheorembox}
\textbf{Name:} Right triangle
\textbf{Statement:} Each leg of the triangle is the mean proportional of the hypotenuse and the segment of the hypotenuse that is adjacent to the leg.
\textbf{General Topic:} Euclidean geometry
\textbf{URL:} https://en.wikipedia.org/wiki/Right_triangle
\end{lemmatheorembox}

\textbf{Usage count:} 5

\section*{Lemma 175}
\begin{lemmatheorembox}
\textbf{Name:} Change of base
\textbf{Statement:} The logarithm $\log _{b}x$ can be computed from the logarithms of $x$ and $b$ with respect to an arbitrary base $k$ using the following formula: $\log _{b}x={\frac {\log _{k}x}{\log _{k}b}}$.
\textbf{General Topic:} Logarithms
\textbf{URL:} https://en.wikipedia.org/wiki/Logarithm#Change_of_base
\end{lemmatheorembox}

\textbf{Usage count:} 5

\section*{Lemma 176}
\begin{lemmatheorembox}
\textbf{Name:} Trapezoid  
\textbf{Statement:} The area $K$ of a trapezoid is given by the product of the midsegment (the average of the two bases) and the height: $K={\tfrac {1}{2}}(a+b)h$ where $a$ and $b$ are the lengths of the bases, and $h$ is the height (the perpendicular distance between these sides).  
\textbf{General Topic:} Geometry  
\textbf{URL:} https://en.wikipedia.org/wiki/Trapezoid
\end{lemmatheorembox}

\textbf{Usage count:} 5

\section*{Lemma 177}
\begin{lemmatheorembox}
\textbf{Name:} Quadratic function
\textbf{Statement:} the average of the two roots, i.e., \frac{r_{1}+r_{2}}{2} is the x-coordinate of the vertex
\textbf{General Topic:} Algebra
\textbf{URL:} https://en.wikipedia.org/wiki/Quadratic_function
\end{lemmatheorembox}

\textbf{Usage count:} 5

\section*{Lemma 178}
\begin{lemmatheorembox}
\textbf{Name:} Parity (mathematics)
\textbf{Statement:} 
\begin{itemize}
\item even $\pm$ even = even;
\item even $\pm$ odd = odd;
\item odd $\pm$ odd = even;
\end{itemize}
\textbf{General Topic:} Number theory
\textbf{URL:} https://en.wikipedia.org/wiki/Parity_%28mathematics%29
\end{lemmatheorembox}

\textbf{Usage count:} 5

\section*{Lemma 179}
\begin{lemmatheorembox}
\textbf{Name:} Ordered field
\textbf{Statement:} One can "multiply inequalities with positive elements": if $a \le b$ and $0 \le c$, then $ac \le bc$.
\textbf{General Topic:} Abstract algebra
\textbf{URL:} https://en.wikipedia.org/wiki/Ordered_field
\end{lemmatheorembox}

\textbf{Usage count:} 5

\section*{Lemma 180}
\begin{lemmatheorembox}
\textbf{Name:} Bipartite graph
\textbf{Statement:} An undirected graph is bipartite if and only if it does not contain an odd cycle.
\textbf{General Topic:} Graph Theory
\textbf{URL:} https://en.wikipedia.org/wiki/Bipartite_graph
\end{lemmatheorembox}

\textbf{Usage count:} 4

\section*{Lemma 181}
\begin{lemmatheorembox}
\textbf{Name:} Archimedean property
\textbf{Statement:} The property, as typically construed, states that given two positive numbers \(x\) and \(y\), there is an integer \(n\) such that \(nx>y\).
\textbf{General Topic:} Analysis
\textbf{URL:} https://en.wikipedia.org/wiki/Archimedean_property
\end{lemmatheorembox}

\textbf{Usage count:} 4

\section*{Lemma 182}
\begin{lemmatheorembox}
\textbf{Name:} Lagrange polynomial
\textbf{Statement:} In numerical analysis, the Lagrange interpolating polynomial is the unique polynomial of lowest degree that interpolates a given set of data.
\textbf{General Topic:} Numerical analysis
\textbf{URL:} https://en.wikipedia.org/wiki/Lagrange_polynomial
\end{lemmatheorembox}

\textbf{Usage count:} 4

\section*{Lemma 183}
\begin{lemmatheorembox}
\textbf{Name:} Multiplicative inverse
\textbf{Statement:} This multiplicative inverse exists if and only if a and n are coprime.
\textbf{General Topic:} Modular arithmetic
\textbf{URL:} https://en.wikipedia.org/wiki/Multiplicative_inverse
\end{lemmatheorembox}

\textbf{Usage count:} 4

\section*{Lemma 184}
\begin{lemmatheorembox}
\textbf{Name:} Jensen's inequality
\textbf{Statement:} For a real convex function \(\varphi\), numbers \(x_{1},x_{2},\ldots ,x_{n}\) in its domain, and positive weights \(a_{i}\), Jensen's inequality can be stated as: \(\varphi \left({\frac {\sum a_{i}x_{i}}{\sum a_{i}}}\right)\leq {\frac {\sum a_{i}\varphi (x_{i})}{\sum a_{i}}}\).
\textbf{General Topic:} Mathematical analysis
\textbf{URL:} https://en.wikipedia.org/wiki/Jensen%27s_inequality
\end{lemmatheorembox}

\textbf{Usage count:} 4

\section*{Lemma 185}
\begin{lemmatheorembox}
\textbf{Name:} Order (group theory)
\textbf{Statement:} For any integer k, we have a\textsuperscript{k} = e if and only if ord(a) divides k.
\textbf{General Topic:} Group theory
\textbf{URL:} https://en.wikipedia.org/wiki/Order_%28group_theory%29
\end{lemmatheorembox}

\textbf{Usage count:} 4

\section*{Lemma 186}
\begin{lemmatheorembox}
\textbf{Name:} Division algorithm
\textbf{Statement:} a = bq + r where 0 \leq r < |b|.
\textbf{General Topic:} Elementary arithmetic
\textbf{URL:} https://en.wikipedia.org/wiki/Division_algorithm
\end{lemmatheorembox}

\textbf{Usage count:} 4

\section*{Lemma 187}
\begin{lemmatheorembox}
\textbf{Name:} Orthocenter
\textbf{Statement:} \detokenize{The orthocenter of a triangle, usually denoted by H, is the point where the three (possibly extended) altitudes intersect.}
\textbf{General Topic:} Triangle geometry
\textbf{URL:} https://en.wikipedia.org/wiki/Orthocenter
\end{lemmatheorembox}

\textbf{Usage count:} 4

\section*{Lemma 188}
\begin{lemmatheorembox}
\textbf{Name:} Interior extremum theorem
\textbf{Statement:} Let f\colon (a,b) \rightarrow \mathbb{R} be a function from an open interval to , and suppose that x_0 \in (a,b) is a point where f has a local extremum. If f is differentiable at x_0, then f'(x_0) = 0.
\textbf{General Topic:} Real analysis
\textbf{URL:} https://en.wikipedia.org/wiki/Interior_extremum_theorem
\end{lemmatheorembox}

\textbf{Usage count:} 4

\section*{Lemma 189}
\begin{lemmatheorembox}
\textbf{Name:} Mean value theorem
\textbf{Statement:} Let \(f:[a,b]\to \mathbb {R} \) be a continuous function on the closed interval \([a,b]\), and differentiable on the open interval \((a,b)\), where \(a<b\). Then there exists some \(c\) in \((a,b)\) such that:
\[
f'(c)={\frac {f(b)-f(a)}{b-a}}.
\]
\textbf{General Topic:} Real analysis
\textbf{URL:} https://en.wikipedia.org/wiki/Mean_value_theorem
\end{lemmatheorembox}

\textbf{Usage count:} 4

\section*{Lemma 190}
\begin{lemmatheorembox}
\textbf{Name:} Complex conjugate root theorem
\textbf{Statement:} In mathematics, the complex conjugate root theorem states that if P is a polynomial in one variable with real coefficients, and a + bi is a root of P with a and b being real numbers, then its complex conjugate a − bi is also a root of P.
\textbf{General Topic:} Algebra
\textbf{URL:} https://en.wikipedia.org/wiki/Complex_conjugate_root_theorem
\end{lemmatheorembox}

\textbf{Usage count:} 4

\section*{Lemma 191}
\begin{lemmatheorembox}
\textbf{Name:} Vandermonde's identity
\textbf{Statement:} In combinatorics, Vandermonde's identity (or Vandermonde's convolution) is the following identity for binomial coefficients: {m+n \choose r}=\sum _{k=0}^{r}{m \choose k}{n \choose r-k} for any nonnegative integers r, m, n.
\textbf{General Topic:} Combinatorics
\textbf{URL:} https://en.wikipedia.org/wiki/Vandermonde%27s_identity
\end{lemmatheorembox}

\textbf{Usage count:} 4

\section*{Lemma 192}
\begin{lemmatheorembox}
\textbf{Name:} Squeeze theorem
\textbf{Statement:} This theorem is also valid for sequences. Let $(a_{n})$, $(c_{n})$ be two sequences converging to $\ell$, and $(b_{n})$ a sequence. If $\forall n\geq N,N\in \mathbb {N}$ we have $a_{n} \leq b_{n} \leq c_{n}$, then $(b_{n})$ also converges to $\ell$.
\textbf{General Topic:} Mathematical Analysis
\textbf{URL:} https://en.wikipedia.org/wiki/Squeeze_theorem
\end{lemmatheorembox}

\textbf{Usage count:} 4

\section*{Lemma 193}
\begin{lemmatheorembox}
\textbf{Name:} Monotone convergence theorem
\textbf{Statement:} Theorem: Let $(a_{n})_{n\in \mathbb {N} }$ be a monotone sequence of real numbers (either $a_{n}\leq a_{n+1}$ for all $n$ or $a_{n}\geq a_{n+1}$ for all $n$). Then the following are equivalent: 1. $(a_{n})$ has a finite limit in $\mathbb {R}$. 2. $(a_{n})$ is bounded. Moreover, if $(a_{n})$ is nondecreasing, then $\lim _{n\to \infty }a_{n}=\sup _{n}a_{n}$; if $(a_{n})$ is nonincreasing, then $\lim _{n\to \infty }a_{n}=\inf _{n}a_{n}$.
\textbf{General Topic:} Real Analysis
\textbf{URL:} https://en.wikipedia.org/wiki/Monotone_convergence_theorem
\end{lemmatheorembox}

\textbf{Usage count:} 4

\section*{Lemma 194}
\begin{lemmatheorembox}
\textbf{Name:} Pisano period
\textbf{Statement:} For any integer n, the sequence of Fibonacci numbers $F_{i}$ taken modulo n is periodic.
\textbf{General Topic:} Number theory
\textbf{URL:} https://en.wikipedia.org/wiki/Pisano_period
\end{lemmatheorembox}

\textbf{Usage count:} 4

\section*{Lemma 195}
\begin{lemmatheorembox}
\textbf{Name:} Midpoint theorem (triangle)
\textbf{Statement:} The midpoint theorem, midsegment theorem, or midline theorem states that if the midpoints of two sides of a triangle are connected, then the resulting line segment will be parallel to the third side and have half of its length.
\textbf{General Topic:} Euclidean geometry
\textbf{URL:} https://en.wikipedia.org/wiki/Midpoint_theorem_%28triangle%29
\end{lemmatheorembox}

\textbf{Usage count:} 4

\section*{Lemma 196}
\begin{lemmatheorembox}
\textbf{Name:} Angle sum and difference identities
\textbf{Statement:} \tan(\alpha \pm \beta ) = {\frac {\tan \alpha \pm \tan \beta }{1\mp \tan \alpha \tan \beta }}
\textbf{General Topic:} Trigonometry
\textbf{URL:} https://en.wikipedia.org/wiki/List_of_trigonometric_identities#Angle_sum_and_difference_identities
\end{lemmatheorembox}

\textbf{Usage count:} 4

\section*{Lemma 197}
\begin{lemmatheorembox}
\textbf{Name:} Circular sector
\textbf{Statement:} A = \pi r^2\, \frac{\theta}{2 \pi} = \frac{r^2 \theta}{2}
\textbf{General Topic:} Geometry
\textbf{URL:} https://en.wikipedia.org/wiki/Circular_sector
\end{lemmatheorembox}

\textbf{Usage count:} 4

\section*{Lemma 198}
\begin{lemmatheorembox}
\textbf{Name:} Euler's Theorem
\textbf{Statement:} A connected graph has an Euler cycle if and only if every vertex has an even number of incident edges.
\textbf{General Topic:} Graph Theory
\textbf{URL:} https://en.wikipedia.org/wiki/Eulerian_path
\end{lemmatheorembox}

\textbf{Usage count:} 4

\section*{Lemma 199}
\begin{lemmatheorembox}
\textbf{Name:} Geometric mean theorem
\textbf{Statement:} It states that the geometric mean of those two segments equals the altitude.
\textbf{General Topic:} Euclidean geometry
\textbf{URL:} https://en.wikipedia.org/wiki/Geometric_mean_theorem
\end{lemmatheorembox}

\textbf{Usage count:} 4

\section*{Lemma 200}
\begin{lemmatheorembox}
\textbf{Name:} Symmetric group
\textbf{Statement:} Every permutation can be written as a product of transpositions.
\textbf{General Topic:} Group Theory
\textbf{URL:} https://en.wikipedia.org/wiki/Symmetric_group
\end{lemmatheorembox}

\textbf{Usage count:} 4

\section*{Lemma 201}
\begin{lemmatheorembox}
\textbf{Name:} Median (geometry)
\textbf{Statement:} Each median divides the area of the triangle in half, hence the name.
\textbf{General Topic:} Geometry
\textbf{URL:} https://en.wikipedia.org/wiki/Median_%28geometry%29
\end{lemmatheorembox}

\textbf{Usage count:} 4

\section*{Lemma 202}
\begin{lemmatheorembox}
\textbf{Name:} Coprime integers
\textbf{Statement:} The least common multiple of a and b is equal to their product ab, i.e. lcm(a, b) = ab.
\textbf{General Topic:} Number theory
\textbf{URL:} https://en.wikipedia.org/wiki/Coprime_integers
\end{lemmatheorembox}

\textbf{Usage count:} 4

\section*{Lemma 203}
\begin{lemmatheorembox}
\textbf{Name:} Pyramid (geometry)
\textbf{Statement:} The volume of a pyramid is the one-third product of the base's area and the height. Given that $B$ is the base's area and $h$ is the height of a pyramid, the volume of a pyramid is: $V = \frac{1}{3}Bh$.
\textbf{General Topic:} Geometry
\textbf{URL:} https://en.wikipedia.org/wiki/Pyramid_%28geometry%29
\end{lemmatheorembox}

\textbf{Usage count:} 4

\section*{Lemma 204}
\begin{lemmatheorembox}
\textbf{Name:} (Permutations of multisets)
\textbf{Statement:} then the number of multiset permutations of $M$ is given by the multinomial coefficient,
\[
{n \choose m_{1},m_{2},\ldots ,m_{l}}=\frac{n!}{m_{1}!\,m_{2}!\,\cdots \,m_{l}!}=\frac{\left(\sum _{i=1}^{l}{m_{i}}\right)!}{\prod _{i=1}^{l}{m_{i}!}}.
\]
\textbf{General Topic:} Combinatorics
\textbf{URL:} https://en.wikipedia.org/wiki/Permutation#Permutations_of_multisets
\end{lemmatheorembox}

\textbf{Usage count:} 4

\section*{Lemma 205}
\begin{lemmatheorembox}
\textbf{Name:} Classical definition of probability
\textbf{Statement:} The probability of an event is the ratio of the number of cases favorable to it, to the number of all cases possible when nothing leads us to expect that any one of these cases should occur more than any other, which renders them, for us, equally possible.
\textbf{General Topic:} Probability Theory
\textbf{URL:} https://en.wikipedia.org/wiki/Classical_definition_of_probability
\end{lemmatheorembox}

\textbf{Usage count:} 4

\section*{Lemma 206}
\begin{lemmatheorembox}
\textbf{Name:} The joint distribution of the order statistics of the uniform distribution
\textbf{Statement:} Perhaps surprisingly, the joint density of the n order statistics turns out to be constant: \[ f_{U_{(1)},U_{(2)},\ldots,U_{(n)}}(u_{1},u_{2},\ldots,u_{n}) = n!. \]
\textbf{General Topic:} Statistics
\textbf{URL:} https://en.wikipedia.org/wiki/Order_statistic
\end{lemmatheorembox}

\textbf{Usage count:} 4

\section*{Lemma 207}
\begin{lemmatheorembox}
\textbf{Name:} Burnside's lemma
\textbf{Statement:} Let G be a finite group that acts on a set X. For each g in G, let X\^g denote the set of elements in X that are fixed by g (left invariant by g): that is, X\^g = \{ x \in X : g\cdot x = x\}. Burnside's lemma asserts the following formula for the number of orbits, denoted |X/G|: |X/G| = \frac{1}{|G|}\sum_{g \in G}|X^g|.
\textbf{General Topic:} Group Theory
\textbf{URL:} https://en.wikipedia.org/wiki/Burnside%27s_lemma
\end{lemmatheorembox}

\textbf{Usage count:} 4

\section*{Lemma 208}
\begin{lemmatheorembox}
\textbf{Name:} Absolute value
\textbf{Statement:} For any real number x, the absolute value or modulus of x is denoted by |x|, with a vertical bar on each side of the quantity, and is defined as \(|x|={\begin{cases}x,&{\text{if }}x\geq 0\\-x,&{\text{if }}x<0.\end{cases}}\)
\textbf{General Topic:} Real analysis
\textbf{URL:} https://en.wikipedia.org/wiki/Absolute_value
\end{lemmatheorembox}

\textbf{Usage count:} 4

\section*{Lemma 209}
\begin{lemmatheorembox}
\textbf{Name:} Markov property
\textbf{Statement:} $P(X_t \in A \mid \mathcal{F}_s) = P(X_t \in A\mid X_s).$
\textbf{General Topic:} Probability theory
\textbf{URL:} https://en.wikipedia.org/wiki/Markov_property
\end{lemmatheorembox}

\textbf{Usage count:} 4

\section*{Lemma 210}
\begin{lemmatheorembox}
\textbf{Name:} Theorem two
\textbf{Statement:} For any pair of positive integers n and k, the number of k-tuples of non-negative integers whose sum is n is equal to the number of multisets of size k − 1 taken from a set of size n + 1, or equivalently, the number of multisets of size n taken from a set of size k, and is given by $\binom{n + k - 1}{k - 1}$.
\textbf{General Topic:} Combinatorics
\textbf{URL:} https://en.wikipedia.org/wiki/Stars_and_bars_(combinatorics)#Theorem_two
\end{lemmatheorembox}

\textbf{Usage count:} 4

\section*{Lemma 211}
\begin{lemmatheorembox}
\textbf{Name:} Regular polygon
\textbf{Statement:} The area A of a convex regular n-sided polygon having side s, circumradius R, apothem a, and perimeter p is given by {\displaystyle {\begin{aligned}A&={\tfrac {1}{2}}nsa\\&={\tfrac {1}{2}}pa\\&={\tfrac {1}{4}}ns^{2}\cot \left({\tfrac {\pi }{n}}\right)\\&=na^{2}\tan \left({\tfrac {\pi }{n}}\right)\\&={\tfrac {1}{2}}nR^{2}\sin \left({\tfrac {2\pi }{n}}\right)\end{aligned}}}
\textbf{General Topic:} Geometry
\textbf{URL:} https://en.wikipedia.org/wiki/Regular_polygon
\end{lemmatheorembox}

\textbf{Usage count:} 4

\section*{Lemma 212}
\begin{lemmatheorembox}
\textbf{Name:} Reflection (mathematics)
\textbf{Statement:} In mathematics, a reflection (also spelled reflexion) is a mapping from a Euclidean space to itself that is an isometry with a hyperplane as the set of fixed points; this set is called the axis (in dimension 2) or plane (in dimension 3) of reflection.
\textbf{General Topic:} Geometry
\textbf{URL:} https://en.wikipedia.org/wiki/Reflection_(mathematics)
\end{lemmatheorembox}

\textbf{Usage count:} 4

\section*{Lemma 213}
\begin{lemmatheorembox}
\textbf{Name:} Rectangle
\textbf{Statement:} A convex quadrilateral is a rectangle if and only if it is any one of the following: * a parallelogram with at least one right angle
\textbf{General Topic:} Euclidean geometry
\textbf{URL:} https://en.wikipedia.org/wiki/Rectangle
\end{lemmatheorembox}

\textbf{Usage count:} 4

\section*{Lemma 214}
\begin{lemmatheorembox}
\textbf{Name:} Transitive relation
\textbf{Statement:} A relation R on a set X is transitive if, for all elements a, b, c in X, whenever R relates a to b and b to c, then R also relates a to c.
\textbf{General Topic:} Set theory
\textbf{URL:} https://en.wikipedia.org/wiki/Transitive_relation
\end{lemmatheorembox}

\textbf{Usage count:} 4

\section*{Lemma 215}
\begin{lemmatheorembox}
\textbf{Name:} Turán's theorem
\textbf{Statement:} Turán's theorem states that every graph G with n vertices that does not contain K\_{r+1} as a subgraph has at most as many edges as the Turán graph T(n,r).
\textbf{General Topic:} Graph Theory
\textbf{URL:} https://en.wikipedia.org/wiki/Tur%C3%A1n%27s_theorem
\end{lemmatheorembox}

\textbf{Usage count:} 3

\section*{Lemma 216}
\begin{lemmatheorembox}
\textbf{Name:} Reductio ad absurdum
\textbf{Statement:} If assuming P to be false implies falsehood, then P is true.
\textbf{General Topic:} Logic
\textbf{URL:} https://en.wikipedia.org/wiki/Reductio_ad_absurdum
\end{lemmatheorembox}

\textbf{Usage count:} 3

\section*{Lemma 217}
\begin{lemmatheorembox}
\textbf{Name:} Ceva's theorem
\textbf{Statement:} Let \(D,E,F\) be points on \(BC,CA,AB\). Then \(AD,BE,CF\) are concurrent iff \(\frac{BD}{DC}\cdot\frac{CE}{EA}\cdot\frac{AF}{FB}=1\).
\textbf{General Topic:} Euclidean geometry
\textbf{URL:} https://en.wikipedia.org/wiki/Ceva%27s_theorem
\end{lemmatheorembox}

\textbf{Usage count:} 3

\section*{Lemma 218}
\begin{lemmatheorembox}
\textbf{Name:} Lagrange's theorem (group theory)  
\textbf{Statement:} In the mathematical field of group theory, Lagrange's theorem states that if H is a subgroup of any finite group G, then |H| is a divisor of |G|.  
\textbf{General Topic:} Group Theory  
\textbf{URL:} https://en.wikipedia.org/wiki/Lagrange%27s_theorem_%28group_theory%29
\end{lemmatheorembox}

\textbf{Usage count:} 3

\section*{Lemma 219}
\begin{lemmatheorembox}
\textbf{Name:} Lifting-the-exponent lemma  
\textbf{Statement:} If \(2\mid x-y\) and \(n\) is even, then \(\nu _{2}(x^{n}-y^{n})=\nu _{2}(x-y)+\nu _{2}(x+y)+\nu _{2}(n)-1=\nu _{2}\left({\frac {x^{2}-y^{2}}{2}}\right)+\nu _{2}(n)\).  
\textbf{General Topic:} Number Theory  
\textbf{URL:} https://en.wikipedia.org/wiki/Lifting-the-exponent_lemma
\end{lemmatheorembox}

\textbf{Usage count:} 3

\section*{Lemma 220}
\begin{lemmatheorembox}
\textbf{Name:} Dirichlet's theorem on arithmetic progressions  
\textbf{Statement:} for any two positive coprime integers a and d, there are infinitely many primes of the form a + nd  
\textbf{General Topic:} Number Theory  
\textbf{URL:} https://en.wikipedia.org/wiki/Dirichlet%27s_theorem_on_arithmetic_progressions
\end{lemmatheorembox}

\textbf{Usage count:} 3

\section*{Lemma 221}
\begin{lemmatheorembox}
\textbf{Name:} Pell's equation
\textbf{Statement:} Joseph Louis Lagrange proved that, as long as n is not a perfect square, Pell's equation has infinitely many distinct integer solutions.
\textbf{General Topic:} Number theory
\textbf{URL:} https://en.wikipedia.org/wiki/Pell%27s_equation
\end{lemmatheorembox}

\textbf{Usage count:} 3

\section*{Lemma 222}
\begin{lemmatheorembox}
\textbf{Name:} Algebraic integer
\textbf{Statement:} The square root \(\sqrt{n}\) of a nonnegative integer \(n\) is an algebraic integer, but is irrational unless \(n\) is a perfect square.
\textbf{General Topic:} Algebraic number theory
\textbf{URL:} https://en.wikipedia.org/wiki/Algebraic_integer
\end{lemmatheorembox}

\textbf{Usage count:} 3

\section*{Lemma 223}
\begin{lemmatheorembox}
\textbf{Name:} Cauchy's functional equation
\textbf{Statement:} Theorem: Let \(f\colon V\to W\) be an additive function. Then \(f\) is \(\mathbb {Q}\)-linear.
\textbf{General Topic:} Functional equations
\textbf{URL:} https://en.wikipedia.org/wiki/Cauchy%27s_functional_equation
\end{lemmatheorembox}

\textbf{Usage count:} 3

\section*{Lemma 224}
\begin{lemmatheorembox}
\textbf{Name:} 30°–60°–90° triangle
\textbf{Statement:} A useful property of such triangles is that their side lengths are in the ratio 1 : √ 3 : 2.
\textbf{General Topic:} Geometry
\textbf{URL:} https://en.wikipedia.org/wiki/Special_right_triangle
\end{lemmatheorembox}

\textbf{Usage count:} 3

\section*{Lemma 225}
\begin{lemmatheorembox}
\textbf{Name:} Corresponding angles
\textbf{Statement:} Two lines are parallel if and only if the two angles of any pair of corresponding angles of any transversal are congruent (equal in measure).
\textbf{General Topic:} Euclidean geometry
\textbf{URL:} https://en.wikipedia.org/wiki/Transversal_(geometry)
\end{lemmatheorembox}

\textbf{Usage count:} 3

\section*{Lemma 226}
\begin{lemmatheorembox}
\textbf{Name:} Orbit–stabilizer theorem
\textbf{Statement:} \(|G\cdot x|=[G\,:\,G_{x}]=|G|/|G_{x}|.\)
\textbf{General Topic:} Group Theory
\textbf{URL:} https://en.wikipedia.org/wiki/Group_action
\end{lemmatheorembox}

\textbf{Usage count:} 3

\section*{Lemma 227}
\begin{lemmatheorembox}
\textbf{Name:} Fundamental theorem of arithmetic
\textbf{Statement:} every integer greater than 1 is either prime or can be represented uniquely as a product of prime numbers, up to the order of the factors. ([en.wikipedia.org](https://en.wikipedia.org/wiki/Fundamental_theorem_of_arithmetic?utm_source=openai))
\textbf{General Topic:} Number Theory
\textbf{URL:} https://en.wikipedia.org/wiki/Fundamental_theorem_of_arithmetic ([en.wikipedia.org](https://en.wikipedia.org/wiki/Fundamental_theorem_of_arithmetic?utm_source=openai))
\end{lemmatheorembox}

\textbf{Usage count:} 3

\section*{Lemma 228}
\begin{lemmatheorembox}
\textbf{Name:} Divisor function
\textbf{Statement:} When x=0, \sigma_0(n) is: \sigma_0(n)=\prod_{i=1}^r (a_i+1). ([en.wikipedia.org](https://en.wikipedia.org/wiki/Divisor_function?utm_source=openai))
\textbf{General Topic:} Number Theory
\textbf{URL:} https://en.wikipedia.org/wiki/Divisor_function ([en.wikipedia.org](https://en.wikipedia.org/wiki/Divisor_function?utm_source=openai))
\end{lemmatheorembox}

\textbf{Usage count:} 3

\section*{Lemma 229}
\begin{lemmatheorembox}
\textbf{Name:} Power rule
\textbf{Statement:} The power rule for integration states that 
\int\! x^r \, dx=\frac{x^{r+1}}{r+1}+C
for any real number r \neq -1.
\textbf{General Topic:} Calculus
\textbf{URL:} https://en.wikipedia.org/wiki/Power_rule
\end{lemmatheorembox}

\textbf{Usage count:} 3

\section*{Lemma 230}
\begin{lemmatheorembox}
\textbf{Name:} Newton's identities
\textbf{Statement:} Let \(x_{1}, \ldots, x_{n}\) be variables, denote for \(k \ge 1\) by \(p_{k}(x_{1},\ldots ,x_{n})\) the \(k\)-th power sum:
\[
p_{k}(x_{1},\ldots ,x_{n})=\sum _{i=1}^{n}x_{i}^{k}=x_{1}^{k}+\cdots +x_{n}^{k},
\]
and for \(k \ge 0\) denote by \(e_{k}(x_{1},\ldots ,x_{n})\) the elementary symmetric polynomial (that is, the sum of all distinct products of \(k\) distinct variables), so
\[
\begin{aligned}
e_{0}(x_{1},\ldots ,x_{n})&=1,\\
e_{1}(x_{1},\ldots ,x_{n})&=x_{1}+x_{2}+\cdots +x_{n},\\
e_{2}(x_{1},\ldots ,x_{n})&=\sum _{1\leq i<j\leq n}x_{i}x_{j},\\
&\;\;\vdots \\
e_{n}(x_{1},\ldots ,x_{n})&=x_{1}x_{2}\cdots x_{n},\\
e_{k}(x_{1},\ldots ,x_{n})&=0,\quad {\text{for}}\ k>n.
\end{aligned}
\]
Then Newton's identities can be stated as
\[
ke_{k}(x_{1},\ldots ,x_{n})=\sum _{i=1}^{k}(-1)^{i-1}e_{k-i}(x_{1},\ldots ,x_{n})p_{i}(x_{1},\ldots ,x_{n}),
\]
valid for all \(n \ge k \ge 1\).
Also, one has
\[
p_{k}(x_{1},\ldots ,x_{n})=\sum _{i=k-n}^{k-1}(-1)^{k-1+i}e_{k-i}(x_{1},\ldots ,x_{n})p_{i}(x_{1},\ldots ,x_{n}),
\]
for all \(k > n \ge 1\).
\textbf{General Topic:} Algebra
\textbf{URL:} https://en.wikipedia.org/wiki/Newton%27s_identities
\end{lemmatheorembox}

\textbf{Usage count:} 3

\section*{Lemma 231}
\begin{lemmatheorembox}
\textbf{Name:} Veblen's theorem
\textbf{Statement:} the set of edges of a finite graph can be written as a union of disjoint simple cycles if and only if every vertex has even degree.
\textbf{General Topic:} Graph Theory
\textbf{URL:} https://en.wikipedia.org/wiki/Veblen%27s_theorem
\end{lemmatheorembox}

\textbf{Usage count:} 3

\section*{Lemma 232}
\begin{lemmatheorembox}
\textbf{Name:} Parabola
\textbf{Statement:} A parabola is the set of the points whose distance to a fixed point, the focus, equals the distance to a fixed line, the directrix.
\textbf{General Topic:} Geometry
\textbf{URL:} https://en.wikipedia.org/wiki/Parabola
\end{lemmatheorembox}

\textbf{Usage count:} 3

\section*{Lemma 233}
\begin{lemmatheorembox}
\textbf{Name:} Chain rule (probability)  
\textbf{Statement:} For two events A and B, the chain rule states that  
\[
\mathbb P(A \cap B) = \mathbb P(B \mid A) \mathbb P(A),
\]
where \mathbb P(B \mid A) denotes the conditional probability of B given A.  
\textbf{General Topic:} Probability theory  
\textbf{URL:} https://en.wikipedia.org/wiki/Chain_rule_%28probability%29
\end{lemmatheorembox}

\textbf{Usage count:} 3

\section*{Lemma 234}
\begin{lemmatheorembox}
\textbf{Name:} Distance between two parallel lines  
\textbf{Statement:} Because the lines are parallel, the perpendicular distance between them is a constant, so it does not matter which point is chosen to measure the distance.  
\textbf{General Topic:} Euclidean Geometry  
\textbf{URL:} https://en.wikipedia.org/wiki/Distance_between_two_parallel_lines
\end{lemmatheorembox}

\textbf{Usage count:} 3

\section*{Lemma 235}
\begin{lemmatheorembox}
\textbf{Name:} Theorem one
\textbf{Statement:} For any pair of positive integers n and k, the number of k-tuples of positive integers whose sum is n is equal to the number of (k − 1)-element subsets of a set with n − 1 elements.
\textbf{General Topic:} Combinatorics
\textbf{URL:} https://en.wikipedia.org/wiki/Stars_and_bars_%28combinatorics%29
\end{lemmatheorembox}

\textbf{Usage count:} 3

\section*{Lemma 236}
\begin{lemmatheorembox}
\textbf{Name:} Tangential quadrilateral
\textbf{Statement:} In terms of the tangent lengths, the incircle has radius $\displaystyle r={\sqrt {\frac {efg+fgh+ghe+hef}{e+f+g+h}}}$.
\textbf{General Topic:} Euclidean geometry
\textbf{URL:} https://en.wikipedia.org/wiki/Tangential_quadrilateral
\end{lemmatheorembox}

\textbf{Usage count:} 3

\section*{Lemma 237}
\begin{lemmatheorembox}
\textbf{Name:} Chain rule (probability)
\textbf{Statement:} For two events $A$ and $B$, the chain rule states that $\mathbb P(A \cap B) = \mathbb P(B \mid A) \mathbb P(A)$, where $\mathbb P(B \mid A)$ denotes the conditional probability of $B$ given $A$.
\textbf{General Topic:} Probability theory
\textbf{URL:} https://en.wikipedia.org/wiki/Chain_rule_(probability)
\end{lemmatheorembox}

\textbf{Usage count:} 3

\section*{Lemma 238}
\begin{lemmatheorembox}
\textbf{Name:} Varignon's theorem
\textbf{Statement:} The midpoints of the sides of an arbitrary quadrilateral form a parallelogram. If the quadrilateral is convex or concave (not complex), then the area of the parallelogram is half the area of the quadrilateral.
\textbf{General Topic:} Euclidean geometry
\textbf{URL:} https://en.wikipedia.org/wiki/Varignon%27s_theorem
\end{lemmatheorembox}

\textbf{Usage count:} 3

\section*{Lemma 239}
\begin{lemmatheorembox}
\textbf{Name:} Euler's theorem in geometry
\textbf{Statement:} In geometry, Euler's theorem states that the distance d between the circumcenter and incenter of a triangle is given by $d^{2}=R(R-2r)$ or equivalently ${\frac {1}{R-d}}+{\frac {1}{R+d}}={\frac {1}{r}}$, where $R$ and $r$ denote the circumradius and inradius respectively (the radii of the circumscribed circle and inscribed circle respectively).
\textbf{General Topic:} Euclidean geometry
\textbf{URL:} https://en.wikipedia.org/wiki/Euler%27s_theorem_in_geometry
\end{lemmatheorembox}

\textbf{Usage count:} 3

\section*{Lemma 240}
\begin{lemmatheorembox}
\textbf{Name:} Distance from a point to a line
\textbf{Statement:} The distance (or perpendicular distance) from a point to a line is the shortest distance from a fixed point to any point on a fixed infinite line in Euclidean geometry. It is the length of the line segment that joins the point to the line and is perpendicular to the line.
\textbf{General Topic:} Euclidean geometry
\textbf{URL:} https://en.wikipedia.org/wiki/Distance_from_a_point_to_a_line
\end{lemmatheorembox}

\textbf{Usage count:} 3

\section*{Lemma 241}
\begin{lemmatheorembox}
\textbf{Name:} Lucas's theorem
\textbf{Statement:} For non-negative integers m and n and a prime p, the following congruence relation holds: $\binom{m}{n}\equiv\prod_{i=0}^k\binom{m_i}{n_i}\pmod p$, where $m=m_kp^k+m_{k-1}p^{k-1}+\cdots +m_1p+m_0$, and $n=n_kp^k+n_{k-1}p^{k-1}+\cdots +n_1p+n_0$ are the base p expansions of m and n respectively. This uses the convention that $\tbinom{m}{n} = 0$ if m<n.
\textbf{General Topic:} Number theory
\textbf{URL:} https://en.wikipedia.org/wiki/Lucas%27s_theorem
\end{lemmatheorembox}

\textbf{Usage count:} 3

\section*{Lemma 242}
\begin{lemmatheorembox}
\textbf{Name:} Integral test for convergence
\textbf{Statement:} Then the infinite series $\sum_{n=N}^{\infty} f(n)$ converges to a real number if and only if the improper integral $\int_{N}^{\infty} f(x)\,dx$ is finite.
\textbf{General Topic:} Calculus
\textbf{URL:} https://en.wikipedia.org/wiki/Integral_test_for_convergence
\end{lemmatheorembox}

\textbf{Usage count:} 3

\section*{Lemma 243}
\begin{lemmatheorembox}
\textbf{Name:} Tail-sum formula
\textbf{Statement:} A more general version holds for any non-negative random variable (discrete or continuous): \operatorname {E} [X]=\int _{0}^{\infty }\Pr(X>t)\,dt, where the integrand is the survival function of X.
\textbf{General Topic:} Probability theory
\textbf{URL:} https://en.wikipedia.org/wiki/Expected_value#Tail-sum_formula
\end{lemmatheorembox}

\textbf{Usage count:} 3

\section*{Lemma 244}
\begin{lemmatheorembox}
\textbf{Name:} Quadratic formula
\textbf{Statement:} Given a general quadratic equation of the form $ax^{2}+bx+c=0$, with $x$ representing an unknown, and coefficients $a$, $b$, and $c$ representing known real or complex numbers with $a\neq 0$, the values of $x$ satisfying the equation, called the roots or zeros, can be found using the quadratic formula, $x={\frac {-b\pm {\sqrt {b^{2}-4ac}}}{2a}},$ where the plus--minus symbol ``$\pm$'' indicates that the equation has two roots. Written separately, these are: $x_{1}={\frac {-b+{\sqrt {b^{2}-4ac}}}{2a}},\qquad x_{2}={\frac {-b-{\sqrt {b^{2}-4ac}}}{2a}}.$ The quantity $\Delta =b^{2}-4ac$ is known as the discriminant of the quadratic equation. If the coefficients $a$, $b$, and $c$ are real numbers then when $\Delta >0$, the equation has two distinct real roots; when $\Delta =0$, the equation has one repeated real root; and when $\Delta <0$, the equation has no real roots but has two distinct complex roots, which are complex conjugates of each other.
\textbf{General Topic:} Elementary algebra
\textbf{URL:} https://en.wikipedia.org/wiki/Quadratic_formula ([en.wikipedia.org](https://en.wikipedia.org/wiki/Quadratic_formula))
\end{lemmatheorembox}

\textbf{Usage count:} 3

\section*{Lemma 245}
\begin{lemmatheorembox}
\textbf{Name:} Double-angle formulae
\textbf{Statement:} $\sin 2\theta =2\sin \theta \cos \theta.$
\textbf{General Topic:} Trigonometry
\textbf{URL:} https://en.wikipedia.org/wiki/List_of_trigonometric_identities#Double-angle_formulae
\end{lemmatheorembox}

\textbf{Usage count:} 3

\section*{Lemma 246}
\begin{lemmatheorembox}
\textbf{Name:} Rank–nullity theorem
\textbf{Statement:} In other words, \dim(\operatorname{Im} T)+\dim(\operatorname{Ker} T)=\dim(\operatorname{Domain} (T)).
\textbf{General Topic:} Linear algebra
\textbf{URL:} https://en.wikipedia.org/wiki/Rank%E2%80%93nullity_theorem
\end{lemmatheorembox}

\textbf{Usage count:} 3

\section*{Lemma 247}
\begin{lemmatheorembox}
\textbf{Name:} Derangement
\textbf{Statement:} This gives us the solution to the hat-check problem: Stated algebraically, the number !n of derangements of an n-element set is !n = \left( n - 1 \right) \bigl({!\left( n - 1 \right)} + {!\left( n - 2 \right)}\bigr) for n \geq 2, where !0 = 1 and !1 = 0.
\textbf{General Topic:} Combinatorics
\textbf{URL:} https://en.wikipedia.org/wiki/Derangement
\end{lemmatheorembox}

\textbf{Usage count:} 3

\section*{Lemma 248}
\begin{lemmatheorembox}
\textbf{Name:} Natural logarithm
\textbf{Statement:} Likewise, when the integral is over an interval where $f(x)\ne 0$, $\int { \frac{f'(x)}{f(x)}\,dx} = \ln|f(x)| + C$.
\textbf{General Topic:} Calculus
\textbf{URL:} https://en.wikipedia.org/wiki/Natural_logarithm
\end{lemmatheorembox}

\textbf{Usage count:} 3

\section*{Lemma 249}
\begin{lemmatheorembox}
\textbf{Name:} Square
\textbf{Statement:} A square's area is $A=\ell ^{2}={\tfrac {1}{2}}d^{2}$.
\textbf{General Topic:} Geometry
\textbf{URL:} https://en.wikipedia.org/wiki/Square
\end{lemmatheorembox}

\textbf{Usage count:} 3

\section*{Lemma 250}
\begin{lemmatheorembox}
\textbf{Name:} Law of trichotomy
\textbf{Statement:} In mathematics, the law of trichotomy states that every real number is either positive, negative, or zero.
\textbf{General Topic:} Real Analysis
\textbf{URL:} https://en.wikipedia.org/wiki/Law_of_trichotomy
\end{lemmatheorembox}

\textbf{Usage count:} 3

\section*{Lemma 251}
\begin{lemmatheorembox}
\textbf{Name:} Carmichael function
\textbf{Statement:} In algebraic terms, $\lambda(n)$ is the exponent of the multiplicative group of integers modulo $n$.
\textbf{General Topic:} Number theory
\textbf{URL:} https://en.wikipedia.org/wiki/Carmichael_function
\end{lemmatheorembox}

\textbf{Usage count:} 3

\section*{Lemma 252}
\begin{lemmatheorembox}
\textbf{Name:} Polynomial remainder theorem
\textbf{Statement:} $f(r)=R.$
\textbf{General Topic:} Algebra
\textbf{URL:} https://en.wikipedia.org/wiki/Polynomial_remainder_theorem
\end{lemmatheorembox}

\textbf{Usage count:} 3

\section*{Lemma 253}
\begin{lemmatheorembox}
\textbf{Name:} Mercator series
\textbf{Statement:} Setting x=1 in the Mercator series yields the alternating harmonic series \sum _{k=1}^{\infty }{\frac {(-1)^{k+1}}{k}}=\ln(2).
\textbf{General Topic:} Mathematical analysis
\textbf{URL:} https://en.wikipedia.org/wiki/Mercator_series
\end{lemmatheorembox}

\textbf{Usage count:} 3

\section*{Lemma 254}
\begin{lemmatheorembox}
\textbf{Name:} Double counting (proof technique)
\textbf{Statement:} Since both expressions equal the size of the same set, they equal each other.
\textbf{General Topic:} Combinatorics
\textbf{URL:} https://en.wikipedia.org/wiki/Double_counting_%28proof_technique%29
\end{lemmatheorembox}

\textbf{Usage count:} 3

\section*{Lemma 255}
\begin{lemmatheorembox}
\textbf{Name:} Square number
\textbf{Statement:} Thus, the number m is a square number if and only if, in its canonical representation, all exponents are even.
\textbf{General Topic:} Number Theory
\textbf{URL:} https://en.wikipedia.org/wiki/Square_number
\end{lemmatheorembox}

\textbf{Usage count:} 3

\section*{Lemma 256}
\begin{lemmatheorembox}
\textbf{Name:} Tree (graph theory)
\textbf{Statement:} A tree is an undirected graph G that satisfies any of the following equivalent conditions:
* G is connected and acyclic (contains no cycles).
If G has finitely many vertices, say n of them, then the above statements are also equivalent to any of the following conditions:
* G is connected and has n − 1 edges.
\textbf{General Topic:} Graph theory
\textbf{URL:} https://en.wikipedia.org/wiki/Tree_(graph_theory)
\end{lemmatheorembox}

\textbf{Usage count:} 3

\section*{Lemma 257}
\begin{lemmatheorembox}
\textbf{Name:} Degree sum formula
\textbf{Statement:} The degree sum formula states that, given a graph $G=(V,E)$,
\[
\sum_{v\in V}\deg(v)=2|E|\, .
\]
\textbf{General Topic:} Graph theory
\textbf{URL:} https://en.wikipedia.org/wiki/Degree_(graph_theory)#Handshaking_lemma
\end{lemmatheorembox}

\textbf{Usage count:} 3

\section*{Lemma 258}
\begin{lemmatheorembox}
\textbf{Name:} Alternating series estimation theorem
\textbf{Statement:} Moreover, let L denote the sum of the series, then the partial sum {\textstyle S_{k}=\sum _{n=0}^{k}(-1)^{n}a_{n}\!} approximates L with error bounded by the next omitted term: \[
\left|S_{k}-L\right\vert \leq \left|S_{k}-S_{k+1}\right\vert =a_{k+1}.\!
\]
\textbf{General Topic:} Mathematical analysis
\textbf{URL:} https://en.wikipedia.org/wiki/Alternating_series_test
\end{lemmatheorembox}

\textbf{Usage count:} 3

\section*{Lemma 259}
\begin{lemmatheorembox}
\textbf{Name:} Parity of a permutation
\textbf{Statement:} $n{\text{ minus the number of disjoint cycles in the decomposition of }}\sigma$
\textbf{General Topic:} Group theory
\textbf{URL:} https://en.wikipedia.org/wiki/Parity_of_a_permutation
\end{lemmatheorembox}

\textbf{Usage count:} 3

\section*{Lemma 260}
\begin{lemmatheorembox}
\textbf{Name:} Hamiltonian path
\textbf{Statement:} The number of different Hamiltonian cycles in a complete undirected graph on n vertices is ⁠(n − 1)!/2⁠ and in a complete directed graph on n vertices is (n − 1)!. These counts assume that cycles that are the same apart from their starting point are not counted separately.
\textbf{General Topic:} Graph Theory
\textbf{URL:} https://en.wikipedia.org/wiki/Hamiltonian_path
\end{lemmatheorembox}

\textbf{Usage count:} 3

\section*{Lemma 261}
\begin{lemmatheorembox}
\textbf{Name:} Composition (combinatorics)
\textbf{Statement:} For weak compositions, the number is ${n+k-1 \choose k-1}={n+k-1 \choose n}$, since each k-composition of $n + k$ corresponds to a weak one of $n$ by the rule
\[
a_{1}+a_{2}+\ldots +a_{k}=n+k\quad \mapsto \quad (a_{1}-1)+(a_{2}-1)+\ldots +(a_{k}-1)=n
\]
\textbf{General Topic:} Combinatorics
\textbf{URL:} https://en.wikipedia.org/wiki/Composition_(combinatorics)
\end{lemmatheorembox}

\textbf{Usage count:} 3

\section*{Lemma 262}
\begin{lemmatheorembox}
\textbf{Name:} Dirichlet convolution
\textbf{Statement:} The Dirichlet convolution of two multiplicative functions is again multiplicative, and every not constantly zero multiplicative function has a Dirichlet inverse which is also multiplicative.
\textbf{General Topic:} Number theory
\textbf{URL:} https://en.wikipedia.org/wiki/Dirichlet_convolution
\end{lemmatheorembox}

\textbf{Usage count:} 3

\section*{Lemma 263}
\begin{lemmatheorembox}
\textbf{Name:} Product rule
\textbf{Statement:} In calculus, the product rule (or Leibniz rule or Leibniz product rule) is a formula used to find the derivatives of products of two or more functions. For two functions, it may be stated in Lagrange's notation as $(u\cdot v)'=u'\cdot v+u\cdot v'$ or in Leibniz's notation as $\frac{d}{dx}(u\cdot v)=\frac{du}{dx}\cdot v+u\cdot \frac{dv}{dx}$.
\textbf{General Topic:} Calculus
\textbf{URL:} https://en.wikipedia.org/wiki/Product_rule
\end{lemmatheorembox}

\textbf{Usage count:} 3

\section*{Lemma 264}
\begin{lemmatheorembox}
\textbf{Name:} Isometry
\textbf{Statement:} In mathematics, an isometry (or congruence, or congruent transformation) is a distance-preserving transformation between metric spaces, usually assumed to be bijective.
\textbf{General Topic:} Metric geometry
\textbf{URL:} https://en.wikipedia.org/wiki/Isometry
\end{lemmatheorembox}

\textbf{Usage count:} 3

\section*{Lemma 265}
\begin{lemmatheorembox}
\textbf{Name:} Quadratic residue
\textbf{Statement:} All odd squares are ≡ 1 (mod 8) and thus also ≡ 1 (mod 4).
\textbf{General Topic:} Number theory
\textbf{URL:} https://en.wikipedia.org/wiki/Quadratic_residue
\end{lemmatheorembox}

\textbf{Usage count:} 3

\section*{Lemma 266}
\begin{lemmatheorembox}
\textbf{Name:} Rolle's theorem
\textbf{Statement:} If a real-valued function f is continuous on a proper closed interval [a, b], differentiable on the open interval (a, b), and f (a) = f (b), then there exists at least one c in the open interval (a, b) such that f'(c) = 0.
\textbf{General Topic:} Calculus
\textbf{URL:} https://en.wikipedia.org/wiki/Rolle%27s_theorem
\end{lemmatheorembox}

\textbf{Usage count:} 3

\section*{Lemma 267}
\begin{lemmatheorembox}
\textbf{Name:} List of logarithmic identities
\textbf{Statement:} $\log _{b}(xy)=\log _{b}(x)+\log _{b}(y)$ and $\log _{b}(x^{d})=d\log _{b}(x)$
\textbf{General Topic:} Logarithms
\textbf{URL:} https://en.wikipedia.org/wiki/List_of_logarithmic_identities
\end{lemmatheorembox}

\textbf{Usage count:} 3

\section*{Lemma 268}
\begin{lemmatheorembox}
\textbf{Name:} Division theorem
\textbf{Statement:} Given two integers $a$ and $b$, with $b\neq 0$, there exist unique integers $q$ and $r$ such that $a=bq+r$, and $0\leq r<|b|$, where $|b|$ denotes the absolute value of $b$.
\textbf{General Topic:} Number theory
\textbf{URL:} https://en.wikipedia.org/wiki/Euclidean_division#Division_theorem
\end{lemmatheorembox}

\textbf{Usage count:} 3

\section*{Lemma 269}
\begin{lemmatheorembox}
\textbf{Name:} Exponent rules
\textbf{Statement:} The following identities, often called exponent rules, hold for all integer exponents, provided that the base is non-zero:
\begin{align} b^m \cdot b^n &= b^{m + n} \\ \left(b^m\right)^n &= b^{m \cdot n} \\ b^n \cdot c^n &= (b \cdot c)^n \end{align}
\textbf{General Topic:} Arithmetic
\textbf{URL:} https://en.wikipedia.org/wiki/Exponentiation#Identities_and_properties
\end{lemmatheorembox}

\textbf{Usage count:} 3

\section*{Lemma 270}
\begin{lemmatheorembox}
\textbf{Name:} Classical definition
\textbf{Statement:} If a random experiment can result in N mutually exclusive and equally likely outcomes and if $N_{A}$ of these outcomes result in the occurrence of the event A, the probability of A is defined by $P(A)={N_{A} \over N}$.
\textbf{General Topic:} Probability theory
\textbf{URL:} https://en.wikipedia.org/wiki/Probability_interpretations
\end{lemmatheorembox}

\textbf{Usage count:} 3

\section*{Lemma 271}
\begin{lemmatheorembox}
\textbf{Name:} Rhombus
\textbf{Statement:} a quadrilateral in which the diagonals are perpendicular and bisect each other
\textbf{General Topic:} Euclidean geometry
\textbf{URL:} https://en.wikipedia.org/wiki/Rhombus
\end{lemmatheorembox}

\textbf{Usage count:} 3

\section*{Lemma 272}
\begin{lemmatheorembox}
\textbf{Name:} Bijection
\textbf{Statement:} It results that two finite sets have the same number of elements if and only if there exists a bijection between them.
\textbf{General Topic:} Set Theory
\textbf{URL:} https://en.wikipedia.org/wiki/Bijection
\end{lemmatheorembox}

\textbf{Usage count:} 3

\section*{Lemma 273}
\begin{lemmatheorembox}
\textbf{Name:} Division
\textbf{Statement:} Given elements a and b of a group G, there is a unique solution x in G to the equation a\cdot x=b, namely a^{-1}\cdot b.
\textbf{General Topic:} Group Theory
\textbf{URL:} https://en.wikipedia.org/wiki/Group_(mathematics)
\end{lemmatheorembox}

\textbf{Usage count:} 3

\section*{Lemma 274}
\begin{lemmatheorembox}
\textbf{Name:} L'Hôpital's rule
\textbf{Statement:} L'Hôpital's rule states that for functions $f$ and $g$ which are defined on an open interval $I$ and differentiable on $I\setminus \{c\}$ for a (possibly infinite) accumulation point $c$ of $I$, if $\lim \limits _{x\to c}f(x)=\lim \limits _{x\to c}g(x)=0$ or $\pm \infty$, and $g'(x)\neq 0$ for all $x$ in $I\setminus \{c\}$, and $\lim \limits _{x\to c}{\frac {f'(x)}{g'(x)}}$ exists, then $\lim _{x\to c}{\frac {f(x)}{g(x)}}=\lim _{x\to c}{\frac {f'(x)}{g'(x)}}.$
\textbf{General Topic:} Calculus
\textbf{URL:} https://en.wikipedia.org/wiki/L%27H%C3%B4pital%27s_rule
\end{lemmatheorembox}

\textbf{Usage count:} 3

\section*{Lemma 275}
\begin{lemmatheorembox}
\textbf{Name:} Cancelling out
\textbf{Statement:} As another example, if a×b=a×c, then the multiplicative term a can be canceled out if a≠0, resulting in the equivalent expression b=c; this is equivalent to dividing through by a.
\textbf{General Topic:} Elementary algebra
\textbf{URL:} https://en.wikipedia.org/wiki/Cancelling_out
\end{lemmatheorembox}

\textbf{Usage count:} 3

\section*{Lemma 276}
\begin{lemmatheorembox}
\textbf{Name:} Cross-multiplication
\textbf{Statement:} Given an equation like \(\frac a b = \frac c d\), where \(b\) and \(d\) are not zero, one can cross-multiply to get \(ad = bc \quad \text{or} \quad a = \frac{bc}d\).
\textbf{General Topic:} Elementary algebra
\textbf{URL:} https://en.wikipedia.org/wiki/Cross-multiplication
\end{lemmatheorembox}

\textbf{Usage count:} 3

\section*{Lemma 277}
\begin{lemmatheorembox}
\textbf{Name:} Slope
\textbf{Statement:} If the slope $m$ of a line and a point $(x_{1},y_{1})$ on the line are both known, then the equation of the line can be found using the point-slope formula: $y-y_{1}=m(x-x_{1}).$
\textbf{General Topic:} Analytic geometry
\textbf{URL:} https://en.wikipedia.org/wiki/Slope
\end{lemmatheorembox}

\textbf{Usage count:} 3

\section*{Lemma 278}
\begin{lemmatheorembox}
\textbf{Name:} Reductio ad absurdum
\textbf{Statement:} A mathematical proof employing proof by contradiction usually proceeds as follows:\par
1. The proposition to be proved is P.\par
2. We assume P to be false, i.e., we assume ¬P.\par
3. It is then shown that ¬P implies falsehood. This is typically accomplished by deriving two mutually contradictory assertions, Q and ¬Q, and appealing to the law of noncontradiction.\par
4. Since assuming P to be false leads to a contradiction, it is concluded that P is in fact true.
\textbf{General Topic:} Logic
\textbf{URL:} https://en.wikipedia.org/wiki/Proof_by_contradiction
\end{lemmatheorembox}

\textbf{Usage count:} 3

\section*{Lemma 279}
\begin{lemmatheorembox}
\textbf{Name:} Product of two integers
\textbf{Statement:} In words:
\begin{itemize}
\item A positive number multiplied by a positive number is positive (product of natural numbers),
\item A positive number multiplied by a negative number is negative,
\item A negative number multiplied by a positive number is negative,
\item A negative number multiplied by a negative number is positive.
\end{itemize}
\textbf{General Topic:} Arithmetic
\textbf{URL:} https://en.wikipedia.org/wiki/Multiplication
\end{lemmatheorembox}

\textbf{Usage count:} 3

\section*{Lemma 280}
\begin{lemmatheorembox}
\textbf{Name:} Rank (linear algebra)
\textbf{Statement:} The rank of A is the largest order of any non-zero minor in A.
\textbf{General Topic:} Linear algebra
\textbf{URL:} https://en.wikipedia.org/wiki/Rank_(linear_algebra)
\end{lemmatheorembox}

\textbf{Usage count:} 3

\section*{Lemma 281}
\begin{lemmatheorembox}
\textbf{Name:} Comparing fractions
\textbf{Statement:} One way to compare fractions with different numerators and denominators is to find a common denominator. To compare a/b and c/d, these are converted to a·d/b·d and b·c/b·d (where the dot signifies multiplication and is an alternative symbol to ×). Then bd is a common denominator and the numerators ad and bc can be compared.
\textbf{General Topic:} Arithmetic
\textbf{URL:} https://en.wikipedia.org/wiki/Fraction
\end{lemmatheorembox}

\textbf{Usage count:} 3

\section*{Lemma 282}
\begin{lemmatheorembox}
\textbf{Name:} Orthocentric system
\textbf{Statement:} The joining of these three orthogonal points into a triangle generates an orthic triangle that is common to all the four possible triangles formed from the four orthocentric points taken three at a time. The incenter of this common orthic triangle must be one of the original four orthocentric points. Furthermore, the three remaining points become the excenters of this common orthic triangle. The orthocentric point that becomes the incenter of the orthic triangle is that orthocentric point closest to the common nine-point center. This relationship between the orthic triangle and the original four orthocentric points leads directly to the fact that the incenter and excenters of a reference triangle form an orthocentric system.[ 3 ] It is normal to distinguish one of the orthocentric points from the others, specifically the one that is the incenter of the orthic triangle; this one is denoted H as the orthocenter of the outer three orthocentric points that are chosen as a reference triangle △ABC.
\textbf{General Topic:} Euclidean geometry
\textbf{URL:} https://en.wikipedia.org/wiki/Orthocentric_system
\end{lemmatheorembox}

\textbf{Usage count:} 2

\section*{Lemma 283}
\begin{lemmatheorembox}
\textbf{Name:} Stirling's approximation
\textbf{Statement:} n! \sim \sqrt{2 \pi n}\left(\frac{n}{e}\right)^n.
\textbf{General Topic:} Asymptotic analysis
\textbf{URL:} https://en.wikipedia.org/wiki/Stirling%27s_approximation
\end{lemmatheorembox}

\textbf{Usage count:} 2

\section*{Lemma 284}
\begin{lemmatheorembox}
\textbf{Name:} Pythagorean theorem

\textbf{Statement:} \(a^{2} + b^{2} = c^{2}\).

\textbf{General Topic:} Euclidean geometry

\textbf{URL:} https://en.wikipedia.org/wiki/Pythagorean_theorem. ([en.wikipedia.org](https://en.wikipedia.org/wiki/Pythagorean_theorem))
\end{lemmatheorembox}

\textbf{Usage count:} 2

\section*{Lemma 285}
\begin{lemmatheorembox}
\textbf{Name:} Hall's marriage theorem
\textbf{Statement:} The marriage theorem in this formulation states that there is an X-perfect matching if and only if for every subset W of X: |W|≤|N\_G(W)|.
\textbf{General Topic:} Graph theory
\textbf{URL:} https://en.wikipedia.org/wiki/Hall%27s_marriage_theorem
\end{lemmatheorembox}

\textbf{Usage count:} 2

\section*{Lemma 286}
\begin{lemmatheorembox}
\textbf{Name:} Pascal's theorem
\textbf{Statement:} Theorem on the collinearity of three points generated from a hexagon inscribed on a conic
\textbf{General Topic:} Projective geometry
\textbf{URL:} https://en.wikipedia.org/wiki/Pascal%27s_theorem
\end{lemmatheorembox}

\textbf{Usage count:} 2

\section*{Lemma 287}
\begin{lemmatheorembox}
\textbf{Name:} Concave function
\textbf{Statement:} If f is twice-differentiable, then f is concave if and only if f ′′ is non-positive (or, informally, if the "acceleration" is non-positive). If f ′′ is negative then f is strictly concave, but the converse is not true, as shown by f(x) = −x^{4}.
\textbf{General Topic:} Mathematical analysis
\textbf{URL:} https://en.wikipedia.org/wiki/Concave_function
\end{lemmatheorembox}

\textbf{Usage count:} 2

\section*{Lemma 288}
\begin{lemmatheorembox}
\textbf{Name:} Quadratic Gauss sum
\textbf{Statement:} The evaluation of the Gauss sum for an integer a not divisible by a prime p > 2 can be reduced to the case a = 1: g(a;p)=\textbackslash left(\textbackslash tfrac\{a\}\{p\}\textbackslash right)g(1;p).
\textbf{General Topic:} Number theory
\textbf{URL:} https://en.wikipedia.org/wiki/Quadratic_Gauss_sum
\end{lemmatheorembox}

\textbf{Usage count:} 2

\section*{Lemma 289}
\begin{lemmatheorembox}
\textbf{Name:} Euler's criterion
\textbf{Statement:} Let p be an odd prime and a be an integer coprime to p. Then a\textasciicircum\textbackslash tfrac\{p-1\}\{2\} \textbackslash equiv \textbackslash begin\{cases\} \textbackslash ;\textbackslash ;\textbackslash ,1\textbackslash pmod\{p\}\& \textbackslash text\{ if there is an integer \}x \textbackslash text\{ such that \}x\textasciicircum 2\textbackslash equiv a \textbackslash pmod\{p\},\textbackslash\textbackslash  -1\textbackslash pmod\{p\}\& \textbackslash text\{ if there is no such integer.\}\textbackslash end\{cases\}
\textbf{General Topic:} Number theory
\textbf{URL:} https://en.wikipedia.org/wiki/Euler%27s_criterion
\end{lemmatheorembox}

\textbf{Usage count:} 2

\section*{Lemma 290}
\begin{lemmatheorembox}
\textbf{Name:} Integer-valued polynomial
\textbf{Statement:} Inside the polynomial ring $\mathbb{Q}[t]$ of polynomials with rational number coefficients, the subring of integer-valued polynomials is a free abelian group. It has as basis the polynomials $P_k(t)=t(t-1)\cdots (t-k+1)/k!$ for $k=0,1,2,\dots$, i.e., the binomial coefficients. In other words, every integer-valued polynomial can be written as an integer linear combination of binomial coefficients in exactly one way.
\textbf{General Topic:} Number Theory
\textbf{URL:} https://en.wikipedia.org/wiki/Integer-valued_polynomial
\end{lemmatheorembox}

\textbf{Usage count:} 2

\section*{Lemma 291}
\begin{lemmatheorembox}
\textbf{Name:} Chebyshev polynomials
\textbf{Statement:} \(x_{k}=\cos \left({\frac {2k+1}{2n}}\pi \right),\quad k=0,\ldots ,n-1.\)
\textbf{General Topic:} Orthogonal polynomials
\textbf{URL:} https://en.wikipedia.org/wiki/Chebyshev_polynomials
\end{lemmatheorembox}

\textbf{Usage count:} 2

\section*{Lemma 292}
\begin{lemmatheorembox}
\textbf{Name:} Prosthaphaeresis
\textbf{Statement:} \(\cos a - \cos b = -2 \sin \left(\frac{a + b}{2} \right) \sin \left(\frac{a - b}{2} \right)\)
\textbf{General Topic:} Trigonometry
\textbf{URL:} https://en.wikipedia.org/wiki/Prosthaphaeresis
\end{lemmatheorembox}

\textbf{Usage count:} 2

\section*{Lemma 293}
\begin{lemmatheorembox}
\textbf{Name:} Euler's product formula
\textbf{Statement:} It states \(\varphi(n)=n\prod _{p\mid n}\left(1-{\frac {1}{p}}\right),\) where the product is over the distinct prime numbers dividing \(n\).
\textbf{General Topic:} Number theory
\textbf{URL:} https://en.wikipedia.org/wiki/Euler%27s_totient_function#Euler's_product_formula
\end{lemmatheorembox}

\textbf{Usage count:} 2

\section*{Lemma 294}
\begin{lemmatheorembox}
\textbf{Name:} Limit of a sequence
\textbf{Statement:} Limits of sequences behave well with respect to the usual arithmetic operations. If \(\lim _{n\to \infty }a_{n}\) and \(\lim _{n\to \infty }b_{n}\) exists, then
\[
\lim _{n\to \infty }(a_{n}\pm b_{n})=\lim _{n\to \infty }a_{n}\pm \lim _{n\to \infty }b_{n}
\]
\[
\lim _{n\to \infty }ca_{n}=c\cdot \lim _{n\to \infty }a_{n}
\]
\[
\lim _{n\to \infty }(a_{n}\cdot b_{n})=\left(\lim _{n\to \infty }a_{n}\right)\cdot \left(\lim _{n\to \infty }b_{n}\right)
\]
\[
\lim _{n\to \infty }\left({\frac {a_{n}}{b_{n}}}\right)={\frac {\lim \limits _{n\to \infty }a_{n}}{\lim \limits _{n\to \infty }b_{n}}}
\]
provided \(\lim _{n\to \infty }b_{n}\neq 0\)
\[
\lim _{n\to \infty }a_{n}^{p}=\left(\lim _{n\to \infty }a_{n}\right)^{p}
\]
\textbf{General Topic:} Mathematical analysis
\textbf{URL:} https://en.wikipedia.org/wiki/Limit_of_a_sequence
\end{lemmatheorembox}

\textbf{Usage count:} 2

\section*{Lemma 295}
\begin{lemmatheorembox}
\textbf{Name:} Eisenstein primes
\textbf{Statement:} an ordinary prime number (or rational prime) which is congruent to 2 mod 3 is also an Eisenstein prime.
\textbf{General Topic:} Algebraic number theory
\textbf{URL:} https://en.wikipedia.org/wiki/Eisenstein_integer
\end{lemmatheorembox}

\textbf{Usage count:} 2

\section*{Lemma 296}
\begin{lemmatheorembox}
\textbf{Name:} Lagrange multiplier theorem
\textbf{Statement:} Let \(f\colon\mathbb{R}^n \to \mathbb{R}\) be the objective function and let \(g\colon\mathbb{R}^n \to \mathbb{R}^c\) be the constraints function, both belonging to \(C^1\) (that is, having continuous first derivatives). Let \(x_\star\) be an optimal solution to the following optimization problem such that, for the matrix of partial derivatives \(\Bigl[ \operatorname{D}g(x_\star) \Bigr]_{j,k} = \frac{\ \partial g_j\ }{\partial x_k}\), \(\operatorname{rank} (\operatorname{D}g(x_\star)) = c \le n\) :
\begin{align} & \text{maximize } f(x) \\ & \text{subject to: } g(x) = 0 \end{align}
Then there exists a unique Lagrange multiplier \(\lambda_\star \in \mathbb{R}^c\) such that \(\operatorname{D}f(x_\star) = \lambda_\star^{\mathsf{T}}\operatorname{D}g(x_\star) ~.\)
\textbf{General Topic:} Mathematical optimization
\textbf{URL:} https://en.wikipedia.org/wiki/Lagrange_multiplier
\end{lemmatheorembox}

\textbf{Usage count:} 2

\section*{Lemma 297}
\begin{lemmatheorembox}
\textbf{Name:} Eisenstein's criterion
\textbf{Statement:} Suppose \(Q(x)=a_{n}x^{n}+a_{n-1}x^{n-1}+\cdots +a_{1}x+a_{0}\) has integer coefficients. If there exists a prime number \(p\) such that \(p\) divides each \(a_i\) for \(0 \le i < n\), \(p\) does not divide \(a_n\), and \(p^{2}\) does not divide \(a_{0}\), then \(Q\) is irreducible over the rational numbers.
\textbf{General Topic:} Algebra
\textbf{URL:} https://en.wikipedia.org/wiki/Eisenstein%27s_criterion
\end{lemmatheorembox}

\textbf{Usage count:} 2

\section*{Lemma 298}
\begin{lemmatheorembox}
\textbf{Name:} Isogonal conjugate
\textbf{Statement:} \detokenize{In geometry, the isogonal conjugate of a point P with respect to a triangle △ABC is constructed by reflecting the lines PA, PB, PC about the angle bisectors of A, B, C respectively. These three reflected lines concur at the isogonal conjugate of P. (This definition applies only to points not on a sideline of triangle △ABC.) This is a direct result of the trigonometric form of Ceva's theorem.}
\textbf{General Topic:} Triangle geometry
\textbf{URL:} https://en.wikipedia.org/wiki/Isogonal_conjugate
\end{lemmatheorembox}

\textbf{Usage count:} 2

\section*{Lemma 299}
\begin{lemmatheorembox}
\textbf{Name:} Leibniz integral rule
\textbf{Statement:} \frac{d}{dx} \left(\int_a^b f(x,t)\,dt \right)= \int_a^b \frac{\partial}{\partial x} f(x,t) \,dt.
\textbf{General Topic:} Calculus
\textbf{URL:} https://en.wikipedia.org/wiki/Leibniz_integral_rule
\end{lemmatheorembox}

\textbf{Usage count:} 2

\section*{Lemma 300}
\begin{lemmatheorembox}
\textbf{Name:} Supporting hyperplane theorem

\textbf{Statement:} This theorem states that if \(S\) is a convex set in the topological vector space \(X=\mathbb{R}^n\), and \(x_0\) is a point on the boundary of \(S\), then there exists a supporting hyperplane containing \(x_0\). If \(x^* \in X^* \backslash \{0\}\) (\(X^*\) is the dual space of \(X\), \(x^*\) is a nonzero linear functional) such that \(x^*\left(x_{0}\right)\geq x^*(x)\) for all \(x\in S\), then \(H=\{x\in X:x^*(x)=x^*\left(x_{0}\right)\}\) defines a supporting hyperplane.

\textbf{General Topic:} Convex geometry

\textbf{URL:} https://en.wikipedia.org/wiki/Supporting_hyperplane
\end{lemmatheorembox}

\textbf{Usage count:} 2

\section*{Lemma 301}
\begin{lemmatheorembox}
\textbf{Name:} Divisor summatory function
\textbf{Statement:} The divisor summatory function is defined as \displaystyle D(x)=\sum _{n\leq x}d(n)=\sum _{j,k \atop jk\leq x}1
\textbf{General Topic:} Number theory
\textbf{URL:} https://en.wikipedia.org/wiki/Divisor_summatory_function
\end{lemmatheorembox}

\textbf{Usage count:} 2

\section*{Lemma 302}
\begin{lemmatheorembox}
\textbf{Name:} Order (group theory)

\textbf{Statement:} For any integer k, we have a\textasciicircum k = e if and only if ord(a) divides k.

\textbf{General Topic:} Group theory

\textbf{URL:} https://en.wikipedia.org/wiki/Order_(group_theory)
\end{lemmatheorembox}

\textbf{Usage count:} 2

\section*{Lemma 303}
\begin{lemmatheorembox}
\textbf{Name:} First-derivative test
\textbf{Statement:} The first-derivative test examines a function's monotonic properties (where the function is increasing or decreasing), focusing on a particular point in its domain. If the function "switches" from increasing to decreasing at the point, then the function will achieve a highest value at that point. Similarly, if the function "switches" from decreasing to increasing at the point, then it will achieve a least value at that point. If the function fails to "switch" and remains increasing or remains decreasing, then no highest or least value is achieved.
\textbf{General Topic:} Calculus
\textbf{URL:} https://en.wikipedia.org/wiki/Derivative_test
\end{lemmatheorembox}

\textbf{Usage count:} 2

\section*{Lemma 304}
\begin{lemmatheorembox}
\textbf{Name:} Binomial series
\textbf{Statement:} In mathematics, the binomial series is a generalization of the binomial formula to cases where the exponent is not a positive integer:
\[
\begin{aligned}
(1+x)^{\alpha }&=\sum _{k=0}^{\infty }\!{\binom {\alpha }{k}}x^{k}\\
&=1+\alpha x+{\frac {\alpha (\alpha -1)}{2!}}x^{2}+{\frac {\alpha (\alpha -1)(\alpha -2)}{3!}}x^{3}+\cdots
\end{aligned}
\]
where \(\alpha\) is any complex number, and the power series on the right-hand side is expressed in terms of the (generalized) binomial coefficients
\[
{\binom {\alpha }{k}}={\frac {\alpha (\alpha -1)(\alpha -2)\cdots (\alpha -k+1)}{k!}}.
\]
The binomial series is the MacLaurin series for the function \(f(x)=(1+x)^{\alpha }\). It converges when \(|x|<1\).
\textbf{General Topic:} Mathematical analysis
\textbf{URL:} https://en.wikipedia.org/wiki/Binomial_series
\end{lemmatheorembox}

\textbf{Usage count:} 2

\section*{Lemma 305}
\begin{lemmatheorembox}
\textbf{Name:} SAS (side-angle-side)

\textbf{Statement:} If two pairs of sides of two triangles are equal in length, and the included angles are equal in measurement, then the triangles are congruent.

\textbf{General Topic:} Euclidean geometry

\textbf{URL:} https://en.wikipedia.org/wiki/Congruence_(geometry)#Determining_congruence
\end{lemmatheorembox}

\textbf{Usage count:} 2

\section*{Lemma 306}
\begin{lemmatheorembox}
\textbf{Name:} Forest (graph theory)
\textbf{Statement:} A forest is an undirected acyclic graph or equivalently a disjoint union of trees.
\textbf{General Topic:} Graph theory
\textbf{URL:} https://en.wikipedia.org/wiki/Forest_%28graph_theory%29
\end{lemmatheorembox}

\textbf{Usage count:} 2

\section*{Lemma 307}
\begin{lemmatheorembox}
\textbf{Name:} Leibniz formula for determinants
\textbf{Statement:} $\det(A)=\sum _{\tau \in S_{n}}\operatorname {sgn} (\tau )\prod _{i=1}^{n}a_{i\tau (i)}=\sum _{\sigma \in S_{n}}\operatorname {sgn} (\sigma )\prod _{i=1}^{n}a_{\sigma (i)i}$
\textbf{General Topic:} Linear algebra
\textbf{URL:} https://en.wikipedia.org/wiki/Leibniz_formula_for_determinants
\end{lemmatheorembox}

\textbf{Usage count:} 2

\section*{Lemma 308}
\begin{lemmatheorembox}
\textbf{Name:} Euclid's formula
\textbf{Statement:} Euclid's formula is a fundamental formula for generating Pythagorean triples given an arbitrary pair of integers m and n with m > n > 0. The formula states that the integers a = m^2 - n^2 ,\ \, b = 2mn ,\ \, c = m^2 + n^2 form a Pythagorean triple.
\textbf{General Topic:} Number theory
\textbf{URL:} https://en.wikipedia.org/wiki/Pythagorean_triple
\end{lemmatheorembox}

\textbf{Usage count:} 2

\section*{Lemma 309}
\begin{lemmatheorembox}
\textbf{Name:} Cone
\textbf{Statement:} For a circular cone with radius $r$ and height $h$, the base is a circle of area $\pi r^{2}$ thus the formula for volume is: $V={\frac {1}{3}}\pi r^{2}h$.
\textbf{General Topic:} Geometry
\textbf{URL:} https://en.wikipedia.org/wiki/Cone
\end{lemmatheorembox}

\textbf{Usage count:} 2

\section*{Lemma 310}
\begin{lemmatheorembox}
\textbf{Name:} Ellipse
\textbf{Statement:} The area $A_{\text{ellipse}}$ enclosed by an ellipse is: $A_{\text{ellipse}}=\pi ab$ where $a$ and $b$ are the lengths of the semi-major and semi-minor axes, respectively.
\textbf{General Topic:} Geometry
\textbf{URL:} https://en.wikipedia.org/wiki/Ellipse
\end{lemmatheorembox}

\textbf{Usage count:} 2

\section*{Lemma 311}
\begin{lemmatheorembox}
\textbf{Name:} Sum/difference of two cubes
\textbf{Statement:} E^3 + F^3 = (E + F)(E^2 - EF + F^2)

E^3 - F^3 = (E - F)(E^2 + EF + F^2)
\textbf{General Topic:} Algebra
\textbf{URL:} https://en.wikipedia.org/wiki/Factorization
\end{lemmatheorembox}

\textbf{Usage count:} 2

\section*{Lemma 312}
\begin{lemmatheorembox}
\textbf{Name:} Bertrand's ballot theorem
\textbf{Statement:} The answer is \frac{p-q}{p+q}.
\textbf{General Topic:} Combinatorics
\textbf{URL:} https://en.wikipedia.org/wiki/Bertrand%27s_ballot_theorem
\end{lemmatheorembox}

\textbf{Usage count:} 2

\section*{Lemma 313}
\begin{lemmatheorembox}
\textbf{Name:} Regular octahedron
\textbf{Statement:} The regular octahedron's dual polyhedron is the cube, and they have the same three-dimensional symmetry groups, the octahedral symmetry $\mathrm {O} _{\mathrm {h} }$.
\textbf{General Topic:} Geometry
\textbf{URL:} https://en.wikipedia.org/wiki/Regular_octahedron
\end{lemmatheorembox}

\textbf{Usage count:} 2

\section*{Lemma 314}
\begin{lemmatheorembox}
\textbf{Name:} Basel problem
\textbf{Statement:} $\sum _{n=1}^{\infty }{\frac {1}{n^{2}}}={\frac {\pi ^{2}}{6}}.$
\textbf{General Topic:} Mathematical analysis
\textbf{URL:} https://en.wikipedia.org/wiki/Basel_problem
\end{lemmatheorembox}

\textbf{Usage count:} 2

\section*{Lemma 315}
\begin{lemmatheorembox}
\textbf{Name:} Isosceles trapezoid
\textbf{Statement:} In Euclidean geometry, an isosceles trapezoid[ a ] is a convex quadrilateral with a line of symmetry bisecting one pair of opposite sides.
\textbf{General Topic:} Euclidean geometry
\textbf{URL:} https://en.wikipedia.org/wiki/Isosceles_trapezoid
\end{lemmatheorembox}

\textbf{Usage count:} 2

\section*{Lemma 316}
\begin{lemmatheorembox}
\textbf{Name:} Dot product
\textbf{Statement:} Two non-zero vectors \mathbf{a} and \mathbf{b} are orthogonal if and only if \mathbf{a} \cdot \mathbf{b} = 0.
\textbf{General Topic:} Linear algebra
\textbf{URL:} https://en.wikipedia.org/wiki/Dot_product
\end{lemmatheorembox}

\textbf{Usage count:} 2

\section*{Lemma 317}
\begin{lemmatheorembox}
\textbf{Name:} Integration by parts
\textbf{Statement:} $\int u \, dv \ =\ uv - \int v \, du.$
\textbf{General Topic:} Calculus
\textbf{URL:} https://en.wikipedia.org/wiki/Integration_by_parts
\end{lemmatheorembox}

\textbf{Usage count:} 2

\section*{Lemma 318}
\begin{lemmatheorembox}
\textbf{Name:} Flat plane assumption
\textbf{Statement:} If two points lie in a plane, the line containing them lies in the plane.
\textbf{General Topic:} Geometry
\textbf{URL:} https://en.wikipedia.org/wiki/Point%E2%80%93line%E2%80%93plane_postulate
\end{lemmatheorembox}

\textbf{Usage count:} 2

\section*{Lemma 319}
\begin{lemmatheorembox}
\textbf{Name:} Cassini's identity
\textbf{Statement:} Cassini's identity, a special case of Catalan's identity, states that for the nth Fibonacci number, $F_{n-1}F_{n+1} - F_n^2 = (-1)^n.$
\textbf{General Topic:} Fibonacci numbers
\textbf{URL:} https://en.wikipedia.org/wiki/Cassini_and_Catalan_identities
\end{lemmatheorembox}

\textbf{Usage count:} 2

\section*{Lemma 320}
\begin{lemmatheorembox}
\textbf{Name:} Automorphism group
\textbf{Statement:} The automorphism group G of a finite cyclic group of order n is isomorphic to (\mathbb {Z} /n\mathbb {Z} )^{\times }
\textbf{General Topic:} Group Theory
\textbf{URL:} https://en.wikipedia.org/wiki/Automorphism_group
\end{lemmatheorembox}

\textbf{Usage count:} 2

\section*{Lemma 321}
\begin{lemmatheorembox}
\textbf{Name:} Affine transformation
\textbf{Statement:} If $f\colon \,{\mathcal {A}}\to {\mathcal {B}}$ is a parallel projection or, more generally, is generated by an axonometry, then $f$ is affine and surjective.
\textbf{General Topic:} Affine Geometry
\textbf{URL:} https://en.wikipedia.org/wiki/Affine_transformation
\end{lemmatheorembox}

\textbf{Usage count:} 2

\section*{Lemma 322}
\begin{lemmatheorembox}
\textbf{Name:} Disjoint union of graphs
\textbf{Statement:} The 2-regular graphs are the disjoint unions of cycle graphs.
\textbf{General Topic:} Graph theory
\textbf{URL:} https://en.wikipedia.org/wiki/Disjoint_union_of_graphs
\end{lemmatheorembox}

\textbf{Usage count:} 2

\section*{Lemma 323}
\begin{lemmatheorembox}
\textbf{Name:} Generating function
\textbf{Statement:} The ordinary generating function for the central binomial coefficients is
\frac{1}{\sqrt{1-4x}} = \sum_{n=0}^\infty \binom{2n}{n} x^n = 1 + 2x + 6x^2 + 20x^3 + 70x^4 + 252x^5 + \cdots.
\textbf{General Topic:} Combinatorics
\textbf{URL:} https://en.wikipedia.org/wiki/Central_binomial_coefficient
\end{lemmatheorembox}

\textbf{Usage count:} 2

\section*{Lemma 324}
\begin{lemmatheorembox}
\textbf{Name:} Transversal (geometry)
\textbf{Statement:} As a consequence of Euclid's parallel postulate, if the two lines are parallel, consecutive angles and linear pairs are supplementary, while corresponding angles, alternate angles, and vertical angles are equal.
\textbf{General Topic:} Euclidean geometry
\textbf{URL:} https://en.wikipedia.org/wiki/Transversal_%28geometry%29
\end{lemmatheorembox}

\textbf{Usage count:} 2

\section*{Lemma 325}
\begin{lemmatheorembox}
\textbf{Name:} Orthogonality
\textbf{Statement:} From the summation formula follows an orthogonality relationship: for \(j = 1, \ldots, n\) and \(j' = 1, \ldots, n\)
\[
\sum_{k=1}^{n}\overline{z^{j\cdot k}}\cdot z^{j'\cdot k}=n\cdot \delta_{j,j'}
\]
where \(\delta\) is the Kronecker delta and \(z\) is any primitive \(n\) th root of unity.
\textbf{General Topic:} Number theory
\textbf{URL:} https://en.wikipedia.org/wiki/Root_of_unity#Orthogonality
\end{lemmatheorembox}

\textbf{Usage count:} 2

\section*{Lemma 326}
\begin{lemmatheorembox}
\textbf{Name:} Tetrahedron
\textbf{Statement:} If $A_1$, $A_2$, $A_3$ and $A_4$ denote the area of each faces, the value of $r$ is given by $r=\frac{3V}{A_1+A_2+A_3+A_4}$.
\textbf{General Topic:} Geometry
\textbf{URL:} https://en.wikipedia.org/wiki/Tetrahedron
\end{lemmatheorembox}

\textbf{Usage count:} 2

\section*{Lemma 327}
\begin{lemmatheorembox}
\textbf{Name:} Subset
\textbf{Statement:} A set A is a subset of B if and only if their intersection is equal to A. Formally: $A\subseteq B{\text{ if and only if }}A\cap B=A.$
\textbf{General Topic:} Set theory
\textbf{URL:} https://en.wikipedia.org/wiki/Subset
\end{lemmatheorembox}

\textbf{Usage count:} 2

\section*{Lemma 328}
\begin{lemmatheorembox}
\textbf{Name:} Desargues's Involution Theorem
\textbf{Statement:} The three pairs of opposite sides of a complete quadrangle meet any line (not through a vertex) in three pairs of an involution.
\textbf{General Topic:} Projective geometry
\textbf{URL:} https://en.wikipedia.org/wiki/Involution_(mathematics)
\end{lemmatheorembox}

\textbf{Usage count:} 2

\section*{Lemma 329}
\begin{lemmatheorembox}
\textbf{Name:} Mutual exclusivity
\textbf{Statement:} The probability of one or both events occurring is denoted $P(A \cup B)$ and in general, it equals $P(A) + P(B) – P(A \cap B)$.
\textbf{General Topic:} Probability Theory
\textbf{URL:} https://en.wikipedia.org/wiki/Mutual_exclusivity
\end{lemmatheorembox}

\textbf{Usage count:} 2

\section*{Lemma 330}
\begin{lemmatheorembox}
\textbf{Name:} Markov chain
\textbf{Statement:} If the Markov chain is irreducible and aperiodic, then there is a unique stationary distribution $\pi$.
\textbf{General Topic:} Markov chains
\textbf{URL:} https://en.wikipedia.org/wiki/Markov_chain
\end{lemmatheorembox}

\textbf{Usage count:} 2

\section*{Lemma 331}
\begin{lemmatheorembox}
\textbf{Name:} Negative binomial distribution
\textbf{Statement:} The probability mass function of the negative binomial distribution is $f(k;r,p)\equiv \Pr(X=k)={\binom {k+r-1}{k}}(1-p)^{k}p^{r}$ where $r$ is the number of successes, $k$ is the number of failures, and $p$ is the probability of success on each trial.
\textbf{General Topic:} Probability theory
\textbf{URL:} https://en.wikipedia.org/wiki/Negative_binomial_distribution
\end{lemmatheorembox}

\textbf{Usage count:} 2

\section*{Lemma 332}
\begin{lemmatheorembox}
\textbf{Name:} Dihedral angle
\textbf{Statement:} Alternatively, if $n_{A}$ and $n_{B}$ are normal vector to the planes, one has $\cos \varphi ={\frac {\left\vert \mathbf {n} _{\mathrm {A} }\cdot \mathbf {n} _{\mathrm {B} }\right\vert }{|\mathbf {n} _{\mathrm {A} }||\mathbf {n} _{\mathrm {B} }|}}$.
\textbf{General Topic:} Geometry
\textbf{URL:} https://en.wikipedia.org/wiki/Dihedral_angle
\end{lemmatheorembox}

\textbf{Usage count:} 2

\section*{Lemma 333}
\begin{lemmatheorembox}
\textbf{Name:} Multiplicative group of integers modulo n
\textbf{Statement:} In modular arithmetic, the integers coprime (relatively prime) to n from the set $\{0,1,\dots ,n-1\}$ of n non-negative integers form a group under multiplication modulo n, called the multiplicative group of integers modulo n.
\textbf{General Topic:} Number theory
\textbf{URL:} https://en.wikipedia.org/wiki/Multiplicative_group_of_integers_modulo_n
\end{lemmatheorembox}

\textbf{Usage count:} 2

\section*{Lemma 334}
\begin{lemmatheorembox}
\textbf{Name:} Double counting (proof technique)  
\textbf{Statement:} In combinatorics, double counting, also called counting in two ways, is a combinatorial proof technique for showing that two expressions are equal by demonstrating that they are two ways of counting the size of one set.  
\textbf{General Topic:} Combinatorics  
\textbf{URL:} https://en.wikipedia.org/wiki/Double_counting_(proof_technique)  
\end{lemmatheorembox}

\textbf{Usage count:} 2

\section*{Lemma 335}
\begin{lemmatheorembox}
\textbf{Name:} Sphere
\textbf{Statement:} In three dimensions, the volume inside a sphere (that is, the volume of a ball, but classically referred to as the volume of a sphere) is \(V={\frac {4}{3}}\pi r^{3}={\frac {\pi }{6}}\ d^{3}\approx 0.5236\cdot d^{3}\) where \(r\) is the radius and \(d\) is the diameter of the sphere.
\textbf{General Topic:} Geometry
\textbf{URL:} https://en.wikipedia.org/wiki/Sphere
\end{lemmatheorembox}

\textbf{Usage count:} 2

\section*{Lemma 336}
\begin{lemmatheorembox}
\textbf{Name:} Bézout's theorem
\textbf{Statement:} Suppose that X and Y are two plane projective curves defined over a field F that do not have a common component (this condition means that X and Y are defined by polynomials, without common divisor of positive degree). Then the total number of intersection points of X and Y with coordinates in an algebraically closed field E that contains F, counted with their multiplicities, is equal to the product of the degrees of X and Y.
\textbf{General Topic:} Algebraic geometry
\textbf{URL:} https://en.wikipedia.org/wiki/B%C3%A9zout%27s_theorem
\end{lemmatheorembox}

\textbf{Usage count:} 2

\section*{Lemma 337}
\begin{lemmatheorembox}
\textbf{Name:} Cramer's rule
\textbf{Statement:} In linear algebra, Cramer's rule is an explicit formula for the solution of a system of linear equations with as many equations as unknowns, valid whenever the system has a unique solution.
\textbf{General Topic:} Linear algebra
\textbf{URL:} https://en.wikipedia.org/wiki/Cramer%27s_rule
\end{lemmatheorembox}

\textbf{Usage count:} 2

\section*{Lemma 338}
\begin{lemmatheorembox}
\textbf{Name:} Theorem one
\textbf{Statement:} For any pair of positive integers $n$ and $k$, the number of $k$-tuples of positive integers whose sum is $n$ is equal to the number of $(k - 1)$-element subsets of a set with $n - 1$ elements.
\textbf{General Topic:} Combinatorics
\textbf{URL:} https://en.wikipedia.org/wiki/Stars_and_bars_(combinatorics)#Theorem_one
\end{lemmatheorembox}

\textbf{Usage count:} 2

\section*{Lemma 339}
\begin{lemmatheorembox}
\textbf{Name:} Freshman's dream
\textbf{Statement:} When p is a prime number and x and y are members of a commutative ring of characteristic p, then (x+y)^p=x^p+y^p.
\textbf{General Topic:} Abstract Algebra
\textbf{URL:} https://en.wikipedia.org/wiki/Freshman%27s_dream
\end{lemmatheorembox}

\textbf{Usage count:} 2

\section*{Lemma 340}
\begin{lemmatheorembox}
\textbf{Name:} Without loss of generality
\textbf{Statement:} Without loss of generality (often abbreviated to WOLOG, WLOG or w.l.o.g.; less commonly stated as without any loss of generality or with no loss of generality) is a frequently used expression in mathematics. The term is used to indicate the assumption that what follows is chosen arbitrarily, narrowing the premise to a particular case, but does not affect the validity of the proof in general.
\textbf{General Topic:} Mathematical terminology
\textbf{URL:} https://en.wikipedia.org/wiki/Without_loss_of_generality
\end{lemmatheorembox}

\textbf{Usage count:} 2

\section*{Lemma 341}
\begin{lemmatheorembox}
\textbf{Name:} Euler characteristic
\textbf{Statement:} If the polyhedron has $V$ vertices (corners), $E$ edges, and $F$ faces, then the Euler characteristic $\chi$ of its surface is $\chi = V - E + F.$ The Euler characteristic of a closed orientable surface can be calculated from its genus $g$ (the number of tori in a connected sum decomposition of the surface; intuitively, the number of "handles") as $\chi =2-2g~.$
\textbf{General Topic:} Algebraic topology
\textbf{URL:} https://en.wikipedia.org/wiki/Euler_characteristic
\end{lemmatheorembox}

\textbf{Usage count:} 2

\section*{Lemma 342}
\begin{lemmatheorembox}
\textbf{Name:} Cycle graph
\textbf{Statement:} A cycle graph is: * 2-vertex colorable, if and only if it has an even number of vertices. More generally, a graph is bipartite if and only if it has no odd cycles (Kőnig, 1936).
\textbf{General Topic:} Graph Theory
\textbf{URL:} https://en.wikipedia.org/wiki/Cycle_graph
\end{lemmatheorembox}

\textbf{Usage count:} 2

\section*{Lemma 343}
\begin{lemmatheorembox}
\textbf{Name:} Orbit–stabilizer theorem
\textbf{Statement:} If G is finite then the orbit–stabilizer theorem, together with Lagrange's theorem, gives $|G\cdot x|=[G\,:\,G_{x}]=|G|/|G_{x}|$. In other words, the length of the orbit of x times the order of its stabilizer is the order of the group. In particular that implies that the orbit length is a divisor of the group order.
\textbf{General Topic:} Group Theory
\textbf{URL:} https://en.wikipedia.org/wiki/Group_action\#Orbit%E2%80%93stabilizer_theorem
\end{lemmatheorembox}

\textbf{Usage count:} 2

\section*{Lemma 344}
\begin{lemmatheorembox}
\textbf{Name:} Cauchy product
\textbf{Statement:} Consider the following two power series
\[
\sum _{i=0}^{\infty }a_{i}x^{i}\ \text{ and }\ \sum _{j=0}^{\infty }b_{j}x^{j}
\]
with complex coefficients $\{a_{i}\}$ and $\{b_{j}\}$. The Cauchy product of these two power series is defined by a discrete convolution as follows:
\[
\left(\sum _{i=0}^{\infty }a_{i}x^{i}\right)\cdot \left(\sum _{j=0}^{\infty }b_{j}x^{j}\right)=\sum _{k=0}^{\infty }c_{k}x^{k}\ \text{ where }\ c_{k}=\sum _{l=0}^{k}a_{l}b_{k-l}.
\]
\textbf{General Topic:} Mathematical analysis
\textbf{URL:} https://en.wikipedia.org/wiki/Cauchy_product
\end{lemmatheorembox}

\textbf{Usage count:} 2

\section*{Lemma 345}
\begin{lemmatheorembox}
\textbf{Name:} Orbit-stabilizer theorem
\textbf{Statement:} $|Gx|=|G:G_{x}|.\!$
\textbf{General Topic:} Group Theory
\textbf{URL:} https://en.wikipedia.org/wiki/Index_of_a_subgroup
\end{lemmatheorembox}

\textbf{Usage count:} 2

\section*{Lemma 346}
\begin{lemmatheorembox}
\textbf{Name:} Arithmetico-geometric sequence
\textbf{Statement:} In general,
\[
\sum _{k=1}^{\infty }kr^{k}={\frac {r}{(1-r)^{2}}}\quad {\text{for }}-1<r<1.
\]
\textbf{General Topic:} Infinite series
\textbf{URL:} https://en.wikipedia.org/wiki/Arithmetico-geometric_sequence
\end{lemmatheorembox}

\textbf{Usage count:} 2

\section*{Lemma 347}
\begin{lemmatheorembox}
\textbf{Name:} Prime number
\textbf{Statement:} No even number $n$ greater than 2 is prime because any such number can be expressed as the product $2\times n/2$. Therefore, every prime number other than 2 is an odd number, and is called an odd prime.
\textbf{General Topic:} Number theory
\textbf{URL:} https://en.wikipedia.org/wiki/Prime_number
\end{lemmatheorembox}

\textbf{Usage count:} 2

\section*{Lemma 348}
\begin{lemmatheorembox}
\textbf{Name:} Tannery's theorem
\textbf{Statement:} Let $S_n = \sum_{k=0}^n a_k(n)$ and suppose that $\lim_{n\to\infty} a_k(n) = b_k$. If $|a_k(n)| \le M_k$ and $\sum_{k=0}^\infty M_k < \infty$, then $\lim_{n\to\infty} S_n = \sum_{k=0}^{\infty} b_k$.
\textbf{General Topic:} Mathematical analysis
\textbf{URL:} https://en.wikipedia.org/wiki/Tannery%27s_theorem
\end{lemmatheorembox}

\textbf{Usage count:} 2

\section*{Lemma 349}
\begin{lemmatheorembox}
\textbf{Name:} Inverse function
\textbf{Statement:} Let $f$ be a function whose domain is the set $X$, and whose codomain is the set $Y$. Then $f$ is invertible if there exists a function $g$ from $Y$ to $X$ such that $g(f(x))=x$ for all $x\in X$ and $f(g(y))=y$ for all $y\in Y$.
\textbf{General Topic:} Functions
\textbf{URL:} https://en.wikipedia.org/wiki/Inverse_function
\end{lemmatheorembox}

\textbf{Usage count:} 2

\section*{Lemma 350}
\begin{lemmatheorembox}
\textbf{Name:} Isosceles triangle theorem
\textbf{Statement:} In geometry, the isosceles triangle theorem states that if two sides of a triangle are congruent, then the angles opposite those sides are congruent.
\textbf{General Topic:} Euclidean Geometry
\textbf{URL:} https://en.wikipedia.org/wiki/Isosceles_triangle_theorem
\end{lemmatheorembox}

\textbf{Usage count:} 2

\section*{Lemma 351}
\begin{lemmatheorembox}
\textbf{Name:} Formal antidifferentiation
\textbf{Statement:} If R is a ring with characteristic zero and the nonzero integers are invertible in R, then given a formal power series f=\sum _{n\geq 0}a_{n}X^{n}\in R[[X]], we define its formal antiderivative or formal indefinite integral by D^{-1}f=\int f\ dX=C+\sum _{n\geq 0}a_{n}{\frac {X^{n+1}}{n+1}}. for any constant C\in R. This operation is R-linear: D^{-1}(af+bg)=a\cdot D^{-1}f+b\cdot D^{-1}g for any a, b in R and any f, g in R[[X]]. Additionally, the formal antiderivative has many of the properties of the usual antiderivative of calculus. For example, the formal antiderivative is the right inverse of the formal derivative: D(D^{-1}(f))=f for any f\in R[[X]].
\textbf{General Topic:} Algebra
\textbf{URL:} https://en.wikipedia.org/wiki/Formal_power_series
\end{lemmatheorembox}

\textbf{Usage count:} 2

\section*{Lemma 352}
\begin{lemmatheorembox}
\textbf{Name:} Reflection (physics)
\textbf{Statement:} The law of reflection says that for specular reflection (for example at a mirror) the angle at which the wave is incident on the surface equals the angle at which it is reflected.
\textbf{General Topic:} Optics
\textbf{URL:} https://en.wikipedia.org/wiki/Reflection_(physics)
\end{lemmatheorembox}

\textbf{Usage count:} 2

\section*{Lemma 353}
\begin{lemmatheorembox}
\textbf{Name:} 2
\textbf{Statement:} It is the smallest and the only even prime number.
\textbf{General Topic:} Number theory
\textbf{URL:} https://en.wikipedia.org/wiki/2
\end{lemmatheorembox}

\textbf{Usage count:} 2

\section*{Lemma 354}
\begin{lemmatheorembox}
\textbf{Name:} Countable additivity
\textbf{Statement:} \mu must satisfy the countable additivity property that for all countable collections E_1, E_2, \ldots of pairwise disjoint sets: \ \mu\left(\bigcup_{i \in \N} E_i\right) = \sum_{i \in \N} \mu(E_i).
\textbf{General Topic:} Probability theory
\textbf{URL:} https://en.wikipedia.org/wiki/Probability_measure
\end{lemmatheorembox}

\textbf{Usage count:} 2

\section*{Lemma 355}
\begin{lemmatheorembox}
\textbf{Name:} Negative hypergeometric distribution
\textbf{Statement:} When counting the number $k$ of successes before $r$ failures, the expected number of successes is $\frac{rK}{N-K+1}$ and can be derived as follows.
\textbf{General Topic:} Probability theory and statistics
\textbf{URL:} https://en.wikipedia.org/wiki/Negative_hypergeometric_distribution
\end{lemmatheorembox}

\textbf{Usage count:} 2

\section*{Lemma 356}
\begin{lemmatheorembox}
\textbf{Name:} Cyclic quadrilateral
\textbf{Statement:} A quadrilateral ABCD with concyclic vertices is called a cyclic quadrilateral; this happens if and only if $\angle CAD=\angle CBD$
\textbf{General Topic:} Euclidean geometry
\textbf{URL:} https://en.wikipedia.org/wiki/Concyclic_points
\end{lemmatheorembox}

\textbf{Usage count:} 2

\section*{Lemma 357}
\begin{lemmatheorembox}
\textbf{Name:} Logarithm
\textbf{Statement:} In other words, the logarithm of x to base b is the unique real number y such that b^{y} = x.
\textbf{General Topic:} Logarithms
\textbf{URL:} https://en.wikipedia.org/wiki/Logarithm#Definition
\end{lemmatheorembox}

\textbf{Usage count:} 2

\section*{Lemma 358}
\begin{lemmatheorembox}
\textbf{Name:} Gauss circle problem
\textbf{Statement:} N(r)=\pi r^2 +E(r)\,
\textbf{General Topic:} Number theory
\textbf{URL:} https://en.wikipedia.org/wiki/Gauss_circle_problem
\end{lemmatheorembox}

\textbf{Usage count:} 2

\section*{Lemma 359}
\begin{lemmatheorembox}
\textbf{Name:} High school exterior angle theorem
\textbf{Statement:} The high school exterior angle theorem (HSEAT) says that the size of an exterior angle at a vertex of a triangle equals the sum of the sizes of the interior angles at the other two vertices of the triangle (remote interior angles).
\textbf{General Topic:} Euclidean geometry
\textbf{URL:} https://en.wikipedia.org/wiki/Exterior_angle_theorem
\end{lemmatheorembox}

\textbf{Usage count:} 2

\section*{Lemma 360}
\begin{lemmatheorembox}
\textbf{Name:} Quadratic reciprocity
\textbf{Statement:} Law of quadratic reciprocity—Let p and q be distinct odd prime numbers, and define the Legendre symbol as
\[
\left({\frac {q}{p}}\right)={\begin{cases}1&{\text{if }}n^{2}\equiv q{\pmod {p}}{\text{ for some integer }}n\\-1&{\text{otherwise}}.\end{cases}}
\]
Then
\[
\left({\frac {p}{q}}\right)\left({\frac {q}{p}}\right)=(-1)^{{\frac {p-1}{2}}{\frac {q-1}{2}}}.
\]
\textbf{General Topic:} Number theory
\textbf{URL:} https://en.wikipedia.org/wiki/Quadratic_reciprocity
\end{lemmatheorembox}

\textbf{Usage count:} 2

\section*{Lemma 361}
\begin{lemmatheorembox}
\textbf{Name:} Probability
\textbf{Statement:} If two events, A and B are independent then the joint probability is \(P(A{\mbox{ and }}B)=P(A\cap B)=P(A)P(B).\)
\textbf{General Topic:} Probability theory
\textbf{URL:} https://en.wikipedia.org/wiki/Probability
\end{lemmatheorembox}

\textbf{Usage count:} 2

\section*{Lemma 362}
\begin{lemmatheorembox}
\textbf{Name:} Law of reflection
\textbf{Statement:} The law of reflection states that the angle of reflection of a ray equals the angle of incidence, and that the incident direction, the surface normal, and the reflected direction are coplanar.
\textbf{General Topic:} Optics
\textbf{URL:} https://en.wikipedia.org/wiki/Specular_reflection
\end{lemmatheorembox}

\textbf{Usage count:} 2

\section*{Lemma 363}
\begin{lemmatheorembox}
\textbf{Name:} Suma o diferencia de cubos
\textbf{Statement:} $a^{3}+b^{3}=(a+b)(a^{2}-ab+b^{2}),\,\!$ and $a^{3}-b^{3}=(a-b)(a^{2}+ab+b^{2}).\,\!$
\textbf{General Topic:} Algebra
\textbf{URL:} https://es.wikipedia.org/wiki/Factorizaci%C3%B3n#Suma_o_diferencia_de_cubos
\end{lemmatheorembox}

\textbf{Usage count:} 2

\section*{Lemma 364}
\begin{lemmatheorembox}
\textbf{Name:} Fermat number
\textbf{Statement:} If $2^{k} + 1$ is prime and $k > 0$, then $k$ itself must be a power of $2$, so $2^{k} + 1$ is a Fermat number; such primes are called Fermat primes.
\textbf{General Topic:} Number theory
\textbf{URL:} https://en.wikipedia.org/wiki/Fermat_number
\end{lemmatheorembox}

\textbf{Usage count:} 2

\section*{Lemma 365}
\begin{lemmatheorembox}
\textbf{Name:} Classical definition of probability
\textbf{Statement:} $P(A)={N_{A} \over N}.$
\textbf{General Topic:} Probability theory
\textbf{URL:} https://en.wikipedia.org/wiki/Probability_interpretations#Classical_definition
\end{lemmatheorembox}

\textbf{Usage count:} 2

\section*{Lemma 366}
\begin{lemmatheorembox}
\textbf{Name:} Geometric distribution
\textbf{Statement:} The mean of the geometric distribution is its expected value which is, as previously discussed in § Moments and cumulants, $\frac{1}{p}$ or $\frac{1-p}{p}$ when defined over $\mathbb{N}$ or $\mathbb{N}_0$ respectively.
\textbf{General Topic:} Probability theory
\textbf{URL:} https://en.wikipedia.org/wiki/Geometric_distribution#Summary_statistics
\end{lemmatheorembox}

\textbf{Usage count:} 2

\section*{Lemma 367}
\begin{lemmatheorembox}
\textbf{Name:} Complex number
\textbf{Statement:} This allows to define the absolute value (or modulus or magnitude) of $z$ to be the square root
\[
|z|=\sqrt{x^2+y^2}.
\]
\textbf{General Topic:} Complex Numbers
\textbf{URL:} https://en.wikipedia.org/wiki/Complex_number
\end{lemmatheorembox}

\textbf{Usage count:} 2

\section*{Lemma 368}
\begin{lemmatheorembox}
\textbf{Name:} Simple polygon
\textbf{Statement:} Thus the sum of the internal angles, for a simple polygon with n sides is (n-2)\pi.
\textbf{General Topic:} Geometry
\textbf{URL:} https://en.wikipedia.org/wiki/Simple_polygon
\end{lemmatheorembox}

\textbf{Usage count:} 2

\section*{Lemma 369}
\begin{lemmatheorembox}
\textbf{Name:} Point reflection  
\textbf{Statement:} In Euclidean geometry, the inversion of a point X with respect to a point P is a point X* such that P is the midpoint of the line segment with endpoints X and X*.  
\textbf{General Topic:} Geometry  
\textbf{URL:} https://en.wikipedia.org/wiki/Point_reflection
\end{lemmatheorembox}

\textbf{Usage count:} 2

\section*{Lemma 370}
\begin{lemmatheorembox}
\textbf{Name:} Strong divisibility sequence
\textbf{Statement:} A strong divisibility sequence is an integer sequence $(a_{n})$ such that for all positive integers $m$ and $n$, $\gcd(a_{m},a_{n})=a_{\gcd(m,n)}$, where gcd is the greatest common divisor function.
\textbf{General Topic:} Number theory
\textbf{URL:} https://en.wikipedia.org/wiki/Divisibility_sequence
\end{lemmatheorembox}

\textbf{Usage count:} 2

\section*{Lemma 371}
\begin{lemmatheorembox}
\textbf{Name:} Median
\textbf{Statement:} If the data set has an odd number of observations, the middle one is selected (after arranging in ascending order).
\textbf{General Topic:} Statistics
\textbf{URL:} https://en.wikipedia.org/wiki/Median
\end{lemmatheorembox}

\textbf{Usage count:} 2

\section*{Lemma 372}
\begin{lemmatheorembox}
\textbf{Name:} Convex hull
\textbf{Statement:} In geometry, the convex hull, convex envelope or convex closure of a shape is the smallest convex set that contains it.
\textbf{General Topic:} Geometry
\textbf{URL:} https://en.wikipedia.org/wiki/Convex_hull
\end{lemmatheorembox}

\textbf{Usage count:} 2

\section*{Lemma 373}
\begin{lemmatheorembox}
\textbf{Name:} Bernoulli's inequality
\textbf{Statement:} An alternative form of Bernoulli's inequality for t\geq 1 and 0\leq x\leq 1 is:
(1-x)^{t}\geq 1-xt.
\textbf{General Topic:} Real Analysis
\textbf{URL:} https://en.wikipedia.org/wiki/Bernoulli%27s_inequality
\end{lemmatheorembox}

\textbf{Usage count:} 2

\section*{Lemma 374}
\begin{lemmatheorembox}
\textbf{Name:} Random permutation
\textbf{Statement:} A random permutation is a sequence where any order of its items is equally likely at random, that is, it is a permutation-valued random variable of a set of objects.
\textbf{General Topic:} Combinatorics
\textbf{URL:} https://en.wikipedia.org/wiki/Random_permutation
\end{lemmatheorembox}

\textbf{Usage count:} 2

\section*{Lemma 375}
\begin{lemmatheorembox}
\textbf{Name:} Similarity (geometry)
\textbf{Statement:} If two angles of a triangle have measures equal to the measures of two angles of another triangle, then the triangles are similar.
\textbf{General Topic:} Geometry
\textbf{URL:} https://en.wikipedia.org/wiki/Similar_triangle
\end{lemmatheorembox}

\textbf{Usage count:} 2

\section*{Lemma 376}
\begin{lemmatheorembox}
\textbf{Name:} Factorial number system
\textbf{Statement:} The factorial number system provides a unique representation for each natural number, with the given restriction on the "digits" used.
\textbf{General Topic:} Numeral systems
\textbf{URL:} https://en.wikipedia.org/wiki/Factorial_number_system
\end{lemmatheorembox}

\textbf{Usage count:} 2

\section*{Lemma 377}
\begin{lemmatheorembox}
\textbf{Name:} Picard--Lindel\"of theorem
\textbf{Statement:} Let $D \subseteq \mathbb{R} \times \mathbb{R}^n$ be a closed rectangle with $(t_0, y_0) \in \operatorname{int} D$, the interior of $D$. Let $f: D \to \mathbb{R}^n$ be a function that is continuous in $t$ and Lipschitz continuous in $y$ (with Lipschitz constant independent from $t$). Then there exists some $\varepsilon > 0$ such that the initial value problem
\[
y'(t)=f(t,y(t)),\qquad y(t_0)=y_0
\]
has a unique solution $y(t)$ on the interval $[t_0-\varepsilon, t_0+\varepsilon]$.
\textbf{General Topic:} Differential equations
\textbf{URL:} https://en.wikipedia.org/wiki/Picard%E2%80%93Lindel%C3%B6f_theorem
\end{lemmatheorembox}

\textbf{Usage count:} 2

\section*{Lemma 378}
\begin{lemmatheorembox}
\textbf{Name:} Convex function
\textbf{Statement:} A convex function $f$ of one real variable defined on some open interval $C$ is continuous on $C$.
\textbf{General Topic:} Convex analysis
\textbf{URL:} https://en.wikipedia.org/wiki/Convex_function
\end{lemmatheorembox}

\textbf{Usage count:} 2

\section*{Lemma 379}
\begin{lemmatheorembox}
\textbf{Name:} Taylor's theorem
\textbf{Statement:} Taylor's theorem—Let $k\geq 1$ be an integer and let the function $f:\mathbb {R} \to \mathbb {R}$ be $k$ times differentiable at the point $a\in \mathbb {R}$. Then there exists a function $h_{k}:\mathbb {R} \to \mathbb {R}$ such that
\[
f(x)=\sum _{i=0}^{k}{\frac {f^{(i)}(a)}{i!}}(x-a)^{i}+h_{k}(x)(x-a)^{k},
\]
and
\[
\lim _{x\to a}h_{k}(x)=0.
\]
This is called the Peano form of the remainder.
\textbf{General Topic:} Calculus
\textbf{URL:} https://en.wikipedia.org/wiki/Taylor%27s_theorem
\end{lemmatheorembox}

\textbf{Usage count:} 2

\section*{Lemma 380}
\begin{lemmatheorembox}
\textbf{Name:} Series and parallel circuits
\textbf{Statement:} The total resistance of two or more resistors connected in series is equal to the sum of their individual resistances:

$R=\sum _{i=1}^{n}R_{i}=R_{1}+R_{2}+R_{3}\cdots +R_{n}.$
\textbf{General Topic:} Electrical circuits
\textbf{URL:} https://en.wikipedia.org/wiki/Series_and_parallel_circuits
\end{lemmatheorembox}

\textbf{Usage count:} 2

\section*{Lemma 381}
\begin{lemmatheorembox}
\textbf{Name:} Nth root
\textbf{Statement:} For a positive real number x, √x denotes the positive square root of x and √[n]{x} denotes the positive real n th root.
\textbf{General Topic:} Arithmetic
\textbf{URL:} https://en.wikipedia.org/wiki/Nth_root
\end{lemmatheorembox}

\textbf{Usage count:} 2

\section*{Lemma 382}
\begin{lemmatheorembox}
\textbf{Name:} Equating coefficients
\textbf{Statement:} In mathematics, the method of equating the coefficients is a way of solving a functional equation of two expressions such as polynomials for a number of unknown parameters. It relies on the fact that two expressions are identical precisely when corresponding coefficients are equal for each different type of term.
\textbf{General Topic:} Elementary algebra
\textbf{URL:} https://en.wikipedia.org/wiki/Equating_coefficients
\end{lemmatheorembox}

\textbf{Usage count:} 2

\section*{Lemma 383}
\begin{lemmatheorembox}
\textbf{Name:} Inverse element
\textbf{Statement:} It is also an involution, since the inverse of the inverse of an element is the element itself.
\textbf{General Topic:} Abstract algebra
\textbf{URL:} https://en.wikipedia.org/wiki/Inverse_element
\end{lemmatheorembox}

\textbf{Usage count:} 2

\section*{Lemma 384}
\begin{lemmatheorembox}
\textbf{Name:} Cube
\textbf{Statement:} The volume of a rectangular cuboid is the product of its length, width, and height. Because all the edges of a cube are equal in length, the formula for the volume of a cube is the third power of its side length.
\textbf{General Topic:} Geometry
\textbf{URL:} https://en.wikipedia.org/wiki/Cube
\end{lemmatheorembox}

\textbf{Usage count:} 2

\section*{Lemma 385}
\begin{lemmatheorembox}
\textbf{Name:} Percentage
\textbf{Statement:} In mathematics, a percentage, percent, or per cent (from Latin per centum 'by a hundred') is a number or ratio expressed as a fraction of 100.
\textbf{General Topic:} Elementary mathematics
\textbf{URL:} https://en.wikipedia.org/wiki/Percentage
\end{lemmatheorembox}

\textbf{Usage count:} 2

\section*{Lemma 386}
\begin{lemmatheorembox}
\textbf{Name:} Translation (geometry)
\textbf{Statement:} In Euclidean geometry, a translation is a geometric transformation that moves every point of a figure, shape or space by the same distance in a given direction. A translation can also be interpreted as the addition of a constant vector to every point, or as shifting the origin of the coordinate system. In a Euclidean space, any translation is an isometry.
\textbf{General Topic:} Euclidean geometry
\textbf{URL:} https://en.wikipedia.org/wiki/Translation_(geometry)
\end{lemmatheorembox}

\textbf{Usage count:} 2

\section*{Lemma 387}
\begin{lemmatheorembox}
\textbf{Name:} Euclidean distance
\textbf{Statement:} The distance between any two points on the real line is the absolute value of the numerical difference of their coordinates, their absolute difference. Thus if p and q are two points on the real line, then the distance between them is given by: d(p,q)=|p-q|.
\textbf{General Topic:} Geometry
\textbf{URL:} https://en.wikipedia.org/wiki/Euclidean_distance
\end{lemmatheorembox}

\textbf{Usage count:} 2

\section*{Lemma 388}
\begin{lemmatheorembox}
\textbf{Name:} Function application
\textbf{Statement:} For every a and b, with some function f(x), if a = b, then f(a)=f(b).
\textbf{General Topic:} Mathematical Logic
\textbf{URL:} https://en.wikipedia.org/wiki/Equality_%28mathematics%29
\end{lemmatheorembox}

\textbf{Usage count:} 2

\section*{Lemma 389}
\begin{lemmatheorembox}
\textbf{Name:} Formula di Grassmann
\textbf{Statement:} $\dim(W+U)=\dim(W)+\dim(U)-\dim(W\cap U)$
\textbf{General Topic:} Linear Algebra
\textbf{URL:} https://it.wikipedia.org/wiki/Formula_di_Grassmann
\end{lemmatheorembox}

\textbf{Usage count:} 2

\section*{Lemma 390}
\begin{lemmatheorembox}
\textbf{Name:} Multiplicative inverse
\textbf{Statement:} If $a \le b$, then $1/a \ge 1/b$.
\textbf{General Topic:} Inequalities
\textbf{URL:} https://en.wikipedia.org/wiki/Inequality_(mathematics)#Multiplicative_inverse
\end{lemmatheorembox}

\textbf{Usage count:} 2

\section*{Lemma 391}
\begin{lemmatheorembox}
\textbf{Name:} Percentage increase and decrease
\textbf{Statement:} In general, a change of x percent in a quantity results in a final amount that is 100 + x percent of the original amount (equivalently, (1 + 0.01 x) times the original amount).
\textbf{General Topic:} Arithmetic
\textbf{URL:} https://en.wikipedia.org/wiki/Percentage#Percentage_increase_and_decrease
\end{lemmatheorembox}

\textbf{Usage count:} 2

\section*{Lemma 392}
\begin{lemmatheorembox}
\textbf{Name:} Berge's theorem
\textbf{Statement:} Graphs where every vertex has degree less than or equal to 2 must consist of either isolated vertices, cycles, and paths.
\textbf{General Topic:} Graph theory
\textbf{URL:} https://en.wikipedia.org/wiki/Berge%27s_theorem
\end{lemmatheorembox}

\textbf{Usage count:} 1

\section*{Lemma 393}
\begin{lemmatheorembox}
\textbf{Name:} Reverse triangle inequality
\textbf{Statement:} Any side of a triangle is greater than or equal to the difference between the other two sides.
\textbf{General Topic:} Geometry
\textbf{URL:} https://en.wikipedia.org/wiki/Triangle_inequality#Reverse_triangle_inequality
\end{lemmatheorembox}

\textbf{Usage count:} 1

\section*{Lemma 394}
\begin{lemmatheorembox}
\textbf{Name:} Well-founded relation
\textbf{Statement:} Equivalently, assuming the axiom of dependent choice, a relation is well-founded when it contains no infinite descending chains,
\textbf{General Topic:} Order theory
\textbf{URL:} https://en.wikipedia.org/wiki/Well-founded_relation
\end{lemmatheorembox}

\textbf{Usage count:} 1

\section*{Lemma 395}
\begin{lemmatheorembox}
\textbf{Name:} Angle bisector theorem

\textbf{Statement:} The theorem states for any triangle ∠ DAB and ∠ DAC where AD is a bisector, then |BD| : |CD| = |AB| : |AC|.

\textbf{General Topic:} Euclidean geometry

\textbf{URL:} https://en.wikipedia.org/wiki/Angle_bisector_theorem. ([en.wikipedia.org](https://en.wikipedia.org/wiki/Angle_bisector_theorem))
\end{lemmatheorembox}

\textbf{Usage count:} 1

\section*{Lemma 396}
\begin{lemmatheorembox}
\textbf{Name:} Thales's theorem

\textbf{Statement:} Thales’ theorem: if AC is a diameter and B is a point on the diameter's circle, the angle ∠ ABC is a right angle.

\textbf{General Topic:} Euclidean geometry

\textbf{URL:} https://en.wikipedia.org/wiki/Thales%27s_theorem. ([en.wikipedia.org](https://en.wikipedia.org/wiki/Thales%27s_theorem))
\end{lemmatheorembox}

\textbf{Usage count:} 1

\section*{Lemma 397}
\begin{lemmatheorembox}
\textbf{Name:} Tangent–secant theorem

\textbf{Statement:} \(|PT|^{2}=|PG_{1}|\cdot |PG_{2}|\)

\textbf{General Topic:} Euclidean geometry

\textbf{URL:} https://en.wikipedia.org/wiki/Tangent%E2%80%93secant_theorem. ([en.wikipedia.org](https://en.wikipedia.org/wiki/Tangent%E2%80%93secant_theorem))
\end{lemmatheorembox}

\textbf{Usage count:} 1

\section*{Lemma 398}
\begin{lemmatheorembox}
\textbf{Name:} Law of cosines

\textbf{Statement:} \(\begin{aligned}c^{2}&=a^{2}+b^{2}-2ab\cos \gamma ,\\[3mu]a^{2}&=b^{2}+c^{2}-2bc\cos \alpha ,\\[3mu]b^{2}&=a^{2}+c^{2}-2ac\cos \beta .\end{aligned}\)

\textbf{General Topic:} Trigonometry

\textbf{URL:} https://en.wikipedia.org/wiki/Law_of_cosines. ([en.wikipedia.org](https://en.wikipedia.org/wiki/Law_of_cosines))
\end{lemmatheorembox}

\textbf{Usage count:} 1

\section*{Lemma 399}
\begin{lemmatheorembox}
\textbf{Name:} Zsigmondy's theorem
\textbf{Statement:} In number theory, Zsigmondy's theorem, named after Karl Zsigmondy, states that if \(a>b>0\) are coprime integers, then for any integer \(n\geq 1\), there is a prime number \(p\) (called a primitive prime divisor) that divides \(a^{n}-b^{n}\) and does not divide \(a^{k}-b^{k}\) for any positive integer \(k<n\), with the following exceptions:
* \(n=1\), \(a-b=1\); then \(a^{n}-b^{n}=1\) which has no prime divisors
* \(n=2\), \(a+b\) a power of two; then any odd prime factors of \(a^{2}-b^{2}=(a+b)(a^{1}-b^{1})\) must be contained in \(a^{1}-b^{1}\), which is also even
* \(n=6\), \(a=2\), \(b=1\); then \(a^{6}-b^{6}=63=3^{2}\times 7=(a^{2}-b^{2})^{2}(a^{3}-b^{3})\)
\textbf{General Topic:} Number theory
\textbf{URL:} https://en.wikipedia.org/wiki/Zsigmondy%27s_theorem
\end{lemmatheorembox}

\textbf{Usage count:} 1

\section*{Lemma 400}
\begin{lemmatheorembox}
\textbf{Name:} Proof by mathematical induction
\textbf{Statement:} Then P(n) is true for all natural numbers n.
\textbf{General Topic:} Proof techniques
\textbf{URL:} https://en.wikipedia.org/wiki/Mathematical_proof\#Proof_by_mathematical_induction
\end{lemmatheorembox}

\textbf{Usage count:} 1

\section*{Lemma 401}
\begin{lemmatheorembox}
\textbf{Name:} Trigonometric form of Ceva's theorem
\textbf{Statement:} For \(D\in BC,E\in CA,F\in AB\), lines \(AD,BE,CF\) are concurrent iff \(\frac{\sin\angle BAD}{\sin\angle CAD}\frac{\sin\angle CBE}{\sin\angle ABE}\frac{\sin\angle ACF}{\sin\angle BCF}=1\).
\textbf{General Topic:} Euclidean geometry
\textbf{URL:} https://en.wikipedia.org/wiki/Trigonometric_form_of_Ceva%27s_theorem
\end{lemmatheorembox}

\textbf{Usage count:} 1

\section*{Lemma 402}
\begin{lemmatheorembox}
\textbf{Name:} Bombieri–Vinogradov theorem

\textbf{Statement:} Let $x$ and $Q$ be any two positive real numbers with
\[
x^{1/2}\log^{-A}x\leq Q\leq x^{1/2}.
\]
Then
\[
\sum_{q\leq Q}\max_{y\le x}\max_{1\le a\le q\atop (a,q)=1}\left|\psi(y;q,a)-{y\over\varphi(q)}\right|=O\left(x^{1/2}Q(\log x)^{5}\right)\!.
\]
Here $\varphi(q)$ is the Euler totient function, which is the number of summands for the modulus $q$, and
\[
\psi(x;q,a)=\sum_{n\le x\atop n\equiv a\bmod q}\Lambda(n),
\]
where $\Lambda$ denotes the von Mangoldt function.

\textbf{General Topic:} Analytic number theory

\textbf{URL:} https://en.wikipedia.org/wiki/Bombieri%E2%80%93Vinogradov_theorem
\end{lemmatheorembox}

\textbf{Usage count:} 1

\section*{Lemma 403}
\begin{lemmatheorembox}
\textbf{Name:} Mertens' second theorem

\textbf{Statement:} Mertens' second theorem is
\[
\lim_{n\to\infty}\left(\sum_{p\le n}\frac1p -\log\log n-M\right) =0,
\]
where $M$ is the Meissel–Mertens constant. More precisely, Mertens proves that the expression under the limit does not in absolute value exceed
\[
\frac 4{\log(n+1)} +\frac 2{n\log n}
\]
for any $n\ge 2$.

\textbf{General Topic:} Analytic number theory

\textbf{URL:} https://en.wikipedia.org/wiki/Mertens%27_theorems
\end{lemmatheorembox}

\textbf{Usage count:} 1

\section*{Lemma 404}
\begin{lemmatheorembox}
\textbf{Name:} Catalan's conjecture
\textbf{Statement:} The theorem states that this is the only case of two consecutive perfect powers. That is to say, that Catalan's conjecture—the only solution in the natural numbers of \(x^{a}-y^{b}=1\) for \(a, b > 1\), \(x, y > 0\) is \(x = 3\), \(a = 2\), \(y = 2\), \(b = 3\).
\textbf{General Topic:} Number theory
\textbf{URL:} https://en.wikipedia.org/wiki/Catalan%27s_conjecture
\end{lemmatheorembox}

\textbf{Usage count:} 1

\section*{Lemma 405}
\begin{lemmatheorembox}
\textbf{Name:} Quadratic reciprocity
\textbf{Statement:} Let \(p\) and \(q\) be distinct odd primes. Using the Legendre symbol, the quadratic reciprocity law can be stated concisely: \(\left(\frac{q}{p}\right)\left(\frac{p}{q}\right)=(-1)^{{\tfrac {p-1}{2}}\cdot {\tfrac {q-1}{2}}}.\)
\textbf{General Topic:} Number theory
\textbf{URL:} https://en.wikipedia.org/wiki/Legendre_symbol
\end{lemmatheorembox}

\textbf{Usage count:} 1

\section*{Lemma 406}
\begin{lemmatheorembox}
\textbf{Name:} Jung's theorem
\textbf{Statement:} Consider a compact set \(K\subset \mathbb {R} ^{n}\) and let \(d=\max _{p,q\,\in \,K}\|p-q\|_{2}\) be the diameter of K, that is, the largest Euclidean distance between any two of its points. Jung's theorem states that there exists a closed ball with radius \(r\leq d{\sqrt {\frac {n}{2(n+1)}}}\) that contains K. The boundary case of equality is attained by the regular n-simplex.
\textbf{General Topic:} Geometry
\textbf{URL:} https://en.wikipedia.org/wiki/Jung%27s_theorem
\end{lemmatheorembox}

\textbf{Usage count:} 1

\section*{Lemma 407}
\begin{lemmatheorembox}
\textbf{Name:} Multiplicity (mathematics)
\textbf{Statement:} For an equation f(x)=0 with a single variable solution x\_*, the multiplicity is k if

f(x\_*)=f'(x\_*) = \cdots = f^{(k-1)}(x\_*)=0 and f^{(k)}(x\_*)\neq0.
\textbf{General Topic:} Algebra
\textbf{URL:} https://en.wikipedia.org/wiki/Multiplicity_%28mathematics%29
\end{lemmatheorembox}

\textbf{Usage count:} 1

\section*{Lemma 408}
\begin{lemmatheorembox}
\textbf{Name:} Harmonic series (mathematics)
\textbf{Statement:} The first \(n\) terms of the series sum to approximately \(\ln n+\gamma\), where \(\ln\) is the natural logarithm and \(\gamma\approx 0.577\) is the Euler--Mascheroni constant.
\textbf{General Topic:} Mathematical analysis
\textbf{URL:} https://en.wikipedia.org/wiki/Harmonic_series_(mathematics)
\end{lemmatheorembox}

\textbf{Usage count:} 1

\section*{Lemma 409}
\begin{lemmatheorembox}
\textbf{Name:} Lobatschewski-Integral
\textbf{Statement:} \(\int _{0}^{\frac {\pi }{2}}{\ln {\sin x}}\,\mathrm {d} x=-{\frac {\pi }{2}}\cdot \ln {2}\)
\textbf{General Topic:} Analysis
\textbf{URL:} https://de.wikipedia.org/wiki/Lobatschewskische_Formeln
\end{lemmatheorembox}

\textbf{Usage count:} 1

\section*{Lemma 410}
\begin{lemmatheorembox}
\textbf{Name:} Loomis--Whitney inequality
\textbf{Statement:} Let $E$ be some measurable subset of $\mathbb{R}^{d}$ and let $f_{j}=\mathbf {1} _{\pi _{j}(E)}$ be the indicator function of the projection of $E$ onto the $j$th coordinate hyperplane. Hence, by the Loomis--Whitney inequality, $\int_{\mathbb{R}^{d}} \mathbf 1_E(x) \, \mathrm{d} x = | E | \leq \prod_{j = 1}^{d} | \pi_{j} (E) |^{1 / (d - 1)}$.
\textbf{General Topic:} Geometry
\textbf{URL:} https://en.wikipedia.org/wiki/Loomis%E2%80%93Whitney_inequality
\end{lemmatheorembox}

\textbf{Usage count:} 1

\section*{Lemma 411}
\begin{lemmatheorembox}
\textbf{Name:} Hölder's inequality
\textbf{Statement:} Hölder's inequality—Let $(S, \Sigma, \mu)$ be a measure space and let $p, q \in [1, \infty]$ with $1/p + 1/q = 1$. Then for all measurable real- or complex-valued functions $f$ and $g$ on $S$, $\|fg\|_{1}\leq \|f\|_{p}\|g\|_{q}$.
\textbf{General Topic:} Mathematical analysis
\textbf{URL:} https://en.wikipedia.org/wiki/H%C3%B6lder%27s_inequality
\end{lemmatheorembox}

\textbf{Usage count:} 1

\section*{Lemma 412}
\begin{lemmatheorembox}
\textbf{Name:} Miquel's theorem
\textbf{Statement:} Formally, let ABC be a triangle, with arbitrary points A´, B´ and C´ on sides BC, AC, and AB respectively (or their extensions). Draw three circumcircles (Miquel's circles) to triangles AB´C´, A´BC´, and A´B´C. Miquel's theorem states that these circles intersect in a single point M, called the Miquel point. In addition, the three angles MA´B, MB´C and MC´A (green in the diagram) are all equal, as are the three supplementary angles MA´C, MB´A and MC´B.
\textbf{General Topic:} Euclidean geometry
\textbf{URL:} https://en.wikipedia.org/wiki/Miquel%27s_theorem
\end{lemmatheorembox}

\textbf{Usage count:} 1

\section*{Lemma 413}
\begin{lemmatheorembox}
\textbf{Name:} Dense set
\textbf{Statement:} In topology and related areas of mathematics, a subset A of a topological space X is said to be dense in X if every point of X either belongs to A or else is arbitrarily "close" to a member of A — for instance, the rational numbers are a dense subset of the real numbers because every real number either is a rational number or has a rational number arbitrarily close to it (see Diophantine approximation).
\textbf{General Topic:} Topology
\textbf{URL:} https://en.wikipedia.org/wiki/Dense_set
\end{lemmatheorembox}

\textbf{Usage count:} 1

\section*{Lemma 414}
\begin{lemmatheorembox}
\textbf{Name:} Karamata's inequality

\textbf{Statement:} Let \(I\) be an interval of the real line and let \(f\) denote a real-valued, convex function defined on \(I\). If \(x_{1}, \dots, x_{n}\) and \(y_{1}, \dots, y_{n}\) are numbers in \(I\) such that \((x_{1}, \dots, x_{n})\) majorizes \((y_{1}, \dots, y_{n})\), then
\[
f(x_{1})+\cdots+f(x_{n})\geq f(y_{1})+\cdots+f(y_{n}).
\]
Here majorization means that \(x_{1}, \dots, x_{n}\) and \(y_{1}, \dots, y_{n}\) satisfies
\[
x_{1}\geq x_{2}\geq \cdots \geq x_{n}\qquad\text{and}\qquad y_{1}\geq y_{2}\geq \cdots \geq y_{n},
\]
and we have the inequalities
\[
x_{1}+\cdots +x_{i}\geq y_{1}+\cdots +y_{i}\qquad\text{for all } i \in \{1,\dots,n-1\},
\]
and the equality
\[
x_{1}+\cdots +x_{n}=y_{1}+\cdots +y_{n}.
\]

\textbf{General Topic:} Inequalities (mathematics)

\textbf{URL:} https://en.wikipedia.org/wiki/Karamata%27s_inequality
\end{lemmatheorembox}

\textbf{Usage count:} 1

\section*{Lemma 415}
\begin{lemmatheorembox}
\textbf{Name:} Difference of two squares
\textbf{Statement:} $a^2-b^2 = (a+b)(a-b)$.
\textbf{General Topic:} Elementary algebra
\textbf{URL:} https://en.wikipedia.org/wiki/Difference_of_two_squares. ([en.wikipedia.org](https://en.wikipedia.org/wiki/Difference_of_two_squares?utm_source=openai))
\end{lemmatheorembox}

\textbf{Usage count:} 1

\section*{Lemma 416}
\begin{lemmatheorembox}
\textbf{Name:} Prime number
\textbf{Statement:} A prime number (or a prime) is a natural number greater than 1 that is not a product of two smaller natural numbers.
\textbf{General Topic:} Number theory
\textbf{URL:} https://en.wikipedia.org/wiki/Prime_number. ([en.wikipedia.org](https://en.wikipedia.org/wiki/Prime_number?utm_source=openai))
\end{lemmatheorembox}

\textbf{Usage count:} 1

\section*{Lemma 417}
\begin{lemmatheorembox}
\textbf{Name:} Proofs of trigonometric identities
\textbf{Statement:} \cot(\alpha +\beta )={\frac {1-\tan \alpha \tan \beta }{\tan \alpha +\tan \beta }}
\textbf{General Topic:} Trigonometry
\textbf{URL:} https://en.wikipedia.org/wiki/Proofs_of_trigonometric_identities
\end{lemmatheorembox}

\textbf{Usage count:} 1

\section*{Lemma 418}
\begin{lemmatheorembox}
\textbf{Name:} Square root of a matrix
\textbf{Statement:} A square real matrix \(A\) is positive semidefinite if and only if \(A=B^{\textsf{T}}B\) for some matrix \(B\).
\textbf{General Topic:} Linear algebra
\textbf{URL:} https://en.wikipedia.org/wiki/Square_root_of_a_matrix
\end{lemmatheorembox}

\textbf{Usage count:} 1

\section*{Lemma 419}
\begin{lemmatheorembox}
\textbf{Name:} Brahmagupta–Fibonacci identity
\textbf{Statement:} \(\left(a^2 + b^2\right)\left(c^2 + d^2\right) = \left(ac-bd\right)^2 + \left(ad+bc\right)^2.\)
\textbf{General Topic:} Algebra
\textbf{URL:} https://en.wikipedia.org/wiki/Brahmagupta%E2%80%93Fibonacci_identity
\end{lemmatheorembox}

\textbf{Usage count:} 1

\section*{Lemma 420}
\begin{lemmatheorembox}
\textbf{Name:} Basis (linear algebra)
\textbf{Statement:} In mathematics, a set B of elements of a vector space V is called a basis (pl.: bases) if every element of V can be written in a unique way as a finite linear combination of elements of B.
\textbf{General Topic:} Linear Algebra
\textbf{URL:} https://en.wikipedia.org/wiki/Basis_%28linear_algebra%29
\end{lemmatheorembox}

\textbf{Usage count:} 1

\section*{Lemma 421}
\begin{lemmatheorembox}
\textbf{Name:} Antipodal point
\textbf{Statement:} In geometry, two points are antipodal if they are diametrically opposite.
\textbf{General Topic:} Geometry
\textbf{URL:} https://en.wikipedia.org/wiki/Antipodal_point
\end{lemmatheorembox}

\textbf{Usage count:} 1

\section*{Lemma 422}
\begin{lemmatheorembox}
\textbf{Name:} Law of excluded middle
\textbf{Statement:} In logic, the law of excluded middle or the principle of excluded middle states that for every proposition, either this proposition or its negation is true.
\textbf{General Topic:} Logic
\textbf{URL:} https://en.wikipedia.org/wiki/Law_of_excluded_middle
\end{lemmatheorembox}

\textbf{Usage count:} 1

\section*{Lemma 423}
\begin{lemmatheorembox}
\textbf{Name:} Pedal circle
\textbf{Statement:} \detokenize{If P does not lie on the circumcircle then its isogonal conjugate Q yields the same pedal circle, that is the six points P_a, P_b, P_c and Q_a, Q_b, Q_c lie on the same circle. Moreover, the midpoint of the line segment PQ is the center of that pedal circle.}
\textbf{General Topic:} Triangle geometry
\textbf{URL:} https://en.wikipedia.org/wiki/Pedal_circle
\end{lemmatheorembox}

\textbf{Usage count:} 1

\section*{Lemma 424}
\begin{lemmatheorembox}
\textbf{Name:} Complete (strong) induction
\textbf{Statement:} Another variant, called complete induction, course of values induction or strong induction (in contrast to which the basic form of induction is sometimes known as weak induction), makes the induction step easier to prove by using a stronger hypothesis: one proves the statement P(m+1) under the assumption that P(n) holds for all natural numbers n less than m+1; by contrast, the basic form only assumes P(m).
\textbf{General Topic:} Mathematical proof
\textbf{URL:} https://en.wikipedia.org/wiki/Mathematical_induction#Complete_(strong)_induction
\end{lemmatheorembox}

\textbf{Usage count:} 1

\section*{Lemma 425}
\begin{lemmatheorembox}
\textbf{Name:} Divisor sum
\textbf{Statement:} The property established by Gauss,[ 18 ] that \sum_{d\mid n}\varphi(d)=n, where the sum is over all positive divisors d of n, can be proven in several ways.
\textbf{General Topic:} Number theory
\textbf{URL:} https://en.wikipedia.org/wiki/Euler%27s_totient_function#Divisor_sum
\end{lemmatheorembox}

\textbf{Usage count:} 1

\section*{Lemma 426}
\begin{lemmatheorembox}
\textbf{Name:} Axiom of induction
\textbf{Statement:} \forall P\,{\Bigl (}P(0)\land \forall k{\bigl (}P(k)\to P(k+1){\bigr )}\to \forall n\,{\bigl (}P(n){\bigr )}{\Bigr )} ([en.wikipedia.org](https://en.wikipedia.org/wiki/Mathematical_induction))
\textbf{General Topic:} Mathematical Logic
\textbf{URL:} https://en.wikipedia.org/wiki/Mathematical_induction ([en.wikipedia.org](https://en.wikipedia.org/wiki/Mathematical_induction))
\end{lemmatheorembox}

\textbf{Usage count:} 1

\section*{Lemma 427}
\begin{lemmatheorembox}
\textbf{Name:} General Leibniz rule
\textbf{Statement:} In calculus, the general Leibniz rule, named after Gottfried Wilhelm Leibniz, generalizes the product rule for the derivative of the product of two functions (which is also known as "Leibniz's rule"). It states that if f and g are n-times differentiable functions, then the product fg is also n-times differentiable and its n-th derivative is given by
(fg)^{(n)}=\sum_{k=0}^n {n \choose k} f^{(n-k)} g^{(k)},
where {n \choose k}={n!\over k! (n-k)!} is the binomial coefficient and f^{(j)} denotes the j-th derivative of f (and in particular f^{(0)}= f).
\textbf{General Topic:} Calculus
\textbf{URL:} https://en.wikipedia.org/wiki/General_Leibniz_rule
\end{lemmatheorembox}

\textbf{Usage count:} 1

\section*{Lemma 428}
\begin{lemmatheorembox}
\textbf{Name:} Lagrange's theorem (group theory)

\textbf{Statement:} Lagrange's theorem states that if H is a subgroup of any finite group G, then |H| is a divisor of |G|.

\textbf{General Topic:} Group theory

\textbf{URL:} https://en.wikipedia.org/wiki/Lagrange%27s_theorem_(group_theory)
\end{lemmatheorembox}

\textbf{Usage count:} 1

\section*{Lemma 429}
\begin{lemmatheorembox}
\textbf{Name:} Mollweide's formula
\textbf{Statement:} Mollweide's formulas are
\[
\begin{aligned}
{\frac {a+b}{c}}&={\frac {\cos {\tfrac {1}{2}}(\alpha -\beta )}{\sin {\tfrac {1}{2}}\gamma }},\\[10mu]
{\frac {a-b}{c}}&={\frac {\sin {\tfrac {1}{2}}(\alpha -\beta )}{\cos {\tfrac {1}{2}}\gamma }}.
\end{aligned}
\]
\textbf{General Topic:} Trigonometry
\textbf{URL:} https://en.wikipedia.org/wiki/Mollweide%27s_formula
\end{lemmatheorembox}

\textbf{Usage count:} 1

\section*{Lemma 430}
\begin{lemmatheorembox}
\textbf{Name:} Dirichlet's approximation theorem
\textbf{Statement:} In number theory, Dirichlet's theorem on Diophantine approximation, also called Dirichlet's approximation theorem, states that for any real numbers $\alpha$ and $N$, with $1\leq N$, there exist integers $p$ and $q$ such that $1\leq q\leq N$ and
\[
\left|q\alpha -p\right|\leq {\frac {1}{\lfloor N\rfloor +1}}<{\frac {1}{N}}.
\]
\textbf{General Topic:} Number theory
\textbf{URL:} https://en.wikipedia.org/wiki/Dirichlet%27s_approximation_theorem
\end{lemmatheorembox}

\textbf{Usage count:} 1

\section*{Lemma 431}
\begin{lemmatheorembox}
\textbf{Name:} Polynomial interpolation
\textbf{Statement:} Generally, if we have n data points, there is exactly one polynomial of degree at most n−1 going through all the data points.
\textbf{General Topic:} Numerical analysis
\textbf{URL:} https://en.wikipedia.org/wiki/Interpolation
\end{lemmatheorembox}

\textbf{Usage count:} 1

\section*{Lemma 432}
\begin{lemmatheorembox}
\textbf{Name:} Polygon triangulation
\textbf{Statement:} The total number of ways to triangulate a convex n -gon by non-intersecting diagonals is the (n−2)nd Catalan number
\textbf{General Topic:} Combinatorics
\textbf{URL:} https://en.wikipedia.org/wiki/Polygon_triangulation
\end{lemmatheorembox}

\textbf{Usage count:} 1

\section*{Lemma 433}
\begin{lemmatheorembox}
\textbf{Name:} Sperner's theorem
\textbf{Statement:} Sperner's theorem states that these examples are the largest possible Sperner families over an n-element set. Formally, the theorem states that, 1. for every Sperner family S whose union has a total of n elements, |S| \leq \binom{n}{\lfloor n/2\rfloor}, and 2. equality holds if and only if S consists of all subsets of an n-element set that have size \lfloor n/2\rfloor or all that have size \lceil n/2\rceil.
\textbf{General Topic:} Extremal set theory
\textbf{URL:} https://en.wikipedia.org/wiki/Sperner%27s_theorem
\end{lemmatheorembox}

\textbf{Usage count:} 1

\section*{Lemma 434}
\begin{lemmatheorembox}
\textbf{Name:} Dilworth's theorem
\textbf{Statement:} Dilworth's theorem states that, in any finite partially ordered set, the largest antichain has the same size as the smallest chain decomposition.
\textbf{General Topic:} Order theory
\textbf{URL:} https://en.wikipedia.org/wiki/Dilworth%27s_theorem
\end{lemmatheorembox}

\textbf{Usage count:} 1

\section*{Lemma 435}
\begin{lemmatheorembox}
\textbf{Name:} Sphere
\textbf{Statement:} The surface area of a sphere of radius r is: A=4\pi r^{2}.
\textbf{General Topic:} Geometry
\textbf{URL:} https://en.wikipedia.org/wiki/Sphere#Surface_area
\end{lemmatheorembox}

\textbf{Usage count:} 1

\section*{Lemma 436}
\begin{lemmatheorembox}
\textbf{Name:} Nicomachus's theorem
\textbf{Statement:} In number theory, the sum of the first n cubes is the square of the n th triangular number. That is, \(1^{3}+2^{3}+3^{3}+\cdots +n^{3}=\left(1+2+3+\cdots +n\right)^{2}.\) The same equation may be written more compactly using the mathematical notation for summation: \(\sum _{k=1}^{n}k^{3}=\left(\sum _{k=1}^{n}k\right)^{2}.\)
\textbf{General Topic:} Number theory
\textbf{URL:} https://en.wikipedia.org/wiki/Squared_triangular_number
\end{lemmatheorembox}

\textbf{Usage count:} 1

\section*{Lemma 437}
\begin{lemmatheorembox}
\textbf{Name:} Infinite product
\textbf{Statement:} In mathematics, for a sequence of complex numbers $a_1, a_2, a_3, \ldots$ the infinite product
\[
\prod_{n=1}^{\infty} a_n = a_1a_2a_3\cdots
\]
is defined to be the limit of the partial products $a_1a_2\cdots a_n$ as $n$ increases without bound. The product is said to converge when the limit exists and is not zero. Otherwise the product is said to diverge.
\textbf{General Topic:} Mathematical analysis
\textbf{URL:} https://en.wikipedia.org/wiki/Infinite_product
\end{lemmatheorembox}

\textbf{Usage count:} 1

\section*{Lemma 438}
\begin{lemmatheorembox}
\textbf{Name:} Tree (graph theory)
\textbf{Statement:} Every connected graph G admits a spanning tree, which is a tree that contains every vertex of G and whose edges are edges of G.
\textbf{General Topic:} Graph theory
\textbf{URL:} https://en.wikipedia.org/wiki/Tree_%28graph_theory%29
\end{lemmatheorembox}

\textbf{Usage count:} 1

\section*{Lemma 439}
\begin{lemmatheorembox}
\textbf{Name:} Parallel (geometry)  
\textbf{Statement:} In geometry, parallel lines are coplanar infinite straight lines that do not intersect at any point.  
\textbf{General Topic:} Euclidean Geometry  
\textbf{URL:} https://en.wikipedia.org/wiki/Parallel_%28geometry%29
\end{lemmatheorembox}

\textbf{Usage count:} 1

\section*{Lemma 440}
\begin{lemmatheorembox}
\textbf{Name:} Equiareal map  
\textbf{Statement:} Every Euclidean isometry of the Euclidean plane is equiareal, but the converse is not true.  
\textbf{General Topic:} Differential Geometry  
\textbf{URL:} https://en.wikipedia.org/wiki/Equiareal_map
\end{lemmatheorembox}

\textbf{Usage count:} 1

\section*{Lemma 441}
\begin{lemmatheorembox}
\textbf{Name:} List of limits
\textbf{Statement:} $\lim _{x\to \infty }{\sqrt[{x}]{a}}=\lim _{x\to \infty }{a}^{1/x}={\begin{cases}1,&a>0\\0,&a=0\\{\text{does not exist}},&a<0\end{cases}}$
\textbf{General Topic:} Calculus
\textbf{URL:} https://en.wikipedia.org/wiki/List_of_limits
\end{lemmatheorembox}

\textbf{Usage count:} 1

\section*{Lemma 442}
\begin{lemmatheorembox}
\textbf{Name:} Chord diagram (mathematics)
\textbf{Statement:} There is a Catalan number of chord diagrams on a given ordered set in which no two chords cross each other.
\textbf{General Topic:} Combinatorics
\textbf{URL:} https://en.wikipedia.org/wiki/Chord_diagram_(mathematics)
\end{lemmatheorembox}

\textbf{Usage count:} 1

\section*{Lemma 443}
\begin{lemmatheorembox}
\textbf{Name:} Déterminant de Vandermonde
\textbf{Statement:} $\begin{vmatrix} 1 &a_1 & {a_1}^2 & \dots & {a_1}^{n-1}\\ 1 & a_2 & {a_2}^2 & \dots & {a_2}^{n-1}\\ 1 & a_3 & {a_3}^2 & \dots & {a_3}^{n-1}\\ \vdots & \vdots & \vdots & &\vdots \\ 1 & a_n & {a_n}^2 & \dots & {a_n}^{n-1} \end{vmatrix} = \prod_{1\le i<j\le n} (a_j-a_i)$
\textbf{General Topic:} Linear algebra
\textbf{URL:} https://fr.wikipedia.org/wiki/Calcul_du_d%C3%A9terminant_d%27une_matrice
\end{lemmatheorembox}

\textbf{Usage count:} 1

\section*{Lemma 444}
\begin{lemmatheorembox}
\textbf{Name:} Normed vector space
\textbf{Statement:} Positive definiteness: for every $x \in V$, \; $\lVert x\rVert=0$ if and only if $x$ is the zero vector.
\textbf{General Topic:} Functional analysis
\textbf{URL:} https://en.wikipedia.org/wiki/Normed_vector_space
\end{lemmatheorembox}

\textbf{Usage count:} 1

\section*{Lemma 445}
\begin{lemmatheorembox}
\textbf{Name:} Sophie Germain's identity
\textbf{Statement:} \(x^4 + 4y^4 = \bigl((x + y)^2 + y^2\bigr)\cdot\bigl((x - y)^2 + y^2\bigr) = (x^2 + 2xy + 2y^2)\cdot(x^2 - 2xy + 2y^2).\)
\textbf{General Topic:} Algebra
\textbf{URL:} https://en.wikipedia.org/wiki/Sophie_Germain%27s_identity
\end{lemmatheorembox}

\textbf{Usage count:} 1

\section*{Lemma 446}
\begin{lemmatheorembox}
\textbf{Name:} De Bruijn sequence
\textbf{Statement:} The de Bruijn sequences can be constructed by taking a Hamiltonian path of an n-dimensional de Bruijn graph over k symbols (or equivalently, an Eulerian cycle of an (n − 1)-dimensional de Bruijn graph).
\textbf{General Topic:} Combinatorics
\textbf{URL:} https://en.wikipedia.org/wiki/De_Bruijn_sequence
\end{lemmatheorembox}

\textbf{Usage count:} 1

\section*{Lemma 447}
\begin{lemmatheorembox}
\textbf{Name:} British flag theorem
\textbf{Statement:} In Euclidean geometry, the British flag theorem says that if a point P is chosen inside a rectangle ABCD then the sum of the squares of the Euclidean distances from P to two opposite corners of the rectangle equals the sum to the other two opposite corners. As an equation: $AP^2 + CP^2 = BP^2 + DP^2$.
\textbf{General Topic:} Euclidean geometry
\textbf{URL:} https://en.wikipedia.org/wiki/British_flag_theorem
\end{lemmatheorembox}

\textbf{Usage count:} 1

\section*{Lemma 448}
\begin{lemmatheorembox}
\textbf{Name:} Fundamental theorem of symmetric polynomials
\textbf{Statement:} For any commutative ring A, denote the ring of symmetric polynomials in the variables X _{1}, ..., X n with coefficients in A by A[X _{1}, ..., X n]S_{n}. This is a polynomial ring in the n elementary symmetric polynomials e_{k}(X _{1}, ..., X n) for k = 1, ..., n.

This means that every symmetric polynomial P(X _{1}, ..., X_{n}) ∈ A[X _{1}, ..., X n]S_{n} has a unique representation
P(X_{1},\ldots ,X_{n})=Q{\big (}e_{1}(X_{1},\ldots ,X_{n}),\ldots ,e_{n}(X_{1},\ldots ,X_{n}){\big )}
for some polynomial Q ∈ A[Y _{1}, ..., Y_{n}]. Another way of saying the same thing is that the ring homomorphism that sends Y_{k} to e_{k}(X _{1}, ..., X_{n}) for k = 1, ..., n defines an isomorphism between A[Y _{1}, ..., Y_{n}] and A[X _{1}, ..., X n]S_{n}.
\textbf{General Topic:} Commutative algebra
\textbf{URL:} https://en.wikipedia.org/wiki/Elementary_symmetric_polynomial#Fundamental_theorem_of_symmetric_polynomials
\end{lemmatheorembox}

\textbf{Usage count:} 1

\section*{Lemma 449}
\begin{lemmatheorembox}
\textbf{Name:} Dual polyhedron
\textbf{Statement:} In geometry, every polyhedron is associated with a second dual structure, wherein the vertices of one correspond to the faces of the other and the edges between pairs of vertices of one correspond to the edges between pairs of faces of the other.
\textbf{General Topic:} Geometry
\textbf{URL:} https://en.wikipedia.org/wiki/Dual_polyhedron
\end{lemmatheorembox}

\textbf{Usage count:} 1

\section*{Lemma 450}
\begin{lemmatheorembox}
\textbf{Name:} Regular tetrahedron
\textbf{Statement:} The vertices of a cube can be grouped into two groups of four, each forming a regular tetrahedron, showing one of the two tetrahedra in the cube.
\textbf{General Topic:} Geometry
\textbf{URL:} https://en.wikipedia.org/wiki/Regular_tetrahedron
\end{lemmatheorembox}

\textbf{Usage count:} 1

\section*{Lemma 451}
\begin{lemmatheorembox}
\textbf{Name:} k-core
\textbf{Statement:} A k-core of a graph G is a maximal connected subgraph of G in which all vertices have degree at least k. Equivalently, it is one of the connected components of the subgraph of G formed by repeatedly deleting all vertices of degree less than k.
\textbf{General Topic:} Graph theory
\textbf{URL:} https://en.wikipedia.org/wiki/Degeneracy_(graph_theory)
\end{lemmatheorembox}

\textbf{Usage count:} 1

\section*{Lemma 452}
\begin{lemmatheorembox}
\textbf{Name:} The Weyl group acts freely and transitively on the Weyl chambers
\textbf{Statement:} Theorem: The Weyl group acts freely and transitively on the Weyl chambers. Thus, the order of the Weyl group is equal to the number of Weyl chambers.
\textbf{General Topic:} Lie theory
\textbf{URL:} https://en.wikipedia.org/wiki/Root_system
\end{lemmatheorembox}

\textbf{Usage count:} 1

\section*{Lemma 453}
\begin{lemmatheorembox}
\textbf{Name:} Parallelogram law
\textbf{Statement:} It states that the sum of the squares of the lengths of the four sides of a parallelogram equals the sum of the squares of the lengths of the two diagonals.
\textbf{General Topic:} Elementary geometry
\textbf{URL:} https://en.wikipedia.org/wiki/Parallelogram_law
\end{lemmatheorembox}

\textbf{Usage count:} 1

\section*{Lemma 454}
\begin{lemmatheorembox}
\textbf{Name:} High school exterior angle theorem
\textbf{Statement:} The high school exterior angle theorem (HSEAT) says that the size of an exterior angle at a vertex of a triangle equals the sum of the sizes of the interior angles at the other two vertices of the triangle (remote interior angles).
\textbf{General Topic:} Euclidean geometry
\textbf{URL:} https://en.wikipedia.org/wiki/Exterior_angle_theorem#High_school_exterior_angle_theorem
\end{lemmatheorembox}

\textbf{Usage count:} 1

\section*{Lemma 455}
\begin{lemmatheorembox}
\textbf{Name:} Tonelli's theorem
\textbf{Statement:} Tonelli's theorem states that if (X, A, μ) and (Y, B, ν) are σ-finite measure spaces, while f:X\times Y\to [0,\infty ] is a non-negative measurable function, then
\[
\int _{X}\left(\int _{Y}f(x,y)\,\mathrm {d} y\right)\,\mathrm {d} x=\int _{Y}\left(\int _{X}f(x,y)\,\mathrm {d} x\right)\,\mathrm {d} y=\int _{X\times Y}f(x,y)\,\mathrm {d} (x,y).
\]
\textbf{General Topic:} Measure theory
\textbf{URL:} https://en.wikipedia.org/wiki/Fubini%27s_theorem#Tonelli's_theorem_for_non-negative_measurable_functions
\end{lemmatheorembox}

\textbf{Usage count:} 1

\section*{Lemma 456}
\begin{lemmatheorembox}
\textbf{Name:} Measure (mathematics)
\textbf{Statement:} Countable additivity (or σ-additivity): For all countable collections $\{E_{k}\}_{k=1}^{\infty}$ of pairwise disjoint sets in $\Sigma,\mu{\left(\bigcup _{k=1}^{\infty }E_{k}\right)}=\sum _{k=1}^{\infty }\mu (E_{k})$
\textbf{General Topic:} Measure theory
\textbf{URL:} https://en.wikipedia.org/wiki/Measure_(mathematics)
\end{lemmatheorembox}

\textbf{Usage count:} 1

\section*{Lemma 457}
\begin{lemmatheorembox}
\textbf{Name:} Barycentric coordinate system
\textbf{Statement:} Let $P$ be a point in the plane, and let $(\lambda _{1},\lambda _{2},\lambda _{3})$ be its normalized barycentric coordinates with respect to the triangle $ABC$, so $P=\lambda _{1}A+\lambda _{2}B+\lambda _{3}C$ and $1=\lambda _{1}+\lambda _{2}+\lambda _{3}$.
\textbf{General Topic:} Geometry
\textbf{URL:} https://en.wikipedia.org/wiki/Barycentric_coordinate_system
\end{lemmatheorembox}

\textbf{Usage count:} 1

\section*{Lemma 458}
\begin{lemmatheorembox}
\textbf{Name:} Affine combination
\textbf{Statement:} The affine combinations commute with any affine transformation $T$ in the sense that $T\sum _{i=1}^{n}{\alpha _{i}\cdot x_{i}}=\sum _{i=1}^{n}{\alpha _{i}\cdot Tx_{i}}$.
\textbf{General Topic:} Affine geometry
\textbf{URL:} https://en.wikipedia.org/wiki/Affine_combination
\end{lemmatheorembox}

\textbf{Usage count:} 1

\section*{Lemma 459}
\begin{lemmatheorembox}
\textbf{Name:} Polar coordinate system
\textbf{Statement:} Let R denote the region enclosed by a curve r(φ) and the rays φ = a and φ = b, where 0 < b − a ≤ 2 π. Then, the area of R is \({\frac {1}{2}}\int _{a}^{b}\left[r(\varphi )\right]^{2}\,d\varphi .\)
\textbf{General Topic:} Calculus
\textbf{URL:} https://en.wikipedia.org/wiki/Polar_coordinate_system
\end{lemmatheorembox}

\textbf{Usage count:} 1

\section*{Lemma 460}
\begin{lemmatheorembox}
\textbf{Name:} Integral of the secant function
\textbf{Statement:} \(\int \sec \theta \, d\theta = \begin{cases} \dfrac12 \ln \dfrac{1+\sin\theta}{1-\sin\theta} + C \\[15mu] \ln{\bigl|\sec\theta + \tan\theta\,\bigr|} + C \\[15mu] \ln{\left|\,{\tan}\biggl(\dfrac\theta2 + \dfrac\pi4\biggr) \right|} + C \end{cases}\)
\textbf{General Topic:} Calculus
\textbf{URL:} https://en.wikipedia.org/wiki/Integral_of_the_secant_function
\end{lemmatheorembox}

\textbf{Usage count:} 1

\section*{Lemma 461}
\begin{lemmatheorembox}
\textbf{Name:} Similarity to the unit parabola
\textbf{Statement:} Thus, any parabola can be mapped to the unit parabola by a similarity.
\textbf{General Topic:} Analytic geometry
\textbf{URL:} https://en.wikipedia.org/wiki/Parabola#Similarity_to_the_unit_parabola
\end{lemmatheorembox}

\textbf{Usage count:} 1

\section*{Lemma 462}
\begin{lemmatheorembox}
\textbf{Name:} Parabola
\textbf{Statement:} The focus  is \left(0, \frac{1}{4a}\right),
\textbf{General Topic:} Analytic geometry
\textbf{URL:} https://en.wikipedia.org/wiki/Parabola#As_a_graph_of_a_function
\end{lemmatheorembox}

\textbf{Usage count:} 1

\section*{Lemma 463}
\begin{lemmatheorembox}
\textbf{Name:} 1 + 2 + 4 + 8 + \ensuremath{\cdots}
\textbf{Statement:} The partial sum of the first \ensuremath{n} terms of \ensuremath{1+2+4+8+\cdots} is \ensuremath{\sum _{k=0}^{n-1}2^{k}=2^{0}+2^{1}+\cdots +2^{n-1}=2^{n}-1.}
\textbf{General Topic:} Series
\textbf{URL:} https://en.wikipedia.org/wiki/1_%2B_2_%2B_4_%2B_8_%2B_%E2%8B%AF
\end{lemmatheorembox}

\textbf{Usage count:} 1

\section*{Lemma 464}
\begin{lemmatheorembox}
\textbf{Name:} Divergent geometric series
\textbf{Statement:} Ordinary summation succeeds only for common ratios $|r|<1.$
\textbf{General Topic:} Infinite series
\textbf{URL:} https://en.wikipedia.org/wiki/Divergent_geometric_series
\end{lemmatheorembox}

\textbf{Usage count:} 1

\section*{Lemma 465}
\begin{lemmatheorembox}
\textbf{Name:} Binet's formula
\textbf{Statement:} The Binet formula is \(\sqrt {5}F_{n}=\varphi ^{n}-\psi ^{n}.\)
\textbf{General Topic:} Integer sequences
\textbf{URL:} https://en.wikipedia.org/wiki/Fibonacci_sequence#Binet_formula_proofs
\end{lemmatheorembox}

\textbf{Usage count:} 1

\section*{Lemma 466}
\begin{lemmatheorembox}
\textbf{Name:} Abel's irreducibility theorem
\textbf{Statement:} Equivalently, if f(x) shares at least one root with g(x) then f is divisible evenly by g(x)
\textbf{General Topic:} Field theory
\textbf{URL:} https://en.wikipedia.org/wiki/Abel%27s_irreducibility_theorem
\end{lemmatheorembox}

\textbf{Usage count:} 1

\section*{Lemma 467}
\begin{lemmatheorembox}
\textbf{Name:} Zonogon
\textbf{Statement:} In geometry, a zonogon is a centrally-symmetric, convex polygon. Equivalently, it is a convex polygon whose sides can be grouped into parallel pairs with equal lengths and opposite orientations, the two-dimensional analog of a zonohedron.
\textbf{General Topic:} Geometry
\textbf{URL:} https://en.wikipedia.org/wiki/Zonogon
\end{lemmatheorembox}

\textbf{Usage count:} 1

\section*{Lemma 468}
\begin{lemmatheorembox}
\textbf{Name:} Polygon
\textbf{Statement:} The sum of the interior angles of a simple n-gon is (n − 2) × π radians or (n − 2) × 180 degrees.
\textbf{General Topic:} Geometry
\textbf{URL:} https://en.wikipedia.org/wiki/Polygon
\end{lemmatheorembox}

\textbf{Usage count:} 1

\section*{Lemma 469}
\begin{lemmatheorembox}
\textbf{Name:} Convex polygon
\textbf{Statement:} Every internal angle is strictly less than 180 degrees.
\textbf{General Topic:} Geometry
\textbf{URL:} https://en.wikipedia.org/wiki/Convex_polygon
\end{lemmatheorembox}

\textbf{Usage count:} 1

\section*{Lemma 470}
\begin{lemmatheorembox}
\textbf{Name:} Regular pentagon
\textbf{Statement:} A regular pentagon has Schläfli symbol {5} and interior angles of 108°.
\textbf{General Topic:} Geometry
\textbf{URL:} https://en.wikipedia.org/wiki/Pentagon
\end{lemmatheorembox}

\textbf{Usage count:} 1

\section*{Lemma 471}
\begin{lemmatheorembox}
\textbf{Name:} Geometric distribution
\textbf{Statement:} The expected value and variance of a geometrically distributed random variable $X$ defined over $\mathbb{N}$ is $\operatorname{E}(X)={\frac {1}{p}},\qquad \operatorname{var}(X)={\frac {1-p}{p^{2}}}.$
\textbf{General Topic:} Probability theory
\textbf{URL:} https://en.wikipedia.org/wiki/Geometric_distribution
\end{lemmatheorembox}

\textbf{Usage count:} 1

\section*{Lemma 472}
\begin{lemmatheorembox}
\textbf{Name:} Binomial coefficient
\textbf{Statement:} In combinatorics the symbol $\tbinom{n}{k}$ is usually read as "n choose k" because there are $\tbinom{n}{k}$ ways to choose an (unordered) subset of $k$ elements from a fixed set of $n$ elements.
\textbf{General Topic:} Combinatorics
\textbf{URL:} https://en.wikipedia.org/wiki/Binomial_coefficient ([en.wikipedia.org](https://en.wikipedia.org/wiki/Binomial_coefficient))
\end{lemmatheorembox}

\textbf{Usage count:} 1

\section*{Lemma 473}
\begin{lemmatheorembox}
\textbf{Name:} Rule of product
\textbf{Statement:} Stated simply, it is the intuitive idea that if there are a ways of doing something and b ways of doing another thing, then there are a · b ways of performing both actions.
\textbf{General Topic:} Combinatorics
\textbf{URL:} https://en.wikipedia.org/wiki/Rule_of_product ([en.wikipedia.org](https://en.wikipedia.org/wiki/Rule_of_product))
\end{lemmatheorembox}

\textbf{Usage count:} 1

\section*{Lemma 474}
\begin{lemmatheorembox}
\textbf{Name:} Cayley's formula
\textbf{Statement:} In mathematics, Cayley's formula is a result in graph theory named after Arthur Cayley. It states that for every positive integer $n$, the number of trees on $n$ labeled vertices is $n^{n-2}$.
\textbf{General Topic:} Graph Theory
\textbf{URL:} https://en.wikipedia.org/wiki/Cayley%27s_formula
\end{lemmatheorembox}

\textbf{Usage count:} 1

\section*{Lemma 475}
\begin{lemmatheorembox}
\textbf{Name:} Spherical segment
\textbf{Statement:} The curved surface area of the spherical zone—which excludes the top and bottom bases—is given by $A=2\pi Rh.$ Thus the surface area of the segment depends only on the distance between the cutting planes, and not their absolute heights.
\textbf{General Topic:} Geometry
\textbf{URL:} https://en.wikipedia.org/wiki/Spherical_segment
\end{lemmatheorembox}

\textbf{Usage count:} 1

\section*{Lemma 476}
\begin{lemmatheorembox}
\textbf{Name:} Palindromic number
\textbf{Statement:} A palindromic number (also known as a numeral palindrome or a numeric palindrome) is a number (such as 16361) that remains the same when its digits are reversed.
\textbf{General Topic:} Number theory
\textbf{URL:} https://en.wikipedia.org/wiki/Palindromic_number
\end{lemmatheorembox}

\textbf{Usage count:} 1

\section*{Lemma 477}
\begin{lemmatheorembox}
\textbf{Name:} Equidistribution theorem
\textbf{Statement:} In mathematics, the equidistribution theorem is the statement that the sequence
\[
a, 2 a, 3 a, \ldots \bmod 1
\]
is uniformly distributed on the circle $\mathbb{R}/\mathbb{Z}$, when $a$ is an irrational number.
\textbf{General Topic:} Ergodic theory
\textbf{URL:} https://en.wikipedia.org/wiki/Equidistribution_theorem
\end{lemmatheorembox}

\textbf{Usage count:} 1

\section*{Lemma 478}
\begin{lemmatheorembox}
\textbf{Name:} Partition function (number theory)

\textbf{Statement:} In number theory, the partition function p(n) represents the number of possible partitions of a non-negative integer n. The first few values of the partition function, starting with p(0) = 1, are 1, 1, 2, 3, 5, 7, 11, 15, 22, 30, 42, 56, 77, 101, 135, 176, 231, 297, 385, 490, 627, 792, 1002, 1255, 1575, 1958, 2436, 3010, 3718, 4565, 5604, ... (sequence A000041 in the OEIS).

\textbf{General Topic:} Number theory

\textbf{URL:} https://en.wikipedia.org/wiki/Partition_function_%28number_theory%29
\end{lemmatheorembox}

\textbf{Usage count:} 1

\section*{Lemma 479}
\begin{lemmatheorembox}
\textbf{Name:} Subgroup test
\textbf{Statement:} The subgroup test provides a necessary and sufficient condition for a nonempty subset $H$ of a group $G$ to be a subgroup: it is sufficient to check that $g^{-1}\cdot h\in H$ for all elements $g$ and $h$ in $H$.
\textbf{General Topic:} Group theory
\textbf{URL:} https://en.wikipedia.org/wiki/Group_%28mathematics%29#Subgroups
\end{lemmatheorembox}

\textbf{Usage count:} 1

\section*{Lemma 480}
\begin{lemmatheorembox}
\textbf{Name:} Euler line
\textbf{Statement:} On the Euler line the centroid G is between the circumcenter O and the orthocenter H and is twice as far from the orthocenter as it is from the circumcenter:

GH=2GO;
OH=3GO.
\textbf{General Topic:} Euclidean geometry
\textbf{URL:} https://en.wikipedia.org/wiki/Euler_line
\end{lemmatheorembox}

\textbf{Usage count:} 1

\section*{Lemma 481}
\begin{lemmatheorembox}
\textbf{Name:} Completing the square
\textbf{Statement:} ax^2 + bx + c = a(x-h)^2 + k, with h = -\frac{b}{2a} \quad\text{and}\quad k = c - ah^2 = c - \frac{b^2}{4a}.
\textbf{General Topic:} Algebra
\textbf{URL:} https://en.wikipedia.org/wiki/Completing_the_square
\end{lemmatheorembox}

\textbf{Usage count:} 1

\section*{Lemma 482}
\begin{lemmatheorembox}
\textbf{Name:} Domino (mathematics)
\textbf{Statement:} The number of tilings of a 2×n rectangle with dominoes is $F_{n}$, the n th Fibonacci number.
\textbf{General Topic:} Combinatorics
\textbf{URL:} https://en.wikipedia.org/wiki/Domino_(mathematics)
\end{lemmatheorembox}

\textbf{Usage count:} 1

\section*{Lemma 483}
\begin{lemmatheorembox}
\textbf{Name:} Minimal polynomial (field theory)  
\textbf{Statement:} The minimal polynomial f of α generates the ideal J\_α, i.e. every  g in J\_α can be factorized as g=fh for some h'  in F[x].  
\textbf{General Topic:} Field Theory  
\textbf{URL:} https://en.wikipedia.org/wiki/Minimal_polynomial_%28field_theory%29
\end{lemmatheorembox}

\textbf{Usage count:} 1

\section*{Lemma 484}
\begin{lemmatheorembox}
\textbf{Name:} Base (geometry)
\textbf{Statement:} The area of a triangle is its half of the product of the base times the height (length of the altitude).
\textbf{General Topic:} Geometry
\textbf{URL:} https://en.wikipedia.org/wiki/Base_(geometry)
\end{lemmatheorembox}

\textbf{Usage count:} 1

\section*{Lemma 485}
\begin{lemmatheorembox}
\textbf{Name:} Hook length formula
\textbf{Statement:} The hook length formula expresses the number of standard Young tableaux of shape $\lambda$, denoted by $f^{\lambda}$ or $d_{\lambda}$, as
\[
f^{\lambda}={\frac {n!}{\prod h_{\lambda }(i,j)}},
\]
where the product is over all cells $(i,j)$ of the Young diagram.
\textbf{General Topic:} Combinatorics
\textbf{URL:} https://en.wikipedia.org/wiki/Hook_length_formula
\end{lemmatheorembox}

\textbf{Usage count:} 1

\section*{Lemma 486}
\begin{lemmatheorembox}
\textbf{Name:} Newton's theorem (quadrilateral)
\textbf{Statement:} In Euclidean geometry Newton's theorem states that in every tangential quadrilateral other than a rhombus, the center of the incircle lies on the Newton line.
\textbf{General Topic:} Euclidean geometry
\textbf{URL:} https://en.wikipedia.org/wiki/Newton%27s_theorem_(quadrilateral)
\end{lemmatheorembox}

\textbf{Usage count:} 1

\section*{Lemma 487}
\begin{lemmatheorembox}
\textbf{Name:} Line–line intersection
\textbf{Statement:} In Euclidean geometry, the intersection of a line and a line can be the empty set, a single point, or a line (if they coincide).
\textbf{General Topic:} Euclidean geometry
\textbf{URL:} https://en.wikipedia.org/wiki/Line%E2%80%93line_intersection
\end{lemmatheorembox}

\textbf{Usage count:} 1

\section*{Lemma 488}
\begin{lemmatheorembox}
\textbf{Name:} Finite vector space
\textbf{Statement:} Then any n-dimensional vector space V over F q will have q n elements.
\textbf{General Topic:} Linear algebra
\textbf{URL:} https://en.wikipedia.org/wiki/Finite_vector_space
\end{lemmatheorembox}

\textbf{Usage count:} 1

\section*{Lemma 489}
\begin{lemmatheorembox}
\textbf{Name:} Relative velocity
\textbf{Statement:} Hence: \mathbf{v}_{B\mid A}=\mathbf{v}_B-\mathbf{v}_A.
\textbf{General Topic:} Classical mechanics
\textbf{URL:} https://en.wikipedia.org/wiki/Relative_velocity
\end{lemmatheorembox}

\textbf{Usage count:} 1

\section*{Lemma 490}
\begin{lemmatheorembox}
\textbf{Name:} Sample maximum and minimum
\textbf{Statement:} The sample maximum and minimum provide a non-parametric prediction interval: in a sample from a population, or more generally an exchangeable sequence of random variables, each observation is equally likely to be the maximum or minimum.
\textbf{General Topic:} Statistics
\textbf{URL:} https://en.wikipedia.org/wiki/Sample_maximum_and_minimum
\end{lemmatheorembox}

\textbf{Usage count:} 1

\section*{Lemma 491}
\begin{lemmatheorembox}
\textbf{Name:} Symédiane
\textbf{Statement:} Pour un triangle ABC, soit D l'intersection des tangentes au cercle circonscrit en B et C. Alors (AD) est la symédiane en A de ABC[ 7 ].
\textbf{General Topic:} Geometry
\textbf{URL:} https://fr.wikipedia.org/wiki/Sym%C3%A9diane
\end{lemmatheorembox}

\textbf{Usage count:} 1

\section*{Lemma 492}
\begin{lemmatheorembox}
\textbf{Name:} Симедиана
\textbf{Statement:} Отрезки, на которые симедиана делит противоположную сторону, пропорциональны квадратам прилежащих сторон.
\textbf{General Topic:} Geometry
\textbf{URL:} https://ru.wikipedia.org/wiki/%D0%A1%D0%B8%D0%BC%D0%B5%D0%B4%D0%B8%D0%B0%D0%BD%D0%B0
\end{lemmatheorembox}

\textbf{Usage count:} 1

\section*{Lemma 493}
\begin{lemmatheorembox}
\textbf{Name:} Standard deviation
\textbf{Statement:} The standard deviation of a random variable, sample, statistical population, data set, or probability distribution is the square root of its variance.
\textbf{General Topic:} Statistics
\textbf{URL:} https://en.wikipedia.org/wiki/Standard_deviation
\end{lemmatheorembox}

\textbf{Usage count:} 1

\section*{Lemma 494}
\begin{lemmatheorembox}
\textbf{Name:} Irwin--Hall distribution
\textbf{Statement:} For n = 3, \[f_X(x)= \begin{cases} \frac{1}{2}x^2 & 0\le x \le 1\\ \frac{1}{2}(-2x^2 + 6x - 3)& 1\le x \le 2\\ \frac{1}{2}(3 - x)^2 & 2\le x \le 3 \end{cases}\]
\textbf{General Topic:} Probability theory
\textbf{URL:} https://en.wikipedia.org/wiki/Irwin%E2%80%93Hall_distribution
\end{lemmatheorembox}

\textbf{Usage count:} 1

\section*{Lemma 495}
\begin{lemmatheorembox}
\textbf{Name:} Linear Diophantine equations
\textbf{Statement:} This Diophantine equation has a solution (where x and y are integers) if and only if c is a multiple of the greatest common divisor of a and b. Moreover, if (x, y) is a solution, then the other solutions have the form (x + kv, y − ku), where k is an arbitrary integer, and u and v are the quotients of a and b (respectively) by the greatest common divisor of a and b.
\textbf{General Topic:} Number theory
\textbf{URL:} https://en.wikipedia.org/wiki/Diophantine_equation
\end{lemmatheorembox}

\textbf{Usage count:} 1

\section*{Lemma 496}
\begin{lemmatheorembox}
\textbf{Name:} Table of divisors
\textbf{Statement:} A divisor of an integer n is an integer m, for which n/m is again an integer (which is necessarily also a divisor of n).
\textbf{General Topic:} Number Theory
\textbf{URL:} https://en.wikipedia.org/wiki/Table_of_divisors
\end{lemmatheorembox}

\textbf{Usage count:} 1

\section*{Lemma 497}
\begin{lemmatheorembox}
\textbf{Name:} Exact trigonometric values
\textbf{Statement:} $\sin\left(\frac{\pi}{12}\right)=\frac{{\sqrt {2}}({\sqrt {3}}-1)}{4}$ and $\cos\left(\frac{\pi}{12}\right)=\frac{{\sqrt {2}}({\sqrt {3}}+1)}{4}$
\textbf{General Topic:} Trigonometry
\textbf{URL:} https://en.wikipedia.org/wiki/Exact_trigonometric_values
\end{lemmatheorembox}

\textbf{Usage count:} 1

\section*{Lemma 498}
\begin{lemmatheorembox}
\textbf{Name:} Empty set
\textbf{Statement:} The empty set is a subset of every set.
\textbf{General Topic:} Set Theory
\textbf{URL:} https://en.wikipedia.org/wiki/Empty_set
\end{lemmatheorembox}

\textbf{Usage count:} 1

\section*{Lemma 499}
\begin{lemmatheorembox}
\textbf{Name:} Change of base formula
\textbf{Statement:} \[
\forall a,b\in \mathbb {R} _{+},a,b\neq 1,\forall x\in \mathbb {R} _{+},\log _{b}(x)={\frac {\log _{a}(x)}{\log _{a}(b)}}.
\]
\textbf{General Topic:} Logarithms
\textbf{URL:} https://en.wikipedia.org/wiki/List_of_logarithmic_identities#Changing_the_base
\end{lemmatheorembox}

\textbf{Usage count:} 1

\section*{Lemma 500}
\begin{lemmatheorembox}
\textbf{Name:} Doubly stochastic matrix
\textbf{Statement:} The stationary distribution of an irreducible aperiodic finite Markov chain is uniform if and only if its transition matrix is doubly stochastic.
\textbf{General Topic:} Linear algebra / Markov chains
\textbf{URL:} https://en.wikipedia.org/wiki/Doubly_stochastic_matrix
\end{lemmatheorembox}

\textbf{Usage count:} 1

\section*{Lemma 501}
\begin{lemmatheorembox}
\textbf{Name:} Carnot's theorem (inradius, circumradius)
\textbf{Statement:} In Euclidean geometry, Carnot's theorem states that the sum of the signed distances from the circumcenter D to the sides of an arbitrary triangle ABC is \(DF + DG + DH = R + r,\) where r is the inradius and R is the circumradius of the triangle.
\textbf{General Topic:} Euclidean geometry
\textbf{URL:} https://en.wikipedia.org/wiki/Carnot%27s_theorem_%28inradius%2C_circumradius%29
\end{lemmatheorembox}

\textbf{Usage count:} 1

\section*{Lemma 502}
\begin{lemmatheorembox}
\textbf{Name:} Partial fraction decomposition
\textbf{Statement:} In algebra, the partial fraction decomposition or partial fraction expansion of a rational fraction (that is, a fraction such that the numerator and the denominator are both polynomials) is an operation that consists of expressing the fraction as a sum of a polynomial (possibly zero) and one or several fractions with a simpler denominator.
\textbf{General Topic:} Algebra
\textbf{URL:} https://en.wikipedia.org/wiki/Partial_fraction_decomposition
\end{lemmatheorembox}

\textbf{Usage count:} 1

\section*{Lemma 503}
\begin{lemmatheorembox}
\textbf{Name:} Polynomial ring
\textbf{Statement:} Two polynomials are equal when the corresponding coefficients of each X k are equal.
\textbf{General Topic:} Algebra
\textbf{URL:} https://en.wikipedia.org/wiki/Polynomial_ring
\end{lemmatheorembox}

\textbf{Usage count:} 1

\section*{Lemma 504}
\begin{lemmatheorembox}
\textbf{Name:} Projected area
\textbf{Statement:} Flat rectangle $A=L\times W$ \quad $A_{\text{proj}}=L\times W\cos {\beta }$.
\textbf{General Topic:} Geometry
\textbf{URL:} https://en.wikipedia.org/wiki/Projected_area
\end{lemmatheorembox}

\textbf{Usage count:} 1

\section*{Lemma 505}
\begin{lemmatheorembox}
\textbf{Name:} 30°–60°–90° triangle
\textbf{Statement:} A useful property of such triangles is that their side lengths are in the ratio 1 : √ 3 : 2.
\textbf{General Topic:} Euclidean geometry
\textbf{URL:} https://en.wikipedia.org/wiki/Special_right_triangle#30%C2%B0%E2%80%9360%C2%B0%E2%80%9390%C2%B0_triangle
\end{lemmatheorembox}

\textbf{Usage count:} 1

\section*{Lemma 506}
\begin{lemmatheorembox}
\textbf{Name:} Direct product of groups
\textbf{Statement:} The order of each element (g, h) is the least common multiple of the orders of g and h: $|(g, h)| = \mathrm{lcm}(|g|, |h|).$
\textbf{General Topic:} Group theory
\textbf{URL:} https://en.wikipedia.org/wiki/Direct_product_of_groups
\end{lemmatheorembox}

\textbf{Usage count:} 1

\section*{Lemma 507}
\begin{lemmatheorembox}
\textbf{Name:} Harary's theorem
\textbf{Statement:} Harary proves that a signed graph is balanced when (1) for every pair of nodes, all paths between them have the same sign, or (2) the vertices partition into a pair of subsets (possibly empty), each containing only positive edges, but connected by negative edges.
\textbf{General Topic:} Graph theory
\textbf{URL:} https://en.wikipedia.org/wiki/Signed_graph
\end{lemmatheorembox}

\textbf{Usage count:} 1

\section*{Lemma 508}
\begin{lemmatheorembox}
\textbf{Name:} Inradius
\textbf{Statement:} The inradius of a triangle is the radius of its incircle; it can be found from the formula $r=\frac{A}{s}$, where $A$ is the area of the triangle and $s$ is its semiperimeter.
\textbf{General Topic:} Euclidean geometry
\textbf{URL:} https://en.wikipedia.org/wiki/Inradius
\end{lemmatheorembox}

\textbf{Usage count:} 1

\section*{Lemma 509}
\begin{lemmatheorembox}
\textbf{Name:} Complete sequence
\textbf{Statement:} Without loss of generality, assume the sequence $a_n$ is in non-decreasing order, and define the partial sums of $a_n$ as: $s_n=\sum_{m=0}^n a_m$. Then the conditions $a_0 = 1$ and $s_{k-1}\ge a_k - 1 \text{ for all } k \ge 1$ are both necessary and sufficient for $a_n$ to be a complete sequence.
\textbf{General Topic:} Number Theory
\textbf{URL:} https://en.wikipedia.org/wiki/Complete_sequence
\end{lemmatheorembox}

\textbf{Usage count:} 1

\section*{Lemma 510}
\begin{lemmatheorembox}
\textbf{Name:} Internal and external angles
\textbf{Statement:} The sum of all the internal angles of a simple polygon is π(n − 2) radians or 180(n − 2) degrees, where n is the number of sides.
\textbf{General Topic:} Euclidean plane geometry
\textbf{URL:} https://en.wikipedia.org/wiki/Internal_and_external_angles
\end{lemmatheorembox}

\textbf{Usage count:} 1

\section*{Lemma 511}
\begin{lemmatheorembox}
\textbf{Name:} Rose (mathematics)
\textbf{Statement:} When \(k\) is a non-zero integer, the curve will be rose-shaped with \(2k\) petals if \(k\) is even, and \(k\) petals when \(k\) is odd.
\textbf{General Topic:} Plane curves
\textbf{URL:} https://en.wikipedia.org/wiki/Rhodonea_curve
\end{lemmatheorembox}

\textbf{Usage count:} 1

\section*{Lemma 512}
\begin{lemmatheorembox}
\textbf{Name:} Fermat's theorem on sums of two squares
\textbf{Statement:} In additive number theory, Fermat's theorem on sums of two squares states that an odd prime p can be expressed as: p = x^{2} + y^{2}, with x and y integers, if and only if p\equiv 1{\pmod {4}}.
\textbf{General Topic:} Number theory
\textbf{URL:} https://en.wikipedia.org/wiki/Fermat%27s_theorem_on_sums_of_two_squares
\end{lemmatheorembox}

\textbf{Usage count:} 1

\section*{Lemma 513}
\begin{lemmatheorembox}
\textbf{Name:} Tangent half-angle formula
\textbf{Statement:} In the unit circle, application of the above shows that $t=\tan {\tfrac {1}{2}}\varphi$. By similarity of triangles,
\[\frac {t}{\sin \varphi }=\frac {1}{1+\cos \varphi }.\]
It follows that
\[t=\frac {\sin \varphi }{1+\cos \varphi }=\frac {\sin \varphi (1-\cos \varphi )}{(1+\cos \varphi )(1-\cos \varphi )}=\frac {1-\cos \varphi }{\sin \varphi }.\]
\textbf{General Topic:} Trigonometry
\textbf{URL:} https://en.wikipedia.org/wiki/Tangent_half-angle_formula
\end{lemmatheorembox}

\textbf{Usage count:} 1

\section*{Lemma 514}
\begin{lemmatheorembox}
\textbf{Name:} Sum of two squares theorem

\textbf{Statement:} An integer greater than one can be written as a sum of two squares if and only if its prime decomposition contains no factor p k, where prime $p\equiv 3{\pmod {4}}$ and k is odd. ([en.wikipedia.org](https://en.wikipedia.org/wiki/Sum_of_two_squares_theorem))

\textbf{General Topic:} Number theory

\textbf{URL:} https://en.wikipedia.org/wiki/Sum_of_two_squares_theorem ([en.wikipedia.org](https://en.wikipedia.org/wiki/Sum_of_two_squares_theorem))
\end{lemmatheorembox}

\textbf{Usage count:} 1

\section*{Lemma 515}
\begin{lemmatheorembox}
\textbf{Name:} 30°–60°–90° triangle
\textbf{Statement:} A useful property of such triangles is that their side lengths are in the ratio 1 : √ 3 : 2.
\textbf{General Topic:} Geometry
\textbf{URL:} https://en.wikipedia.org/wiki/Special_right_triangle\#30%C2%B0%E2%80%9360%C2%B0%E2%80%9390%C2%B0_triangle
\end{lemmatheorembox}

\textbf{Usage count:} 1

\section*{Lemma 516}
\begin{lemmatheorembox}
\textbf{Name:} Half-angle formulae
\textbf{Statement:} $\tan \frac{\theta}{2} = \frac{1 - \cos \theta}{\sin \theta} = \frac{\sin \theta}{1 + \cos \theta} = \csc \theta - \cot \theta = \frac{\tan\theta}{1 + \sec{\theta}}$
\textbf{General Topic:} Trigonometry
\textbf{URL:} https://en.wikipedia.org/wiki/List_of_trigonometric_identities#Half-angle_formulae
\end{lemmatheorembox}

\textbf{Usage count:} 1

\section*{Lemma 517}
\begin{lemmatheorembox}
\textbf{Name:} Feuerbach's theorem
\textbf{Statement:} In 1822 Karl Feuerbach discovered that any triangle's nine-point circle is externally tangent to that triangle's three excircles and internally tangent to its incircle; this result is known as Feuerbach's theorem.
\textbf{General Topic:} Euclidean geometry
\textbf{URL:} https://en.wikipedia.org/wiki/Nine-point_circle
\end{lemmatheorembox}

\textbf{Usage count:} 1

\section*{Lemma 518}
\begin{lemmatheorembox}
\textbf{Name:} Émile Lemoine
\textbf{Statement:} Other results in the paper included the idea that the symmedian from a vertex of the triangle divides the opposite side into segments whose ratio is equal to the ratio of the squares of the other two sides.
\textbf{General Topic:} Geometry
\textbf{URL:} https://en.wikipedia.org/wiki/%C3%89mile_Lemoine
\end{lemmatheorembox}

\textbf{Usage count:} 1

\section*{Lemma 519}
\begin{lemmatheorembox}
\textbf{Name:} Polynomial long division
\textbf{Statement:} A = BQ + R, and either R = 0 or the degree of R is lower than the degree of B.
\textbf{General Topic:} Algebra
\textbf{URL:} https://en.wikipedia.org/wiki/Polynomial_long_division
\end{lemmatheorembox}

\textbf{Usage count:} 1

\section*{Lemma 520}
\begin{lemmatheorembox}
\textbf{Name:} Induction principle
\textbf{Statement:} ${\mathsf {I}}\varphi :\quad {\big [}\varphi (0)\land \forall x{\big (}\varphi (x)\to \varphi (x+1){\big )}{\big ]}\to \forall x\ \varphi (x)$
\textbf{General Topic:} Mathematical logic
\textbf{URL:} https://en.wikipedia.org/wiki/Induction%2C_bounding_and_least_number_principles
\end{lemmatheorembox}

\textbf{Usage count:} 1

\section*{Lemma 521}
\begin{lemmatheorembox}
\textbf{Name:} Circumradius
\textbf{Statement:} The diameter of the circumcircle can also be expressed as
\[
{\begin{aligned}{\text{diameter}}&{}={\frac {abc}{2\cdot {\text{area}}}}={\frac {|AB||BC||CA|}{2|\Delta ABC|}}\\[5pt]&{}={\frac {abc}{2{\sqrt {s(s-a)(s-b)(s-c)}}}}\\[5pt]&{}={\frac {2abc}{\sqrt {(a+b+c)(-a+b+c)(a-b+c)(a+b-c)}}}\end{aligned}}
\]
\textbf{General Topic:} Euclidean geometry
\textbf{URL:} https://en.wikipedia.org/wiki/Circumradius
\end{lemmatheorembox}

\textbf{Usage count:} 1

\section*{Lemma 522}
\begin{lemmatheorembox}
\textbf{Name:} Triangular number
\textbf{Statement:} $T_{n}={\frac {n(n+1)}{2}}$
\textbf{General Topic:} Number theory
\textbf{URL:} https://en.wikipedia.org/wiki/Triangular_number#Formula
\end{lemmatheorembox}

\textbf{Usage count:} 1

\section*{Lemma 523}
\begin{lemmatheorembox}
\textbf{Name:} Fundamental theorem of linear programming
\textbf{Statement:} Consider the optimization problem \(\min c^{T}x{\text{ subject to }}x\in P\). Where \(P=\{x\in \mathbb {R} ^{n}:Ax\leq b\}\). If \(P\) is a bounded polyhedron (and thus a polytope) and \(x^{\ast }\) is an optimal solution to the problem, then \(x^{\ast }\) is either an extreme point (vertex) of \(P\), or lies on a face \(F\subset P\) of optimal solutions.
\textbf{General Topic:} Mathematical optimization
\textbf{URL:} https://en.wikipedia.org/wiki/Fundamental_theorem_of_linear_programming
\end{lemmatheorembox}

\textbf{Usage count:} 1

\section*{Lemma 524}
\begin{lemmatheorembox}
\textbf{Name:} Rayleigh's theorem (also known as Beatty's theorem)
\textbf{Statement:} Rayleigh's theorem (also known as Beatty's theorem) states that given an irrational number $r>1\,, $ there exists $s>1$ so that the Beatty sequences ${\mathcal {B}}_{r}$ and ${\mathcal {B}}_{s}$ partition the set of positive integers: each positive integer belongs to exactly one of the two sequences.
\textbf{General Topic:} Number theory
\textbf{URL:} https://en.wikipedia.org/wiki/Beatty_sequence
\end{lemmatheorembox}

\textbf{Usage count:} 1

\section*{Lemma 525}
\begin{lemmatheorembox}
\textbf{Name:} Degree of a polynomial
\textbf{Statement:} The degree of the composition of two non-constant polynomials P and Q over a field or integral domain is the product of their degrees: \deg(P \circ Q) = \deg(P)\deg(Q).
\textbf{General Topic:} Polynomial Theory
\textbf{URL:} https://en.wikipedia.org/wiki/Degree_of_a_polynomial
\end{lemmatheorembox}

\textbf{Usage count:} 1

\section*{Lemma 526}
\begin{lemmatheorembox}
\textbf{Name:} Frobenius endomorphism
\textbf{Statement:} Thus F(r + s) = (r + s)^p = r^p + s^p = F(r) + F(s).
\textbf{General Topic:} Commutative Algebra
\textbf{URL:} https://en.wikipedia.org/wiki/Frobenius_endomorphism
\end{lemmatheorembox}

\textbf{Usage count:} 1

\section*{Lemma 527}
\begin{lemmatheorembox}
\textbf{Name:} Congruence linéaire
\textbf{Statement:} Lorsque l'ensemble des solutions est non vide, il forme donc une classe mod m, soit d classes mod M.
\textbf{General Topic:} Number theory
\textbf{URL:} https://fr.wikipedia.org/wiki/Congruence_lin%C3%A9aire
\end{lemmatheorembox}

\textbf{Usage count:} 1

\section*{Lemma 528}
\begin{lemmatheorembox}
\textbf{Name:} Octahedral symmetry
\textbf{Statement:} The cube has 48 isometries (symmetry elements), forming the symmetry group \(O_{h}\), isomorphic to \(S_{4} \times Z_{2}\).
\textbf{General Topic:} Geometry
\textbf{URL:} https://en.wikipedia.org/wiki/Octahedral_symmetry
\end{lemmatheorembox}

\textbf{Usage count:} 1

\section*{Lemma 529}
\begin{lemmatheorembox}
\textbf{Name:} Permutation
\textbf{Statement:} The order of a permutation $\sigma$ is the smallest positive integer $m$ so that $\sigma^m = \mathrm{id}$. It is the least common multiple of the lengths of its cycles.
\textbf{General Topic:} Group Theory
\textbf{URL:} https://en.wikipedia.org/wiki/Permutation#Order_of_a_permutation
\end{lemmatheorembox}

\textbf{Usage count:} 1

\section*{Lemma 530}
\begin{lemmatheorembox}
\textbf{Name:} Symmetric difference
\textbf{Statement:} The same fact can be stated as the indicator function (denoted here by \chi) of the symmetric difference, being the XOR (or addition mod 2) of the indicator functions of its two arguments: \chi_{(A\, \Delta\,B)} = \chi_A \oplus \chi_B or using the Iverson bracket notation [x \in A\, \Delta\,B] = [x \in A] \oplus [x \in B].
\textbf{General Topic:} Set Theory
\textbf{URL:} https://en.wikipedia.org/wiki/Symmetric_difference
\end{lemmatheorembox}

\textbf{Usage count:} 1

\section*{Lemma 531}
\begin{lemmatheorembox}
\textbf{Name:} Prime number theorem

\textbf{Statement:} Let $\pi(x)$ be the prime-counting function defined to be the number of primes less than or equal to $x$, for any real number $x$. The prime number theorem then states that $\frac{x}{\log(x)}$ is a good approximation to $\pi(x)$ (where log here means the natural logarithm), in the sense that the limit of the quotient of the two functions $\pi(x)$ and $\left[\frac{x}{\log(x)}\right]$ as $x$ increases without bound is 1:
\[
\lim_{x\to\infty}\frac{\;\pi(x)\;}{\;\left[ \frac{x}{\log(x)}\right]\;} = 1,
\]
known as the asymptotic law of distribution of prime numbers. Using asymptotic notation this result can be restated as
\[
\pi(x)\sim \frac{x}{\log x}.
\]

\textbf{General Topic:} Number theory

\textbf{URL:} https://en.wikipedia.org/wiki/Prime_number_theorem
\end{lemmatheorembox}

\textbf{Usage count:} 1

\section*{Lemma 532}
\begin{lemmatheorembox}
\textbf{Name:} Dual graph
\textbf{Statement:} A connected planar graph is Eulerian (has even degree at every vertex) if and only if its dual graph is bipartite.
\textbf{General Topic:} Graph Theory
\textbf{URL:} https://en.wikipedia.org/wiki/Dual_graph
\end{lemmatheorembox}

\textbf{Usage count:} 1

\section*{Lemma 533}
\begin{lemmatheorembox}
\textbf{Name:} Characterizations of the exponential function
\textbf{Statement:} The exponential function {\displaystyle e^{x}} is the unique function f with the multiplicative property {\displaystyle f(x+y)=f(x)f(y)} for all {\displaystyle x,y} and {\displaystyle f'(0)=1}.
\textbf{General Topic:} Real Analysis
\textbf{URL:} https://en.wikipedia.org/wiki/Characterizations_of_the_exponential_function
\end{lemmatheorembox}

\textbf{Usage count:} 1

\section*{Lemma 534}
\begin{lemmatheorembox}
\textbf{Name:} Monge's theorem
\textbf{Statement:} In geometry, Monge's theorem, named after Gaspard Monge, states that for any three circles in a plane, none of which is completely inside one of the others, the intersection points of each of the three pairs of external tangent lines are collinear.
\textbf{General Topic:} Euclidean geometry
\textbf{URL:} https://en.wikipedia.org/wiki/Monge%27s_theorem
\end{lemmatheorembox}

\textbf{Usage count:} 1

\section*{Lemma 535}
\begin{lemmatheorembox}
\textbf{Name:} Discrete uniform distribution
\textbf{Statement:} \textbf{Mean} $\displaystyle {\frac {a+b}{2}}$
\textbf{General Topic:} Probability theory
\textbf{URL:} https://en.wikipedia.org/wiki/Discrete_uniform_distribution
\end{lemmatheorembox}

\textbf{Usage count:} 1

\section*{Lemma 536}
\begin{lemmatheorembox}
\textbf{Name:} Hadamard's inequality
\textbf{Statement:} Specifically, Hadamard's inequality states that if $N$ is the matrix having columns $v_i$, then
\[
\left| \det(N) \right| \le \prod_{i=1}^n \|v_i\|.
\]
If the $n$ vectors are non-zero, equality in Hadamard's inequality is achieved if and only if the vectors are orthogonal.
\textbf{General Topic:} Linear Algebra
\textbf{URL:} https://en.wikipedia.org/wiki/Hadamard%27s_inequality
\end{lemmatheorembox}

\textbf{Usage count:} 1

\section*{Lemma 537}
\begin{lemmatheorembox}
\textbf{Name:} Cycle type
\textbf{Statement:} The number of permutations of a given cycle type is
\[
{\frac {n!}{1^{\alpha _{1}}2^{\alpha _{2}}\dotsm n^{\alpha _{n}}\alpha _{1}!\alpha _{2}!\dotsm \alpha _{n}!}}.
\]
\textbf{General Topic:} Combinatorics
\textbf{URL:} https://en.wikipedia.org/wiki/Permutation#Cycle_type
\end{lemmatheorembox}

\textbf{Usage count:} 1

\section*{Lemma 538}
\begin{lemmatheorembox}
\textbf{Name:} Bayes' theorem
\textbf{Statement:} Bayes' theorem is stated mathematically as the following equation:
\[
P(A\vert B)={\frac {P(B\vert A)P(A)}{P(B)}}
\]
where A and B are events and P(B)\neq 0.
\textbf{General Topic:} Probability theory
\textbf{URL:} https://en.wikipedia.org/wiki/Bayes%27_theorem
\end{lemmatheorembox}

\textbf{Usage count:} 1

\section*{Lemma 539}
\begin{lemmatheorembox}
\textbf{Name:} GCD domain
\textbf{Statement:} For every pair of elements x, y of a GCD domain R, a GCD d of x and y and an LCM m of x and y can be chosen such that dm = xy, or stated differently, if x and y are nonzero elements and d is any GCD d of x and y, then xy/d is an LCM of x and y, and vice versa.
\textbf{General Topic:} Ring Theory
\textbf{URL:} https://en.wikipedia.org/wiki/GCD_domain
\end{lemmatheorembox}

\textbf{Usage count:} 1

\section*{Lemma 540}
\begin{lemmatheorembox}
\textbf{Name:} Trigonometric functions
\textbf{Statement:} \(\cot \theta ={\frac {\cos \theta }{\sin \theta }}=\tan \left({\frac {\pi }{2}}-\theta \right)={\frac {1}{\tan \theta }}\)
\textbf{General Topic:} Trigonometry
\textbf{URL:} https://en.wikipedia.org/wiki/Cotangent
\end{lemmatheorembox}

\textbf{Usage count:} 1

\section*{Lemma 541}
\begin{lemmatheorembox}
\textbf{Name:} Reciprocal polynomial
\textbf{Statement:} If a is a root of a polynomial that is either palindromic or antipalindromic, then 1/a is also a root and has the same multiplicity.
\textbf{General Topic:} Algebra
\textbf{URL:} https://en.wikipedia.org/wiki/Reciprocal_polynomial
\end{lemmatheorembox}

\textbf{Usage count:} 1

\section*{Lemma 542}
\begin{lemmatheorembox}
\textbf{Name:} Cofunction identities
\textbf{Statement:} $\tan \left({\frac {\pi }{2}}-A\right)=\cot(A)$
\textbf{General Topic:} Trigonometry
\textbf{URL:} https://en.wikipedia.org/wiki/Cofunction
\end{lemmatheorembox}

\textbf{Usage count:} 1

\section*{Lemma 543}
\begin{lemmatheorembox}
\textbf{Name:} Binary number
\textbf{Statement:} In the binary system, each bit represents an increasing power of 2, with the rightmost bit representing $2^{0}$, the next representing $2^{1}$, then $2^{2}$, and so on. The value of a binary number is the sum of the powers of 2 represented by each "1" bit.
\textbf{General Topic:} Numeral systems
\textbf{URL:} https://en.wikipedia.org/wiki/Binary_number
\end{lemmatheorembox}

\textbf{Usage count:} 1

\section*{Lemma 544}
\begin{lemmatheorembox}
\textbf{Name:} Parking function
\textbf{Statement:} The number of parking functions of length n is exactly (n+1)^{n-1}.
\textbf{General Topic:} Combinatorics
\textbf{URL:} https://en.wikipedia.org/wiki/Parking_function
\end{lemmatheorembox}

\textbf{Usage count:} 1

\section*{Lemma 545}
\begin{lemmatheorembox}
\textbf{Name:} Cramer's rule
\textbf{Statement:} Consider a system of n linear equations for n unknowns, represented in matrix multiplication form as follows: \(A\mathbf{x}=\mathbf{b}\) where the \(n\times n\) matrix \(A\) has a nonzero determinant, and the vector \(\mathbf{x}=(x_1,\ldots,x_n)^{\mathsf{T}}\) is the column vector of the variables. Then the theorem states that in this case the system has a unique solution, whose individual values for the unknowns are given by: \(x_i=\frac{\det(A_i)}{\det(A)}\qquad i=1,\ldots,n\) where \(A_i\) is the matrix formed by replacing the i-th column of \(A\) by the column vector \(b\).
\textbf{General Topic:} Linear algebra
\textbf{URL:} https://en.wikipedia.org/wiki/Cramer's_rule
\end{lemmatheorembox}

\textbf{Usage count:} 1

\section*{Lemma 546}
\begin{lemmatheorembox}
\textbf{Name:} Phi is a multiplicative function
\textbf{Statement:} This means that if $\gcd(m,n)=1$, then $\varphi(m)\varphi(n)=\varphi(mn)$.
\textbf{General Topic:} Number theory
\textbf{URL:} https://en.wikipedia.org/wiki/Euler%27s_totient_function#Phi_is_a_multiplicative_function
\end{lemmatheorembox}

\textbf{Usage count:} 1

\section*{Lemma 547}
\begin{lemmatheorembox}
\textbf{Name:} Value of phi for a prime power argument
\textbf{Statement:} If p is prime and $k\geq 1$, then $\varphi \left(p^{k}\right)=p^{k}-p^{k-1}=p^{k-1}(p-1)=p^{k}\left(1-{\tfrac {1}{p}}\right).$
\textbf{General Topic:} Number theory
\textbf{URL:} https://en.wikipedia.org/wiki/Euler%27s_totient_function#Value_of_phi_for_a_prime_power_argument
\end{lemmatheorembox}

\textbf{Usage count:} 1

\section*{Lemma 548}
\begin{lemmatheorembox}
\textbf{Name:} Circular permutations
\textbf{Statement:} There are (n – 1)! circular permutations of a set with n elements.
\textbf{General Topic:} Combinatorics
\textbf{URL:} https://en.wikipedia.org/wiki/Permutation#Circular_permutations
\end{lemmatheorembox}

\textbf{Usage count:} 1

\section*{Lemma 549}
\begin{lemmatheorembox}
\textbf{Name:} Regular graph
\textbf{Statement:} Regular graphs of degree at most 2 are easy to classify: a 0-regular graph consists of disconnected vertices, a 1-regular graph consists of disconnected edges, and a 2-regular graph consists of a disjoint union of cycles and infinite chains.
\textbf{General Topic:} Graph Theory
\textbf{URL:} https://en.wikipedia.org/wiki/Regular_graph
\end{lemmatheorembox}

\textbf{Usage count:} 1

\section*{Lemma 550}
\begin{lemmatheorembox}
\textbf{Name:} Cardinality
\textbf{Statement:} Two sets A and B are said to have the same cardinality if their elements can be paired one-to-one. That is, if there exists a function f:A\mapsto B which is bijective.
\textbf{General Topic:} Set theory
\textbf{URL:} https://en.wikipedia.org/wiki/Cardinality
\end{lemmatheorembox}

\textbf{Usage count:} 1

\section*{Lemma 551}
\begin{lemmatheorembox}
\textbf{Name:} Contraposition
\textbf{Statement:} The law of contraposition says that a conditional statement is true if, and only if, its contrapositive is true.
\textbf{General Topic:} Mathematical logic
\textbf{URL:} https://en.wikipedia.org/wiki/Contraposition
\end{lemmatheorembox}

\textbf{Usage count:} 1

\section*{Lemma 552}
\begin{lemmatheorembox}
\textbf{Name:} Graph automorphism
\textbf{Statement:} Formally, an automorphism of a graph $G = (V, E)$ is a permutation $\sigma$ of the vertex set $V$, such that the pair of vertices $(u, v)$ form an edge if and only if the pair $(\sigma(u), \sigma(v))$ also form an edge.
\textbf{General Topic:} Graph theory
\textbf{URL:} https://en.wikipedia.org/wiki/Graph_automorphism
\end{lemmatheorembox}

\textbf{Usage count:} 1

\section*{Lemma 553}
\begin{lemmatheorembox}
\textbf{Name:} Generating function
\textbf{Statement:} The ordinary generating function of a sequence can be expressed as a rational function (the ratio of two finite-degree polynomials) if and only if the sequence is a linear recursive sequence with constant coefficients; this generalizes the examples above. Conversely, every sequence generated by a fraction of polynomials satisfies a linear recurrence with constant coefficients; these coefficients are identical to the coefficients of the fraction denominator polynomial (so they can be directly read off).
\textbf{General Topic:} Combinatorics
\textbf{URL:} https://en.wikipedia.org/wiki/Generating_function
\end{lemmatheorembox}

\textbf{Usage count:} 1

\section*{Lemma 554}
\begin{lemmatheorembox}
\textbf{Name:} Polynomial interpolation
\textbf{Statement:} For any $n+1$ bivariate data points $(x_{0},y_{0}),\dotsc,(x_{n},y_{n})\in \mathbb {R} ^{2}$, where no two $x_{j}$ are the same, there exists a unique polynomial $p(x)$ of degree at most $n$ that interpolates these points, i.e. $p(x_{0})=y_{0},\ldots ,p(x_{n})=y_{n}$.
\textbf{General Topic:} Numerical analysis
\textbf{URL:} https://en.wikipedia.org/wiki/Polynomial_interpolation
\end{lemmatheorembox}

\textbf{Usage count:} 1

\section*{Lemma 555}
\begin{lemmatheorembox}
\textbf{Name:} Square-free polynomial
\textbf{Statement:} The product rule implies that, if p^{2} divides f, then p divides the formal derivative f ′ of f. The converse is also true and hence, f is square-free if and only if 1 is a greatest common divisor of the polynomial and its derivative.
\textbf{General Topic:} Algebra
\textbf{URL:} https://en.wikipedia.org/wiki/Square-free_polynomial
\end{lemmatheorembox}

\textbf{Usage count:} 1

\section*{Lemma 556}
\begin{lemmatheorembox}
\textbf{Name:} Greedy algorithm for Egyptian fractions
\textbf{Statement:} Fibonacci's algorithm expands the fraction $\frac{x}{y}$ to be represented, by repeatedly performing the replacement $\frac {x}{y}=\frac {1}{\left\lceil {\frac {y}{x}}\right\rceil }+\frac {(-y){\bmod {x}}}{y\left\lceil {\frac {y}{x}}\right\rceil }$ (simplifying the second term in this replacement as necessary).
\textbf{General Topic:} Number theory
\textbf{URL:} https://en.wikipedia.org/wiki/Greedy_algorithm_for_Egyptian_fractions
\end{lemmatheorembox}

\textbf{Usage count:} 1

\section*{Lemma 557}
\begin{lemmatheorembox}
\textbf{Name:} Fermat point
\textbf{Statement:} The angles subtended by the sides of the triangle at X(13) are all equal to 120° (Case 2), or 60°, 60°, 120° (Case 1).
\textbf{General Topic:} Euclidean geometry
\textbf{URL:} https://en.wikipedia.org/wiki/Fermat_point
\end{lemmatheorembox}

\textbf{Usage count:} 1

\section*{Lemma 558}
\begin{lemmatheorembox}
\textbf{Name:} Function composition
\textbf{Statement:} The composition of functions is always associative—a property inherited from the composition of relations. That is, if $f$, $g$, and $h$ are composable, then $f \circ (g \circ h) = (f \circ g) \circ h.
\textbf{General Topic:} Functions
\textbf{URL:} https://en.wikipedia.org/wiki/Function_composition
\end{lemmatheorembox}

\textbf{Usage count:} 1

\section*{Lemma 559}
\begin{lemmatheorembox}
\textbf{Name:} Frobenius automorphism
\textbf{Statement:} In GF(q), the identity (x + y)p = x^{p} + y^{p} implies that the map $\varphi:x\mapsto x^{p}$ is a GF(p)-linear endomorphism and a field automorphism of GF(q), which fixes every element of the subfield GF(p).
\textbf{General Topic:} Finite fields
\textbf{URL:} https://en.wikipedia.org/wiki/Finite_field
\end{lemmatheorembox}

\textbf{Usage count:} 1

\section*{Lemma 560}
\begin{lemmatheorembox}
\textbf{Name:} Circumcenter
\textbf{Statement:} In geometry, the circumcenter of a triangle is the point of intersection of the perpendicular bisectors of the sides of the triangle; this point is equidistant from the triangle's three vertices.
\textbf{General Topic:} Euclidean Geometry
\textbf{URL:} https://en.wikipedia.org/wiki/Circumcenter
\end{lemmatheorembox}

\textbf{Usage count:} 1

\section*{Lemma 561}
\begin{lemmatheorembox}
\textbf{Name:} Tangent–chord angle theorem
\textbf{Statement:} The angle between a tangent and a chord through the point of contact is equal to the angle in the alternate segment.
\textbf{General Topic:} Euclidean Geometry
\textbf{URL:} https://en.wikipedia.org/wiki/Tangent%E2%80%93chord_angle_theorem
\end{lemmatheorembox}

\textbf{Usage count:} 1

\section*{Lemma 562}
\begin{lemmatheorembox}
\textbf{Name:} Krein–Milman theorem
\textbf{Statement:} Suppose $X$ is a Hausdorff locally convex topological vector space (for example, a normed space) and $K$ is a compact and convex subset of $X.$ Then $K$ is equal to the closed convex hull of its extreme points: $K~=~{\overline {\operatorname {co} }}(\operatorname {extreme} (K)).$
\textbf{General Topic:} Functional analysis
\textbf{URL:} https://en.wikipedia.org/wiki/Krein%E2%80%93Milman_theorem
\end{lemmatheorembox}

\textbf{Usage count:} 1

\section*{Lemma 563}
\begin{lemmatheorembox}
\textbf{Name:} Identity function
\textbf{Statement:} The identity function on the positive integers is a completely multiplicative function (essentially multiplication by 1), considered in number theory.
\textbf{General Topic:} Number theory
\textbf{URL:} https://en.wikipedia.org/wiki/Identity_function
\end{lemmatheorembox}

\textbf{Usage count:} 1

\section*{Lemma 564}
\begin{lemmatheorembox}
\textbf{Name:} Euler's formula
\textbf{Statement:} Euler's formula states that if a finite, connected, planar graph is drawn in the plane without any edge intersections, and v is the number of vertices, e is the number of edges and f is the number of faces (regions bounded by edges, including the outer, infinitely large region), then v-e+f=2.
\textbf{General Topic:} Graph theory
\textbf{URL:} https://en.wikipedia.org/wiki/Planar_graph
\end{lemmatheorembox}

\textbf{Usage count:} 1

\section*{Lemma 565}
\begin{lemmatheorembox}
\textbf{Name:} Separation of variables
\textbf{Statement:} A differential equation for the unknown $f(x)$ is separable if it can be written in the form $\frac{d}{dx}f(x)=g(x)h(f(x))$ where $g$ and $h$ are given functions. This is perhaps more transparent when written using $y=f(x)$ as: $\frac{dy}{dx}=g(x)h(y).$ So now as long as $h(y)\neq 0,$ we can rearrange terms to obtain: ${dy \over h(y)}=g(x)\,dx,$ where the two variables $x$ and $y$ have been separated. Integrating both sides of the equation with respect to $x,$ we have, or equivalently, $\int {\frac {1}{h(y)}}\,dy=\int g(x)\,dx$ because of the substitution rule for integrals.
\textbf{General Topic:} Differential equations
\textbf{URL:} https://en.wikipedia.org/wiki/Separation_of_variables
\end{lemmatheorembox}

\textbf{Usage count:} 1

\section*{Lemma 566}
\begin{lemmatheorembox}
\textbf{Name:} Sinc function
\textbf{Statement:} $\operatorname {sinc} (0):=\lim _{x\to 0}{\frac {\sin(ax)}{ax}}=1$
\textbf{General Topic:} Mathematical analysis
\textbf{URL:} https://en.wikipedia.org/wiki/Sinc_function
\end{lemmatheorembox}

\textbf{Usage count:} 1

\section*{Lemma 567}
\begin{lemmatheorembox}
\textbf{Name:} Fundamental theorem of symmetric polynomials
\textbf{Statement:} Indeed, a theorem called the fundamental theorem of symmetric polynomials states that any symmetric polynomial can be expressed in terms of elementary symmetric polynomials.
\textbf{General Topic:} Commutative algebra
\textbf{URL:} https://en.wikipedia.org/wiki/Fundamental_theorem_of_symmetric_polynomials
\end{lemmatheorembox}

\textbf{Usage count:} 1

\section*{Lemma 568}
\begin{lemmatheorembox}
\textbf{Name:} Sagitta (geometry)
\textbf{Statement:} Generally, for a known value of $\theta$, any of $s$, $l$, and $r$, can be computed from one of the others: ${\begin{aligned}s&=r\operatorname {versin} \theta =r\left(1-\cos \theta \right)=2r\sin ^{2}{\frac {\theta }{2}}\,,\\s&={\frac {l\operatorname {versin} \theta }{2\sin \theta }}={\frac {l(1-\cos \theta )}{2\sin \theta }}={\frac {l\sin ^{2}{\frac {\theta }{2}}}{\sin \theta }}\,,\\l&=2r\sin \theta \,.\end{aligned}}$
\textbf{General Topic:} Geometry
\textbf{URL:} https://en.wikipedia.org/wiki/Sagitta_(geometry)
\end{lemmatheorembox}

\textbf{Usage count:} 1

\section*{Lemma 569}
\begin{lemmatheorembox}
\textbf{Name:} Logarithmic identities
\textbf{Statement:} \[\log _{b}(xy)=\log _{b}x+\log _{b}y.\]
\textbf{General Topic:} Logarithms
\textbf{URL:} https://en.wikipedia.org/wiki/Logarithm#Product,_quotient,_power,_and_root
\end{lemmatheorembox}

\textbf{Usage count:} 1

\section*{Lemma 570}
\begin{lemmatheorembox}
\textbf{Name:} Cube (algebra)
\textbf{Statement:} In real numbers, the cube function preserves the order: larger numbers have larger cubes.
\textbf{General Topic:} Algebra
\textbf{URL:} https://en.wikipedia.org/wiki/Cube_(algebra)
\end{lemmatheorembox}

\textbf{Usage count:} 1

\section*{Lemma 571}
\begin{lemmatheorembox}
\textbf{Name:} Trinomial triangle
\textbf{Statement:} ${0 \choose 0}_{2}=1$, ${n+1 \choose k}_{2}={n \choose k-1}_{2}+{n \choose k}_{2}+{n \choose k+1}_{2}$ for $n\geq 0$, where ${n \choose k}_{2}=0$ for $\ k<-n$ and $\ k>n$.
\textbf{General Topic:} Combinatorics
\textbf{URL:} https://en.wikipedia.org/wiki/Trinomial_triangle
\end{lemmatheorembox}

\textbf{Usage count:} 1

\section*{Lemma 572}
\begin{lemmatheorembox}
\textbf{Name:} Periodicity and asymptotes
\textbf{Statement:} The functions sine and cosine also have semiperiods \(\pi\), and \(\sin(z+\pi )=-\sin(z),\quad \cos(z+\pi )=-\cos(z)\) and consequently \(\tan(z+\pi )=\tan(z),\quad \cot(z+\pi )=\cot(z).\) The tangent function \(\tan(z)=\sin(z)/\cos(z)\) has a simple zero at \(z=0\) and vertical asymptotes at \(z=\pm \pi /2\), where it has a simple pole of residue \(-1\). Again, owing to the periodicity, the zeros are all the integer multiples of \(\pi\) and the poles are odd multiples of \(\pi /2\), all having the same residue.
\textbf{General Topic:} Trigonometry
\textbf{URL:} https://en.wikipedia.org/wiki/Trigonometric_functions#Periodicity_and_asymptotes
\end{lemmatheorembox}

\textbf{Usage count:} 1

\section*{Lemma 573}
\begin{lemmatheorembox}
\textbf{Name:} Sum and difference formulas
\textbf{Statement:} The sum and difference formulas allow expanding the sine, the cosine, and the tangent of a sum or a difference of two angles in terms of sines and cosines and tangents of the angles themselves. Sum \(\displaystyle {\begin{aligned}\sin \left(x+y\right)&=\sin x\cos y+\cos x\sin y,\\[5mu]\cos \left(x+y\right)&=\cos x\cos y-\sin x\sin y,\\[5mu]\tan(x+y)&={\frac {\tan x+\tan y}{1-\tan x\tan y}}.\end{aligned}}\) When the two angles are equal, the sum formulas reduce to simpler equations known as the double-angle formulae. \(\displaystyle {\begin{aligned}\sin 2x&=2\sin x\cos x={\frac {2\tan x}{1+\tan ^{2}x}},\\[5mu]\cos 2x&=\cos ^{2}x-\sin ^{2}x=2\cos ^{2}x-1=1-2\sin ^{2}x={\frac {1-\tan ^{2}x}{1+\tan ^{2}x}},\\[5mu]\tan 2x&={\frac {2\tan x}{1-\tan ^{2}x}}.\end{aligned}}\)
\textbf{General Topic:} Trigonometry
\textbf{URL:} https://en.wikipedia.org/wiki/Trigonometric_functions#Sum_and_difference_formulas
\end{lemmatheorembox}

\textbf{Usage count:} 1

\section*{Lemma 574}
\begin{lemmatheorembox}
\textbf{Name:} Basic identities
\textbf{Statement:} The cosine and the secant are even functions; the other trigonometric functions are odd functions. That is: \(\displaystyle {\begin{aligned}\sin(-x)&=-\sin x\\\cos(-x)&=\cos x\\\tan(-x)&=-\tan x\\\cot(-x)&=-\cot x\\\csc(-x)&=-\csc x\\\sec(-x)&=\sec x.\end{aligned}}\) All trigonometric functions are periodic functions of period \(2 \pi\). This is the smallest period, except for the tangent and the cotangent, which have \(\pi\) as smallest period. This means that, for every integer \(k\), one has \(\displaystyle {\begin{array}{lrl}\sin(x+&2k\pi )&=\sin x\\\cos(x+&2k\pi )&=\cos x\\\tan(x+&k\pi )&=\tan x\\\cot(x+&k\pi )&=\cot x\\\csc(x+&2k\pi )&=\csc x\\\sec(x+&2k\pi )&=\sec x.\end{array}}\)
\textbf{General Topic:} Trigonometry
\textbf{URL:} https://en.wikipedia.org/wiki/Trigonometric_functions#Basic_identities
\end{lemmatheorembox}

\textbf{Usage count:} 1

\section*{Lemma 575}
\begin{lemmatheorembox}
\textbf{Name:} Euler's polyhedron formula
\textbf{Statement:} Any convex polyhedron's surface has Euler characteristic $V - E + F = 2$, where $V$ is the number of vertices, $E$ is the number of edges, and $F$ is the number of faces.
\textbf{General Topic:} Geometry
\textbf{URL:} https://en.wikipedia.org/wiki/Euler%27s_polyhedron_formula
\end{lemmatheorembox}

\textbf{Usage count:} 1

\section*{Lemma 576}
\begin{lemmatheorembox}
\textbf{Name:} Fermat's Last Theorem
\textbf{Statement:} For any integer $n>2$, the equation $a^n+b^n=c^n$ has no positive integer solutions.
\textbf{General Topic:} Number theory
\textbf{URL:} https://en.wikipedia.org/wiki/Fermat%27s_Last_Theorem
\end{lemmatheorembox}

\textbf{Usage count:} 1

\section*{Lemma 577}
\begin{lemmatheorembox}
\textbf{Name:} Tangent
\textbf{Statement:} or internally tangent if the distance between their centres is equal to the difference between their radii:
\[
\left(x_{1}-x_{2}\right)^{2}+\left(y_{1}-y_{2}\right)^{2}=\left(r_{1}-r_{2}\right)^{2}.
\]
\textbf{General Topic:} Euclidean geometry
\textbf{URL:} https://en.wikipedia.org/wiki/Tangent
\end{lemmatheorembox}

\textbf{Usage count:} 1

\section*{Lemma 578}
\begin{lemmatheorembox}
\textbf{Name:} Inequality of arithmetic and geometric means
\textbf{Statement:} The simplest non-trivial case is for two non-negative numbers x and y, that is, \(\frac{x+y}{2}\geq \sqrt{xy}\) with equality if and only if x = y.
\textbf{General Topic:} Inequalities
\textbf{URL:} https://en.wikipedia.org/wiki/Inequality_of_arithmetic_and_geometric_means
\end{lemmatheorembox}

\textbf{Usage count:} 1

\section*{Lemma 579}
\begin{lemmatheorembox}
\textbf{Name:} Quotient
\textbf{Statement:} {\textstyle \log _{b}\!{\frac {x}{y}}=\log _{b}x-\log _{b}y}
\textbf{General Topic:} Logarithms
\textbf{URL:} https://en.wikipedia.org/wiki/Logarithm#Logarithmic_identities
\end{lemmatheorembox}

\textbf{Usage count:} 1

\section*{Lemma 580}
\begin{lemmatheorembox}
\textbf{Name:} Surface area
\textbf{Statement:} More rigorously, if a surface $S$ is a union of finitely many pieces $S_1, \dots, S_r$ which do not overlap except at their boundaries, then $A(S)=A(S_{1})+\cdots +A(S_{r}).$
\textbf{General Topic:} Geometry
\textbf{URL:} https://en.wikipedia.org/wiki/Surface_area
\end{lemmatheorembox}

\textbf{Usage count:} 1

\section*{Lemma 581}
\begin{lemmatheorembox}
\textbf{Name:} Division theorem
\textbf{Statement:} Given two integers $a$ and $b$, with $b\neq 0$, there exist unique integers $q$ and $r$ such that $a=bq+r$, and $0\leq r<|b|$, where $|b|$ denotes the absolute value of $b$.
\textbf{General Topic:} Number theory
\textbf{URL:} https://en.wikipedia.org/wiki/Euclidean_division\#Division_theorem
\end{lemmatheorembox}

\textbf{Usage count:} 1

\section*{Lemma 582}
\begin{lemmatheorembox}
\textbf{Name:} Symmetry of the binomial coefficients
\textbf{Statement:} The symmetry of the binomial coefficients states that ${n \choose k}={n \choose n-k}$.
\textbf{General Topic:} Combinatorics
\textbf{URL:} https://en.wikipedia.org/wiki/Bijective_proof
\end{lemmatheorembox}

\textbf{Usage count:} 1

\section*{Lemma 583}
\begin{lemmatheorembox}
\textbf{Name:} Vector algebra relations
\textbf{Statement:} The area \Sigma of a parallelogram with sides A and B containing the angle \theta is: \Sigma = AB\sin \theta ,
\textbf{General Topic:} Vector calculus
\textbf{URL:} https://en.wikipedia.org/wiki/Vector_algebra_relations
\end{lemmatheorembox}

\textbf{Usage count:} 1

\section*{Lemma 584}
\begin{lemmatheorembox}
\textbf{Name:} Locus (mathematics)
\textbf{Statement:} The set of points equidistant from two points is a perpendicular bisector to the line segment connecting the two points.
\textbf{General Topic:} Geometry
\textbf{URL:} https://en.wikipedia.org/wiki/Locus_(mathematics)
\end{lemmatheorembox}

\textbf{Usage count:} 1

\section*{Lemma 585}
\begin{lemmatheorembox}
\textbf{Name:} Divisibility properties
\textbf{Statement:} $\gcd(F_{a},F_{b},F_{c},\ldots )=F_{\gcd(a,b,c,\ldots )}\,$
\textbf{General Topic:} Number theory
\textbf{URL:} https://en.wikipedia.org/wiki/Fibonacci_sequence#Divisibility_properties
\end{lemmatheorembox}

\textbf{Usage count:} 1

\section*{Lemma 586}
\begin{lemmatheorembox}
\textbf{Name:} Expectation of product of random variables
\textbf{Statement:} When two random variables are statistically independent, the expectation of their product is the product of their expectations.
\textbf{General Topic:} Probability theory
\textbf{URL:} https://en.wikipedia.org/wiki/Distribution_of_the_product_of_two_random_variables
\end{lemmatheorembox}

\textbf{Usage count:} 1

\section*{Lemma 587}
\begin{lemmatheorembox}
\textbf{Name:} Repeating decimal
\textbf{Statement:} It is possible to get a general formula expressing a repeating decimal with an n-digit period (repetend length), beginning right after the decimal point, as a fraction:
\begin{align}
x&=0.{\overline {a_{1}a_{2}\cdots a_{n}}}\\
10^{n}x&=a_{1}a_{2}\cdots a_{n}.{\overline {a_{1}a_{2}\cdots a_{n}}}\\
\left(10^{n}-1\right)x=99\cdots 99x&=a_{1}a_{2}\cdots a_{n}\\
x&={\frac {a_{1}a_{2}\cdots a_{n}}{10^{n}-1}}={\frac {a_{1}a_{2}\cdots a_{n}}{99\cdots 99}}
\end{align}
\textbf{General Topic:} Number theory
\textbf{URL:} https://en.wikipedia.org/wiki/Repeating_decimal
\end{lemmatheorembox}

\textbf{Usage count:} 1

\section*{Lemma 588}
\begin{lemmatheorembox}
\textbf{Name:} Linear independence
\textbf{Statement:} A sequence of vectors $\mathbf{v}_1,\mathbf{v}_2,\dots,\mathbf{v}_n$ is said to be linearly independent if it is not linearly dependent, that is, if the equation $a_1\mathbf{v}_1+a_2\mathbf{v}_2+\cdots+a_n\mathbf{v}_n=\mathbf{0}$ can only be satisfied by $a_i=0$ for $i=1,\dots,n.$
\textbf{General Topic:} Linear algebra
\textbf{URL:} https://en.wikipedia.org/wiki/Linear_independence
\end{lemmatheorembox}

\textbf{Usage count:} 1

\section*{Lemma 589}
\begin{lemmatheorembox}
\textbf{Name:} Galileo's law of odd numbers
\textbf{Statement:} The middle figure in the diagram is a visual proof that the sum of the first n odd numbers is \(n^{2}\).
\textbf{General Topic:} Number Theory
\textbf{URL:} https://en.wikipedia.org/wiki/Galileo%27s_law_of_odd_numbers
\end{lemmatheorembox}

\textbf{Usage count:} 1

\section*{Lemma 590}
\begin{lemmatheorembox}
\textbf{Name:} Sinus und Kosinus
\textbf{Statement:} \cos 60^{\circ }=\sin 30^{\circ }={\tfrac {1}{2}}
\textbf{General Topic:} Trigonometry
\textbf{URL:} https://de.wikipedia.org/wiki/Sinus_und_Kosinus
\end{lemmatheorembox}

\textbf{Usage count:} 1

\section*{Lemma 591}
\begin{lemmatheorembox}
\textbf{Name:} Geometric probability
\textbf{Statement:} His analysis established the fundamental principle that chance is proportional to area fraction, noted that probability can be irrational, and proposed a frequency experiment for chance estimation, leading to the founding of stereology.
\textbf{General Topic:} Probability theory
\textbf{URL:} https://en.wikipedia.org/wiki/Geometric_probability
\end{lemmatheorembox}

\textbf{Usage count:} 1

\section*{Lemma 592}
\begin{lemmatheorembox}
\textbf{Name:} Tail-sum formula
\textbf{Statement:} \operatorname{E}[X] = \sum_{k=0}^{\infty} \Pr(X > k).
\textbf{General Topic:} Probability theory
\textbf{URL:} https://en.wikipedia.org/wiki/Expected_value\#Tail-sum_formula
\end{lemmatheorembox}

\textbf{Usage count:} 1

\section*{Lemma 593}
\begin{lemmatheorembox}
\textbf{Name:} Bertrand's postulate
\textbf{Statement:} A less restrictive formulation is: for every $n>1$, there is always at least one prime $p$ such that $n<p<2n$.
\textbf{General Topic:} Number theory
\textbf{URL:} https://en.wikipedia.org/wiki/Bertrand%27s_postulate
\end{lemmatheorembox}

\textbf{Usage count:} 1

\section*{Lemma 594}
\begin{lemmatheorembox}
\textbf{Name:} Axiom of induction
\textbf{Statement:} $\forall P\,\Bigl( P(0) \land \forall k \bigl( P(k) \to P(k+1)\bigr ) \to \forall n \,\bigl(P(n)\bigr)\Bigr),$
\textbf{General Topic:} Logic
\textbf{URL:} https://en.wikipedia.org/wiki/Mathematical\_induction
\end{lemmatheorembox}

\textbf{Usage count:} 1

\section*{Lemma 595}
\begin{lemmatheorembox}
\textbf{Name:} A theorem of Kronecker
\textbf{Statement:} A theorem of Kronecker states that if α is a nonzero algebraic integer such that α and all of its conjugates in the complex numbers have absolute value at most 1, then α is a root of unity.
\textbf{General Topic:} Algebraic Number Theory
\textbf{URL:} https://en.wikipedia.org/wiki/Conjugate_element_%28field_theory%29
\end{lemmatheorembox}

\textbf{Usage count:} 1

\section*{Lemma 596}
\begin{lemmatheorembox}
\textbf{Name:} Multinomial coefficient
\textbf{Statement:} Binomial coefficients can be generalized to multinomial coefficients defined to be the number: $\binom{n}{k_{1},k_{2},\ldots,k_{r}}=\frac{n!}{k_{1}!k_{2}!\cdots k_{r}!}$ where $\sum_{i=1}^{r}k_{i}=n.$ The combinatorial interpretation of multinomial coefficients is distribution of $n$ distinguishable elements over $r$ (distinguishable) containers, each containing exactly $k_{i}$ elements, where $i$ is the index of the container.
\textbf{General Topic:} Combinatorics
\textbf{URL:} https://en.wikipedia.org/wiki/Binomial_coefficient\#Generalization_to_multinomials
\end{lemmatheorembox}

\textbf{Usage count:} 1

\section*{Lemma 597}
\begin{lemmatheorembox}
\textbf{Name:} Vieta's formulas
\textbf{Statement:} Vieta's formulas (named after François Viète) are the relations $x_{1}+x_{2}=-{\frac {b}{a}},\quad x_{1}x_{2}={\frac {c}{a}}$ between the roots of a quadratic polynomial and its coefficients.
\textbf{General Topic:} Algebra
\textbf{URL:} https://en.wikipedia.org/wiki/Quadratic_equation#Vieta's_formulas
\end{lemmatheorembox}

\textbf{Usage count:} 1

\section*{Lemma 598}
\begin{lemmatheorembox}
\textbf{Name:} Particular values of the Riemann zeta function
\textbf{Statement:} The first few values are given by: {\begin{aligned}\zeta (2)&=1+{\frac {1}{2^{2}}}+{\frac {1}{3^{2}}}+\cdots ={\frac {\pi ^{2}}{6}}\\[4pt]\zeta (4)&=1+{\frac {1}{2^{4}}}+{\frac {1}{3^{4}}}+\cdots ={\frac {\pi ^{4}}{90}}\\[4pt]\zeta (6)&=1+{\frac {1}{2^{6}}}+{\frac {1}{3^{6}}}+\cdots ={\frac {\pi ^{6}}{945}}\\[4pt]\zeta (8)&=1+{\frac {1}{2^{8}}}+{\frac {1}{3^{8}}}+\cdots ={\frac {\pi ^{8}}{9450}}\\[4pt]\zeta (10)&=1+{\frac {1}{2^{10}}}+{\frac {1}{3^{10}}}+\cdots ={\frac {\pi ^{10}}{93555}}\\[4pt]\zeta (12)&=1+{\frac {1}{2^{12}}}+{\frac {1}{3^{12}}}+\cdots ={\frac {691\pi ^{12}}{638512875}}\\[4pt]\zeta (14)&=1+{\frac {1}{2^{14}}}+{\frac {1}{3^{14}}}+\cdots ={\frac {2\pi ^{14}}{18243225}}\\[4pt]\zeta (16)&=1+{\frac {1}{2^{16}}}+{\frac {1}{3^{16}}}+\cdots ={\frac {3617\pi ^{16}}{325641566250}}\,.\end{aligned}}
\textbf{General Topic:} Complex analysis
\textbf{URL:} https://en.wikipedia.org/wiki/Particular_values_of_the_Riemann_zeta_function
\end{lemmatheorembox}

\textbf{Usage count:} 1

\section*{Lemma 599}
\begin{lemmatheorembox}
\textbf{Name:} Bernoulli distribution
\textbf{Statement:} The variance of a Bernoulli distributed X is
\operatorname{Var}[X] = pq = p(1-p)
\textbf{General Topic:} Probability theory
\textbf{URL:} https://en.wikipedia.org/wiki/Bernoulli_distribution
\end{lemmatheorembox}

\textbf{Usage count:} 1

\section*{Lemma 600}
\begin{lemmatheorembox}
\textbf{Name:} Pythagorean theorem
\textbf{Statement:} Given a triangle with sides of length a, b, and c, if a^{2} + b^{2} = c^{2}, then the angle between sides a and b is a right angle.
\textbf{General Topic:} Euclidean geometry
\textbf{URL:} https://en.wikipedia.org/wiki/Pythagorean_theorem#Converse
\end{lemmatheorembox}

\textbf{Usage count:} 1

\section*{Lemma 601}
\begin{lemmatheorembox}
\textbf{Name:} Integral domain
\textbf{Statement:} If a divides b and b divides a, then a and b are associated elements or associates. Equivalently, a and b are associates if a = ub for some unit u.
\textbf{General Topic:} Ring theory
\textbf{URL:} https://en.wikipedia.org/wiki/Integral_domain
\end{lemmatheorembox}

\textbf{Usage count:} 1

\section*{Lemma 602}
\begin{lemmatheorembox}
\textbf{Name:} Unit (ring theory)
\textbf{Statement:} In the ring of integers Z, the only units are 1 and −1.
\textbf{General Topic:} Ring theory
\textbf{URL:} https://en.wikipedia.org/wiki/Unit_(ring_theory)
\end{lemmatheorembox}

\textbf{Usage count:} 1

\section*{Lemma 603}
\begin{lemmatheorembox}
\textbf{Name:} Orbit–stabilizer theorem
\textbf{Statement:} $|G\cdot x|=[G\,:\,G_{x}]=|G|/|G_{x}|.$
\textbf{General Topic:} Group Theory
\textbf{URL:} https://en.wikipedia.org/wiki/Orbit-stabilizer_theorem
\end{lemmatheorembox}

\textbf{Usage count:} 1

\section*{Lemma 604}
\begin{lemmatheorembox}
\textbf{Name:} Similarity (geometry)
\textbf{Statement:} The ratio between the areas of similar figures is equal to the square of the ratio of corresponding lengths of those figures (for example, when the side of a square or the radius of a circle is multiplied by three, its area is multiplied by nine — i.e. by three squared).
\textbf{General Topic:} Euclidean geometry
\textbf{URL:} https://en.wikipedia.org/wiki/Similarity\_%28geometry%29
\end{lemmatheorembox}

\textbf{Usage count:} 1

\section*{Lemma 605}
\begin{lemmatheorembox}
\textbf{Name:} Chessboard
\textbf{Statement:} It consists of 64 squares, 8 rows by 8 columns, on which the chess pieces are placed.
\textbf{General Topic:} Chess
\textbf{URL:} https://en.wikipedia.org/wiki/Chessboard
\end{lemmatheorembox}

\textbf{Usage count:} 1

\section*{Lemma 606}
\begin{lemmatheorembox}
\textbf{Name:} Knight (chess)
\textbf{Statement:} Consequently, a knight alternates between light and dark squares with each move.
\textbf{General Topic:} Chess
\textbf{URL:} https://en.wikipedia.org/wiki/Knight_(chess)
\end{lemmatheorembox}

\textbf{Usage count:} 1

\section*{Lemma 607}
\begin{lemmatheorembox}
\textbf{Name:} Binomische Formeln
\textbf{Statement:} {\displaystyle (a-b)^{2}=a^{2}-2ab+b^{2}}
\textbf{General Topic:} Algebra
\textbf{URL:} https://de.wikipedia.org/wiki/Binomische_Formeln
\end{lemmatheorembox}

\textbf{Usage count:} 1

\section*{Lemma 608}
\begin{lemmatheorembox}
\textbf{Name:} Sigma-additive set function
\textbf{Statement:} In mathematics, an additive set function is a function $\mu$ mapping sets to numbers, with the property that its value on a union of two disjoint sets equals the sum of its values on these sets, namely, $\mu(A\cup B)=\mu(A)+\mu(B)$.
\textbf{General Topic:} Measure theory
\textbf{URL:} https://en.wikipedia.org/wiki/Sigma-additive_set_function
\end{lemmatheorembox}

\textbf{Usage count:} 1

\section*{Lemma 609}
\begin{lemmatheorembox}
\textbf{Name:} Kummer's theorem
\textbf{Statement:} Kummer's theorem states that for given integers n \ge m \ge 0 and a prime number p, the p-adic valuation \nu_{p}\!{\tbinom {n}{m}} of the binomial coefficient {\tbinom {n}{m}} is equal to the number of carries when m is added to n - m in base p.
\textbf{General Topic:} Number theory
\textbf{URL:} https://en.wikipedia.org/wiki/Kummer%27s_theorem
\end{lemmatheorembox}

\textbf{Usage count:} 1

\section*{Lemma 610}
\begin{lemmatheorembox}
\textbf{Name:} Indicator function
\textbf{Statement:} $\ \operatorname {\mathbb {E} } (\mathbf {1} _{A}(\omega ))=\operatorname {\mathbb {P} } (A)\ $
\textbf{General Topic:} Probability theory
\textbf{URL:} https://en.wikipedia.org/wiki/Indicator_function
\end{lemmatheorembox}

\textbf{Usage count:} 1

\section*{Lemma 611}
\begin{lemmatheorembox}
\textbf{Name:} Number of digits
\textbf{Statement:} The number of digits in base b of a positive integer k is \lfloor \log _{b}{k}\rfloor +1=\lceil \log _{b}{(k+1)}\rceil .
\textbf{General Topic:} Number theory
\textbf{URL:} https://en.wikipedia.org/wiki/Floor_and_ceiling_functions\#Number\_of\_digits
\end{lemmatheorembox}

\textbf{Usage count:} 1

\section*{Lemma 612}
\begin{lemmatheorembox}
\textbf{Name:} Classical definition
\textbf{Statement:} it states that probability is shared equally between all the possible outcomes, provided these outcomes can be deemed equally likely.
\textbf{General Topic:} Probability theory
\textbf{URL:} https://en.wikipedia.org/wiki/Probability_interpretations\#Classical_definition
\end{lemmatheorembox}

\textbf{Usage count:} 1

\section*{Lemma 613}
\begin{lemmatheorembox}
\textbf{Name:} Square root of non-square is irrational
\textbf{Statement:} The answer to this is that the square root of any natural number that is not a square number is irrational.
\textbf{General Topic:} Number Theory
\textbf{URL:} https://en.wikipedia.org/wiki/Quadratic_irrational_number\#Square_root_of_non-square_is_irrational
\end{lemmatheorembox}

\textbf{Usage count:} 1

\section*{Lemma 614}
\begin{lemmatheorembox}
\textbf{Name:} Problem of Apollonius
\textbf{Statement:} By contrast, an internal tangency is one in which the two circles curve in the same way at their point of contact; the two circles lie on the same side of the tangent line, and one circle encloses the other. In this case, the distance between their centers equals the difference of their radii.
\textbf{General Topic:} Euclidean geometry
\textbf{URL:} https://en.wikipedia.org/wiki/Problem_of_Apollonius
\end{lemmatheorembox}

\textbf{Usage count:} 1

\section*{Lemma 615}
\begin{lemmatheorembox}
\textbf{Name:} Steiner's calculus problem
\textbf{Statement:} Translating this solution back to Steiner's formulation, $e^{1/e}\approx 1.44467$ is the unique maximum of $f(x)=x^{1/x}$.
\textbf{General Topic:} Calculus
\textbf{URL:} https://en.wikipedia.org/wiki/Optimal_radix_choice
\end{lemmatheorembox}

\textbf{Usage count:} 1

\section*{Lemma 616}
\begin{lemmatheorembox}
\textbf{Name:} Rigid transformation
\textbf{Statement:} In mathematics, a rigid transformation (also called Euclidean transformation or Euclidean isometry) is a geometric transformation of a Euclidean space that preserves the Euclidean distance between every pair of points. The rigid transformations include rotations, translations, reflections, or any sequence of these.
\textbf{General Topic:} Euclidean geometry
\textbf{URL:} https://en.wikipedia.org/wiki/Rigid_transformation
\end{lemmatheorembox}

\textbf{Usage count:} 1

\section*{Lemma 617}
\begin{lemmatheorembox}
\textbf{Name:} Absolute value (algebra)
\textbf{Statement:} In algebra, an absolute value[ a ] is a function that generalizes the usual absolute value.[ 1 ] More precisely, if D is a field or (more generally) an integral domain, an absolute value on D is a function, commonly denoted |x|, from D to the real numbers satisfying: • |x|\geq 0 (non-negativity) • |x|=0 if and only if x=0 (positive definiteness) • |xy|=\left|x\right|\left|y\right| (multiplicativity) • \left|x+y\right|\leq \left|x\right|+\left|y\right| (triangle inequality)
\textbf{General Topic:} Algebra
\textbf{URL:} https://en.wikipedia.org/wiki/Absolute_value_%28algebra%29
\end{lemmatheorembox}

\textbf{Usage count:} 1

\section*{Lemma 618}
\begin{lemmatheorembox}
\textbf{Name:} Relationship to the max and min functions
\textbf{Statement:} Let $s,t\in \mathbb {R}$, then the following relationship to the minimum and maximum functions hold: $|t-s|=-2\min(s,t)+s+t$ and $|t-s|=2\max(s,t)-s-t$.
\textbf{General Topic:} Real analysis
\textbf{URL:} https://en.wikipedia.org/wiki/Absolute_value#Relationship_to_the_max_and_min_functions
\end{lemmatheorembox}

\textbf{Usage count:} 1

\section*{Lemma 619}
\begin{lemmatheorembox}
\textbf{Name:} Reflection symmetry
\textbf{Statement:} In mathematics, reflection symmetry, line symmetry, mirror symmetry, or mirror-image symmetry is symmetry with respect to a reflection.
\textbf{General Topic:} Geometry
\textbf{URL:} https://en.wikipedia.org/wiki/Reflection_symmetry
\end{lemmatheorembox}

\textbf{Usage count:} 1

\section*{Lemma 620}
\begin{lemmatheorembox}
\textbf{Name:} Orthodiagonal quadrilateral
\textbf{Statement:} The area K of an orthodiagonal quadrilateral equals one half the product of the lengths of the diagonals p and q: K = \frac{pq}{2}.
\textbf{General Topic:} Euclidean geometry
\textbf{URL:} https://en.wikipedia.org/wiki/Orthodiagonal_quadrilateral
\end{lemmatheorembox}

\textbf{Usage count:} 1

\section*{Lemma 621}
\begin{lemmatheorembox}
\textbf{Name:} Medial triangle
\textbf{Statement:} In Euclidean geometry, the medial triangle or midpoint triangle of a triangle △ABC is the triangle with vertices at the midpoints of the triangle's sides AB, AC, BC. It also follows from this that the perimeter of the medial triangle equals the semiperimeter of triangle △ABC, and that the area is one quarter of the area of triangle △ABC.
\textbf{General Topic:} Euclidean geometry
\textbf{URL:} https://en.wikipedia.org/wiki/Medial_triangle
\end{lemmatheorembox}

\textbf{Usage count:} 1

\section*{Lemma 622}
\begin{lemmatheorembox}
\textbf{Name:} Exponentiation
\textbf{Statement:} For positive real numbers, exponentiation to real powers can be defined in two equivalent ways, either by extending the rational powers to reals by continuity (\S\ Limits of rational exponents, below), or in terms of the logarithm of the base and the exponential function (\S\ Powers via logarithms, below). The result is always a positive real number, and the identities and properties shown above for integer exponents remain true with these definitions for real exponents.
\textbf{General Topic:} Exponentiation
\textbf{URL:} https://en.wikipedia.org/wiki/Exponentiation#Real_exponents
\end{lemmatheorembox}

\textbf{Usage count:} 1

\section*{Lemma 623}
\begin{lemmatheorembox}
\textbf{Name:} Desargues's theorem
\textbf{Statement:} Two triangles are in perspective axially if and only if they are in perspective centrally.
\textbf{General Topic:} Projective Geometry
\textbf{URL:} https://en.wikipedia.org/wiki/Desargues%27s_theorem
\end{lemmatheorembox}

\textbf{Usage count:} 1

\section*{Lemma 624}
\begin{lemmatheorembox}
\textbf{Name:} Multilinear polynomial
\textbf{Statement:} For example, in two variables:$f(x,y)=\sum_{i=0}^{1}\sum_{j=0}^{1}a_{ij}x^{i}y^{j}=a_{00}+a_{10}x+a_{01}y+a_{11}xy={\begin{pmatrix}1&x\end{pmatrix}}{\begin{pmatrix}a_{00}&a_{01}\\a_{10}&a_{11}\end{pmatrix}}{\begin{pmatrix}1\\y\end{pmatrix}}$
\textbf{General Topic:} Algebra
\textbf{URL:} https://en.wikipedia.org/wiki/Multilinear_polynomial
\end{lemmatheorembox}

\textbf{Usage count:} 1

\section*{Lemma 625}
\begin{lemmatheorembox}
\textbf{Name:} Two-point form
\textbf{Statement:} Given two different points $(x_{1}, y_{1})$ and $(x_{2}, y_{2})$, there is exactly one line that passes through them.
\textbf{General Topic:} Analytic geometry
\textbf{URL:} https://en.wikipedia.org/wiki/Linear_equation#Two-point_form
\end{lemmatheorembox}

\textbf{Usage count:} 1

\section*{Lemma 626}
\begin{lemmatheorembox}
\textbf{Name:} Linearity of expectation
\textbf{Statement:} {\begin{aligned}\operatorname {E} [X+Y]&=\operatorname {E} [X]+\operatorname {E} [Y],\\\operatorname {E} [aX]&=a\operatorname {E} [X],\end{aligned}}
\textbf{General Topic:} Probability theory
\textbf{URL:} https://en.wikipedia.org/wiki/Linearity_of_expectation
\end{lemmatheorembox}

\textbf{Usage count:} 1

\section*{Lemma 627}
\begin{lemmatheorembox}
\textbf{Name:} Convex quadrilateral
\textbf{Statement:} In a convex quadrilateral all interior angles are less than 180°, and the two diagonals both lie inside the quadrilateral.
\textbf{General Topic:} Euclidean geometry
\textbf{URL:} https://en.wikipedia.org/wiki/Quadrilateral
\end{lemmatheorembox}

\textbf{Usage count:} 1

\section*{Lemma 628}
\begin{lemmatheorembox}
\textbf{Name:} Coupon collector's problem
\textbf{Statement:} {\begin{aligned}\operatorname {E} (T)&{}=\operatorname {E} (t_{1}+t_{2}+\cdots +t_{n})\\&{}=\operatorname {E} (t_{1})+\operatorname {E} (t_{2})+\cdots +\operatorname {E} (t_{n})\\&{}={\frac {1}{p_{1}}}+{\frac {1}{p_{2}}}+\cdots +{\frac {1}{p_{n}}}\\&{}={\frac {n}{n}}+{\frac {n}{n-1}}+\cdots +{\frac {n}{1}}\\&{}=n\cdot \left({\frac {1}{1}}+{\frac {1}{2}}+\cdots +{\frac {1}{n}}\right)\\&{}=n\cdot H_{n}.\end{aligned}}
\textbf{General Topic:} Probability theory
\textbf{URL:} https://en.wikipedia.org/wiki/Coupon_collector%27s_problem
\end{lemmatheorembox}

\textbf{Usage count:} 1

\section*{Lemma 629}
\begin{lemmatheorembox}
\textbf{Name:} Negative binomial series
\textbf{Statement:} Closely related is the negative binomial series defined by the MacLaurin series for the function {\displaystyle g(x)=(1-x)^{-\alpha }}, where {\displaystyle \alpha \in \mathbb {C} } and {\displaystyle |x|<1}. Explicitly, {\displaystyle {\begin{aligned}{\frac {1}{(1-x)^{\alpha }}}&=\sum _{k=0}^{\infty }\;{\frac {g^{(k)}(0)}{k!}}\;x^{k}\\&=1+\alpha x+{\frac {\alpha (\alpha +1)}{2!}}x^{2}+{\frac {\alpha (\alpha +1)(\alpha +2)}{3!}}x^{3}+\cdots ,\end{aligned}}}
\textbf{General Topic:} Mathematical analysis
\textbf{URL:} https://en.wikipedia.org/wiki/Binomial_series#Negative_binomial_series
\end{lemmatheorembox}

\textbf{Usage count:} 1

\section*{Lemma 630}
\begin{lemmatheorembox}
\textbf{Name:} Tangent--secant theorem
\textbf{Statement:} $|PT|^2=|PG_1|\cdot|PG_2|$
\textbf{General Topic:} Euclidean geometry
\textbf{URL:} https://en.wikipedia.org/wiki/Tangent%E2%80%93secant_theorem ([en.wikipedia.org](https://en.wikipedia.org/wiki/Tangent%E2%80%93secant_theorem?utm_source=openai))
\end{lemmatheorembox}

\textbf{Usage count:} 1

\section*{Lemma 631}
\begin{lemmatheorembox}
\textbf{Name:} Law of cosines
\textbf{Statement:} $c^{2}=a^{2}+b^{2}-2ab\cos \gamma .$
\textbf{General Topic:} Trigonometry
\textbf{URL:} https://en.wikipedia.org/wiki/Law_of_cosines ([en.wikipedia.org](https://en.wikipedia.org/wiki/Law_of_cosines))
\end{lemmatheorembox}

\textbf{Usage count:} 1

\section*{Lemma 632}
\begin{lemmatheorembox}
\textbf{Name:} Circular arc
\textbf{Statement:} $L=\theta r.$
\textbf{General Topic:} Geometry
\textbf{URL:} https://en.wikipedia.org/wiki/Circular_arc
\end{lemmatheorembox}

\textbf{Usage count:} 1

\section*{Lemma 633}
\begin{lemmatheorembox}
\textbf{Name:} Equivalent conditions
\textbf{Statement:} The following statements are equivalent for a tournament $T$ on $n$ vertices:
1. $T$ is transitive.
2. $T$ is a strict total ordering.
3. $T$ is acyclic.
4. $T$ does not contain a cycle of length 3.
5. The score sequence (set of outdegrees) of $T$ is $\{0,1,2,\ldots,n-1\}$.
6. $T$ has exactly one Hamiltonian path.
\textbf{General Topic:} Graph theory
\textbf{URL:} https://en.wikipedia.org/wiki/Tournament_(graph_theory)
\end{lemmatheorembox}

\textbf{Usage count:} 1

\section*{Lemma 634}
\begin{lemmatheorembox}
\textbf{Name:} Non-standard positional numeral systems
\textbf{Statement:} The value of a digit string like pqrs in base b is given by the polynomial form $p\times b^{3}+q\times b^{2}+r\times b+s$.
\textbf{General Topic:} Numeral systems
\textbf{URL:} https://en.wikipedia.org/wiki/Non-standard_positional_numeral_systems
\end{lemmatheorembox}

\textbf{Usage count:} 1

\section*{Lemma 635}
\begin{lemmatheorembox}
\textbf{Name:} Locus (mathematics)
\textbf{Statement:} The set of points equidistant from two intersecting lines is the union of their two angle bisectors.
\textbf{General Topic:} Geometry
\textbf{URL:} https://en.wikipedia.org/wiki/Locus_%28mathematics%29
\end{lemmatheorembox}

\textbf{Usage count:} 1

\section*{Lemma 636}
\begin{lemmatheorembox}
\textbf{Name:} Euler's formula
\textbf{Statement:} Euler's formula states that if a finite, connected, planar graph is drawn in the plane without any edge intersections, and v is the number of vertices, e is the number of edges and f is the number of faces (regions bounded by edges, including the outer, infinitely large region), then v-e+f=2.
\textbf{General Topic:} Graph Theory
\textbf{URL:} https://en.wikipedia.org/wiki/Planar_graph#Euler's_formula
\end{lemmatheorembox}

\textbf{Usage count:} 1

\section*{Lemma 637}
\begin{lemmatheorembox}
\textbf{Name:} Bauer maximum principle
\textbf{Statement:} Any function that is convex and continuous, and defined on a set that is convex and compact, attains its maximum at some extreme point of that set.
\textbf{General Topic:} Mathematical optimization
\textbf{URL:} https://en.wikipedia.org/wiki/Bauer_maximum_principle
\end{lemmatheorembox}

\textbf{Usage count:} 1

\section*{Lemma 638}
\begin{lemmatheorembox}
\textbf{Name:} Altitude (triangle)
\textbf{Statement:} Also the altitude having the incongruent side as its base will be the angle bisector of the vertex angle.
\textbf{General Topic:} Geometry
\textbf{URL:} https://en.wikipedia.org/wiki/Altitude_(triangle)
\end{lemmatheorembox}

\textbf{Usage count:} 1

\section*{Lemma 639}
\begin{lemmatheorembox}
\textbf{Name:} Arithmetic function
\textbf{Statement:} $\sigma _{k}(n)$ is the sum of the k th powers of the positive divisors of n, including 1 and n, where k is a complex number. $\sigma _{k}(n)=\prod _{i=1}^{\omega (n)}{\frac {p_{i}^{(a_{i}+1)k}-1}{p_{i}^{k}-1}}=\prod _{i=1}^{\omega (n)}\left(1+p_{i}^{k}+p_{i}^{2k}+\cdots +p_{i}^{a_{i}k}\right)$.
\textbf{General Topic:} Number theory
\textbf{URL:} https://en.wikipedia.org/wiki/Arithmetic_function
\end{lemmatheorembox}

\textbf{Usage count:} 1

\section*{Lemma 640}
\begin{lemmatheorembox}
\textbf{Name:} Rule of division (combinatorics)
\textbf{Statement:} It states that there are $n/d$ ways to do a task if it can be done using a procedure that can be carried out in $n$ ways, and for each way $w$, exactly $d$ of the $n$ ways correspond to the way $w$.
\textbf{General Topic:} Combinatorics
\textbf{URL:} https://en.wikipedia.org/wiki/Rule_of_division_%28combinatorics%29
\end{lemmatheorembox}

\textbf{Usage count:} 1

\section*{Lemma 641}
\begin{lemmatheorembox}
\textbf{Name:} Frullani integral
\textbf{Statement:} The following formula for their general solution holds if f is continuous on (0,\infty), has finite limit at \infty, and a,b>0:
\int _{0}^{\infty }{\frac {f(ax)-f(bx)}{x}}\,{\rm {d}}x={\Big (}f(\infty )-f(0){\Big )}\ln {\frac {a}{b}}.
If f(\infty) does not exist, but \int_c^\infty \frac{f(x)}{x}\,dx exists for some c>0, then \int _{0}^{\infty }{\frac {f(ax)-f(bx)}{x}}\,{\rm {d}}x=-f(0)\ln {\frac {a}{b}}.
\textbf{General Topic:} Real Analysis
\textbf{URL:} https://en.wikipedia.org/wiki/Frullani_integral
\end{lemmatheorembox}

\textbf{Usage count:} 1

\section*{Lemma 642}
\begin{lemmatheorembox}
\textbf{Name:} Vector area
\textbf{Statement:} The projected area onto a plane is given by the dot product of the vector area $\mathbf{S}$ and the target plane unit normal $\hat{\mathbf{m}}$: $A_{\parallel}=\mathbf{S}\cdot \hat{\mathbf{m}}$
\textbf{General Topic:} Vector calculus
\textbf{URL:} https://en.wikipedia.org/wiki/Vector_area
\end{lemmatheorembox}

\textbf{Usage count:} 1

\section*{Lemma 643}
\begin{lemmatheorembox}
\textbf{Name:} Absolutely continuous probability distribution
\textbf{Statement:} In particular, the probability for $X$ to take any single value $a$ (that is, $a\leq X\leq a$) is zero, because an integral with coinciding upper and lower limits is always equal to zero.
\textbf{General Topic:} Probability theory
\textbf{URL:} https://en.wikipedia.org/wiki/Probability_distribution
\end{lemmatheorembox}

\textbf{Usage count:} 1

\section*{Lemma 644}
\begin{lemmatheorembox}
\textbf{Name:} Area of a triangle
\textbf{Statement:} T={\tfrac {1}{2}}ab\sin \gamma ={\tfrac {1}{2}}bc\sin \alpha ={\tfrac {1}{2}}ca\sin \beta
\textbf{General Topic:} Geometry
\textbf{URL:} https://en.wikipedia.org/wiki/Area_of_a_triangle#Knowing_SAS_(side-angle-side)
\end{lemmatheorembox}

\textbf{Usage count:} 1

\section*{Lemma 645}
\begin{lemmatheorembox}
\textbf{Name:} Kolmogorov axioms
\textbf{Statement:} This is the assumption of σ-additivity: Any countable sequence of disjoint sets (synonymous with mutually exclusive events) $E_{1},E_{2},\ldots$ satisfies $P\left(\bigcup _{i=1}^{\infty }E_{i}\right)=\sum _{i=1}^{\infty }P(E_{i}).$
\textbf{General Topic:} Probability theory
\textbf{URL:} https://en.wikipedia.org/wiki/Probability_axioms ([en.wikipedia.org](https://en.wikipedia.org/wiki/Probability_axioms))
\end{lemmatheorembox}

\textbf{Usage count:} 1

\section*{Lemma 646}
\begin{lemmatheorembox}
\textbf{Name:} Statistical independence
\textbf{Statement:} Events A and B are defined to be statistically independent if the probability of the intersection of A and B is equal to the product of the probabilities of A and B: $P(A\cap B)=P(A)P(B).$
\textbf{General Topic:} Probability theory
\textbf{URL:} https://en.wikipedia.org/wiki/Conditional_probability#Statistical_independence ([en.wikipedia.org](https://en.wikipedia.org/wiki/Conditional_probability))
\end{lemmatheorembox}

\textbf{Usage count:} 1

\section*{Lemma 647}
\begin{lemmatheorembox}
\textbf{Name:} Euler product formula for the Riemann zeta function
\textbf{Statement:} The Euler product formula for the Riemann zeta function reads $\zeta (s)=\sum _{n=1}^{\infty }{\frac {1}{n^{s}}}=\prod _{p{\text{ prime}}}{\frac {1}{1-p^{-s}}}$.
\textbf{General Topic:} Number theory
\textbf{URL:} https://en.wikipedia.org/wiki/Proof_of_the_Euler_product_formula_for_the_Riemann_zeta_function
\end{lemmatheorembox}

\textbf{Usage count:} 1

\section*{Lemma 648}
\begin{lemmatheorembox}
\textbf{Name:} k-permutations of n
\textbf{Statement:} computed by the formula:
\(P(n,k)=\underbrace{n\cdot (n-1)\cdot (n-2)\cdots (n-k+1)}_{k\ \mathrm{factors}},\)
which is 0 when \(k > n\), and otherwise is equal to
\(\frac{n!}{(n-k)!}.\)
\textbf{General Topic:} Combinatorics
\textbf{URL:} https://en.wikipedia.org/wiki/Permutation\#k-permutations\_of\_n
\end{lemmatheorembox}

\textbf{Usage count:} 1

\section*{Lemma 649}
\begin{lemmatheorembox}
\textbf{Name:} Addition principle
\textbf{Statement:} Stated simply, it is the intuitive idea that if we have A number of ways of doing something and B number of ways of doing another thing and we can not do both at the same time, then there are \(A+B\) ways to choose one of the actions. In mathematical terms, the addition principle states that, for disjoint sets A and B, we have \(|A\cup B|=|A|+|B|\), provided that the intersection of the sets is without any elements.
\textbf{General Topic:} Combinatorics
\textbf{URL:} https://en.wikipedia.org/wiki/Addition\_principle
\end{lemmatheorembox}

\textbf{Usage count:} 1

\section*{Lemma 650}
\begin{lemmatheorembox}
\textbf{Name:} Rule of product
\textbf{Statement:} In combinatorics, the rule of product or multiplication principle is a basic counting principle (a.k.a. the fundamental principle of counting). Stated simply, it is the intuitive idea that if there are a ways of doing something and b ways of doing another thing, then there are a · b ways of performing both actions.
\textbf{General Topic:} Combinatorics
\textbf{URL:} https://en.wikipedia.org/wiki/Rule\_of\_product
\end{lemmatheorembox}

\textbf{Usage count:} 1

\section*{Lemma 651}
\begin{lemmatheorembox}
\textbf{Name:} Proof by contradiction
\textbf{Statement:} For a set of consistent premises \Sigma and a proposition \varphi, it is true in classical logic that \Sigma \vdash\varphi (i.e., \Sigma proves \varphi) if and only if \Sigma \cup \{\neg\varphi\} \vdash \bot (i.e., \Sigma and \neg\varphi leads to a contradiction).
\textbf{General Topic:} Logic
\textbf{URL:} https://en.wikipedia.org/wiki/Contradiction
\end{lemmatheorembox}

\textbf{Usage count:} 1

\section*{Lemma 652}
\begin{lemmatheorembox}
\textbf{Name:} Fundamental theorem of equivalence relations
\textbf{Statement:} An equivalence relation \textasciitilde{} on a set X partitions X. Conversely, corresponding to any partition of X, there exists an equivalence relation \textasciitilde{} on X.
\textbf{General Topic:} Set Theory
\textbf{URL:} https://en.wikipedia.org/wiki/Equivalence_relation\#Fundamental_theorem_of_equivalence_relations
\end{lemmatheorembox}

\textbf{Usage count:} 1

\section*{Lemma 653}
\begin{lemmatheorembox}
\textbf{Name:} Playfair's axiom
\textbf{Statement:} In a plane, given a line and a point not on it, at most one line parallel to the given line can be drawn through the point.
\textbf{General Topic:} Euclidean geometry
\textbf{URL:} https://en.wikipedia.org/wiki/Playfair%27s_axiom
\end{lemmatheorembox}

\textbf{Usage count:} 1

\section*{Lemma 654}
\begin{lemmatheorembox}
\textbf{Name:} Cavalieri's principle
\textbf{Statement:} If every plane parallel to these two planes intersects both regions in cross-sections of equal area, then the two regions have equal volumes.
\textbf{General Topic:} Geometry
\textbf{URL:} https://en.wikipedia.org/wiki/Cavalieri%27s_principle
\end{lemmatheorembox}

\textbf{Usage count:} 1

\section*{Lemma 655}
\begin{lemmatheorembox}
\textbf{Name:} Harmonic number
\textbf{Statement:} $\lim _{n\to \infty }\left(H_{n}-\ln n\right)=\gamma ,$
\textbf{General Topic:} Number Theory
\textbf{URL:} https://en.wikipedia.org/wiki/Harmonic_number
\end{lemmatheorembox}

\textbf{Usage count:} 1

\section*{Lemma 656}
\begin{lemmatheorembox}
\textbf{Name:} Dini's theorem
\textbf{Statement:} If $X$ is a compact topological space, and $(f_n)_{n\in\mathbb{N}}$ is a monotonically increasing sequence (meaning $f_n(x)\leq f_{n+1}(x)$ for all $n\in\mathbb{N}$ and $x\in X$) of continuous real-valued functions on $X$ which converges pointwise to a continuous function $f\colon X\to \mathbb{R}$, then the convergence is uniform.
\textbf{General Topic:} Real analysis
\textbf{URL:} https://en.wikipedia.org/wiki/Dini%27s_theorem
\end{lemmatheorembox}

\textbf{Usage count:} 1

\section*{Lemma 657}
\begin{lemmatheorembox}
\textbf{Name:} Niven's theorem
\textbf{Statement:} In mathematics, Niven's theorem, named after Ivan Niven, states that the only rational values of $\theta$ in the interval $0^\circ \leq \theta \leq 90^\circ$ for which the sine of $\theta$ degrees is also a rational number are:
\[
\begin{aligned}
\sin 0^{\circ }& = 0, \\[10pt]
\sin 30^{\circ }& = \frac 12, \\[10pt]
\sin 90^{\circ }& = 1.
\end{aligned}
\]
In radians, one would require that $0^\circ \leq x \leq \pi/2$, that $x/\pi$ be rational, and that $\sin(x)$ be rational. The conclusion is then that the only such values are $\sin(0) = 0$, $\sin(\pi/6) = 1/2$, and $\sin(\pi/2) = 1$.
\textbf{General Topic:} Trigonometry
\textbf{URL:} https://en.wikipedia.org/wiki/Niven%27s_theorem
\end{lemmatheorembox}

\textbf{Usage count:} 1

\section*{Lemma 658}
\begin{lemmatheorembox}
\textbf{Name:} Irrational rotation
\textbf{Statement:} If $\theta$ is irrational, then the orbit of any element of $[0, 1]$ under the rotation $T_{\theta}$ is dense in $[0, 1]$. Therefore, irrational rotations are topologically transitive.
\textbf{General Topic:} Dynamical systems
\textbf{URL:} https://en.wikipedia.org/wiki/Irrational_rotation
\end{lemmatheorembox}

\textbf{Usage count:} 1

\section*{Lemma 659}
\begin{lemmatheorembox}
\textbf{Name:} Change of variables formula in terms of Lebesgue measure
\textbf{Statement:} Suppose that $\Omega$ is an open subset of $\mathbb{R}^{n}$ and $G:\Omega \to \mathbb{R}^{n}$ is a $C^{1}$ diffeomorphism.
\begin{itemize}
\item If $f$ is a Lebesgue measurable function on $G(\Omega)$, then $f\circ G$ is Lebesgue measurable on $\Omega$. If $f\geq 0$ or $f\in L^{1}(G(\Omega),m),$ then $\int _{G(\Omega )}f(x)dx=\int _{\Omega }f\circ G(x)|\text{det}D_{x}G|dx$.
\item If $E\subset \Omega$ and $E$ is Lebesgue measurable, then $G(E)$ is Lebesgue measurable, then $m(G(E))=\int _{E}|\text{det}D_{x}G|dx$.
\end{itemize}
\textbf{General Topic:} Measure Theory
\textbf{URL:} https://en.wikipedia.org/wiki/Change_of_variables
\end{lemmatheorembox}

\textbf{Usage count:} 1

\section*{Lemma 660}
\begin{lemmatheorembox}
\textbf{Name:} Cycle (graph theory)
\textbf{Statement:} In graph theory, a cycle in a graph is a non-empty trail in which only the first and last vertices are equal.
\textbf{General Topic:} Graph theory
\textbf{URL:} https://en.wikipedia.org/wiki/Cycle_%28graph_theory%29
\end{lemmatheorembox}

\textbf{Usage count:} 1

\section*{Lemma 661}
\begin{lemmatheorembox}
\textbf{Name:} Dihedral group
\textbf{Statement:} A regular polygon with n sides has 2n different symmetries: n rotational symmetries and n reflection symmetries; here, n \ge 3.
\textbf{General Topic:} Group theory
\textbf{URL:} https://en.wikipedia.org/wiki/Dihedral_group
\end{lemmatheorembox}

\textbf{Usage count:} 1

\section*{Lemma 662}
\begin{lemmatheorembox}
\textbf{Name:} Sum of first n cubes
\textbf{Statement:} The sum of the first n cubes is the n th triangle number squared: \(1^{3}+2^{3}+\dots +n^{3}=(1+2+\dots +n)^{2}=\left({\frac {n(n+1)}{2}}\right)^{2}.\)
\textbf{General Topic:} Elementary arithmetic
\textbf{URL:} https://en.wikipedia.org/wiki/Cube_%28algebra%29
\end{lemmatheorembox}

\textbf{Usage count:} 1

\section*{Lemma 663}
\begin{lemmatheorembox}
\textbf{Name:} Rule of sum
\textbf{Statement:} If there are $a$ ways to do one thing and $b$ ways to do another thing, and these cannot both be done, then there are $a+b$ ways to choose one of the things.
\textbf{General Topic:} Combinatorics
\textbf{URL:} https://en.wikipedia.org/wiki/Rule_of_sum
\end{lemmatheorembox}

\textbf{Usage count:} 1

\section*{Lemma 664}
\begin{lemmatheorembox}
\textbf{Name:} Arithmetic progression
\textbf{Statement:} {\displaystyle {\frac {n(a_{1}+a_{n})}{2}}}. This formula works for any arithmetic progression of real numbers beginning with {\displaystyle a_{1}} and ending with {\displaystyle a_{n}}.
\textbf{General Topic:} Sequences and series
\textbf{URL:} https://en.wikipedia.org/wiki/Arithmetic_progression#Sum
\end{lemmatheorembox}

\textbf{Usage count:} 1

\section*{Lemma 665}
\begin{lemmatheorembox}
\textbf{Name:} Laws of reflection
\textbf{Statement:} If the reflecting surface is very smooth, the reflection of light that occurs is called specular or regular reflection. The laws of reflection are as follows:
1 The incident ray, the reflected ray and the normal to the reflection surface at the point of the incidence lie in the same plane.
2 The angle which the incident ray makes with the normal is equal to the angle which the reflected ray makes to the same normal.
3 The reflected ray and the incident ray are on the opposite sides of the normal.
\textbf{General Topic:} Optics
\textbf{URL:} https://en.wikipedia.org/wiki/Reflection_(physics)#Laws_of_reflection
\end{lemmatheorembox}

\textbf{Usage count:} 1

\section*{Lemma 666}
\begin{lemmatheorembox}
\textbf{Name:} Midpoint
\textbf{Statement:} In a right triangle, the circumcenter is the midpoint of the hypotenuse.
\textbf{General Topic:} Euclidean geometry
\textbf{URL:} https://en.wikipedia.org/wiki/Midpoint
\end{lemmatheorembox}

\textbf{Usage count:} 1

\section*{Lemma 667}
\begin{lemmatheorembox}
\textbf{Name:} Cayley--Hamilton theorem
\textbf{Statement:} In linear algebra, the Cayley--Hamilton theorem (named after the mathematicians Arthur Cayley and William Rowan Hamilton) states that every square matrix over a commutative ring (such as the real or complex numbers or the integers) satisfies its own characteristic equation.
\textbf{General Topic:} Linear algebra
\textbf{URL:} https://en.wikipedia.org/wiki/Cayley%E2%80%93Hamilton_theorem
\end{lemmatheorembox}

\textbf{Usage count:} 1

\section*{Lemma 668}
\begin{lemmatheorembox}
\textbf{Name:} Minimal polynomial (linear algebra)
\textbf{Statement:} In linear algebra, the minimal polynomial \(\mu_{A}\) of an \(n\times n\) matrix \(A\) over a field \(F\) is the monic polynomial \(P\) over \(F\) of least degree such that \(P(A) = 0\). Any other polynomial \(Q\) with \(Q(A) = 0\) is a (polynomial) multiple of \(\mu_{A}\).
\textbf{General Topic:} Linear algebra
\textbf{URL:} https://en.wikipedia.org/wiki/Minimal_polynomial_%28linear_algebra%29
\end{lemmatheorembox}

\textbf{Usage count:} 1

\section*{Lemma 669}
\begin{lemmatheorembox}
\textbf{Name:} Riemann–Stieltjes integral
\textbf{Statement:} The Riemann–Stieltjes integral admits integration by parts in the form

$\int_a^b f(x) \, \mathrm{d}g(x)=f(b)g(b)-f(a)g(a)-\int_a^b g(x) \, \mathrm{d}f(x)$

and the existence of either integral implies the existence of the other.
\textbf{General Topic:} Mathematical analysis
\textbf{URL:} https://en.wikipedia.org/wiki/Riemann%E2%80%93Stieltjes_integral
\end{lemmatheorembox}

\textbf{Usage count:} 1

\section*{Lemma 670}
\begin{lemmatheorembox}
\textbf{Name:} Euclidean planes in three-dimensional space
\textbf{Statement:} In a Euclidean space of any number of dimensions, a plane is uniquely determined by any of the following:
\begin{itemize}
  \item Three non-collinear points (points not on a single line).
  \item A line and a point not on that line.
  \item Two distinct but intersecting lines.
  \item Two distinct but parallel lines.
\end{itemize}
\textbf{General Topic:} Euclidean geometry
\textbf{URL:} https://en.wikipedia.org/wiki/Euclidean_planes_in_three-dimensional_space
\end{lemmatheorembox}

\textbf{Usage count:} 1

\section*{Lemma 671}
\begin{lemmatheorembox}
\textbf{Name:} Bienaymé's identity
\textbf{Statement:} $\operatorname {Var} \left(\sum _{i=1}^{n}X_{i}\right)=\sum _{k=1}^{n}\operatorname {Var} (X_{k}).$
\textbf{General Topic:} Probability theory
\textbf{URL:} https://en.wikipedia.org/wiki/Bienaym%C3%A9%27s_identity
\end{lemmatheorembox}

\textbf{Usage count:} 1

\section*{Lemma 672}
\begin{lemmatheorembox}
\textbf{Name:} Multiplicative function
\textbf{Statement:} A multiplicative function is completely determined by its values at the powers of prime numbers, a consequence of the fundamental theorem of arithmetic. Thus, if n is a product of powers of distinct primes, say n = p^a q^b ..., then f(n) = f(p^a) f(q^b) ...
\textbf{General Topic:} Number theory
\textbf{URL:} https://en.wikipedia.org/wiki/Multiplicative_function
\end{lemmatheorembox}

\textbf{Usage count:} 1

\section*{Lemma 673}
\begin{lemmatheorembox}
\textbf{Name:} fundamental theorem of cyclic groups
\textbf{Statement:} In abstract algebra, every subgroup of a cyclic group is cyclic. Moreover, for a finite cyclic group of order n, every subgroup's order is a divisor of n, and there is exactly one subgroup for each divisor.
\textbf{General Topic:} Group theory
\textbf{URL:} https://en.wikipedia.org/wiki/Subgroups_of_cyclic_groups
\end{lemmatheorembox}

\textbf{Usage count:} 1

\section*{Lemma 674}
\begin{lemmatheorembox}
\textbf{Name:} Cauchy's theorem (group theory)
\textbf{Statement:} if G is a finite group and p is a prime number dividing the order of G, then G contains an element of order p.
\textbf{General Topic:} Group theory
\textbf{URL:} https://en.wikipedia.org/wiki/Cauchy%27s_theorem_%28group_theory%29
\end{lemmatheorembox}

\textbf{Usage count:} 1

\section*{Lemma 675}
\begin{lemmatheorembox}
\textbf{Name:} First isomorphism theorem
\textbf{Statement:} The image of f is isomorphic to the quotient group G/ker f.
\textbf{General Topic:} Group theory
\textbf{URL:} https://en.wikipedia.org/wiki/Isomorphism_theorems
\end{lemmatheorembox}

\textbf{Usage count:} 1

\section*{Lemma 676}
\begin{lemmatheorembox}
\textbf{Name:} Breadth-first search
\textbf{Statement:} Finding the shortest path between two nodes u and v, with path length measured by number of edges (an advantage over depth-first search)
\textbf{General Topic:} Graph theory
\textbf{URL:} https://en.wikipedia.org/wiki/Breadth-first_search
\end{lemmatheorembox}

\textbf{Usage count:} 1

\section*{Lemma 677}
\begin{lemmatheorembox}
\textbf{Name:} Casting out nines
\textbf{Statement:} Adding the decimal digits of a positive whole number, while optionally ignoring any 9s or digits which sum to 9 or a multiple of 9. The result of this procedure is a number which is smaller than the original whenever the original has more than one digit, leaves the same remainder as the original after division by nine, and may be obtained from the original by subtracting a multiple of 9 from it.
\textbf{General Topic:} Arithmetic
\textbf{URL:} https://en.wikipedia.org/wiki/Casting_out_nines
\end{lemmatheorembox}

\textbf{Usage count:} 1

\section*{Lemma 678}
\begin{lemmatheorembox}
\textbf{Name:} Perfect matching
\textbf{Statement:} The number of perfect matchings in a complete graph $K_{n}$ (with $n$ even) is given by the double factorial: $(n-1)!!$
\textbf{General Topic:} Graph theory
\textbf{URL:} https://en.wikipedia.org/wiki/Perfect_matching
\end{lemmatheorembox}

\textbf{Usage count:} 1

\section*{Lemma 679}
\begin{lemmatheorembox}
\textbf{Name:} Coin problem
\textbf{Statement:} If n = 2, the Frobenius number can be found from the formula $g(a_{1},a_{2})=a_{1}a_{2}-a_{1}-a_{2}$, which was discovered by James Joseph Sylvester in 1882.
\textbf{General Topic:} Number Theory
\textbf{URL:} https://en.wikipedia.org/wiki/Coin_problem
\end{lemmatheorembox}

\textbf{Usage count:} 1

\section*{Lemma 680}
\begin{lemmatheorembox}
\textbf{Name:} Angular defect
\textbf{Statement:} However, on a convex polyhedron, the angles of the faces meeting at a vertex add up to less than 360° (a defect), while the angles at some vertices of a nonconvex polyhedron may add up to more than 360° (an excess).
\textbf{General Topic:} Geometry
\textbf{URL:} https://en.wikipedia.org/wiki/Angular_defect
\end{lemmatheorembox}

\textbf{Usage count:} 1

\section*{Lemma 681}
\begin{lemmatheorembox}
\textbf{Name:} Euler's polyhedron formula
\textbf{Statement:} Any convex polyhedron's surface has Euler characteristic $V-E+F=2,$ where $V$ is the number of vertices, $E$ is the number of edges, and $F$ is the number of faces.
\textbf{General Topic:} Polyhedral combinatorics
\textbf{URL:} https://en.wikipedia.org/wiki/Edge_(geometry)
\end{lemmatheorembox}

\textbf{Usage count:} 1

\section*{Lemma 682}
\begin{lemmatheorembox}
\textbf{Name:} Fractional part
\textbf{Statement:} The fractional part defined via difference from $\lfloor\ \rfloor$ is usually denoted by curly braces:
\[
\{x\}:=x-\lfloor x\rfloor .
\]
\textbf{General Topic:} Arithmetic
\textbf{URL:} https://en.wikipedia.org/wiki/Fractional_part
\end{lemmatheorembox}

\textbf{Usage count:} 1

\section*{Lemma 683}
\begin{lemmatheorembox}
\textbf{Name:} 1 + 2 + 3 + 4 + ⋯
\textbf{Statement:} 
\[
\sum _{k=1}^{n}k={\frac {n(n+1)}{2}},
\]
\textbf{General Topic:} Series (Mathematics)
\textbf{URL:} https://en.wikipedia.org/wiki/1_%2B_2_%2B_3_%2B_4_%2B_%E2%8B%AF
\end{lemmatheorembox}

\textbf{Usage count:} 1

\section*{Lemma 684}
\begin{lemmatheorembox}
\textbf{Name:} Lagrange's theorem (number theory)

\textbf{Statement:} This can be stated with congruence classes as follows: for all polynomials \(f \in (\mathbb{Z}/p\mathbb{Z})[x]\) with \(p\) prime, either:
* every coefficient of \(f\) is null, or
* \(f(x)=0\) has at most \(\deg(f)\) solutions in \(\mathbb{Z}/p\mathbb{Z}\).

\textbf{General Topic:} Number Theory

\textbf{URL:} https://en.wikipedia.org/wiki/Lagrange%27s_theorem_%28number_theory%29
\end{lemmatheorembox}

\textbf{Usage count:} 1

\section*{Lemma 685}
\begin{lemmatheorembox}
\textbf{Name:} Radix
\textbf{Statement:} This representation is unique. Let b be a positive integer greater than 1. Then every positive integer a can be expressed uniquely in the form {\displaystyle a=r_{m}b^{m}+r_{m-1}b^{m-1}+\dotsb +r_{1}b+r_{0},} where m is a nonnegative integer and the r's are integers such that {\displaystyle 0<r_{m}<b} and {\displaystyle 0\leq r_{i}<b} for i = 0, 1, ... , m − 1.
\textbf{General Topic:} Number Theory
\textbf{URL:} https://en.wikipedia.org/wiki/Radix
\end{lemmatheorembox}

\textbf{Usage count:} 1

\section*{Lemma 686}
\begin{lemmatheorembox}
\textbf{Name:} Tangent circles
\textbf{Statement:} Two circles are mutually and externally tangent if the distance between their centers is equal to the sum of their radii.
\textbf{General Topic:} Geometry
\textbf{URL:} https://en.wikipedia.org/wiki/Tangent_circles
\end{lemmatheorembox}

\textbf{Usage count:} 1

\section*{Lemma 687}
\begin{lemmatheorembox}
\textbf{Name:} Pythagorean theorem
\textbf{Statement:} The sum of the areas of the two squares on the legs (a and b) equals the area of the square on the hypotenuse (c).
\textbf{General Topic:} Euclidean geometry
\textbf{URL:} https://en.wikipedia.org/wiki/Pythagorean_theorem ([en.wikipedia.org](https://en.wikipedia.org/wiki/Pythagorean_theorem))
\end{lemmatheorembox}

\textbf{Usage count:} 1

\section*{Lemma 688}
\begin{lemmatheorembox}
\textbf{Name:} Triangle inequality
\textbf{Statement:} d(A,\ C)\le d(A,\ B)+d(B,\ C)\ ,
\textbf{General Topic:} Metric spaces
\textbf{URL:} https://en.wikipedia.org/wiki/Triangle_inequality ([en.wikipedia.org](https://en.wikipedia.org/wiki/Triangle_inequality))
\end{lemmatheorembox}

\textbf{Usage count:} 1

\section*{Lemma 689}
\begin{lemmatheorembox}
\textbf{Name:} Arithmetic progression
\textbf{Statement:} Arithmetic progression, a sequence of numbers such that the difference between any two successive members of the sequence is a constant
\textbf{General Topic:} Sequences
\textbf{URL:} https://en.wikipedia.org/wiki/Progression
\end{lemmatheorembox}

\textbf{Usage count:} 1

\section*{Lemma 690}
\begin{lemmatheorembox}
\textbf{Name:} Minimum bounding box algorithms
\textbf{Statement:} It is based on the observation that a side of a minimum-area enclosing box must be colinear with a side of the convex polygon.
\textbf{General Topic:} Computational geometry
\textbf{URL:} https://en.wikipedia.org/wiki/Minimum_bounding_box_algorithms
\end{lemmatheorembox}

\textbf{Usage count:} 1

\section*{Lemma 691}
\begin{lemmatheorembox}
\textbf{Name:} Measure (mathematics)
\textbf{Statement:} Countable additivity (or σ-additivity): For all countable collections \{E_{k}\}_{k=1}^{\infty } of pairwise disjoint sets in \Sigma,\mu {\left(\bigcup _{k=1}^{\infty }E_{k}\right)}=\sum _{k=1}^{\infty }\mu (E_{k})
\textbf{General Topic:} Measure theory
\textbf{URL:} https://en.wikipedia.org/wiki/Measure_%28mathematics%29
\end{lemmatheorembox}

\textbf{Usage count:} 1

\section*{Lemma 692}
\begin{lemmatheorembox}
\textbf{Name:} Euclidean plane isometry
\textbf{Statement:} As Alice found through the looking-glass, a single mirror causes left and right hands to switch. (In formal terms, topological orientation is reversed.)
\textbf{General Topic:} Geometry
\textbf{URL:} https://en.wikipedia.org/wiki/Euclidean_plane_isometry
\end{lemmatheorembox}

\textbf{Usage count:} 1

\section*{Lemma 693}
\begin{lemmatheorembox}
\textbf{Name:} Linearity
\textbf{Statement:} \(\int _{a}^{b}(\alpha f+\beta g)(x)\,dx=\alpha \int _{a}^{b}f(x)\,dx+\beta \int _{a}^{b}g(x)\,dx.\,\)
\textbf{General Topic:} Calculus
\textbf{URL:} https://en.wikipedia.org/wiki/Integral
\end{lemmatheorembox}

\textbf{Usage count:} 1

\section*{Lemma 694}
\begin{lemmatheorembox}
\textbf{Name:} Powers of negative one
\textbf{Statement:} Since a negative number times another negative is positive, we have: \[(-1)^{n}=\left\{\begin{array}{rl}1&{\text{for even }}n,\\-1&{\text{for odd }}n.\\\end{array}\right.\]
\textbf{General Topic:} Exponentiation
\textbf{URL:} https://en.wikipedia.org/wiki/Exponentiation\#Powers\_of\_negative\_one
\end{lemmatheorembox}

\textbf{Usage count:} 1

\section*{Lemma 695}
\begin{lemmatheorembox}
\textbf{Name:} Rule of product
\textbf{Statement:} The rule of product is another intuitive principle stating that if there are a ways to do something and b ways to do another thing, then there are a·b ways to do both things.
\textbf{General Topic:} Combinatorics
\textbf{URL:} https://en.wikipedia.org/wiki/Combinatorial_principles\#Rule_of_product
\end{lemmatheorembox}

\textbf{Usage count:} 1

\section*{Lemma 696}
\begin{lemmatheorembox}
\textbf{Name:} Substitution property of equality
\textbf{Statement:} (a=b) \implies \bigl[ \phi(a) \Rightarrow \phi(b) \bigr]
\textbf{General Topic:} Logic
\textbf{URL:} https://en.wikipedia.org/wiki/Substitution_(logic)
\end{lemmatheorembox}

\textbf{Usage count:} 1

\section*{Lemma 697}
\begin{lemmatheorembox}
\textbf{Name:} Complement rule
\textbf{Statement:} $P(A^{c})=1-P(A).$
\textbf{General Topic:} Probability Theory
\textbf{URL:} https://en.wikipedia.org/wiki/Complementary_event#Complement_rule
\end{lemmatheorembox}

\textbf{Usage count:} 1

\section*{Lemma 698}
\begin{lemmatheorembox}
\textbf{Name:} Euclid's Elements
\textbf{Statement:} For example, in the first construction of Book 1, of an equilateral triangle, Euclid used a premise that was neither postulated nor proved: that two circles sharing the same line segment as a radius will cross each other in two points, rather than somehow not crossing.
\textbf{General Topic:} Euclidean geometry
\textbf{URL:} https://en.wikipedia.org/wiki/Euclid%27s_Elements
\end{lemmatheorembox}

\textbf{Usage count:} 1

\section*{Lemma 699}
\begin{lemmatheorembox}
\textbf{Name:} Trairāśika
\textbf{Statement:} In the contemporary mathematical literature, the term "rule of three" refers to the principle of cross-multiplication which states that if $\tfrac{a}{b}=\tfrac{c}{d}$ then $ad=bc$ or $a=\tfrac{bc}{d}$.
\textbf{General Topic:} Arithmetic
\textbf{URL:} https://en.wikipedia.org/wiki/Trair%C4%81%C5%9Bika
\end{lemmatheorembox}

\textbf{Usage count:} 1

\section*{Lemma 700}
\begin{lemmatheorembox}
\textbf{Name:} Fundamental rule of proportion
\textbf{Statement:} If \ \frac ab=\frac cd, then \ ad=bc
\textbf{General Topic:} Elementary Algebra
\textbf{URL:} https://en.wikipedia.org/wiki/Proportion_%28mathematics%29
\end{lemmatheorembox}

\textbf{Usage count:} 1

\section*{Lemma 701}
\begin{lemmatheorembox}
\textbf{Name:} Properties of equality
\textbf{Statement:} if $a=b$ and $c=d$ then $a+c=b+d$ and $ac=bd$;
\textbf{General Topic:} Elementary algebra
\textbf{URL:} https://en.wikipedia.org/wiki/Elementary_algebra
\end{lemmatheorembox}

\textbf{Usage count:} 1

\section*{Lemma 702}
\begin{lemmatheorembox}
\textbf{Name:} Sign function
\textbf{Statement:} $\operatorname {sgn} (xy)=(\operatorname {sgn} x)(\operatorname {sgn} y)\,,$
\textbf{General Topic:} Real Analysis
\textbf{URL:} https://en.wikipedia.org/wiki/Sign_function
\end{lemmatheorembox}

\textbf{Usage count:} 1

\section*{Lemma 703}
\begin{lemmatheorembox}
\textbf{Name:} Function application
\textbf{Statement:} For every a and b, with some function f(x), if a = b, then f(a) = f(b).
\textbf{General Topic:} Logic
\textbf{URL:} https://en.wikipedia.org/wiki/Equality_(mathematics)#Basic_properties
\end{lemmatheorembox}

\textbf{Usage count:} 1

\section*{Lemma 704}
\begin{lemmatheorembox}
\textbf{Name:} Direct proportionality
\textbf{Statement:} Given an independent variable $x$ and a dependent variable $y$, $y$ is directly proportional to $x$[ 1 ] if there is a positive constant $k$ such that: $y=kx$.
\textbf{General Topic:} Proportionality (Mathematics)
\textbf{URL:} https://en.wikipedia.org/wiki/Proportionality_(mathematics)
\end{lemmatheorembox}

\textbf{Usage count:} 1

\section*{Lemma 705}
\begin{lemmatheorembox}
\textbf{Name:} Addition and subtraction
\textbf{Statement:} 
* even ± even = even;[ 1 ]  
* even ± odd = odd;  
* odd ± odd = even;
\textbf{General Topic:} Number Theory
\textbf{URL:} https://en.wikipedia.org/wiki/Parity_(mathematics)#Addition_and_subtraction
\end{lemmatheorembox}

\textbf{Usage count:} 1

\section*{Lemma 706}
\begin{lemmatheorembox}
\textbf{Name:} Multiplication
\textbf{Statement:} 
* even × even = even;  
* even × odd = even;  
* odd × odd = odd.
\textbf{General Topic:} Number Theory
\textbf{URL:} https://en.wikipedia.org/wiki/Parity_(mathematics)#Multiplication
\end{lemmatheorembox}

\textbf{Usage count:} 1

\section*{Lemma 707}
\begin{lemmatheorembox}
\textbf{Name:} Definition
\textbf{Statement:} An even number is an integer of the form $x=2k$ where k is an integer;[ 4 ] an odd number is an integer of the form $x=2k+1.$
\textbf{General Topic:} Number Theory
\textbf{URL:} https://en.wikipedia.org/wiki/Parity_(mathematics)#Definition
\end{lemmatheorembox}

\textbf{Usage count:} 1

\section*{Lemma 708}
\begin{lemmatheorembox}
\textbf{Name:} Subtraction principle
\textbf{Statement:} Similarly, for a given finite set S, and given another set A, if $A\subset S$, then $|A^{c}|=|S|-|A|$.
\textbf{General Topic:} Combinatorics
\textbf{URL:} https://en.wikipedia.org/wiki/Addition_principle\#Subtraction_principle
\end{lemmatheorembox}

\textbf{Usage count:} 1

\section*{Lemma 709}
\begin{lemmatheorembox}
\textbf{Name:} Pi
\textbf{Statement:} The circumference of a circle with radius r is 2π r.
\textbf{General Topic:} Geometry
\textbf{URL:} https://en.wikipedia.org/wiki/Pi
\end{lemmatheorembox}

\textbf{Usage count:} 1

\section*{Lemma 710}
\begin{lemmatheorembox}
\textbf{Name:} Turn (angle)
\textbf{Statement:} One turn is equal to 2π radians, 360 degrees or 400 gradians.
\textbf{General Topic:} Geometry
\textbf{URL:} https://en.wikipedia.org/wiki/Turn_(angle)
\end{lemmatheorembox}

\textbf{Usage count:} 1

\section*{Lemma 711}
\begin{lemmatheorembox}
\textbf{Name:} Circular uniform distribution
\textbf{Statement:} The probability density function (pdf) of the circular uniform distribution, e.g. with $\theta \in [0,2\pi )$, is:
$f_{UC}(\theta )={\frac {1}{2\pi }}.$
\textbf{General Topic:} Probability theory
\textbf{URL:} https://en.wikipedia.org/wiki/Circular_uniform_distribution
\end{lemmatheorembox}

\textbf{Usage count:} 1

\section*{Lemma 712}
\begin{lemmatheorembox}
\textbf{Name:} Trailing zero
\textbf{Statement:} The number of trailing zeros in a non-zero base-b integer n equals the exponent of the highest power of b that divides n.
\textbf{General Topic:} Number theory
\textbf{URL:} https://en.wikipedia.org/wiki/Trailing_zero
\end{lemmatheorembox}

\textbf{Usage count:} 1

\section*{Lemma 713}
\begin{lemmatheorembox}
\textbf{Name:} Proof by exhaustion
\textbf{Statement:} Proof by exhaustion, also known as proof by cases, proof by case analysis, complete induction or the brute force method, is a method of mathematical proof in which the statement to be proved is split into a finite number of cases or sets of equivalent cases, and where each type of case is checked to see if the proposition in question holds.
\textbf{General Topic:} Mathematical logic
\textbf{URL:} https://en.wikipedia.org/wiki/Proof_by_exhaustion
\end{lemmatheorembox}

\textbf{Usage count:} 1

\section*{Lemma 714}
\begin{lemmatheorembox}
\textbf{Name:} Order of operations
\textbf{Statement:} For example, multiplication is granted a higher precedence than addition, and it has been this way since the introduction of modern algebraic notation.
\textbf{General Topic:} Arithmetic
\textbf{URL:} https://en.wikipedia.org/wiki/Order_of_operations
\end{lemmatheorembox}

\textbf{Usage count:} 1

\section*{Lemma 715}
\begin{lemmatheorembox}
\textbf{Name:} The indiscernibility of identicals
\textbf{Statement:} The indiscernibility of identicals: \forall x \, \forall y \, [x=y \rightarrow \forall F(Fx \leftrightarrow Fy)]\newline
For any x and y, if x is identical to y, then x and y have all the same properties.
\textbf{General Topic:} Philosophical logic
\textbf{URL:} https://en.wikipedia.org/wiki/Identity_of_indiscernibles
\end{lemmatheorembox}

\textbf{Usage count:} 1

\section*{Lemma 716}
\begin{lemmatheorembox}
\textbf{Name:} Rectangular cuboid
\textbf{Statement:} If a rectangular cuboid has length a, width b, and height c, then: its volume is the product of the rectangular area and its height: V=abc.
\textbf{General Topic:} Geometry
\textbf{URL:} https://en.wikipedia.org/wiki/Rectangular_cuboid
\end{lemmatheorembox}

\textbf{Usage count:} 1

\section*{Lemma 717}
\begin{lemmatheorembox}
\textbf{Name:} Line coordinates
\textbf{Statement:} There are several possible ways to specify the position of a line in the plane. A simple way is by the pair (m, b) where the equation of the line is y = mx + b. Here m is the slope and b is the y -intercept.
\textbf{General Topic:} Analytic geometry
\textbf{URL:} https://en.wikipedia.org/wiki/Line_coordinates
\end{lemmatheorembox}

\textbf{Usage count:} 1

\section*{Lemma 718}
\begin{lemmatheorembox}
\textbf{Name:} Translation (geometry)
\textbf{Statement:} In Euclidean geometry, a translation is a geometric transformation that moves every point of a figure, shape or space by the same distance in a given direction. A translation can also be interpreted as the addition of a constant vector to every point, or as shifting the origin of the coordinate system. In a Euclidean space, any translation is an isometry.
\textbf{General Topic:} Euclidean geometry
\textbf{URL:} https://en.wikipedia.org/wiki/Translation_%28geometry%29
\end{lemmatheorembox}

\textbf{Usage count:} 1

\section*{Lemma 719}
\begin{lemmatheorembox}
\textbf{Name:} Y-intercept
\textbf{Statement:} Analogously, an x-intercept is a point where the graph of a function or relation intersects with the x-axis. As such, these points satisfy y = 0.
\textbf{General Topic:} Analytic geometry
\textbf{URL:} https://en.wikipedia.org/wiki/Y-intercept
\end{lemmatheorembox}

\textbf{Usage count:} 1

\section*{Lemma 720}
\begin{lemmatheorembox}
\textbf{Name:} Material conditional
\textbf{Statement:} When the conditional symbol $\to$ is interpreted as material implication, a formula $P\to Q$ is true unless $P$ is true and $Q$ is false.
\textbf{General Topic:} Logic
\textbf{URL:} https://en.wikipedia.org/wiki/Material_conditional
\end{lemmatheorembox}

\textbf{Usage count:} 1

\section*{Lemma 721}
\begin{lemmatheorembox}
\textbf{Name:} Division (mathematics)\\
\textbf{Statement:} The result of dividing two rational numbers is another rational number when the divisor is not 0. The division of two rational numbers p/q and r/s can be computed as $\frac{p/q}{r/s}=\frac{p}{q}\times\frac{s}{r}=\frac{ps}{qr}$.\\
\textbf{General Topic:} Arithmetic\\
\textbf{URL:} https://en.wikipedia.org/wiki/Division_(mathematics)
\end{lemmatheorembox}

\textbf{Usage count:} 1

\section*{Lemma 722}
\begin{lemmatheorembox}
\textbf{Name:} Associative property
\textbf{Statement:} In mathematics, the associative property[ 1 ] is a property of some binary operations that rearranging the parentheses in an expression will not change the result. In propositional logic, associativity is a valid rule of replacement for expressions in logical proofs.
\textbf{General Topic:} Elementary algebra
\textbf{URL:} https://en.wikipedia.org/wiki/Associativity
\end{lemmatheorembox}

\textbf{Usage count:} 1

\section*{Lemma 723}
\begin{lemmatheorembox}
\textbf{Name:} Power of 10
\textbf{Statement:} In decimal notation the n th power of ten is written as '1' followed by n zeroes.
\textbf{General Topic:} Decimal notation
\textbf{URL:} https://en.wikipedia.org/wiki/Power_of_10
\end{lemmatheorembox}

\textbf{Usage count:} 1

\section*{Lemma 724}
\begin{lemmatheorembox}
\textbf{Name:} Pie chart
\textbf{Statement:} The size of each central angle is proportional to the size of the corresponding quantity, here the number of seats. Since the sum of the central angles has to be 360°, the central angle for a quantity that is a fraction Q of the total is 360 Q degrees.
\textbf{General Topic:} Statistics
\textbf{URL:} https://en.wikipedia.org/wiki/Pie_chart
\end{lemmatheorembox}

\textbf{Usage count:} 1

\section*{Lemma 725}
\begin{lemmatheorembox}
\textbf{Name:} Square function preserves the order of positive numbers
\textbf{Statement:} The square function preserves the order of positive numbers: larger numbers have larger squares. In other words, the square is a monotonic function on the interval [0, +∞).
\textbf{General Topic:} Real Analysis
\textbf{URL:} https://en.wikipedia.org/wiki/Square_%28algebra%29
\end{lemmatheorembox}

\textbf{Usage count:} 1

\section*{Lemma 726}
\begin{lemmatheorembox}
\textbf{Name:} Natural logarithm of 2
\textbf{Statement:} $\ln 2 = \sum_{n=1}^\infty \frac{(-1)^{n+1}}{n}=1-\frac12+\frac13-\frac14+\frac15-\frac16+\cdots.$
\textbf{General Topic:} Mathematical analysis
\textbf{URL:} https://en.wikipedia.org/wiki/Natural_logarithm_of_2
\end{lemmatheorembox}

\textbf{Usage count:} 1

\section*{Lemma 727}
\begin{lemmatheorembox}
\textbf{Name:} Arctangent series
\textbf{Statement:} $\arctan x=x-{\frac {x^{3}}{3}}+{\frac {x^{5}}{5}}-{\frac {x^{7}}{7}}+\cdots =\sum _{k=0}^{\infty }{\frac {(-1)^{k}x^{2k+1}}{2k+1}}.$
\textbf{General Topic:} Mathematical analysis
\textbf{URL:} https://en.wikipedia.org/wiki/Arctangent_series
\end{lemmatheorembox}

\textbf{Usage count:} 1

\section*{Lemma 728}
\begin{lemmatheorembox}
\textbf{Name:} Power-reduction formulae
\textbf{Statement:} $\cos^2\theta = \frac{1 + \cos (2\theta)}{2}$
\textbf{General Topic:} Trigonometry
\textbf{URL:} https://en.wikipedia.org/wiki/List_of_trigonometric_identities#Power-reduction_formulae
\end{lemmatheorembox}

\textbf{Usage count:} 1

\section*{Lemma 729}
\begin{lemmatheorembox}
\textbf{Name:} Sum-to-product identities
\textbf{Statement:} $\begin{align}
\sin \theta +\sin \varphi &=2\sin {\tfrac {1}{2}}(\theta +\varphi )\,\cos {\tfrac {1}{2}}(\theta -\varphi ),\\[5mu]
\sin \theta -\sin \varphi &=2\cos {\tfrac {1}{2}}(\theta +\varphi )\,\sin {\tfrac {1}{2}}(\theta -\varphi ),\\[5mu]
\cos \theta +\cos \varphi &=2\cos {\tfrac {1}{2}}(\theta +\varphi )\,\cos {\tfrac {1}{2}}(\theta -\varphi ),\\[5mu]
\cos \theta -\cos \varphi &=-2\sin {\tfrac {1}{2}}(\theta +\varphi )\,\sin {\tfrac {1}{2}}(\theta -\varphi ).
\end{align}$
\textbf{General Topic:} Trigonometry
\textbf{URL:} https://en.wikipedia.org/wiki/List_of_trigonometric_identities#Sum-to-product_identities
\end{lemmatheorembox}

\textbf{Usage count:} 1

\section*{Lemma 730}
\begin{lemmatheorembox}
\textbf{Name:} Limit of a function
\textbf{Statement:} $\displaystyle \lim_{x \to 0} \frac{\sin x}{x} = 1$
\textbf{General Topic:} Calculus
\textbf{URL:} https://en.wikipedia.org/wiki/Limit_of_a_function
\end{lemmatheorembox}

\textbf{Usage count:} 1

\section*{Lemma 731}
\begin{lemmatheorembox}
\textbf{Name:} First Isomorphism Theorem
\textbf{Statement:} Let V,W be K-Vector Spaces and T:V->W linear. Define the map $\overline T:V/\ker T\to \operatorname {im} (T)$ by $\overline T([v])=T(v)$.  Then $\overline T$ is well-defined and an isomorphism.
\textbf{General Topic:} Linear Algebra
\textbf{URL:} https://en.wikipedia.org/wiki/Quotient_space_%28linear_algebra%29
\end{lemmatheorembox}

\textbf{Usage count:} 1

\section*{Lemma 732}
\begin{lemmatheorembox}
\textbf{Name:} Arithmetic mean
\textbf{Statement:} The arithmetic mean, also known as "arithmetic average", is the sum of the values divided by the number of values.
\textbf{General Topic:} Statistics
\textbf{URL:} https://en.wikipedia.org/wiki/Mean
\end{lemmatheorembox}

\textbf{Usage count:} 1

\section*{Lemma 733}
\begin{lemmatheorembox}
\textbf{Name:} Product rule
\textbf{Statement:} $(f \cdot g)'(x) = f'(x) \cdot g(x) + f(x) \cdot g'(x).$
\textbf{General Topic:} Calculus
\textbf{URL:} https://en.wikipedia.org/wiki/Formal_derivative
\end{lemmatheorembox}

\textbf{Usage count:} 1

\section*{Lemma 734}
\begin{lemmatheorembox}
\textbf{Name:} Triangle inequality
\textbf{Statement:} $\|\mathbf {u} +\mathbf {v} \|\leq \|\mathbf {u} \|+\|\mathbf {v} \|\quad \forall \,\mathbf {u} ,\mathbf {v} \in V.$
\textbf{General Topic:} Mathematical Analysis
\textbf{URL:} https://en.wikipedia.org/wiki/Triangle_inequality#Normed_vector_space
\end{lemmatheorembox}

\textbf{Usage count:} 1

\section*{Lemma 735}
\begin{lemmatheorembox}
\textbf{Name:} Absolute convergence
\textbf{Statement:} When a series of real or complex numbers is absolutely convergent, any rearrangement or reordering of that series' terms will still converge to the same value.
\textbf{General Topic:} Mathematical analysis
\textbf{URL:} https://en.wikipedia.org/wiki/Absolute_convergence
\end{lemmatheorembox}

\textbf{Usage count:} 1

\section*{Lemma 736}
\begin{lemmatheorembox}
\textbf{Name:} Law of the unconscious statistician
\textbf{Statement:} The form of the law depends on the type of random variable $X$ in question. If the distribution of $X$ is discrete and one knows its probability mass function $p_X$, then the expected value of $g(X)$ is $\operatorname{E}[g(X)]=\sum_x g(x)p_X(x),\,$ where the sum is over all possible values $x$ of $X$.
\textbf{General Topic:} Probability theory
\textbf{URL:} https://en.wikipedia.org/wiki/Law_of_the_unconscious_statistician
\end{lemmatheorembox}

\textbf{Usage count:} 1

\section*{Lemma 737}
\begin{lemmatheorembox}
\textbf{Name:} Steinitz exchange lemma
\textbf{Statement:} Let U and W be finite subsets of a vector space V. If U is a set of linearly independent vectors, and W spans V, then:
1. |U| \leq |W|;
2. There is a set W' \subseteq W with |W'|=|W|-|U| such that U \cup W' spans V.
\textbf{General Topic:} Linear algebra
\textbf{URL:} https://en.wikipedia.org/wiki/Steinitz_exchange_lemma
\end{lemmatheorembox}

\textbf{Usage count:} 1

\section*{Lemma 738}
\begin{lemmatheorembox}
\textbf{Name:} Trial division
\textbf{Statement:} The essential idea behind trial division tests to see if an integer n, the integer to be factored, can be divided by each number in turn that is less than or equal to the square root of n.
\textbf{General Topic:} Number theory
\textbf{URL:} https://en.wikipedia.org/wiki/Trial_division
\end{lemmatheorembox}

\textbf{Usage count:} 1

\section*{Lemma 739}
\begin{lemmatheorembox}
\textbf{Name:} Heine–Borel theorem
\textbf{Statement:} For a subset $S$ of Euclidean space $\mathbb {R} ^{n}$, the following two statements are equivalent:
* $S$ is compact, that is, every open cover of $S$ has a finite subcover
* $S$ is closed and bounded.
\textbf{General Topic:} Real analysis
\textbf{URL:} https://en.wikipedia.org/wiki/Heine%E2%80%93Borel_theorem
\end{lemmatheorembox}

\textbf{Usage count:} 1

\section*{Lemma 740}
\begin{lemmatheorembox}
\textbf{Name:} Examples of vector spaces
\textbf{Statement:} Then any n-dimensional vector space V over F\_q will have q\^n elements.
\textbf{General Topic:} Linear algebra
\textbf{URL:} https://en.wikipedia.org/wiki/Examples_of_vector_spaces
\end{lemmatheorembox}

\textbf{Usage count:} 1

\section*{Lemma 741}
\begin{lemmatheorembox}
\textbf{Name:} Closed set
\textbf{Statement:} If $f:X\to Y$ is a function between topological spaces then $f$ is continuous if and only if preimages of closed sets in $Y$ are closed in $X$.
\textbf{General Topic:} Topology
\textbf{URL:} https://en.wikipedia.org/wiki/Closed_set
\end{lemmatheorembox}

\textbf{Usage count:} 1

\section*{Lemma 742}
\begin{lemmatheorembox}
\textbf{Name:} Quotient rule
\textbf{Statement:} In calculus, the quotient rule is a method of finding the derivative of a function that is the ratio of two differentiable functions. Let $h(x)={\frac {f(x)}{g(x)}}$, where both $f$ and $g$ are differentiable and $g(x)\neq 0.$ The quotient rule states that the derivative of $h(x)$ is $h'(x) = {\frac {f'(x)g(x)-f(x)g'(x)}{(g(x))^{2}}}.$
\textbf{General Topic:} Calculus
\textbf{URL:} https://en.wikipedia.org/wiki/Quotient_rule
\end{lemmatheorembox}

\textbf{Usage count:} 1

\section*{Lemma 743}
\begin{lemmatheorembox}
\textbf{Name:} Inverse trigonometric functions
\textbf{Statement:} ${\frac {d}{dz}}\arctan(z) = {\frac {1}{1+z^{2}}}\; ;\; z\neq -i,+i$.
\textbf{General Topic:} Calculus
\textbf{URL:} https://en.wikipedia.org/wiki/Inverse_trigonometric_functions
\end{lemmatheorembox}

\textbf{Usage count:} 1

\section*{Lemma 744}
\begin{lemmatheorembox}
\textbf{Name:} Moore-Osgood theorem
\textbf{Statement:} $\displaystyle \lim _{y\to b}\lim _{x\to a}f(x,y)=\lim _{x\to a}\lim _{y\to b}f(x,y)=\lim _{\begin{smallmatrix}x\to a\\y\to b\end{smallmatrix}}f(x,y).$
\textbf{General Topic:} Mathematical analysis
\textbf{URL:} https://en.wikipedia.org/wiki/Iterated_limit
\end{lemmatheorembox}

\textbf{Usage count:} 1

\section*{Lemma 745}
\begin{lemmatheorembox}
\textbf{Name:} Minimax theorem
\textbf{Statement:} $\max_{x\in X} \min_{y\in Y} f(x,y) = \min_{y\in Y} \max_{x\in X}f(x,y)$
\textbf{General Topic:} Game theory; Convex optimization
\textbf{URL:} https://en.wikipedia.org/wiki/Minimax_theorem
\end{lemmatheorembox}

\textbf{Usage count:} 1

\section*{Lemma 746}
\begin{lemmatheorembox}
\textbf{Name:} Absorbing element
\textbf{Statement:} Formally, let (S, •) be a set S with a closed binary operation • on it (known as a magma). A zero element (or an absorbing/annihilating element) is an element z such that for all s in S, z • s = s • z = z.
\textbf{General Topic:} Semigroup theory
\textbf{URL:} https://en.wikipedia.org/wiki/Absorbing_element
\end{lemmatheorembox}

\textbf{Usage count:} 1

\section*{Lemma 747}
\begin{lemmatheorembox}
\textbf{Name:} Equivalences
\textbf{Statement:} {\begin{alignedat}{3}\lfloor x\rfloor &=m\ \ &&{\mbox{ if and only if }}&m&\leq x<m+1,\\\lceil x\rceil &=n&&{\mbox{ if and only if }}&\ \ n-1&<x\leq n,\\\lfloor x\rfloor &=m&&{\mbox{ if and only if }}&x-1&<m\leq x,\\\lceil x\rceil &=n&&{\mbox{ if and only if }}&x&\leq n<x+1.\end{alignedat}}
{\begin{aligned}x<n&\;\;{\mbox{ if and only if }}&\lfloor x\rfloor &<n,\\n<x&\;\;{\mbox{ if and only if }}&n&<\lceil x\rceil ,\\x\leq n&\;\;{\mbox{ if and only if }}&\lceil x\rceil &\leq n,\\n\leq x&\;\;{\mbox{ if and only if }}&n&\leq \lfloor x\rfloor .\end{aligned}}
\textbf{General Topic:} Elementary Number Theory
\textbf{URL:} https://en.wikipedia.org/wiki/Floor_and_ceiling_functions#Equivalences
\end{lemmatheorembox}

\textbf{Usage count:} 1

\section*{Lemma 748}
\begin{lemmatheorembox}
\textbf{Name:} Monotonicity
\textbf{Statement:} Both floor and ceiling functions are monotonically non-decreasing functions:
{\begin{aligned}x_{1}\leq x_{2}&\Rightarrow \lfloor x_{1}\rfloor \leq \lfloor x_{2}\rfloor ,\\x_{1}\leq x_{2}&\Rightarrow \lceil x_{1}\rceil \leq \lceil x_{2}\rceil .\end{aligned}}
\textbf{General Topic:} Real Analysis
\textbf{URL:} https://en.wikipedia.org/wiki/Floor_and_ceiling_functions#Monotonicity
\end{lemmatheorembox}

\textbf{Usage count:} 1

\section*{Lemma 749}
\begin{lemmatheorembox}
\textbf{Name:} Negative base
\textbf{Statement:} Denoting the base as −r, every integer a can be written uniquely as \(a=\sum _{i=0}^{n}d_{i}(-r)^{i}\) where each digit \(d_{k}\) is an integer from 0 to \(r-1\) and the leading digit \(d_{n}>0\) (unless \(n=0\)). The base −r expansion of a is then given by the string \(d_{n}d_{n-1}\ldots d_{1}d_{0}\).
\textbf{General Topic:} Numeral systems
\textbf{URL:} https://en.wikipedia.org/wiki/Negative_base
\end{lemmatheorembox}

\textbf{Usage count:} 1

\section*{Lemma 750}
\begin{lemmatheorembox}
\textbf{Name:} 142857
\textbf{Statement:} 142857 is the best-known cyclic number in base 10, being the six repeating digits of 1/7 (0.142857).
\textbf{General Topic:} Number Theory
\textbf{URL:} https://en.wikipedia.org/wiki/142857
\end{lemmatheorembox}

\textbf{Usage count:} 1

\end{document}
