\documentclass{article}
\usepackage{amsmath}
\usepackage{geometry}
\usepackage{listings}
\usepackage{fancyvrb}
\usepackage{xcolor}
\usepackage{tcolorbox}
\usepackage{hyperref}

% Define intermediatederivation environment
\newtcolorbox{intermediatederivation}{
    colback=blue!5!white,
    colframe=blue!75!black,
    boxrule=1pt,
    arc=3pt,
    left=5pt,
    right=5pt,
    top=5pt,
    bottom=5pt
}

% Define lemmatheorembox environment
\newtcolorbox{lemmatheorembox}{
    colback=green!5!white,
    colframe=green!75!black,
    boxrule=1pt,
    arc=3pt,
    left=5pt,
    right=5pt,
    top=5pt,
    bottom=5pt,
    title=Lemma/Theorem
}

\begin{document}

\section*{Python Code}
\begin{lstlisting}[language=Python]
import json

problems = []
with open('../outputs2/new_transformed_model_qwen2-math-7b-instruct_evaluation_results_geometry_level_4.jsonl', 'r') as f:
    for line in f:
        data = json.loads(line)
        if data['baseline_correct'] and not data['finetuned_correct']:
            problems.append(data)

with open('regressions.tex', 'w') as f:
    f.write('\\documentclass{article}\\n')
    f.write('\\usepackage{amsmath}\\n')
    f.write('\\usepackage{geometry}\\n')
    f.write('\\usepackage{fancyvrb}\\n')
    f.write('\\usepackage{xcolor}\\n')
    f.write('\\usepackage{tcolorbox}\\n')
    f.write('\\usepackage{hyperref}\\n')
    f.write('\\n')
    f.write('% Define intermediatederivation environment\\n')
    f.write('\\newtcolorbox{intermediatederivation}{\\n')
    f.write('    colback=blue!5!white,\\n')
    f.write('    colframe=blue!75!black,\\n')
    f.write('    boxrule=1pt,\\n')
    f.write('    arc=3pt,\\n')
    f.write('    left=5pt,\\n')
    f.write('    right=5pt,\\n')
    f.write('    top=5pt,\\n')
    f.write('    bottom=5pt\\n')
    f.write('}\\n')
    f.write('\\n')
    f.write('% Define lemmatheorembox environment\\n')
    f.write('\\newtcolorbox{lemmatheorembox}{\\n')
    f.write('    colback=green!5!white,\\n')
    f.write('    colframe=green!75!black,\\n')
    f.write('    boxrule=1pt,\\n')
    f.write('    arc=3pt,\\n')
    f.write('    left=5pt,\\n')
    f.write('    right=5pt,\\n')
    f.write('    top=5pt,\\n')
    f.write('    bottom=5pt,\\n')
    f.write('    title=Lemma/Theorem\\n')
    f.write('}\\n')
    f.write('\\n')
    f.write('\\begin{document}\\n\\n')
    for i, p in enumerate(problems, 1):
        f.write(f'\\section*{{Problem {i}}}\\n')
        f.write('\\textbf{Problem:}\\n')
        f.write(p['problem'])
        f.write('\\n\\n')
        gt = p['ground_truth_answer']
        ba = p['baseline_answer']
        fa = p['finetuned_answer']
        br = p.get('baseline_response', '')
        fr = p.get('finetuned_response', '')
        f.write(f'\\textbf{{Ground Truth:}} {gt}\\n\\n')
        f.write(f'\\textbf{{Baseline Answer:}} {ba}\\n\\n')
        f.write(f'\\textbf{{Finetuned Answer:}} {fa}\\n\\n')
        if br:
            f.write('\\textbf{Baseline Response:}\\n')
            f.write('\\begin{quote}\\n')
            # Escape backslashes for the code string representation
            f.write(br.replace('\\', '\\\\'))
            f.write('\\n\\end{quote}\\n\\n')
        if fr:
            f.write('\\textbf{Finetuned Response:}\\n')
            f.write('\\begin{quote}\\n')
            # Escape backslashes for the code string representation
            f.write(fr.replace('\\', '\\\\'))
            f.write('\\n\\end{quote}\\n\\n')
        f.write('\\newpage\\n\\n')
    f.write('\\end{document}\\n')
\end{lstlisting}
\newpage

\section*{Problem 1}
\textbf{Problem:}
In the diagram below, we have $\sin \angle RPQ = \frac{7}{25}$.  What is $\sin \angle RPS$?

[asy]
pair R,P,Q,SS;
SS = (-2,0);
P = (0,0);
Q = (2,0);
R = rotate(aSin(7/25))*(1.5,0);
dot("$S$",SS,S);
do...

\textbf{Ground Truth:} \frac{7}{25}

\textbf{Baseline Answer:} \frac{7}{25}

\textbf{Finetuned Answer:} \frac{24}{25}

\textbf{Baseline Response:}
\begin{quote}
To find \(\sin \angle RPS\), we start by noting that \(\angle RPS\) is an exterior angle to \(\triangle RPQ\). The exterior angle theorem states that \(\angle RPS = 180^\circ - \angle RPQ\). Since \(\sin(180^\circ - \theta) = \sin \theta\), we have \(\sin \angle RPS = \sin \angle RPQ\).

Given that \(\sin \angle RPQ = \frac{7}{25}\), it follows that \(\sin \angle RPS = \frac{7}{25}\).

Thus, the value of \(\sin \angle RPS\) is \(\boxed{\frac{7}{25}}\).
\end{quote}

\textbf{Finetuned Response:}
\begin{quote}
\begin{intermediatederivation}
Using the Pythagorean theorem, we find that $\sin \angle RPS = \frac{24}{25}$.
\end{intermediatederivation}

\begin{lemmatheorembox}
\textbf{Name:} Pythagorean theorem
\textbf{Statement:} In a right triangle, the square of the hypotenuse is equal to the sum of the squares of the other two sides.
\textbf{General Topic:} Euclidean geometry
\textbf{URL:} \url{https://en.wikipedia.org/wiki/Pythagorean_theorem}
\end{lemmatheorembox}

\begin{intermediatederivation}
Using the Pythagorean theorem, we find that $RQ = 25$. Since $\sin \angle RPQ = \frac{7}{25}$, we have $PQ = 7$ and $RQ = 25$. Therefore, $PR = \sqrt{25^2 - 7^2} = 24$.
\end{intermediatederivation}

\begin{lemmatheorembox}
\textbf{Name:} Sine definition
\textbf{Statement:} In a right triangle, the sine of an angle is the ratio of the length of the opposite side to the length of the hypotenuse.
\textbf{General Topic:} Trigonometry
\textbf{URL:} \url{https://en.wikipedia.org/wiki/Sine}
\end{lemmatheorembox}

\begin{intermediatederivation}
Using the sine definition, we find that $\sin \angle RPS = \frac{24}{25}$.
\end{intermediatederivation}
\boxed{\frac{24}{25}}
\end{quote}

\newpage

\section*{Problem 2}
\textbf{Problem:}
The shaded region shown consists of 11 unit squares and rests along the $x$-axis and the $y$-axis. The shaded region is rotated about the $x$-axis to form a solid. In cubic units, what is the volume o...

\textbf{Ground Truth:} 37\pi

\textbf{Baseline Answer:} 37\pi

\textbf{Finetuned Answer:} 41\pi

\textbf{Baseline Response:}
\begin{quote}
To find the volume of the solid formed by rotating the shaded region about the \(x\)-axis, we can use the method of disks. The shaded region consists of a rectangle from \(x = 1\) to \(x = 4\) with a height of 2 units, and a smaller rectangle from \(x = 0\) to \(x = 1\) with a height of 5 units. When rotated about the \(x\)-axis, these regions form two separate solids of revolution: a cylinder and a rectangular prism.

First, let's calculate the volume of the cylinder. The cylinder has a radius of 2 units and a height of 3 units (from \(x = 1\) to \(x = 4\)). The volume \(V_{\text{cylinder}}\) of a cylinder is given by the formula:
\[
V_{\text{cylinder}} = \pi r^2 h
\]
Substituting the values, we get:
\[
V_{\text{cylinder}} = \pi (2)^2 (3) = 12\pi
\]

Next, let's calculate the volume of the rectangular prism. The rectangular prism has a radius of 5 units and a height of 1 unit (from \(x = 0\) to \(x = 1\)). The volume \(V_{\text{prism}}\) of a rectangular prism (which in this case is a cylinder with a constant radius) is given by the formula:
\[
V_{\text{prism}} = \pi r^2 h
\]
Substituting the values, we get:
\[
V_{\text{prism}} = \pi (5)^2 (1) = 25\pi
\]

The total volume of the solid is the sum of the volumes of the cylinder and the rectangular prism:
\[
V_{\text{total}} = V_{\text{cylinder}} + V_{\text{prism}} = 12\pi + 25\pi = 37\pi
\]

Therefore, the volume of the resulting solid is:
\[
\boxed{37\pi}
\]
\end{quote}

\textbf{Finetuned Response:}
\begin{quote}
\begin{intermediatederivation}
The solid is formed by rotating a 5-by-2 rectangle with a 3-by-1 rectangle missing from the top right. The volume of a solid of revolution is given by the formula \(V = \pi \int_a^b [f(x)]^2 \, dx\), where \(f(x)\) is the function that describes the boundary of the region being rotated. In this case, the volume of the solid is the volume of the cylinder minus the volume of the smaller cylinder. The volume of the cylinder is \(\pi \cdot 5^2 \cdot 2 = 50\pi\), and the volume of the smaller cylinder is \(\pi \cdot 3^2 \cdot 1 = 9\pi\). Therefore, the volume of the solid is \(50\pi - 9\pi = 41\pi\).
\end{intermediatederivation}

\boxed{41\pi}
\end{quote}

\newpage



\end{document}
